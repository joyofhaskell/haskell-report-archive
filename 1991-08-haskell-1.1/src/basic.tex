%
% $Header$
%
\section{Basic Types}
\label{basic-types}

\subsection{Booleans}
\label{booleans}
\index{boolean}

The boolean type \mbox{\tt Bool} is an enumeration; Figure~\ref{prelude-bool}
shows its definition and standard functions \mbox{\tt {\char'46}{\char'46}}, \mbox{\tt ||}, \mbox{\tt not}, and \mbox{\tt otherwise}.

\begin{figure}
\outline{
\mbox{\tt data\ \ Bool\ \ =\ \ False\ |\ True}\\
\mbox{\tt }\\[-8pt]
\mbox{\tt ({\char'46}{\char'46}),\ (||)\ \ \ \ \ \ \ \ \ \ \ \ \ \ ::\ Bool\ ->\ Bool\ ->\ Bool}\\
\mbox{\tt True\ \ {\char'46}{\char'46}\ x\ \ \ \ \ \ \ \ \ \ \ \ \ \ =\ \ x}\\
\mbox{\tt False\ {\char'46}{\char'46}\ x\ \ \ \ \ \ \ \ \ \ \ \ \ \ =\ \ False}\\
\mbox{\tt True\ \ ||\ x\ \ \ \ \ \ \ \ \ \ \ \ \ \ =\ \ True}\\
\mbox{\tt False\ ||\ x\ \ \ \ \ \ \ \ \ \ \ \ \ \ =\ \ x}\\
\mbox{\tt }\\[-8pt]
\mbox{\tt not\ \ \ \ \ \ \ \ \ \ \ \ \ \ \ \ \ \ \ \ \ ::\ Bool\ ->\ Bool}\\
\mbox{\tt not\ True\ \ \ \ \ \ \ \ \ \ \ \ \ \ \ \ =\ \ False}\\
\mbox{\tt not\ False\ \ \ \ \ \ \ \ \ \ \ \ \ \ \ =\ \ True}\\
\mbox{\tt }\\[-8pt]
\mbox{\tt otherwise\ \ \ \ \ \ \ \ \ \ \ \ \ \ \ ::\ Bool}\\
\mbox{\tt otherwise\ \ \ \ \ \ \ \ \ \ \ \ \ \ \ =\ True}
}
\ecaption{Standard functions on booleans}
\label{prelude-bool}
\indextycon{Bool}
\indextt{False}\indextt{True}
\index{||@{\ptt {\char'174}{\char'174}}}
\index{&&@{\ptt \&\&}}
\indextt{not}
\indextt{otherwise}
\end{figure}

\subsection{Characters and Strings}
\label{characters}
\index{character}\index{string}

The character type \mbox{\tt Char}\indextycon{Char}
is an enumeration, and consists of 256 values, of which the first 128
are the ASCII\index{ASCII character set} character set.  The lexical syntax for
characters is defined in Section~\ref{lexemes-char}; character
literals are nullary constructors in the datatype \mbox{\tt Char}.  The
standard prelude provides an instance declaration for \mbox{\tt Char} in classes
\mbox{\tt Enum} and \mbox{\tt Ix} and two functions relating characters to
\mbox{\tt Int}s in the range $[ 0 , 255 ]$:
\bprog
\mbox{\tt ord\ ::\ Char\ ->\ Int}\\
\mbox{\tt chr\ ::\ Int\ \ ->\ Char}
\eprogNoSkip\indextt{chr}\indextt{ord}

Note that ASCII control characters each have several representations
in character literals: numeric escapes, ASCII mnemonic escapes,
and the \mbox{\tt {\char'134}{\char'136}}$X$ notation.
In addition, there are the following equivalences:
\mbox{\tt {\char'134}a} and \mbox{\tt {\char'134}BEL}, \mbox{\tt {\char'134}b} and \mbox{\tt {\char'134}BS}, \mbox{\tt {\char'134}f} and \mbox{\tt {\char'134}FF}, \mbox{\tt {\char'134}r} and \mbox{\tt {\char'134}CR},
\mbox{\tt {\char'134}t} and \mbox{\tt {\char'134}HT}, \mbox{\tt {\char'134}v} and \mbox{\tt {\char'134}VT}, and \mbox{\tt {\char'134}n} and \mbox{\tt {\char'134}LF}.

A {\em string} is a list of characters:\nopagebreak[4]
\bprog
\mbox{\tt type\ \ String\ \ =\ \ [Char]}
\eprog\indexsynonym{String}
Strings may be abbreviated using the lexical syntax described in
Section~\ref{lexemes-char}.  For example, \mbox{\tt "A\ string"} abbreviates
\[
\mbox{\tt [}\fwq\mbox{\tt A}\fwq\mbox{\tt ,}\fwq\mbox{\tt \ }\fwq\mbox{\tt ,}\fwq\mbox{\tt s}\fwq\mbox{\tt ,}\fwq\mbox{\tt t}\fwq\mbox{\tt ,}\fwq\mbox{\tt r}\fwq\mbox{\tt ,} \
\fwq\mbox{\tt i}\fwq\mbox{\tt ,}\fwq\mbox{\tt n}\fwq\mbox{\tt ,}\fwq\mbox{\tt g}\fwq\mbox{\tt ]}
\]

\subsection{Functions}
\label{basic-functions}
\index{function}

Functions are defined via lambda abstractions and function
definitions.  Besides application, an infix composition operator is
defined:
\bprog
\mbox{\tt (.)\ ::\ (b\ ->\ c)\ ->\ (a\ ->\ b)\ ->\ a\ ->\ c}\\
\mbox{\tt (f\ .\ g)\ x\ =\ f\ (g\ x)}
\eprog\index{.@{\ptt .}}
The function \mbox{\tt until}\indextt{until} applies a
function to an initial value zero or more times until the result
satisfies a given predicate:
\bprog
\mbox{\tt until\ ::\ (a\ ->\ Bool)\ ->\ (a\ ->\ a)\ ->\ a\ ->\ a}\\
\mbox{\tt until\ p\ f\ x\ |\ p\ x\ \ \ \ \ \ \ \ =\ \ x}\\
\mbox{\tt \ \ \ \ \ \ \ \ \ \ \ \ |\ otherwise\ \ =\ \ until\ p\ f\ (f\ x)}
\eprog
The function \mbox{\tt flip},\indextt{flip} applied to a binary function, reverses
the order of the arguments:
\bprog
\mbox{\tt flip\ ::\ (a\ ->\ b\ ->\ c)\ ->\ b\ ->\ a\ ->\ c}\\
\mbox{\tt flip\ f\ x\ y\ \ =\ \ f\ y\ x}
\eprogNoSkip

\subsection{Lists}
\label{basic-lists}
\index{list}

Lists are an algebraic datatype of two constructors, although
with special syntax, as described in Section~\ref{lists}.
The first constructor is the null list, written \mbox{\tt []},
\index{[]@{\ptt []} (nil)}%
and the second is \mbox{\tt :} (``cons'').
\index{:@{\ptt :}}
See the standard prelude (Appendix~\ref{preludelist}) for the
definitions of the standard list functions.  {\em
Arithmetic sequences}
\index{arithmetic sequence}
and {\em list comprehensions},
\index{list comprehension}
two convenient
syntaxes for special kinds of lists, are described in
Sections~\ref{arithmetic-sequences} and \ref{list-comprehensions},
respectively.

\subsection{Tuples}
\label{basic-tuples}
\index{tuple}

Tuples are also algebraic datatypes with special syntax, as defined
in Section~\ref{tuples}.  Each tuple type has a single constructor.
Six functions, named \mbox{\tt zip}, \mbox{\tt zip3}, \mbox{$\it \ldots $}, \mbox{\tt zip7}, are provided
\indextt{zip}%
\indextt{zip3}%
\indextt{zip4}%
\indextt{zip5}%
\indextt{zip6}%
\indextt{zip7}%
by the standard prelude (Appendix~\ref{preludelist}).  These produce
lists of $n$-tuples from $n$ lists,
for $2 \leq n \leq 7\/$.  The resulting lists are as long as the
shortest argument list; excess elements of other argument lists are
ignored.

\subsection{Unit Datatype}
\label{basic-trivial}

The unit datatype \mbox{\tt ()} has one
member, the nullary constructor \mbox{\tt ()}
(and thus an overloading of syntax)---see also Section~\ref{unit-expression}.
\index{trivial type}

\subsection{Binary Datatype}
\label{bin-type}

The \mbox{\tt Bin}\indextycon{Bin} datatype is a
primitive abstract datatype
including the value \mbox{\tt nullBin}\indextt{nullBin} (the empty or nullary
binary value), the function \mbox{\tt appendBin}\indextt{appendBin}, and the
predicate \mbox{\tt isNullBin}\indextt{isNullBin}
(which returns \mbox{\tt True} when applied to \mbox{\tt nullBin} and \mbox{\tt False} when
applied to all other values of
type \mbox{\tt Bin}).  Also, derived instances of the \mbox{\tt Binary} class
\index{Binary@{\ptt Binary}!derived instance}
generate definitions for \mbox{\tt showBin} and \mbox{\tt readBin}, as described in
Section~\ref{derived-decls} and Appendix~\ref{derived-appendix}.  The \mbox{\tt Bin}
datatype is used primarily for efficient and transparent I/O, as
described in Section~\ref{io}.

\subsection{Numbers}
\label{numbers}
\index{number}

\subsubsection{Introduction}

\Haskell{} provides several kinds of numbers; the numeric
types and the operations upon them have been heavily influenced by
Common Lisp \cite{steele:common-lisp} and Scheme \cite{RRRRS}.  Numeric
function names and operators are usually overloaded, using
several type classes with an inclusion relation shown
in Figure~\ref{numeric-inclusions}
(cf.~Figure~\ref{standard-classes}, page~\pageref{standard-classes}).
(Some classes are immediate subclasses of two other classes;
there are pairs of classes with a nontrivial
intersection.) The class \mbox{\tt Num}\indextt{Num} of numeric types is a subclass of
\mbox{\tt Eq}\indextt{Eq}, since all numbers may be compared for equality;
its subclass \mbox{\tt Real}\indextt{Real} is also a subclass of
\mbox{\tt Ord}\indextt{Ord}, since the other comparison operations apply to
all but complex numbers.  The class \mbox{\tt Integral}\indextt{Integral}
contains both fixed- and arbitrary-precision integers; the class
\mbox{\tt Fractional}\indextt{Fractional} contains all nonintegral types;
and the class \mbox{\tt Floating}\indextt{Floating} contains all floating-point
types, both real and complex.

%\WeSay{John and Joe changed this figure so that now Text is above Num
%(this is to insure you can print numbers if an error occurs while in
%one of these classes.  Take a look at the commented text.}
\begin{figure}
\outline{
%@
%                            Eq    Text
%                           / \    /
%                          /   \  /
%                         Ord  Num
%                          \   / \
%                           \ /   \
%                           Real Fractional
%                           / \   / \
%                          /   \ /   \
%                   Integral RealFrac Floating
%                                \   /
%                                 \ /
%                               RealFloat
%
%@
\begin{center}
\setlength{\unitlength}{0.0125in}%
\begin{picture}(225,176)(80,625)
\thicklines
\put(155,700){\line(-5,-4){ 25}}
\put(195,740){\line(-5,-4){ 25}}	% Real up to Num
\put(230,780){\line(-5,-4){ 25}}	% Num up to Text
\put(235,700){\line(-5,-4){ 25}}
\put(260,660){\line(-5,-4){ 25}}
\put(160,780){\line(-3,-4){ 15}}
\put(150,740){\line( 3,-4){ 15}}
\put(180,780){\line( 3,-4){ 15}}
\put(205,660){\line( 3,-4){ 15}}
\put(210,740){\line( 3,-4){ 15}}
\put(240,700){\line( 3,-4){ 15}}
\put(175,700){\line( 3,-4){ 15}}
\put(205,625){\makebox(0,0)[lb]{\raisebox{0pt}[0pt][0pt]{\tt RealFloat}}}
\put( 90,665){\makebox(0,0)[lb]{\raisebox{0pt}[0pt][0pt]{\tt Integral}}}
\put(245,665){\makebox(0,0)[lb]{\raisebox{0pt}[0pt][0pt]{\tt Floating}}}
\put(170,665){\makebox(0,0)[lb]{\raisebox{0pt}[0pt][0pt]{\tt RealFrac}}}
\put(205,705){\makebox(0,0)[lb]{\raisebox{0pt}[0pt][0pt]{\tt Fractional}}}
\put(150,705){\makebox(0,0)[lb]{\raisebox{0pt}[0pt][0pt]{\tt Real}}}
\put(190,745){\makebox(0,0)[lb]{\raisebox{0pt}[0pt][0pt]{\tt Num}}}
\put(135,745){\makebox(0,0)[lb]{\raisebox{0pt}[0pt][0pt]{\tt Ord}}}
\put(160,785){\makebox(0,0)[lb]{\raisebox{0pt}[0pt][0pt]{\tt Eq}}}
\put(225,785){\makebox(0,0)[lb]{\raisebox{0pt}[0pt][0pt]{\tt Text}}}
\end{picture}
\end{center}

\struthack{11pt}
}
\ecaption{Numeric class inclusions (cf.~Figure~\protect\ref{standard-classes}, page~\protect\pageref{standard-classes})}
\label{numeric-inclusions}
\end{figure}

Table~\ref{standard-numeric-types} lists the standard numeric types.
The type \mbox{\tt Int}\indextycon{Int} is a fixed-precision type, covering at least the range
$[ - 2^{29} + 1, 2^{29} - 1]\/$. The range chosen by an implementation must either be
symmetric about zero or contain one more negative value than positive
(to accommodate twos-complement representation) and should be large
enough to serve as array indices.  The constants
\mbox{\tt minInt}\indextt{minInt} and \mbox{\tt maxInt}\indextt{maxInt}
(Figure~\ref{basic-numeric-2}, page~\pageref{basic-numeric-2}) define the limits of
\mbox{\tt Int}\indextycon{Int} in each implementation.
\mbox{\tt Float}\indextycon{Float} is a
floating-point type, also implementation-defined; it is desirable that
this type be at least equal in range and precision to the IEEE
single-precision type.  Similarly, \mbox{\tt Double}\indextycon{Double} should
cover IEEE double-precision.  An implementation may provide other
numeric types, such as additional
precisions of integer and
floating-point.  The results of exceptional
conditions (such as overflow or underflow) on the fixed-precision
numeric types are undefined; an implementation may choose error
(\mbox{$\it \bot$}, semantically), a truncated value, or a special value such as
infinity, indefinite, etc.

\begin{table}
\[
\begin{tabular}{|l|l|l|}
\hline
\mc{1}{|c|}{Type} & 
        \mc{1}{c|}{Class} &
        \mc{1}{c|}{Description} \\ \hline
\mbox{\tt Integer} & \mbox{\tt Integral} & Arbitrary-precision integers \\
\mbox{\tt Int} & \mbox{\tt Integral} & Fixed-precision integers \\
\mbox{\tt (Integral\ a)\ =>\ Ratio\ a} & \mbox{\tt RealFrac} & Rational numbers \\
\mbox{\tt Float} & \mbox{\tt RealFloat} & Real floating-point, single precision \\
\mbox{\tt Double} & \mbox{\tt RealFloat} & Real floating-point, double precision \\
\mbox{\tt (RealFloat\ a)\ =>\ Complex\ a} & \mbox{\tt Floating} & Complex floating-point \\
\hline
\end{tabular}
\]
\ecaption{Standard numeric types}
\label{standard-numeric-types}
\index{numeric type}
\end{table}

The interface text (Section~\ref{module-interfaces}) associated with
the standard numeric classes, types, and operations is shown
in Figures~\ref{basic-numeric-1}--\ref{basic-numeric-3}.

\begin{figure}
\outline{
\mbox{\tt class\ \ (Eq\ a,\ Text\ a)\ =>\ Num\ a\ \ where}\\
\mbox{\tt \ \ \ \ (+),\ (-),\ (*)\ \ \ \ \ \ \ ::\ a\ ->\ a\ ->\ a}\\
\mbox{\tt \ \ \ \ negate\ \ \ \ \ \ \ \ \ \ \ \ \ \ ::\ a\ ->\ a}\\
\mbox{\tt \ \ \ \ abs,\ signum\ \ \ \ \ \ \ \ \ ::\ a\ ->\ a}\\
\mbox{\tt \ \ \ \ fromInteger\ \ \ \ \ \ \ \ \ ::\ Integer\ ->\ a}\\
\mbox{\tt \ \ \ \ x\ -\ y\ \ \ \ \ \ \ \ \ \ \ \ \ \ \ =\ \ x\ +\ negate\ y}\\
\mbox{\tt }\\[-8pt]
\mbox{\tt class\ \ (Num\ a,\ Ord\ a)\ =>\ Real\ a\ where}\\
\mbox{\tt \ \ \ \ toRational\ \ \ \ \ \ \ \ \ \ ::\ \ a\ ->\ Rational}\\
\mbox{\tt }\\[-8pt]
\mbox{\tt class\ \ (Real\ a)\ =>\ Integral\ a\ \ where}\\
\mbox{\tt \ \ \ \ div,\ rem,\ mod\ \ \ \ \ \ \ ::\ a\ ->\ a\ ->\ a}\\
\mbox{\tt \ \ \ \ divRem\ \ \ \ \ \ \ \ \ \ \ \ \ \ ::\ a\ ->\ a\ ->\ (a,a)}\\
\mbox{\tt \ \ \ \ even,\ odd\ \ \ \ \ \ \ \ \ \ \ ::\ a\ ->\ Bool}\\
\mbox{\tt \ \ \ \ toInteger\ \ \ \ \ \ \ \ \ \ \ ::\ a\ ->\ Integer}\\
\mbox{\tt \ \ \ \ x\ `div`\ y\ \ \ \ \ \ \ \ \ \ \ =\ \ q\ \ where\ (q,r)\ =\ divRem\ x\ y}\\
\mbox{\tt \ \ \ \ x\ `rem`\ y\ \ \ \ \ \ \ \ \ \ \ =\ \ r\ \ where\ (q,r)\ =\ divRem\ x\ y}\\
\mbox{\tt \ \ \ \ x\ `mod`\ y\ \ \ \ \ \ \ \ \ \ \ =\ \ if\ signum\ x\ ==\ -\ (signum\ y)\ then\ r\ +\ y\ else\ r}\\
\mbox{\tt \ \ \ \ \ \ \ \ \ \ \ \ \ \ \ \ \ \ \ \ \ \ \ \ \ \ \ where\ r\ =\ x\ `rem`\ y}\\
\mbox{\tt \ \ \ \ even\ x\ \ \ \ \ \ \ \ \ \ \ \ \ \ =\ \ x\ `rem`\ 2\ ==\ 0}\\
\mbox{\tt \ \ \ \ odd\ \ \ \ \ \ \ \ \ \ \ \ \ \ \ \ \ =\ \ not\ .\ even}\\
\mbox{\tt }\\[-8pt]
\mbox{\tt class\ \ (Num\ a)\ =>\ Fractional\ a\ \ where}\\
\mbox{\tt \ \ \ \ (/)\ \ \ \ \ \ \ \ \ \ \ \ \ \ \ \ \ ::\ a\ ->\ a\ ->\ a}\\
\mbox{\tt \ \ \ \ fromRational\ \ \ \ \ \ \ \ ::\ Rational\ ->\ a}\\
\mbox{\tt }\\[-8pt]
\mbox{\tt class\ \ (Fractional\ a)\ =>\ Floating\ a\ \ where}\\
\mbox{\tt \ \ \ \ pi\ \ \ \ \ \ \ \ \ \ \ \ \ \ \ \ \ \ ::\ a}\\
\mbox{\tt \ \ \ \ exp,\ log,\ sqrt\ \ \ \ \ \ ::\ a\ ->\ a}\\
\mbox{\tt \ \ \ \ (**),\ logBase\ \ \ \ \ \ \ ::\ a\ ->\ a\ ->\ a}\\
\mbox{\tt \ \ \ \ sin,\ cos,\ tan\ \ \ \ \ \ \ ::\ a\ ->\ a}\\
\mbox{\tt \ \ \ \ asin,\ acos,\ atan\ \ \ \ ::\ a\ ->\ a}\\
\mbox{\tt \ \ \ \ sinh,\ cosh,\ tanh\ \ \ \ ::\ a\ ->\ a}\\
\mbox{\tt \ \ \ \ asinh,\ acosh,\ atanh\ ::\ a\ ->\ a}\\
\mbox{\tt \ \ \ \ x\ **\ y\ \ \ \ \ \ \ \ \ \ \ \ \ \ =\ \ exp\ (log\ x\ *\ y)}\\
\mbox{\tt \ \ \ \ logBase\ x\ y\ \ \ \ \ \ \ \ \ =\ \ log\ y\ /\ log\ x}\\
\mbox{\tt \ \ \ \ sqrt\ x\ \ \ \ \ \ \ \ \ \ \ \ \ \ =\ \ x\ **\ 0.5}\\
\mbox{\tt \ \ \ \ tan\ \ x\ \ \ \ \ \ \ \ \ \ \ \ \ \ =\ \ sin\ \ x\ /\ cos\ \ x}\\
\mbox{\tt \ \ \ \ tanh\ x\ \ \ \ \ \ \ \ \ \ \ \ \ \ =\ \ sinh\ x\ /\ cosh\ x}\\
\mbox{\tt }\\[-8pt]
\mbox{\tt class\ \ (Real\ a,\ Fractional\ a)\ =>\ RealFrac\ a\ \ where}\\
\mbox{\tt \ \ \ \ properFraction\ \ \ \ \ \ ::\ (Integral\ b)\ =>\ a\ ->\ (b,a)}\\
\mbox{\tt \ \ \ \ approxRational\ \ \ \ \ \ ::\ a\ ->\ a\ ->\ Rational}
}
\ecaption{Numeric classes and related operations}
\label{basic-numeric-1}
\indextt{Num}\indextt{+}\indextt{-}\indextt{*}
\indextt{negate}\indextt{abs}\indextt{signum}         
\indextt{fromInteger}
\indextt{Real}\indextt{toRational}
\indextt{Integral}\indextt{divRem}\indextt{mod}\indextt{div}\indextt{rem}
\indextt{even}\indextt{odd}
\indextt{Fractional}\indextt{/}\indextt{fromRational}       
\indextt{Floating}\indextt{pi}\indextt{exp}\indextt{log}\indextt{sqrt} 
\indextt{**}\indextt{logBase}
\indextt{sin}\indextt{cos}\indextt{tan}                        
\indextt{asin}\indextt{acos}\indextt{atan}               
\indextt{sinh}\indextt{cosh}\indextt{tanh}               
\indextt{asinh}\indextt{acosh}\indextt{atanh}      
\indextt{RealFrac}\indextt{properFraction}\indextt{approxRational}
\end{figure}

\begin{figure}
\outline{
\mbox{\tt class\ \ (RealFrac\ a,\ Floating\ a)\ =>\ RealFloat\ a\ \ where}\\
\mbox{\tt \ \ \ \ floatRadix\ \ \ \ \ \ \ \ \ \ ::\ a\ ->\ Integer}\\
\mbox{\tt \ \ \ \ floatDigits\ \ \ \ \ \ \ \ \ ::\ a\ ->\ Int}\\
\mbox{\tt \ \ \ \ floatRange\ \ \ \ \ \ \ \ \ \ ::\ a\ ->\ (Int,Int)}\\
\mbox{\tt \ \ \ \ decodeFloat\ \ \ \ \ \ \ \ \ ::\ a\ ->\ (Integer,Int)}\\
\mbox{\tt \ \ \ \ encodeFloat\ \ \ \ \ \ \ \ \ ::\ Integer\ ->\ Int\ ->\ a}\\
\mbox{\tt \ \ \ \ exponent\ \ \ \ \ \ \ \ \ \ \ \ ::\ a\ ->\ Int}\\
\mbox{\tt \ \ \ \ significand\ \ \ \ \ \ \ \ \ ::\ a\ ->\ a}\\
\mbox{\tt \ \ \ \ scaleFloat\ \ \ \ \ \ \ \ \ \ ::\ Int\ ->\ a\ ->\ a}\\
\mbox{\tt }\\[-8pt]
\mbox{\tt \ \ \ \ exponent\ x\ \ \ \ \ \ \ \ \ \ =\ \ if\ m\ ==\ 0\ then\ 0\ else\ n\ +\ floatDigits\ x}\\
\mbox{\tt \ \ \ \ \ \ \ \ \ \ \ \ \ \ \ \ \ \ \ \ \ \ \ \ \ \ \ where\ (m,n)\ =\ decodeFloat\ x}\\
\mbox{\tt \ \ \ \ significand\ x\ \ \ \ \ \ \ =\ \ encodeFloat\ m\ (-\ (floatDigits\ x))}\\
\mbox{\tt \ \ \ \ \ \ \ \ \ \ \ \ \ \ \ \ \ \ \ \ \ \ \ \ \ \ \ where\ (m,{\char'137})\ =\ decodeFloat\ x}\\
\mbox{\tt \ \ \ \ scaleFloat\ k\ x\ \ \ \ \ \ =\ \ encodeFloat\ m\ (n+k)}\\
\mbox{\tt \ \ \ \ \ \ \ \ \ \ \ \ \ \ \ \ \ \ \ \ \ \ \ \ \ \ \ where\ (m,n)\ =\ decodeFloat\ x}\\
\mbox{\tt }\\[-8pt]
\mbox{\tt instance\ \ Integral\ Int}\\
\mbox{\tt instance\ \ Integral\ Integer}\\
\mbox{\tt }\\[-8pt]
\mbox{\tt minInt,\ maxInt\ \ \ \ \ \ \ \ \ \ ::\ \ Int}\\
\mbox{\tt fromIntegral\ \ \ \ \ \ \ \ \ \ \ \ ::\ \ (Integral\ a,\ Num\ b)\ =>\ a\ ->\ b}\\
\mbox{\tt gcd,\ lcm\ \ \ \ \ \ \ \ \ \ \ \ \ \ \ \ ::\ \ (Integral\ a)\ =>\ a\ ->\ a->\ a}\\
\mbox{\tt ({\char'136})\ \ \ \ \ \ \ \ \ \ \ \ \ \ \ \ \ \ \ \ \ ::\ \ (Num\ a,\ Integral\ b)\ =>\ a\ ->\ b\ ->\ a}\\
\mbox{\tt ({\char'136}{\char'136})\ \ \ \ \ \ \ \ \ \ \ \ \ \ \ \ \ \ \ \ ::\ \ (Fractional\ a,\ Integral\ b)\ =>\ a\ ->\ b\ ->\ a}\\
\mbox{\tt }\\[-8pt]
\mbox{\tt data\ \ (Integral\ a)\ \ \ \ \ \ =>\ Ratio\ a}\\
\mbox{\tt type\ \ Rational\ \ \ \ \ \ \ \ \ \ =\ \ Ratio\ Integer}\\
\mbox{\tt instance\ \ (Integral\ a)\ \ =>\ RealFrac\ (Ratio\ a)}\\
\mbox{\tt }\\[-8pt]
\mbox{\tt ({\char'45})\ \ \ \ \ \ \ \ \ \ \ \ \ \ \ \ \ \ \ \ \ ::\ \ (Integral\ a)\ =>\ a\ ->\ a\ ->\ Ratio\ a}\\
\mbox{\tt numerator,\ denominator\ \ ::\ \ (Integral\ a)\ =>\ Ratio\ a\ ->\ a}\\
\mbox{\tt }\\[-8pt]
\mbox{\tt instance\ \ RealFloat\ Float}\\
\mbox{\tt instance\ \ RealFloat\ Double}\\
\mbox{\tt }\\[-8pt]
\mbox{\tt fromRealFrac\ \ \ \ \ \ \ \ \ \ \ \ ::\ (RealFrac\ a,\ Fractional\ b)\ =>\ a\ ->\ b}\\
\mbox{\tt }\\[-8pt]
\mbox{\tt truncate,\ round\ \ \ \ \ \ \ \ \ ::\ \ (RealFrac\ a,\ Integral\ b)\ =>\ a\ ->\ b}\\
\mbox{\tt ceiling,\ floor\ \ \ \ \ \ \ \ \ \ ::\ \ (RealFrac\ a,\ Integral\ b)\ =>\ a\ ->\ b}\\
\mbox{\tt atan2\ \ \ \ \ \ \ \ \ \ \ \ \ \ \ \ \ \ \ ::\ \ (RealFloat\ a)\ =>\ a\ ->\ a\ ->\ a}
}
\ecaption{Numeric classes and related operations (continued)}
\label{basic-numeric-2}
\indextt{RealFloat}
\indextt{floatRadix}\indextt{floatDigits}\indextt{floatRange} 
\indextt{decodeFloat}\indextt{encodeFloat}
\indextt{exponent}\indextt{significand}\indextt{scaleFloat}
\indextt{Int}\indextt{Integer}\indextt{minInt}\indextt{maxInt}                
\indextt{fromIntegral}
\indextt{gcd}\indextt{lcm} 
\index{^@{\ptt {\char'136}}} % this is ^
\index{^^@{\ptt {\char'136}{\char'136}}} % this is ^^
\indextycon{Ratio}\indexsynonym{Rational}           
\indextt{RealFrac}
\index{%@{\ptt {\char'045}}}\indextt{numerator}\indextt{denominator}         
\indextycon{Float}\indextycon{Double}
\indextt{fromRealFrac}
\indextt{truncate}\indextt{round}
\indextt{ceiling}\indextt{floor}    
\indextt{atan2}
\end{figure}
\begin{figure}
\outline{
\mbox{\tt data\ \ (RealFloat\ a)\ \ \ \ =>\ Complex\ a\ =\ a\ :+\ a\ \ deriving\ (Eq,\ Binary,\ Text)}\\
\mbox{\tt instance\ (RealFloat\ a)\ =>\ Floating\ (Complex\ a)}\\
\mbox{\tt }\\[-8pt]
\mbox{\tt realPart,\ imagPart\ \ \ \ \ ::\ \ (RealFloat\ a)\ =>\ Complex\ a\ ->\ a}\\
\mbox{\tt conjugate\ \ \ \ \ \ \ \ \ \ \ \ \ \ ::\ \ (RealFloat\ a)\ =>\ Complex\ a\ ->\ Complex\ a}\\
\mbox{\tt mkPolar\ \ \ \ \ \ \ \ \ \ \ \ \ \ \ \ ::\ \ (RealFloat\ a)\ =>\ a\ ->\ a\ ->\ Complex\ a}\\
\mbox{\tt cis\ \ \ \ \ \ \ \ \ \ \ \ \ \ \ \ \ \ \ \ ::\ \ (RealFloat\ a)\ =>\ a\ ->\ Complex\ a}\\
\mbox{\tt polar\ \ \ \ \ \ \ \ \ \ \ \ \ \ \ \ \ \ ::\ \ (RealFloat\ a)\ =>\ Complex\ a\ ->\ (a,a)}\\
\mbox{\tt magnitude,\ phase\ \ \ \ \ \ \ ::\ \ (RealFloat\ a)\ =>\ Complex\ a\ ->\ a}
}
\ecaption{Numeric classes and related operations (continued)}
\label{basic-numeric-3}
\indextycon{Complex}
\indextt{:+}\indextt{Floating}
\indextt{realPart}\indextt{imagPart}
\indextt{conjugate}\indextt{mkPolar}\indextt{cis}
\indextt{polar}\indextt{magnitude}\indextt{phase}
\end{figure}

\subsubsection{Numeric Literals}
\label{numeric-literals}

The syntax of numeric literals is given in
Section~\ref{lexemes-numeric}.  An integer literal represents the
application
of the function \mbox{\tt fromInteger}\indextt{fromInteger} to the appropriate value of type
\mbox{\tt Integer}.  Similarly, a floating literal stands for an application of
\mbox{\tt fromRational}\indextt{fromRational} to a value of type \mbox{\tt Rational} (that is, 
\mbox{\tt Ratio\ Integer}).  Given the typings:
\bprog
\mbox{\tt fromInteger\ \ ::\ (Num\ a)\ =>\ Integer\ ->\ a}\\
\mbox{\tt fromRational\ ::\ (Fractional\ a)\ =>\ Rational\ ->\ a}
\eprog\indextt{fromInteger}\indextt{fromRational}%
integer and floating literals have the
typings \mbox{\tt (Num\ a)\ =>\ a} and \mbox{\tt (Fractional\ a)\ =>\ a}, respectively.
Numeric literals are defined in this indirect way so that they may be
interpreted as values of any appropriate numeric type.
For example, \mbox{\tt fromInteger} for complex
numbers is defined as follows:
\bprog
\mbox{\tt fromInteger\ n\ =\ fromInteger\ n\ :+\ 0}
\eprog
See Section~\ref{default-decls} for a discussion of overloading ambiguity.

\subsubsection{Constructed Numbers}
\label{constructed-numbers}

There are two kinds of numeric types formed by data constructors:
namely, \mbox{\tt Ratio} and \mbox{\tt Complex}.  For each \mbox{\tt Integral} type $t$, there is
a type \mbox{\tt Ratio}\indextycon{Ratio} $t$ of rational pairs with components of
type $t$. (The type name \mbox{\tt Rational}\indexsynonym{Rational} is a synonym for
\mbox{\tt Ratio\ Integer}.)  Similarly, for each real floating-point type $t$,
\mbox{\tt Complex}\indextycon{Complex} $t$ is a type of complex numbers with real
and imaginary components of type $t$.

The operator \mbox{\tt ({\char'45})}\index{%@{\ptt {\char'045}}} forms the ratio of two
integral numbers.  The functions \mbox{\tt numerator}\indextt{numerator}
and \mbox{\tt denominator}\indextt{denominator} extract the components of a
ratio; these are in reduced form with a positive denominator.

Complex numbers are an algebraic type:
\index{algebraic datatype}
\bprog
\mbox{\tt data\ \ (RealFloat\ a)\ =>\ Floating\ (Complex\ a)\ \ =\ \ a\ :+\ a}
\eprog\indextycon{Floating}%
The constructor \mbox{\tt (:+)}\indextt{:+} forms a complex number from its
real and imaginary rectangular components.  A complex number may also
be formed from polar components of magnitude and phase by the function
\mbox{\tt mkPolar}\indextt{mkPolar}.  The function \mbox{\tt cis}\indextt{polar}
produces a complex number from an angle \mbox{$\it t$}:
\bprog
\mbox{\tt cis\ t\ =\ cos\ t\ :+\ sin\ t}
\eprog
Put another way, \mbox{\tt cis} \mbox{$\it t$} is a complex value with magnitude \mbox{$\it 1$}
and phase \mbox{$\it t$} (modulo \mbox{$\it 2\pi$}).

The function \mbox{\tt polar}\indextt{polar} takes a complex number and
returns a (magnitude, phase) pair in canonical form: The magnitude is
nonnegative, and the phase, in the range $(- \pi , \pi ]$; if the
magnitude is zero, then so is the phase.  Several
component-extraction functions are provided:
\bprog
\mbox{\tt realPart\ (x:+y)\ =\ \ x}\\
\mbox{\tt imagPart\ (x:+y)\ =\ \ y}\\
\mbox{\tt magnitude\ z\ \ \ \ \ =\ \ r\ \ where\ (r,t)\ =\ polar\ z}\\
\mbox{\tt phase\ z\ \ \ \ \ \ \ \ \ =\ \ t\ \ where\ (r,t)\ =\ polar\ z}
\eprog\indextt{realPart}\indextt{imagPart}%
\indextt{magnitude}\indextt{phase}%
Also defined on complex numbers is the conjugate function:
\bprog
\mbox{\tt conjugate\ (x:+y)\ =\ \ x:+(-y)}
\eprogNoSkip\indextt{conjugate}

\subsubsection{Arithmetic and Number-Theoretic Operations}
\label{arithmetic-operators}
\index{arithmetic}

The infix operations \mbox{\tt (+)}\indextt{+}, \mbox{\tt (*)}\indextt{*},
\mbox{\tt (-)}\indextt{-} and the unary function
\mbox{\tt negate}\indextt{negate} (which can also be written as a prefix minus sign; see
section~\ref{operators}) apply to all numbers.  The operations
\mbox{\tt div}\indextt{div}, \mbox{\tt rem}\indextt{rem}, and
\mbox{\tt mod}\indextt{mod} apply only to integral numbers, while the operation
\mbox{\tt (/)}\indextt{/} applies only to fractional ones.  The \mbox{\tt div} and
\mbox{\tt rem} operations satisfy the law:
\[\ba{c}
\mbox{\tt (x\ }\bkqB\mbox{\tt div}\bkqA\mbox{\tt \ y)*y\ +\ (x\ }\bkqB\mbox{\tt rem}\bkqA\mbox{\tt \ y)\ ==\ x}
\ea\]
The result of \mbox{\tt x\ }\bkqB\mbox{\tt div}\bkqA\mbox{\tt \ y} has the same sign as \mbox{\tt x\ *\ y}
and is truncated toward zero.  The modulo function differs from the
remainder function when the signs of the dividend and divisor differ, the
remainder always having the sign of the dividend, and the modulo
having the sign of the divisor.  For example,
\bprog
\mbox{\tt -13\ }\bkqB\mbox{\tt rem}\bkqA\mbox{\tt \ 4\ ==\ -1}\\
\mbox{\tt -13\ }\bkqB\mbox{\tt mod}\bkqA\mbox{\tt \ 4\ ==\ 3}\\
\mbox{\tt }\\[-8pt]
\mbox{\tt 13\ }\bkqB\mbox{\tt rem}\bkqA\mbox{\tt \ -4\ ==\ 1}\\
\mbox{\tt 13\ }\bkqA\mbox{\tt mod}\bkqA\mbox{\tt \ -4\ ==\ -3}
\eprog
The \mbox{\tt divRem} operation takes a dividend and a divisor as arguments
and returns a (quotient, remainder) pair:
\bprog
\mbox{\tt divRem\ x\ y\ \ =\ \ (x\ }\bkqB\mbox{\tt div}\bkqA\mbox{\tt \ y,\ x\ }\bkqB\mbox{\tt rem}\bkqA\mbox{\tt \ y)}
\eprog
Also available on integers are the even and odd predicates:
\bprog
\mbox{\tt even\ x\ \ \ \ \ \ =\ \ x\ }\bkqB\mbox{\tt rem}\bkqA\mbox{\tt \ 2\ ==\ 0}\\
\mbox{\tt odd\ \ \ \ \ \ \ \ \ =\ \ not\ .\ even}
\eprog\indextt{even}\indextt{odd}
Finally, there are the greatest common divisor and least common
multiple functions: \mbox{\tt gcd}\indextt{gcd} $x$ $y$ is the greatest
integer that divides both $x$ and $y$.  \mbox{\tt lcm}\indextt{lcm} $x$ $y$
is the smallest positive integer that both $x$ and $y$ divide.

\subsubsection{Exponentiation and Logarithms}
\index{exponentiation}\index{logarithm}

The one-argument exponential function \mbox{\tt exp}\indextt{exp} and the
logarithm function \mbox{\tt log}\indextt{log} act on floating-point numbers and
use base $e$.  \mbox{\tt logBase}\indextt{logBase} $a$ $x$ returns the
logarithm of $x$ in base $a$.  \mbox{\tt sqrt}\indextt{sqrt} returns the
principal square root of a floating-point number.
There are three two-argument exponentiation operations:
\mbox{\tt ({\char'136})}\index{^@{\ptt {\char'136}}} raises any
number to a nonnegative integer power,
\mbox{\tt ({\char'136}{\char'136})}\index{^^@{\ptt {\char'136}{\char'136}}} raises a
fractional number to any integer power, and \mbox{\tt (**)}\indextt{**}
takes two floating-point arguments.  The value of $x$\mbox{\tt {\char'136}0} or $x$\mbox{\tt {\char'136}{\char'136}0}
is \mbox{\tt 1} for any $x$, including zero; \mbox{\tt 0**}$y$ is undefined.
  
\subsubsection{Magnitude and Sign}
\label{magnitude-sign}
\index{magnitude}\index{sign}

A number has a {\em magnitude}
and a {\em sign}.  The functions \mbox{\tt abs}\indextt{abs} and
\mbox{\tt signum}\indextt{signum} apply to any number and satisfy the law:
\bprog
\mbox{\tt abs\ x\ *\ signum\ x\ ==\ x}
\eprog
For real numbers, these functions are defined by:
\bprog
\mbox{\tt abs\ x\ \ \ \ |\ x\ >=\ 0\ \ =\ x}\\
\mbox{\tt \ \ \ \ \ \ \ \ \ |\ x\ <\ \ 0\ \ =\ -x}\\
\mbox{\tt }\\[-8pt]
\mbox{\tt signum\ x\ |\ x\ >\ \ 0\ \ =\ 1}\\
\mbox{\tt \ \ \ \ \ \ \ \ \ |\ x\ ==\ 0\ \ =\ 0}\\
\mbox{\tt \ \ \ \ \ \ \ \ \ |\ x\ <\ \ 0\ \ =\ -1}
\eprog
For complex numbers, the definitions are different:\nopagebreak
\bprog
\mbox{\tt abs\ z\ \ \ \ \ \ \ \ \ \ \ \ \ =\ \ magnitude\ z\ :+\ 0}\\
\mbox{\tt signum\ 0\ \ \ \ \ \ \ \ \ \ =\ \ 0}\\
\mbox{\tt signum\ z@(x:+y)\ \ \ =\ \ x/r\ :+\ y/r\ \ where\ r\ =\ magnitude\ z}
\eprog
That is, \mbox{\tt abs} $z$ is a number with the magnitude of $z$, but oriented
in the positive real direction, whereas \mbox{\tt signum} $z$ has the phase of
$z$, but unit magnitude.  (\mbox{\tt abs} for a complex number differs from
\mbox{\tt magnitude} only in type.  See Section~\ref{constructed-numbers}.)

\subsubsection{Trigonometric Functions}
\index{trigonometric function}

The circular and hyperbolic sine\index{sine}, cosine\index{cosine},
and tangent\index{tangent} functions and their inverses are provided
for floating-point numbers.  A version of arctangent\index{arctangent}
taking two real floating-point arguments is also provided: For real floating
$x$ and $y$, \mbox{\tt atan2}\indextt{atan2} $y$ $x$ differs from 
\mbox{\tt atan\ (}$y\/$\mbox{\tt /}$x\/$\mbox{\tt )}\indextt{atan} in that its range is 
$( -\pi , \pi ]$ rather than $(- \pi / 2 , \pi / 2 )$ (because the signs
of the arguments provide quadrant information), and that it is defined
when $x$ is zero.

The precise definition of the above functions is as in Common Lisp
\cite{steele:common-lisp}, which in turn follows Penfield's proposal
for APL
\cite{penfield:complex-apl}.
See these references for discussions of branch cuts, discontinuities,
and implementation.

\subsubsection{Coercions and Component Extraction}
\label{coercion}
\index{coercion}

The \mbox{\tt ceiling}\indextt{ceiling}, \mbox{\tt floor}\indextt{floor},
\mbox{\tt truncate}\indextt{truncate}, and \mbox{\tt round}\indextt{round}
functions each take a real fractional argument and return an integral
result.  \mbox{\mbox{\tt ceiling} $x$} returns the least integer not less than $x$, and
\mbox{\mbox{\tt floor} $x$}, the greatest integer not greater than $x$.  \mbox{\mbox{\tt truncate} $x$}
yields the integer nearest $x$ between $0$ and $x$, inclusive.
\mbox{\mbox{\tt round} $x$} returns the nearest integer to $x$, the even integer if
$x$ is equidistant between two integers.

The function \mbox{\tt properFraction}\indextt{properFraction} takes a real
fractional number $x$ and returns a pair comprising $x$ as a
proper fraction: an integral number with the same sign as $x$ and a
fraction with the same type and sign as $x$ and with absolute
value less than 1.  The \mbox{\tt ceiling}, \mbox{\tt floor}, \mbox{\tt truncate}, and \mbox{\tt round}
functions can be defined in terms of this one.

Two functions convert numbers to type \mbox{\tt Rational}:
\mbox{\tt toRational}\indextt{toRational} returns the rational equivalent of
its real argument with full precision;
\mbox{\tt approxRational}\indextt{approxRational} takes two real fractional arguments and returns an
approximation to the first within the tolerance given by the second.
Subject to the tolerance constraint, the result has the smallest
denominator possible.

The operations of class \mbox{\tt RealFloat}\indextt{RealFloat} allow efficient, machine-independent
access to the components of a floating-point number.
The functions \mbox{\tt floatRadix}\indextt{floatRadix},
\mbox{\tt floatDigits}\indextt{floatDigits}, and
\mbox{\tt floatRange}\indextt{floatRange} give the parameters of a
floating-point type:  the radix of the representation, the number of
digits of this radix in the significand, and the lowest and highest
values the exponent may assume, respectively.
The function \mbox{\tt decodeFloat}\indextt{decodeFloat}
applied to a real floating-point number returns the significand
expressed as an \mbox{\tt Integer} and an appropriately scaled exponent (an
\mbox{\tt Int}).  If \mbox{\mbox{\tt decodeFloat\ x}} yields \mbox{\mbox{\tt (}\mbox{$\it m$}\mbox{\tt ,}\mbox{$\it n$}\mbox{\tt )}}, then \mbox{\tt x} is
equal in value to \mbox{$\it mb^n$}, where \mbox{$\it b$} is the floating-point radix, and
furthermore, either \mbox{$\it m$} and \mbox{$\it n$} are both zero or else
\mbox{$\it b^{d-1}\leq m<b^d$}, where \mbox{$\it d$} is the value of \mbox{\mbox{\tt floatDigits\ x}}.
\mbox{\tt encodeFloat}\indextt{encodeFloat} performs the inverse of this
transformation.  The functions \mbox{\tt significand}\indextt{significand}
and \mbox{\tt exponent}\indextt{exponent} together provide the same
information as \mbox{\tt decodeFloat},  but rather than an \mbox{\tt Integer},
\mbox{\mbox{\tt significand\ x}} yields a value of the same type as \mbox{\tt x}, scaled to lie
in the open interval \mbox{$\it (-1,1)$}.  \mbox{\mbox{\tt exponent\ 0}} is zero.  \mbox{\tt scaleFloat}
multiplies a floating-point number by an integer power of the radix.

Also available are the following coercion functions:
\bprog
\mbox{\tt fromIntegral\ ::\ (Integral\ a,\ Num\ b)\ =>\ a\ ->\ b}\\
\mbox{\tt fromRealFrac\ ::\ (RealFrac\ a,\ Fractional\ b)\ =>\ a\ ->\ b}
\eprogNoSkip\indextt{fromIntegral}\indextt{fromRealFrac}

\subsection{Arrays}
\label{arrays}
\index{array}

\Haskell{} provides indexable {\em arrays}, which may be thought of as
functions whose domains are isomorphic to contiguous subsets of the
integers.
Functions restricted in this way can be
implemented efficiently; in particular, a programmer may
reasonably expect rapid access to the components.  To ensure
the possibility of such an implementation, arrays are treated as data, not as
general functions.

Types that are instances of class \mbox{\tt Ix}\indextt{Ix} (see
Section~\ref{instance-decls}) may be indices of arrays; a one-dimensional 
array might have index type \mbox{\tt Int}, a two-dimensional array
\mbox{\tt (Int,Char)} etc.

\subsubsection{Array Construction}

If \mbox{\tt a} is an index type and \mbox{\tt b} is any type, the type of arrays with
indices in \mbox{\tt a} and elements in \mbox{\tt b} is written \mbox{\tt Array\ a\ b}.\indextycon{Array}
An array may be created by the function \mbox{\tt array}:
\bprog
\mbox{\tt array\ ::\ (Ix\ a)\ =>\ (a,a)\ \ ->\ [Assoc\ a\ b]\ ->\ Array\ a\ b}\\
\mbox{\tt data\ \ Assoc\ a\ b\ \ =\ \ a\ :=\ b}
\eprog\indextt{array}\indextycon{Assoc}\indextt{:=}
The first argument of \mbox{\tt array} is a pair of {\em bounds}, each of the
index type of the array.  These bounds are the lowest and
highest indices in the array, in that order.  For example, a
one-origin vector of length \mbox{\tt 10} has bounds \mbox{\tt (1,10)}, and a one-origin \mbox{\tt 10}
by \mbox{\tt 10} matrix has bounds \mbox{\tt ((1,1),(10,10))}.

The second argument of \mbox{\tt array} is a list of {\em associations}
of the form $index$ \mbox{\tt :=} $value$.  Typically, this list will
be expressed as a comprehension.  An association \mbox{\tt i\ :=\ x} defines the
value of the array at index \mbox{\tt i} to be \mbox{\tt x}.  The array is undefined if
any index in the list is out of bounds.  If any two associations in the
list have the same index, the value at that index is undefined.
Because the indices must be checked for these errors, \mbox{\tt array} is
strict in the bounds argument and in the indices of the association list,
but nonstrict in the values.  Thus, recurrences such as the following are
possible:
\bprog
\mbox{\tt a\ =\ array\ (1,100)\ ((1\ :=\ 1)\ :\ [i\ :=\ i\ *\ a!(i-1)\ |\ i\ <-\ [2..100]])}
\eprog
Not every index within the bounds of the array need
appear in the association list, but the values associated with indices
that do not appear will be undefined.
Figure~\ref{array-examples} shows some examples that use the
\mbox{\tt array} constructor.

\begin{figure}
\outline{
\mbox{\tt --\ Scaling\ an\ array\ of\ numbers\ by\ a\ given\ number:}\\
\mbox{\tt scale\ ::\ (Num\ a,\ Ix\ b)\ =>\ a\ ->\ Array\ b\ a\ ->\ Array\ b\ a}\\
\mbox{\tt scale\ x\ a\ =\ array\ b\ [i\ :=\ a!i\ *\ x\ |\ i\ <-\ range\ b]}\\
\mbox{\tt \ \ \ \ \ \ \ \ \ \ \ \ where\ b\ =\ bounds\ a}\\
\mbox{\tt }\\[-8pt]
\mbox{\tt --\ Inverting\ an\ array\ that\ holds\ a\ permutation\ of\ its\ indices}\\
\mbox{\tt invPerm\ ::\ (Ix\ a)\ =>\ Array\ a\ a\ ->\ Array\ a\ a}\\
\mbox{\tt invPerm\ a\ =\ array\ b\ [a!i\ :=\ i\ |\ i\ <-\ range\ b]}\\
\mbox{\tt \ \ \ \ \ \ \ \ \ \ \ \ where\ b\ =\ bounds\ a}\\
\mbox{\tt }\\[-8pt]
\mbox{\tt --\ The\ inner\ product\ of\ two\ vectors}\\
\mbox{\tt inner\ ::\ (Ix\ a,\ Num\ b)\ =>\ Array\ a\ b\ ->\ Array\ a\ b\ ->\ b}\\
\mbox{\tt inner\ v\ w\ =\ if\ b\ ==\ bounds\ w}\\
\mbox{\tt \ \ \ \ \ \ \ \ \ \ \ \ \ \ \ \ then\ sum\ [v!i\ *\ w!i\ |\ i\ <-\ range\ b]}\\
\mbox{\tt \ \ \ \ \ \ \ \ \ \ \ \ \ \ \ \ else\ error\ "inconformable\ arrays\ for\ inner\ product"}\\
\mbox{\tt \ \ \ \ \ \ \ \ \ \ \ \ where\ b\ =\ bounds\ v}
}
\ecaption{Array examples}
\label{array-examples}
\end{figure}

\mbox{\tt (!)}\index{!@{\ptt {\char'041}}} denotes
array subscripting; the \mbox{\tt bounds}\indextt{bounds} function
applied to an array returns its bounds:
\bprog
\mbox{\tt (!)\ \ \ \ ::\ (Ix\ a)\ =>\ Array\ a\ b\ ->\ a\ ->\ b}\\
\mbox{\tt bounds\ ::\ (Ix\ a)\ =>\ Array\ a\ b\ ->\ (a,a)}
\eprog
The functions \mbox{\tt indices}\indextt{indices}, \mbox{\tt elems}\indextt{elems}, and
\mbox{\tt assocs},\indextt{assocs} when applied to an array, return lists of
the indices, elements, or associations, respectively, in index order:
\bprog
\mbox{\tt indices::\ (Ix\ a)\ =>\ Array\ a\ b\ ->\ [a]}\\
\mbox{\tt indices\ =\ range\ .\ bounds}\\
\mbox{\tt }\\[-8pt]
\mbox{\tt elems::\ (Ix\ a)\ =>\ Array\ a\ b\ ->\ [b]}\\
\mbox{\tt elems\ a\ =\ [a!i\ |\ i\ <-\ indices\ a]}\\
\mbox{\tt }\\[-8pt]
\mbox{\tt assocs:\ (Ix\ a)\ =>\ Array\ a\ b\ ->\ [Assoc\ a\ b]}\\
\mbox{\tt assocs\ a\ =\ [\ i\ :=\ a!i\ |\ i\ <-\ indices\ a]}
\eprog
An array may be constructed from a pair of bounds and a list
of values in index order using the function \mbox{\tt listArray}:\nopagebreak
\bprog
\mbox{\tt listArray::\ (Ix\ a)\ =>\ (a,a)\ ->\ [b]\ ->\ Array\ a\ b}\\
\mbox{\tt listArray\ bnds\ xs\ =\ array\ bnds\ (zipWith\ (:=)\ (range\ bnds)\ xs)}
\eprogNoSkip

\subsubsection{Accumulated Arrays}
\index{array!accumulated}

Another array creation function, \mbox{\tt accumArray},\indextt{accumArray}
relaxes the restriction that a given index may appear at most once in
the association list, using an {\em accumulating function} which
combines the values of associations with the same index
\cite{nikhil:id-nouveau,wadler:array-primitive}:
\bprog
\mbox{\tt accumArray::(Ix\ a)\ =>\ (b->c->b)\ ->\ b\ ->\ (a,a)\ ->\ [Assoc\ a\ c]\ ->\ Array\ a\ b}
\eprog\indextt{accumArray}%
The first argument of \mbox{\tt accumArray} is the accumulating function; the
second is an initial value; the remaining two arguments are a bounds
pair and an association list, as for the \mbox{\tt array} function.
For example, given a list of values of some index type, \mbox{\tt hist}
produces a histogram of the number of occurrences of each index within
a specified range:
\bprog
\mbox{\tt hist\ ::\ (Ix\ a,\ Num\ b)\ =>\ (a,a)\ ->\ [a]\ ->\ Array\ a\ b}\\
\mbox{\tt hist\ bnds\ is\ =\ accumArray\ (+)\ 0\ bnds\ [i\ :=\ 1\ |\ i<-is,\ inRange\ bnds\ i]}
\eprog
If the accumulating function is strict, then \mbox{\tt accumArray} is
strict in the values, as well as the indices, in the
association list.  Thus, unlike ordinary arrays,
accumulated arrays should not in general be recursive.

\subsubsection{Incremental Array Updates}

% no indent, please!
\bprogB
\mbox{\tt (//)\ \ ::\ (Ix\ a)\ =>\ Array\ a\ b\ ->\ [Assoc\ a\ b]\ ->\ Array\ a\ b}\\
\mbox{\tt accum\ ::\ (Ix\ a)\ =>\ (b\ ->\ c\ ->\ b)\ ->\ Array\ a\ b\ ->\ [Assoc\ a\ c]\ ->\ Array\ a\ b}
\eprogNoSkip\indextt{//}\indextt{accum}

The operator \mbox{\tt (//)} takes an array and a list of \mbox{\tt Assoc} pairs and returns
an array identical to the left argument except that it has
been updated by the associations in the right argument.  (As with
the \mbox{\tt array} function, the indices in the association list must
be unique for the updated elements to be defined.)  For example,
if \mbox{\tt m} is a 1-origin, \mbox{\tt n} by \mbox{\tt n} matrix, then
\mbox{\tt m//[(i,i)\ :=\ 0\ |\ i\ <-\ [1..n]]} is the same matrix, except with
the diagonal zeroed.

\mbox{\tt accum} \mbox{$\it f$} takes an array
and an association list and accumulates pairs from the list into
the array with the accumulating function \mbox{$\it f$}.  Thus \mbox{\tt accumArray}
can be defined using \mbox{\tt accum}:\nopagebreak[4]
\bprog
\mbox{\tt accumArray\ f\ z\ b\ =\ accum\ f\ (array\ b\ [i\ :=\ z\ |\ i\ <-\ range\ b])}
\eprogNoSkip

\subsubsection{Derived Arrays}
\index{array!derived}

The two functions \mbox{\tt amap}\indextt{amap} and \mbox{\tt ixmap}\indextt{ixmap}
derive new arrays from existing ones; they may be
thought of as providing function composition on the left and right,
respectively, with the mapping that the original array embodies:
\bprog
\mbox{\tt amap\ ::\ (Ix\ a)\ =>\ (b\ ->\ c)\ ->\ Array\ a\ b\ ->\ Array\ a\ c}\\
\mbox{\tt amap\ f\ a\ =\ array\ b\ [i\ :=\ f\ (a!i)\ |\ i\ <-\ range\ b]}\\
\mbox{\tt \ \ \ \ \ \ \ \ \ \ \ where\ b\ =\ bounds\ a}\\
\mbox{\tt }\\[-8pt]
\mbox{\tt ixmap\ ::\ (Ix\ a,\ Ix\ a')\ =>\ (a',a')\ ->\ (a'->a)\ ->\ Array\ a\ b\ ->\ Array\ a'\ b}\\
\mbox{\tt ixmap\ bnds\ f\ a\ =\ array\ bnds\ [i\ :=\ a\ !\ f\ i\ |\ i\ <-\ range\ bnds]}
\eprog\indextt{amap}\indextt{ixmap}%
\mbox{\tt amap} is the array analogue of the \mbox{\tt map} function on lists, while
\mbox{\tt ixmap} allows for transformations on array indices.
Figure~\ref{derived-array-examples} shows some examples.

\begin{figure}
\outline{
\mbox{\tt --\ A\ rectangular\ subarray}\\
\mbox{\tt subArray\ ::\ (Ix\ a)\ =>\ (a,a)\ ->\ Array\ a\ b\ ->\ Array\ a\ b}\\
\mbox{\tt subArray\ bnds\ =\ ixmap\ bnds\ ({\char'134}i->i)}\\
\mbox{\tt }\\[-8pt]
\mbox{\tt --\ A\ row\ of\ a\ matrix}\\
\mbox{\tt row\ ::\ (Ix\ a,\ Ix\ b)\ =>\ a\ ->\ Array\ (a,b)\ c\ ->\ Array\ b\ c}\\
\mbox{\tt row\ i\ x\ =\ ixmap\ (l',u')\ ({\char'134}j->(i,j))\ x\ where\ ((l,l'),(u,u'))\ =\ bounds\ x}\\
\mbox{\tt }\\[-8pt]
\mbox{\tt --\ Diagonal\ of\ a\ square\ matrix}\\
\mbox{\tt diag\ ::\ (Ix\ a)\ =>\ Array\ (a,a)\ b\ ->\ Array\ a\ b}\\
\mbox{\tt diag\ x\ =\ ixmap\ (l,u)\ ({\char'134}i->(i,i))\ x}\\
\mbox{\tt \ \ \ \ \ \ \ \ \ where\ ((l,l'),(u,u'))\ |\ l\ ==\ l'\ {\char'46}{\char'46}\ u\ ==\ u'\ \ =\ bounds\ x}\\
\mbox{\tt }\\[-8pt]
\mbox{\tt --\ Projection\ of\ first\ components\ of\ an\ array\ of\ pairs}\\
\mbox{\tt firstArray\ ::\ (Ix\ a)\ =>\ Array\ a\ (b,c)\ ->\ Array\ a\ b}\\
\mbox{\tt firstArray\ =\ amap\ ({\char'134}(x,y)->x)}
}
\ecaption{Derived array examples}
\label{derived-array-examples}
\end{figure}

\subsection{Errors}
\label{basic-errors}\index{error}

All errors in \Haskell{} are semantically equivalent to $\bot$.
\mbox{\tt error::\ String\ ->\ a}\indextt{error} takes a string
argument and returns $\bot$.  An application of
\mbox{\tt error} terminates evaluation of the
program and displays the string as appropriate.

% Local Variables: 
% mode: latex
% End:

