%
% $Header$
%
\section{Introduction}
\label{introduction}

\Haskell{}\index{Haskell@\Haskell{}} is a general purpose, purely functional
programming language incorporating many recent innovations in
programming language research,
including higher-order functions,
non-strict semantics, static polymorphic typing, user-defined
algebraic datatypes, pattern-matching, list comprehensions, a module
system, and a rich set of primitive datatypes, including lists,
arrays, arbitrary and fixed precision integers, and floating-point
numbers.  \Haskell{} is both the culmination
and solidification of many years of research on functional
languages---the design has been influenced by languages as old as
ISWIM and as new as Miranda.

Although the initial emphasis was on standardisation, \Haskell{}
also has several new features that both simplify and
generalise the design.  For example,
\begin{enumerate}
\item Rather than using {\em ad hoc} techniques for
overloading,\index{overloading}
\Haskell{} provides an explicit overloading facility, integrated with
the polymorphic type system\index{type system}, that allows the
precise definition of overloading behaviour for any operator or function.

%For example, this facility provides a simple way to handle
%``equality types'' (types whose elements may be tested for generic
%equality), a problem that arises with Hindley-Milner style
%polymorphism, and that is usually treated in an {\em ad hoc} manner.

\item The conventional notion of ``abstract data
type''\index{abstract datatype}
has been unbundled
into two orthogonal components:
data abstraction\index{data abstraction}
and information hiding.\index{information hiding}

\item \Haskell{} has a flexible I/O facility that unifies two
popular styles of purely functional I/O---the {\em stream} model and
the {\em continuation} model---and both styles can be mixed within the same
program.  The system supports most of the standard operations provided by
conventional operating systems while retaining referential
transparency within a program.

\item Recognising the importance of arrays, \Haskell{} has a
family of multidimensional non-strict immutable arrays\index{array}
whose special interaction with list comprehensions provides a
convenient ``array comprehension'' syntax for defining arrays
monolithically.
\end{enumerate}

This report defines the syntax for \Haskell{} programs and an
informal abstract semantics for the meaning of such
programs; the formal abstract semantics is in preparation.%
\index{formal semantics}
We leave as implementation
dependent the ways in which \Haskell{} programs are to be
manipulated, interpreted, compiled, etc.  This includes such issues as
the nature of batch versus interactive programming environments, and
the nature of error messages returned for undefined programs
(i.e.~programs that formally evaluate to $\bot$).

\subsection{Program Structure}\index{program structure}
\label{programs}

In this section, we describe the abstract syntactic and semantic structure of
\Haskell{}, as well as how it relates to the organisation of the
rest of the report.
\begin{enumerate}
\item At the topmost level a \Haskell{} program is a set
of {\em modules} (described in Section~\ref{modules}).  Modules provide
a way to control namespaces
and to re-use software in large programs.

\item The top level of a module consists of a collection of
{\em declarations}, of which there are several kinds, all described
in Section~\ref{declarations}.  Declarations
define things such as ordinary values, datatypes, type
classes, and fixity information.

\item At the next lower level are {\em expressions}, described
in Section~\ref{expressions}.  An expression denotes a {\em value}
and has a {\em static type}; expressions are at the heart of
\Haskell{} programming ``in the small.''

\item At the bottom level is \Haskell{}'s {\em
lexical structure}, defined in Section~\ref{lexical-structure}.  The
lexical structure captures the concrete
representation of \Haskell{} programs in text files.

\end{enumerate}
This report proceeds bottom-up with respect
to \Haskell{}'s syntactic structure.

The sections not mentioned above are Section~\ref{basic-types}, which
describes the standard built-in datatypes in \Haskell{}, and
Section~\ref{io}, which discusses the I/O facility in \Haskell{}
(i.e.~how \Haskell{} programs communicate with the outside world).
Also, there are several appendices describing the
standard prelude, the
concrete syntax,
%% static and dynamic formal semantics,
the semantics of I/O,
and the specification of derived instances.

Examples of \Haskell{} program fragments in running text are given
in typewriter font:
% highlighted with a vertical line to the left of the text:
% as in:
\bprog
\mbox{\tt \ let\ x\ =\ 1}\\
\mbox{\tt \ \ \ \ \ z\ =\ x+y}\\
\mbox{\tt \ in\ \ z+1}
\eprog
``Holes'' in program fragments representing arbitrary
pieces of \Haskell{} code are written in italics, as in 
\mbox{$\it \makebox{\tt if}\ e_1\ \makebox{\tt then}\ e_2\ \makebox{\tt else}\ e_3$}.  Generally the italicised names will
be mnemonic, such as \mbox{$\it e$} for expressions, \mbox{$\it d$} for
declarations, \mbox{$\it t$} for types, etc.

\subsection{The \Haskell{} Kernel}
\index{Haskell kernel@\Haskell{} kernel}
\label{intro-kernel}

\Haskell{} has adopted many of the convenient syntactic structures
that have become popular
in functional programming.  In all cases, their formal
semantics can be given via translation into a proper subset of
\Haskell{} called the \Haskell{} {\em kernel}.  It is
essentially a slightly sugared variant of the lambda calculus with
a straightforward denotational semantics.  The translation of each
syntactic structure into the kernel is given as the syntax is
introduced.
% and the formal semantics of the kernel is given in
% Appendix~\ref{formal-semantics}.
This modular design facilitates
reasoning about \Haskell{} programs and provides useful guidelines
for implementors of the language.

% In specifying the translation of special syntax, named entities are
% often referred to ``as defined in the standard prelude.''  This means
% that even if these names are rebound (i.e.~the standard prelude name
% is not currently in scope), the translation still takes on the meaning
% as defined in the standard prelude.  In other words, the meaning of
% \Haskell{}'s syntax is invariant.

\subsection{Values and Types}
\index{value}\index{type}
\label{errors}\index{error}

An expression\index{expression} evaluates to a {\em value} and has a
static {\em type}.  Values and types are not mixed in
\Haskell{}.
However, the type system
allows user-defined datatypes of various sorts, and permits not only
parametric polymorphism\index{polymorphism} (using a
traditional Hindley-Milner\index{Hindley-Milner type system} type structure) but
also {\em ad hoc} polymorphism, or {\em overloading} (using
{\em type classes}).

Errors in \Haskell{} are semantically equivalent to
$\bot$.  Technically, they are not distinguishable
from nontermination, so the language includes no mechanism
for detecting or acting upon errors.  Of course, implementations
will probably try to provide useful information about
errors.
% A more elaborate treatment of errors is left as a future extension to
% \Haskell{}.

\subsection{Namespaces}
\index{namespaces}
\label{namespaces}

There are six kinds of names in \Haskell{}: those for {\em variables} and
{\em constructors} denote values; those for {\em type variables}, {\em
type constructors}, and {\em type classes} refer to entities related
to the type system; and {\em module names} refer to modules.
There are three constraints on naming:\nopagebreak[4]
\begin{enumerate}
\item Names for variables and type variables are identifiers beginning
      with small letters; the other four kinds of names are
      identifiers beginning with capitals.
\item Constructor operators are operators beginning with ``\mbox{\tt :}'';
      variable operators are operators not beginning with ``\mbox{\tt :}''.
\item An identifier must not be used as the name of a type constructor
      and a class in the same scope.
\end{enumerate}
These are the only constraints; for example,
\mbox{\tt Int} may simultaneously be the name of a module, class, and constructor
within a single scope.

\Haskell{} provides a lexical syntax for infix {\em
operators} (either functions or constructors).  To emphasise
that operators are bound to the same things as identifiers, and to
allow the two to be used interchangeably, there is a simple way to
convert between the two: any function or constructor identifier may be
converted into an operator by enclosing it in grave accents, and any
operator may be converted into an identifier by enclosing it in
parentheses.  For example, \mbox{\tt x\ +\ y} is equivalent to
\mbox{\tt (+)\ x\ y}, and \mbox{\tt f\ x\ y} is the same as
\mbox{\mbox{\tt x} \bkqB\mbox{\tt f}\bkqA\mbox{\tt \ y}}.
These lexical matters are discussed further in
Section~\ref{lexical-structure}.

\subsection{Layout}\index{layout}
\label{lexemes-layout}

In the syntax given in the rest of the report, {\em declaration
lists} are always preceded by one of the keywords \mbox{\tt let}, \mbox{\tt where},
or \mbox{\tt of}, and are enclosed within curly braces (\mbox{\tt {\char'173}\ {\char'175}}) with
the individual declarations separated (or terminated) by semicolons
(\mbox{\tt ;}). For example, the syntax of a \mbox{\tt let} expression is:
\[
\mbox{$\it \makebox{\tt let\ {\char'173}}\ decl_1\ \makebox{\tt ;}\ decl_2\ \makebox{\tt ;}\ \ldots \ \makebox{\tt ;}\ decl_n\ \makebox{\tt {\char'175}\ in\ }\ exp$}
\]

%
% $Header$
%
% partain:
% in a separate file, because it is included twice (for now);
%  in intro.verb and syntax.verb

\Haskell{} permits the omission of the braces and semicolons by
using {\em layout} to convey the same information.  This allows both
layout-sensitive and -insensitive styles of coding, which
can be freely mixed within one program.  Because layout is
not required, \Haskell{} programs can be straightforwardly
produced by other programs.
% without worrying about deeply nested layout difficulties.

The layout (or ``off-side'') rule\index{off-side rule} takes effect whenever the
open brace is omitted after the keyword \mbox{\tt where}, \mbox{\tt let} or \mbox{\tt of}.
When this happens, the indentation of the next lexeme (whether or not
on a new line) is remembered and the omitted open brace is inserted
(the whitespace preceding the lexeme may include comments).
For each subsequent line, if it contains only whitespace or is
indented more, then the previous item is continued (nothing is
inserted); if it is indented the same amount, then a new item begins
(a semicolon is inserted); and if it is indented less, then the
declaration list ends (a close brace is inserted).  A close brace is
also inserted whenever the syntactic category containing the
declaration list ends; that is, if an illegal lexeme is encountered at a
point where a close brace would be legal, a close brace is inserted.
The layout rule will match only those open braces
that it has inserted; an
open brace that the user has inserted must be
matched by a close brace inserted by the user.

Given these rules, a single newline may actually terminate several
declaration lists.  Also, these rules permit:
\bprog
\mbox{\tt f\ x\ =\ let\ a\ =\ 1;\ b\ =\ 2\ }\\
\mbox{\tt \ \ \ \ \ \ \ \ \ \ g\ y\ =\ exp2\ in\ exp1}
\eprog
making \mbox{\tt a}, \mbox{\tt b} and \mbox{\tt g} all part of the same declaration
list.

To facilitate the use of layout at the top level of a module
(several modules may reside in one file), the keywords
\mbox{\tt module} and \mbox{\tt interface} and the end-of-file token are assumed to occur in column
0 (whereas normally the first column is 1).  Otherwise, all
top-level declarations would have to be indented.


As an example, Figure~\ref{layout-before} shows a (somewhat contrived)
module and Figure~\ref{layout-after} shows the result of applying the
layout rule to it.  Note in particular: (a)~the line beginning \mbox{\tt {\char'175}{\char'175};pop},
where the termination of the previous line invokes three applications
of the layout rule, corresponding to the depth (3) of the nested
\mbox{\tt where} clauses, (b)~the close braces in the \mbox{\tt where} clause nested
within the tuple and \mbox{\tt case} expression, inserted because the end of the
tuple was detected, and (c)~the close brace at the very end, inserted
because of the column 0 indentation of the end-of-file token.

\begin{figure}
\outline{
\mbox{\tt module\ AStack(\ Stack,\ push,\ pop,\ top,\ size\ )\ where}\\
\mbox{\tt data\ Stack\ a\ =\ Empty\ }\\
\mbox{\tt \ \ \ \ \ \ \ \ \ \ \ \ \ |\ MkStack\ a\ (Stack\ a)}\\
\mbox{\tt }\\[-8pt]
\mbox{\tt push\ ::\ a\ ->\ Stack\ a\ ->\ Stack\ a}\\
\mbox{\tt push\ x\ s\ =\ MkStack\ x\ s}\\
\mbox{\tt }\\[-8pt]
\mbox{\tt size\ ::\ Stack\ a\ ->\ Integer}\\
\mbox{\tt size\ s\ =\ length\ (stkToLst\ s)\ \ where}\\
\mbox{\tt \ \ \ \ \ \ \ \ \ \ \ stkToLst\ \ Empty\ \ \ \ \ \ \ \ \ =\ []}\\
\mbox{\tt \ \ \ \ \ \ \ \ \ \ \ stkToLst\ (MkStack\ x\ s)\ \ =\ x:xs\ where\ xs\ =\ stkToLst\ s}\\
\mbox{\tt }\\[-8pt]
\mbox{\tt pop\ ::\ Stack\ a\ ->\ (a,\ Stack\ a)}\\
\mbox{\tt pop\ (MkStack\ x\ s)}\\
\mbox{\tt \ \ =\ (x,\ case\ s\ of\ r\ ->\ i\ r\ where\ i\ x\ =\ x)\ --\ (pop\ Empty)\ is\ an\ error}\\
\mbox{\tt }\\[-8pt]
\mbox{\tt top\ ::\ Stack\ a\ ->\ a}\\
\mbox{\tt top\ (MkStack\ x\ s)\ =\ x\ \ \ \ \ \ \ \ \ \ \ \ \ \ \ \ \ \ \ \ \ --\ (top\ Empty)\ is\ an\ error}
}
\ecaption{A sample program}
\label{layout-before}
\outline{
\mbox{\tt module\ AStack(\ Stack,\ push,\ pop,\ top,\ size\ )\ where}\\
\mbox{\tt {\char'173}data\ Stack\ a\ =\ Empty\ }\\
\mbox{\tt \ \ \ \ \ \ \ \ \ \ \ \ \ |\ MkStack\ a\ (Stack\ a)}\\
\mbox{\tt }\\[-8pt]
\mbox{\tt ;push\ ::\ a\ ->\ Stack\ a\ ->\ Stack\ a}\\
\mbox{\tt ;push\ x\ s\ =\ MkStack\ x\ s}\\
\mbox{\tt }\\[-8pt]
\mbox{\tt ;size\ ::\ Stack\ a\ ->\ Integer}\\
\mbox{\tt ;size\ s\ =\ length\ (stkToLst\ s)\ \ where}\\
\mbox{\tt \ \ \ \ \ \ \ \ \ \ \ {\char'173}stkToLst\ \ Empty\ \ \ \ \ \ \ \ \ =\ []}\\
\mbox{\tt \ \ \ \ \ \ \ \ \ \ \ ;stkToLst\ (MkStack\ x\ s)\ \ =\ x:xs\ where\ {\char'173}xs\ =\ stkToLst\ s}\\
\mbox{\tt }\\[-8pt]
\mbox{\tt {\char'175}{\char'175};pop\ ::\ Stack\ a\ ->\ (a,\ Stack\ a)}\\
\mbox{\tt ;pop\ (MkStack\ x\ s)}\\
\mbox{\tt \ \ =\ (x,\ case\ s\ of\ {\char'173}r\ ->\ i\ r\ where\ {\char'173}i\ x\ =\ x{\char'175}{\char'175})\ --\ (pop\ Empty)\ is\ an\ error}\\
\mbox{\tt }\\[-8pt]
\mbox{\tt ;top\ ::\ Stack\ a\ ->\ a}\\
\mbox{\tt ;top\ (MkStack\ x\ s)\ =\ x\ \ \ \ \ \ \ \ \ \ \ \ \ \ \ \ \ \ \ \ \ \ \ \ --\ (top\ Empty)\ is\ an\ error}\\
\mbox{\tt {\char'175}}
}
\ecaption{Sample program with layout expanded}
\label{layout-after}

\end{figure}

When comparing indentations for standard \Haskell{} programs, a
fixed-width font with this tab convention is assumed: tab
stops are 8 characters apart (with the first tab stop in column 9),
and a tab character causes the insertion of enough spaces (always
$\geq 1$) to align the current position with the next tab stop.
Particular implementations may alter this rule to accommodate
variable-width fonts and alternate tab conventions, but standard
\Haskell{} (i.e., portable) programs must observe this rule.

% Local Variables: 
% mode: latex
% End:
