%
% $Header$
%
\section{Declarations and Bindings}
\index{declaration}
\index{binding}
\label{declarations}

In this section, we describe the syntax and informal semantics of \Haskell{}
{\em declarations}.
% including their translations into
% the \Haskell{} kernel where appropriate.
% (see Appendix~\ref{formal-semantics} for a complete formal semantics).

\begin{flushleft}\it\begin{tabbing}
\hspace{0.5in}\=\hspace{3.0in}\=\kill
$\it module$\>\makebox[3.5em]{$\rightarrow$}$\it \makebox{\tt module}\ modid\ [exports]\ \makebox{\tt where}\ body$\\ 
$\it $\>\makebox[3.5em]{$|$}$\it body$\\ 
$\it body$\>\makebox[3.5em]{$\rightarrow$}$\it \makebox{\tt {\char'173}}\ [impdecls\ \makebox{\tt ;}]\ [[fixdecls\ \makebox{\tt ;}]\ topdecls\ [\makebox{\tt ;}]]\ \makebox{\tt {\char'175}}$\\ 
$\it $\>\makebox[3.5em]{$|$}$\it \makebox{\tt {\char'173}}\ impdecls\ [\makebox{\tt ;}]\ \makebox{\tt {\char'175}}$\\ 
$\it $\\ 
$\it topdecls$\>\makebox[3.5em]{$\rightarrow$}$\it topdecl_1\ \makebox{\tt ;}\ \ldots \ \makebox{\tt ;}\ topdecl_n$\>\makebox[3em]{}$\it \qquad\ (n\geq 1)$\\ 
$\it topdecl$\>\makebox[3.5em]{$\rightarrow$}$\it \makebox{\tt type}\ simple\ \makebox{\tt =}\ type$\\ 
$\it $\>\makebox[3.5em]{$|$}$\it \makebox{\tt data}\ [context\ \makebox{\tt =>}]\ simple\ \makebox{\tt =}\ constrs\ [\makebox{\tt deriving}\ (tycls\ |\ \makebox{\tt (}tyclses\makebox{\tt )})]$\\ 
$\it $\>\makebox[3.5em]{$|$}$\it \makebox{\tt class}\ [context\ \makebox{\tt =>}]\ class\ [\makebox{\tt where}\ \makebox{\tt {\char'173}}\ cbody\ [\makebox{\tt ;}]\ \makebox{\tt {\char'175}}]$\\ 
$\it $\>\makebox[3.5em]{$|$}$\it \makebox{\tt instance}\ [context\ \makebox{\tt =>}]\ tycls\ inst\ [\makebox{\tt where}\ \makebox{\tt {\char'173}}\ valdefs\ [\makebox{\tt ;}]\ \makebox{\tt {\char'175}}]$\\ 
$\it $\>\makebox[3.5em]{$|$}$\it \makebox{\tt default}\ (type\ |\ \makebox{\tt (}type_1\ \makebox{\tt ,}\ \ldots \ \makebox{\tt ,}\ type_n\makebox{\tt )})$\>\makebox[3em]{}$\it \qquad\ (n\geq 0)$\\ 
$\it $\>\makebox[3.5em]{$|$}$\it decl$\\ 
$\it $\\ 
$\it decls$\>\makebox[3.5em]{$\rightarrow$}$\it decl_1\ \makebox{\tt ;}\ \ldots \ \makebox{\tt ;}\ decl_n$\>\makebox[3em]{}$\it \qquad\ (n\geq 0)$\\ 
$\it decl$\>\makebox[3.5em]{$\rightarrow$}$\it vars\ \makebox{\tt ::}\ [context\ \makebox{\tt =>}]\ type$\\ 
$\it $\>\makebox[3.5em]{$|$}$\it valdef$
\end{tabbing}\end{flushleft}
\indexsyn{module}%
\indexsyn{body}%
\indexsyn{topdecls}%
\indexsyn{topdecl}%
\indexsyn{decls}%
\indexsyn{decl}%

The declarations in the syntactic category \mbox{$\it topdecls$} are only allowed
at the top level of a \Haskell{} module (see
Section~\ref{modules}), whereas \mbox{$\it decls$} may be used either at the top level or
in nested scopes (i.e.~those within a \mbox{\tt let} or \mbox{\tt where} construct).

For exposition, we divide the declarations into
three groups: user-defined datatypes, consisting of \mbox{\tt type} and \mbox{\tt data}
declarations (Section~\ref{user-defined-datatypes}); type classes and
overloading, consisting of \mbox{\tt class}, \mbox{\tt instance}, and \mbox{\tt default}
declarations (Section~\ref{overloading}); and nested declarations,
consisting of value bindings and type signatures
(Section~\ref{nested}).

%The \mbox{\tt module} declaration, along with \mbox{\tt import} and
%infix declarations, is described in Section~\ref{modules}.

\Haskell{} has several primitive datatypes that are ``hard-wired''
(such as integers and arrays), but most ``built-in'' datatypes are
defined in the standard prelude with normal \Haskell{} code, using
\mbox{\tt type} and \mbox{\tt data} declarations. % (see Section~\ref{user-defined-datatypes}).
These ``built-in'' datatypes are described in 
detail in Section~\ref{basic-types}.

\subsection{Overview of Types and Classes}
\label{types-overview}

\Haskell{} uses a traditional
Hindley-Milner\index{Hindley-Milner type system}
polymorphic type system to provide a static type semantics
\cite{damas-milner82,hindley69}, but the type system has been extended with
{\em type classes} (or just {\em classes}\index{class}) that provide
a structured way to introduce {\em overloaded} functions.
This is the major technical innovation in \Haskell{}.

A \mbox{\tt class} declaration (Section~\ref{class-decls}) introduces a new
{\em type class} and the overloaded {\em operations} that must be
supported by any type that is an instance of that class.  An
\mbox{\tt instance} declaration (Section~\ref{instance-decls}) declares that a
type is an {\em instance} of a class and includes
the definitions of the overloaded operations---called {\em
methods}---instantiated on the named type.
\index{class method}

For example, suppose we wish to overload the operations \mbox{\tt (+)} and
\mbox{\tt negate} on types \mbox{\tt Int} and \mbox{\tt Float}.  We introduce a new
type class called \mbox{\tt Num}:\nopagebreak[4]
\bprog
\mbox{\tt class\ Num\ a\ \ where\ \ \ \ \ \ \ \ \ \ --\ simplified\ class\ declaration\ for\ Num}\\
\mbox{\tt \ \ (+)\ \ \ \ ::\ a\ ->\ a\ ->\ a}\\
\mbox{\tt \ \ negate\ ::\ a\ ->\ a}
\eprog
This declaration may be read ``a type \mbox{\tt a} is an instance of the class
\mbox{\tt Num} if there are (overloaded) operations \mbox{\tt (+)} and \mbox{\tt negate}, of the
appropriate types, defined on it.''

We may then declare \mbox{\tt Int} and \mbox{\tt Float} to be instances of this class:
\bprog
\mbox{\tt instance\ Num\ Int\ \ where\ \ \ \ \ --\ simplified\ instance\ of\ Num\ Int}\\
\mbox{\tt \ \ x\ +\ y\ \ \ \ \ \ \ =\ \ addInt\ x\ y}\\
\mbox{\tt \ \ negate\ x\ \ \ \ =\ \ negateInt\ x}\\
\mbox{\tt }\\[-8pt]
\mbox{\tt instance\ Num\ Float\ \ where\ \ \ --\ simplified\ instance\ of\ Num\ Float}\\
\mbox{\tt \ \ x\ +\ y\ \ \ \ \ \ \ =\ \ addFloat\ x\ y}\\
\mbox{\tt \ \ negate\ x\ \ \ \ =\ \ negateFloat\ x}
\eprog
where \mbox{\tt addInt}, \mbox{\tt negateInt}, \mbox{\tt addFloat}, and \mbox{\tt negateFloat} are assumed
in this case to be primitive functions, but in general could be any
user-defined function.  The first declaration above may be read
``\mbox{\tt Int} is an instance of the class \mbox{\tt Num} as witnessed by these
definitions (i.e.~methods)\index{class method} for \mbox{\tt (+)} and \mbox{\tt negate}.''

More examples can be found in Wadler and Blott's paper
\cite{wadler:classes}.

\subsubsection{Syntax of Types}
\label{type-syntax}
\index{type}
\label{types}

\begin{flushleft}\it\begin{tabbing}
\hspace{0.5in}\=\hspace{3.0in}\=\kill
$\it type$\>\makebox[3.5em]{$\rightarrow$}$\it atype$\\ 
$\it $\>\makebox[3.5em]{$|$}$\it type_1\ \makebox{\tt ->}\ type_2$\\ 
$\it $\>\makebox[3.5em]{$|$}$\it tycon\ atype_1\ \ldots \ atype_k$\>\makebox[3em]{}$\it (\arity{tycon}=k\geq 1)$\\ 
$\it $\\ 
$\it atype$\>\makebox[3.5em]{$\rightarrow$}$\it tyvar$\\ 
$\it $\>\makebox[3.5em]{$|$}$\it tycon$\>\makebox[3em]{}$\it (\arity{tycon}=0)$\\ 
$\it $\>\makebox[3.5em]{$|$}$\it \makebox{\tt ()}$\>\makebox[3em]{}$\it (\tr{unit\ type})$\\ 
$\it $\>\makebox[3.5em]{$|$}$\it \makebox{\tt (}\ type\ \makebox{\tt )}$\>\makebox[3em]{}$\it (\tr{parenthesised\ type})$\\ 
$\it $\>\makebox[3.5em]{$|$}$\it \makebox{\tt (}\ type_1\ \makebox{\tt ,}\ \ldots \ \makebox{\tt ,}\ type_k\ \makebox{\tt )}$\>\makebox[3em]{}$\it (\tr{tuple\ type},\ k\geq 2)$\\ 
$\it $\>\makebox[3.5em]{$|$}$\it \makebox{\tt [}\ type\ \makebox{\tt ]}$
\end{tabbing}\end{flushleft}
\indexsyn{type}%
\indexsyn{atype}%
%\ToDo{I left off \mbox{$\it tyvar$} and \mbox{$\it tycon$}}

\noindent
The syntax for \Haskell{}
{\em type expressions} is given above.\index{type expression} 
They are built in the usual way
from type variables, function types, type constructors, tuple types,
and list types.  Type variables are identifiers beginning with a lower-case
letter and type constructors are identifiers beginning with an upper-case
letter.  A type is one of:\nopagebreak[4]
\begin{enumerate}
\item A {\em function type}\index{function type} having form 
\mbox{$\it t_1\ \makebox{\tt ->}\ t_2$}.  Function arrows associate to the right.

\item A {\em constructed type}\index{constructed type} having form 
\mbox{$\it T\ t_1\ \ldots \ t_k$}, where \mbox{$\it T$} is a type constructor of arity \mbox{$\it k$}.

\item A {\em tuple type}\index{tuple type} having form 
\mbox{$\it \makebox{\tt (}t_1\makebox{\tt ,}\ \ldots \makebox{\tt ,}\ t_k\makebox{\tt )}$} where \mbox{$\it k\geq 2$}.  It denotes the type of
\mbox{$\it k$}-tuples with the first component of type \mbox{$\it t_1$}, the second
component of type \mbox{$\it t_2$}, and so on (see Sections~\ref{tuples}
and \ref{basic-tuples}).

\item A {\em list type}\index{list type} has the form \mbox{$\it \makebox{\tt [}t\makebox{\tt ]}$}.  
It denotes the type of lists with elements of type \mbox{$\it t$} (see
Sections~\ref{lists} and \ref{basic-lists}).

\item The {\em trivial type}\index{trivial type} having form \mbox{\tt ()}.
% was ``degenerate tuple'' (mismatched exps.verb)
It denotes the ``nullary tuple'' type, and has exactly one value,
also written \mbox{\tt ()} (see Sections~\ref{unit-expression}
and~\ref{basic-trivial}).

\item A {\em parenthesised type}, having form \mbox{$\it \makebox{\tt (}t\makebox{\tt )}$}, is identical to
the type \mbox{$\it t$}.
\end{enumerate}

Although the tuple, list, and trivial types have special syntax, they
are not different from user-defined types with equivalent
functionality.

% \outline{
% \paragraph{Translation:}  
% The list type \mbox{$\it \makebox{\tt [}t\makebox{\tt ]}$} is equivalent to the constructed type 
% \mbox{$\it \makebox{\tt List}\ t$}, where \mbox{\tt List} is a datatype defined in the standard prelude
% (see Section~\ref{lists}).  Similarly, the tuple type
% \mbox{$\it \makebox{\tt (}t_1\makebox{\tt ,}\ldots \makebox{\tt ,}t_k\makebox{\tt )}$} is syntax for \mbox{$\it \makebox{\tt Tuple}k\ t_1\ \ldots \ t_k$}, where
% \mbox{$\it \makebox{\tt Tuple}k$} is a type constructor implicitly defined for k-tuples in
% the standard prelude (see Section~\ref{tuples}).  Finally, the
% trivial type \mbox{\tt ()} is syntax for the type \mbox{\tt Triv} defined in the
% standard prelude (see Section~\ref{basic-trivial}).
% }

Expressions and types have a consistent syntax.
If \mbox{$\it t_i$} is the type of
expression or pattern \mbox{$\it e_i$}, then the expressions \mbox{$\it \makebox{\tt ({\char'134}}\ e_1\ \makebox{\tt ->}\ e_2\makebox{\tt )}$},
\mbox{$\it \makebox{\tt [}e_1\makebox{\tt ]}$}, and \mbox{$\it \makebox{\tt (}e_1,e_2\makebox{\tt )}$} have the types \mbox{$\it \makebox{\tt (}t_1\ \makebox{\tt ->}\ t_2\makebox{\tt )}$},
\mbox{$\it \makebox{\tt [}t_1\makebox{\tt ]}$}, and \mbox{$\it \makebox{\tt (}t_1,t_2\makebox{\tt )}$}, respectively.

With one exception, the type variables in a \Haskell{} type expression
are all assumed to be universally quantified; there is no explicit
syntax for universal quantification \cite{damas-milner82,reynolds90}.
For example, the type expression
\mbox{\tt a\ ->\ a} denotes the type $\forall a.~a \rightarrow a$.
For clarity, however, we will often write quantification explicitly
when discussing the types of \Haskell{} programs.

The exception referred to is that of the distinguished type variable
in a class declaration (Section~\ref{class-decls}).

%Every type variable appearing in a signature
%is universally quantified over that signature.  This last
%constraint implies that signatures such as:
%\bprog
%@
%       \ x -> ([x] :: [a])
%@
%\eprog
%are not valid, because this declares \mbox{\tt [x]} to be of type 
%\mbox{$\it \forall\ a.\makebox{\tt [}a\makebox{\tt ]}$}.  In contrast, this {\em is} a valid signature:
%\mbox{\tt ({\char'134}\ x\ ->\ [x])\ ::\ a\ ->\ [a]}; it declares that \mbox{\tt ({\char'134}\ x\ ->\ [x])} has type
%\mbox{$\it \forall\ a.a\ \makebox{\tt ->}\ \makebox{\tt [}a\makebox{\tt ]}$}.

\subsubsection{Syntax of Class Assertions and Contexts}
\index{class assertion}
\index{context}
\label{classes&contexts}

\begin{flushleft}\it\begin{tabbing}
\hspace{0.5in}\=\hspace{3.0in}\=\kill
$\it context$\>\makebox[3.5em]{$\rightarrow$}$\it class$\\ 
$\it $\>\makebox[3.5em]{$|$}$\it \makebox{\tt (}\ class_1\ \makebox{\tt ,}\ \ldots \ \makebox{\tt ,}\ class_n\ \makebox{\tt )}$\>\makebox[3em]{}$\it (n\geq 1)$\\ 
$\it class$\>\makebox[3.5em]{$\rightarrow$}$\it tycls\ tyvar$\\ 
$\it tycls$\>\makebox[3.5em]{$\rightarrow$}$\it aconid$\\ 
$\it tyvar$\>\makebox[3.5em]{$\rightarrow$}$\it avarid$
\end{tabbing}\end{flushleft}
\indexsyn{context}%
\indexsyn{class}%
\indexsyn{tycls}%
\indexsyn{tyvar}%
A {\em class assertion} has form \mbox{$\it tycls\ tyvar$}, and
indicates the membership of the parameterised type \mbox{$\it tyvar$} in the class
\mbox{$\it tycls$}.  A class identifier begins with a capital
letter.

A {\em context} consists of one or more class assertions,
and has the general form
\[
\mbox{$\it \makebox{\tt (}\ C_1\ u_1,\ \ldots ,\ C_n\ u_n\ \makebox{\tt )}$}
\]
where \mbox{$\it C_1,\ \ldots ,\ C_n$} are class identifiers, and \mbox{$\it u_1,\ \ldots ,\ u_n$} are
type variables; the parentheses may be omitted when \mbox{$\it n=1$}.  In
general, we use \mbox{$\it c$} to denote a context and we write \mbox{$\it c\ \makebox{\tt =>}\ t$} to
indicate the type \mbox{$\it t$} restricted by the context \mbox{$\it c$}.
The context \mbox{$\it c$} must only contain type variables referenced in \mbox{$\it t$}.
For convenience,
we write \mbox{$\it c\ \makebox{\tt =>}\ t$} even if the context \mbox{$\it c$} is empty, although in this
case the concrete syntax contains no \mbox{\tt =>}.

\subsubsection{Semantics of Types and Classes}
\label{type-semantics}

In this subsection, we provide informal details of the type system.
% the formal semantics is described in Appendix~\ref{static-semantics}
(Wadler and Blott \cite{wadler:classes} discuss type
classes further.)

%A type is a {\em monotype\/}\index{monotype} if it contains no type
%variables, and is {\em monomorphic\/}
%\index{monomorphic type}
%if it contains type variables
%but is not polymorphic (in Milner's original terminology,
%it is monomorphic if it contains no generic type variables).

The \Haskell{} type system attributes a {\em type} to each
\index{type}
expression in the program.  In general, a type is of the form
$\forall \overline{u}.~c \Rightarrow t$,
where $\overline{u}$ is a set of type variables \mbox{$\it u_1,\ \ldots ,\ u_n$}.
In any such type, any of the universally-quantified type variables \mbox{$\it u_i$}
which are free in \mbox{$\it c$} must also be free in \mbox{$\it t$}.

The type of an expression $e$
depends on a {\em type environment}
\index{type environment}
that gives types for the free variables in \mbox{$\it e$}, and a
{\em class environment}
\index{class environment}
that declares which types are instances of which classes (a type becomes
an instance of a class only via the presence of an
\mbox{\tt instance} declaration or a \mbox{\tt deriving} clause).

Types are related by a generalisation order
\index{generalisation order}
(specified below);
the most general type that can be assigned to a particular
expression (in a given environment) is called its {\em
principal type}.
\index{principal type}
\Haskell{}'s extended Hindley-Milner type system can infer the principal
type of all expressions, including the proper use of overloaded
operations (although certain ambiguous overloadings could arise, as
described in Section~\ref{default-decls}).  Therefore, explicit typings (called
{\em type signatures})
\index{type signature}
are optional (see
Sections~\ref{expression-type-sigs} and~\ref{type-signatures}).

The type $\forall \overline{u}.~c_1 \Rightarrow t_1$ is
{\em more general than}
the type $\forall \overline{w}.~c_2 \Rightarrow t_2$ if and only if there is 
a substitution \mbox{$\it S$} whose domain is $\overline{u}$ such that:
\begin{itemize}
\item \mbox{$\it t_2$} is identical to \mbox{$\it S(t_1)$}.
\item Whenever \mbox{$\it c_2$} holds in the class environment, \mbox{$\it S(c_1)$} also holds.
\end{itemize}

The main point about contexts above is that, given the type
$\forall \overline{u}.~c \Rightarrow t$,
the presence of \mbox{$\it C\ u_i$} in the context \mbox{$\it c$} expresses the
constraint that the type variable \mbox{$\it u_i$} may be instantiated as \mbox{$\it t'$}
within the type expression \mbox{$\it t$} only if \mbox{$\it t'$} is a member of the class
\mbox{$\it C$}.  For example, consider the function \mbox{\tt double}:
\bprog
\mbox{\tt \ \ \ \ \ \ \ \ double\ x\ =\ x\ +\ x}
\eprog
The most general type of \mbox{\tt double} is
$\forall a.~\mbox{\tt Num}~a \Rightarrow a \rightarrow a$.
\mbox{\tt double} may be applied to values of type \mbox{\tt Int} (instantiating \mbox{$\it a$} to
\mbox{\tt Int}), since \mbox{\tt Int} is an instance of the class \mbox{\tt Num}.  However,
\mbox{\tt double} may not be applies to values of type \mbox{\tt Char}, because \mbox{\tt Char}
is not an instance of class \mbox{\tt Num}.


\subsection{User-Defined Datatypes}
\index{datatype}
\label{user-defined-datatypes}

In this section, we describe algebraic datatypes (\mbox{\tt data} declarations)
and type synonyms (\mbox{\tt type} declarations).  These declarations
may only appear at the top level of a module.

\subsubsection{Algebraic Datatype Declarations}
\index{algebraic datatype}
\label{datatype-decls}

\begin{flushleft}\it\begin{tabbing}
\hspace{0.5in}\=\hspace{3.0in}\=\kill
$\it topdecl$\>\makebox[3.5em]{$\rightarrow$}$\it \makebox{\tt data}\ [context\ \makebox{\tt =>}]\ simple\ \makebox{\tt =}\ constrs\ [\makebox{\tt deriving}\ (tycls\ |\ \makebox{\tt (}tyclses\makebox{\tt )})]$\\ 
$\it simple$\>\makebox[3.5em]{$\rightarrow$}$\it tycon\ tyvar_1\ \ldots \ tyvar_k$\>\makebox[3em]{}$\it (\arity{tycon}=k\geq 0)$\\ 
$\it constrs$\>\makebox[3.5em]{$\rightarrow$}$\it constr_1\ \makebox{\tt |}\ \ldots \ \makebox{\tt |}\ constr_n$\>\makebox[3em]{}$\it (n\geq 1)$\\ 
$\it constr$\>\makebox[3.5em]{$\rightarrow$}$\it con\ atype_1\ \ldots \ atype_k$\>\makebox[3em]{}$\it (\arity{con}=k\geq 0)$\\ 
$\it $\>\makebox[3.5em]{$|$}$\it type_1\ conop\ type_2$\>\makebox[3em]{}$\it (\infix{conop})$\\ 
$\it tyclses$\>\makebox[3.5em]{$\rightarrow$}$\it tycls_1\makebox{\tt ,}\ \ldots \makebox{\tt ,}\ tycls_n$\>\makebox[3em]{}$\it (n\geq 0)$
\end{tabbing}\end{flushleft}
\index{topdecl@{\em topdecl} (\mbox{\tt data})}%
\indexsyn{simple}%
\indexsyn{constrs}%
\indexsyn{constr}%
\indexsyn{tyclses}%
\index{data declaration@{\ptt data} declaration}
The precedence\index{precedence} for \mbox{$\it constr$} is the same as that for
expressions---normal constructor application has higher precedence
than infix constructor application (thus \mbox{\tt a\ :\ Foo\ a} parses as 
\mbox{\tt a\ :\ (Foo\ a)}).

An algebraic datatype declaration introduces a new type
and constructors over that type and has the form:
\[
\mbox{$\it \makebox{\tt data}\ c\ \makebox{\tt =>}\ T\ u_1\ \ldots \ u_k\ \makebox{\tt =}\ K_1\ t_{11}\ \ldots \ t_{1k_1}\ \makebox{\tt |}\ \cdots\ \makebox{\tt |}\ K_n\ t_{n1}\ \ldots \ t_{nk_n}$}
\]
where \mbox{$\it c$} is a context.
\index{context!in data declaration@in {\ptt data} declaration}
This declaration
introduces a new type constructor \mbox{$\it T$} with constituent data
constructors \mbox{$\it K_1,\ \ldots ,\ K_n$} whose types are given by:
\[
\mbox{$\it K_i\ ::\ \forall\ u_1\ \ldots \ u_k.~\ c_i\ \Rightarrow\ t_{i1}\ \rightarrow\ \cdots\ \rightarrow\ t_{ik_i}\ \rightarrow\ (T\ u_1\ \ldots \ u_k)$}
\]
where \mbox{$\it c_i$} is the largest subset of \mbox{$\it c$} that constrains only those type
variables free in the types \mbox{$\it t_{i1},\ \ldots ,\ t_{ik_i}$}.
The type variables \mbox{$\it u_1$} through \mbox{$\it u_k$} must be distinct and may appear
in \mbox{$\it c$} and the $t_{ij}$; it is a static error
for any other type variable to appear in \mbox{$\it c$} or on the right-hand-side.

For example, the declaration
\bprog
\mbox{\tt \ \ \ \ \ \ \ \ data\ Eq\ a\ =>\ Set\ a\ =\ NilSet\ |\ ConsSet\ a\ (Set\ a)}
\eprog
introduces a type constructor \mbox{\tt Set}, and constructors \mbox{\tt NilSet} and
\mbox{\tt ConsSet} with types
\[\begin{array}{ll}
\mbox{\tt NilSet}  & \mbox{$\it ::\ \forall\ a.~\ \makebox{\tt Set}~\ a$} \\
\mbox{\tt ConsSet} & \mbox{$\it ::\ \forall\ a.~\ \makebox{\tt Eq}~\ a\ \Rightarrow\ a\ \rightarrow\ \makebox{\tt Set}~\ a\ \rightarrow\ \makebox{\tt Set}~\ a$}
\end{array}\]
In the example given, the overloaded
type for \mbox{\tt ConsSet} ensures that \mbox{\tt ConsSet} can only be applied to values whose
type is an instance of the class \mbox{\tt Eq}.  The context in the \mbox{\tt data}
declaration has no other effect whatsoever.  In particular, pattern
matching is unaffected.

The visibility of a datatype's constructors (i.e.~the ``abstractness''
of the datatype) outside of the module in which the datatype is
defined is controlled by the form of the datatype's name in the export
list as described in Section~\ref{abstract-types}.

The optional \mbox{$\it \makebox{\tt deriving}$} part of a \mbox{\tt data} declaration has to do
with {\em derived instances}, and is described in
Section~\ref{derived-decls}.

\subsubsection{Type Synonym Declarations}
\index{type synonym}
\label{type-synonym-decls}

\begin{flushleft}\it\begin{tabbing}
\hspace{0.5in}\=\hspace{3.0in}\=\kill
$\it topdecl$\>\makebox[3.5em]{$\rightarrow$}$\it \makebox{\tt type}\ simple\ \makebox{\tt =}\ type$\\ 
$\it simple$\>\makebox[3.5em]{$\rightarrow$}$\it tycon\ tyvar_1\ \ldots \ tyvar_k$\>\makebox[3em]{}$\it (\arity{tycon}=k\geq 0)$
\end{tabbing}\end{flushleft}
\index{topdecl@{\em topdecl} (\mbox{\tt type})}%
\indexsyn{simple}%
A type synonym declaration introduces a new type that
is equivalent to an old type and has the form
\[
\mbox{$\it \makebox{\tt type}\ T\ u_1\ \ldots \ u_k\ \makebox{\tt =}\ t$}
\]
which introduces a new type constructor, \mbox{$\it T$}.  The type \mbox{$\it (T\ t_1\ \ldots \\
\it t_k)$} is equivalent to the type \mbox{$\it t[t_1/u_1,\ \ldots ,\ t_k/u_k]$}.  The type
variables \mbox{$\it u_1$} through \mbox{$\it u_k$} must be distinct and are scoped only
over \mbox{$\it t$}; it is a static error for any other type variable to appear
in \mbox{$\it t$}.

Although recursive and mutually recursive datatypes are allowed,
\index{recursive datatype}
\index{type synonym!recursive}
this is not so for type synonyms, {\em unless an algebraic datatype
intervenes}.  For example,
\bprog
\mbox{\tt type\ Rec\ a\ \ \ =\ \ [Circ\ a]}\\
\mbox{\tt data\ Circ\ a\ \ =\ \ Tag\ [Rec\ a]}
\eprog
is allowed, whereas
\bprog
\mbox{\tt type\ Rec\ a\ \ \ =\ \ [Circ\ a]\ \ \ \ \ \ \ \ --\ ILLEGAL}\\
\mbox{\tt type\ Circ\ a\ \ =\ \ [Rec\ a]\ \ \ \ \ \ \ \ \ --}
\eprog
is not. Similarly, \mbox{\tt type\ Rec\ a\ =\ [Rec\ a]} is not allowed.

\subsection{Type Classes and Overloading}
\index{class}
\index{overloading}
\label{overloading}
\label{classes}

\subsubsection{Class Declarations}
\index{class declaration}
%\index{class declaration@{\ptt class} declaration}
\label{class-decls}

\begin{flushleft}\it\begin{tabbing}
\hspace{0.5in}\=\hspace{3.0in}\=\kill
$\it topdecl$\>\makebox[3.5em]{$\rightarrow$}$\it \makebox{\tt class}\ [context\ \makebox{\tt =>}]\ class\ [\makebox{\tt where}\ \makebox{\tt {\char'173}}\ cbody\ [\makebox{\tt ;}]\ \makebox{\tt {\char'175}}]$\\ 
$\it cbody$\>\makebox[3.5em]{$\rightarrow$}$\it [csigns\ \makebox{\tt ;}]\ [valdefs]$\\ 
$\it csigns$\>\makebox[3.5em]{$\rightarrow$}$\it csign_1\ \makebox{\tt ;}\ \ldots \ \makebox{\tt ;}\ csign_n$\>\makebox[3em]{}$\it (n\geq 1)$\\ 
$\it csign$\>\makebox[3.5em]{$\rightarrow$}$\it vars\ \makebox{\tt ::}\ [context\ \makebox{\tt =>}]\ type$\\ 
$\it vars$\>\makebox[3.5em]{$\rightarrow$}$\it var_1\ \makebox{\tt ,}\ \ldots \makebox{\tt ,}\ var_n$\>\makebox[3em]{}$\it (n\geq 1)$
\end{tabbing}\end{flushleft}
\index{topdecl@{\em topdecl} (\mbox{\tt class})}%
\indexsyn{cbody}%
\indexsyn{csigns}%
\indexsyn{csign}%
\indexsyn{vars}%
\index{declaration!within a {\ptt class} declaration}
%\ToDo{tycls and tyvar left off above}

\noindent
A {\em class declaration} introduces a new class and the operations on it.
A class declaration has the general form:
\[\begin{array}{rl}
\mbox{$\it \makebox{\tt class}\ c\ \makebox{\tt =>}\ C\ u\ \makebox{\tt where}\ \makebox{\tt {\char'173}}$}&\mbox{$\it v_1\ \makebox{\tt ::}\ c_1\ \makebox{\tt =>}\ t_1\ \makebox{\tt ;}\ \ldots \ \makebox{\tt ;}\ v_n\ \makebox{\tt ::}\ c_n\ \makebox{\tt =>}\ t_n\ \makebox{\tt ;}$}\\
                                &\mbox{$\it valdef_1\ \makebox{\tt ;}\ \ldots \ \makebox{\tt ;}\ valdef_m\ \makebox{\tt {\char'175}}$}
\end{array}\]
This introduces a new class name \mbox{$\it C$}; the type variable \mbox{$\it u$} is
scoped only over the method signatures in the class body.
The context \mbox{$\it c$} specifies the superclasses\index{superclass} of \mbox{$\it C$}, as
described below; the only type variable that may be referred to in \mbox{$\it c$}
is \mbox{$\it u$}.
The class declaration introduces new {\em class methods}
\index{class method}
\mbox{$\it v_1,\ \ldots ,\ v_n$}, whose scope extends outside the \mbox{\tt class} declaration,
with types:
\[
v_i :: \forall u,\overline{w}.~(C u, c_i) \Rightarrow t_i
\]
%old:
%Note the implicit context in the types for each \mbox{$\it v_i$}.  
The \mbox{$\it t_i$} must mention \mbox{$\it u$}; they may mention type variables
$\overline{w}$ other than \mbox{$\it u$}, and the type of \mbox{$\it v_i$} is
polymorphic in both \mbox{$\it u$} and $\overline{w}\/$.
The \mbox{$\it c_i$} may constrain only $\overline{w}$; in particular,
the \mbox{$\it c_i$} may not constrain \mbox{$\it u$}.
For example:
\bprog
\mbox{\tt \ \ \ \ \ \ \ \ class\ Foo\ a\ where}\\
\mbox{\tt \ \ \ \ \ \ \ \ \ \ \ \ \ \ \ \ op\ ::\ Num\ b\ =>\ a\ ->\ b\ ->\ a}
\eprog
Here the type of \mbox{\tt op} is
$\forall a, b.~(\mbox{\tt Foo}~a,~\mbox{\tt Num}~b)~ \Rightarrow a \rightarrow b \rightarrow a$.

{\em Default methods}
\index{default method}
for any of the \mbox{$\it v_i$} may be included in the
\mbox{\tt class} declaration as a normal \mbox{$\it valdef$}; no other definitions are
permitted.  The default method for \mbox{$\it v_i$} is used if no binding for it
is given in a particular \mbox{\tt instance} declaration (see
Section~\ref{instance-decls}).

Two classes in scope at the same time may not share any of the same
methods.

\begin{figure}
\outline{
\mbox{\tt class\ \ Eq\ a\ \ where\ \ \ \ \ \ \ \ \ \ \ \ \ \ \ \ \ \ \ \ \ \ \ \ \ \ \ \ \ \ \ --\ \ \ Eq}\\
\mbox{\tt \ \ \ \ \ \ \ \ (==),\ (/=)\ \ ::\ \ a\ ->\ a\ ->\ Bool\ \ \ \ \ \ \ \ \ \ \ --\ \ \ \ |}\\
\mbox{\tt \ \ \ \ \ \ \ \ \ \ \ \ \ \ \ \ \ \ \ \ \ \ \ \ \ \ \ \ \ \ \ \ \ \ \ \ \ \ \ \ \ \ \ \ \ \ \ \ \ --\ \ \ Ord}\\
\mbox{\tt \ \ \ \ \ \ \ \ x\ /=\ y\ \ \ =\ \ not\ (x\ ==\ y)\ \ \ \ \ \ \ \ \ \ \ \ \ \ \ \ \ --\ \ \ /\ {\char'134}}\\
\mbox{\tt \ \ \ \ \ \ \ \ \ \ \ \ \ \ \ \ \ \ \ \ \ \ \ \ \ \ \ \ \ \ \ \ \ \ \ \ \ \ \ \ \ \ \ \ \ \ \ \ \ --\ \ Ix\ Enum}\\
\mbox{\tt class\ \ (Eq\ a)\ =>\ Ord\ a\ \ where}\\
\mbox{\tt \ \ \ \ \ \ \ \ (<),\ (<=),\ (>=),\ (>)\ ::\ \ a\ ->\ a\ ->\ Bool}\\
\mbox{\tt \ \ \ \ \ \ \ \ max,\ min\ \ \ \ \ \ \ \ \ \ \ \ \ ::\ \ a\ ->\ a\ ->\ a}\\
\mbox{\tt }\\[-8pt]
\mbox{\tt \ \ \ \ \ \ \ \ x\ <\ \ y\ \ \ \ \ \ \ \ \ \ \ \ \ \ \ \ =\ \ x\ <=\ y\ {\char'46}{\char'46}\ x\ /=\ y}\\
\mbox{\tt \ \ \ \ \ \ \ \ x\ >=\ y\ \ \ \ \ \ \ \ \ \ \ \ \ \ \ \ =\ \ y\ <=\ x}\\
\mbox{\tt \ \ \ \ \ \ \ \ x\ >\ \ y\ \ \ \ \ \ \ \ \ \ \ \ \ \ \ \ =\ \ y\ <\ \ x}\\
\mbox{\tt \ \ \ \ \ \ \ \ max\ x\ y\ |\ x\ >=\ y\ \ \ \ \ \ =\ \ x}\\
\mbox{\tt \ \ \ \ \ \ \ \ \ \ \ \ \ \ \ \ |\ y\ >=\ x\ \ \ \ \ \ =\ \ y}\\
\mbox{\tt \ \ \ \ \ \ \ \ min\ x\ y\ |\ x\ <=\ y\ \ \ \ \ \ =\ \ x}\\
\mbox{\tt \ \ \ \ \ \ \ \ \ \ \ \ \ \ \ \ |\ y\ <=\ x\ \ \ \ \ \ =\ \ y}\\
\mbox{\tt }\\[-8pt]
\mbox{\tt class\ \ Text\ a\ \ where}\\
\mbox{\tt \ \ \ \ \ \ \ \ showsPrec\ ::\ Int\ ->\ a\ ->\ String\ ->\ String}\\
\mbox{\tt \ \ \ \ \ \ \ \ readsPrec\ ::\ Int\ ->\ String\ ->\ [(a,String)]}\\
\mbox{\tt \ \ \ \ \ \ \ \ showList\ \ ::\ [a]\ ->\ String\ ->\ String}\\
\mbox{\tt \ \ \ \ \ \ \ \ readList\ \ ::\ String\ ->\ [([a],String)]}\\
\mbox{\tt }\\[-8pt]
\mbox{\tt \ \ \ \ \ \ \ \ showList\ =\ ...\ --\ see\ Appendix\ A\ \ }\\
\mbox{\tt \ \ \ \ \ \ \ \ readList\ =\ ...\ --\ see\ Appendix\ A\ \ }\\
\mbox{\tt }\\[-8pt]
\mbox{\tt class\ \ Binary\ a\ \ where\ \ \ \ \ \ \ \ \ \ \ \ \ \ \ \ \ \ \ \ \ \ }\\
\mbox{\tt \ \ \ \ \ \ \ \ showBin\ ::\ a\ ->\ Bin\ ->\ Bin\ }\\
\mbox{\tt \ \ \ \ \ \ \ \ readBin\ ::\ Bin\ ->\ (a,Bin)\ \ }\\
\mbox{\tt }\\[-8pt]
\mbox{\tt class\ \ (Ord\ a)\ =>\ Ix\ a\ \ where}\\
\mbox{\tt \ \ \ \ \ \ \ \ range\ \ \ ::\ (a,a)\ ->\ [a]}\\
\mbox{\tt \ \ \ \ \ \ \ \ index\ \ \ ::\ (a,a)\ ->\ a\ ->\ Int}\\
\mbox{\tt \ \ \ \ \ \ \ \ inRange\ ::\ (a,a)\ ->\ a\ ->\ Bool}\\
\mbox{\tt }\\[-8pt]
\mbox{\tt class\ \ (Ord\ a)\ =>\ Enum\ a\ \ where}\\
\mbox{\tt \ \ \ \ \ \ \ \ enumFrom\ \ \ \ \ \ \ ::\ a\ ->\ [a]\ \ \ \ \ \ \ \ \ \ \ \ \ --\ [n..]}\\
\mbox{\tt \ \ \ \ \ \ \ \ enumFromThen\ \ \ ::\ a\ ->\ a\ ->\ [a]\ \ \ \ \ \ \ \ --\ [n,n'..]}\\
\mbox{\tt \ \ \ \ \ \ \ \ enumFromTo\ \ \ \ \ ::\ a\ ->\ a\ ->\ [a]\ \ \ \ \ \ \ \ --\ [n..m]}\\
\mbox{\tt \ \ \ \ \ \ \ \ enumFromThenTo\ ::\ a\ ->\ a\ ->\ a\ ->\ [a]\ \ \ --\ [n,n'..m]}\\
\mbox{\tt }\\[-8pt]
\mbox{\tt \ \ \ \ \ \ \ \ enumFromTo\ n\ m\ \ \ \ \ \ \ \ =\ takeWhile\ ((>=)\ m)\ (enumFrom\ n)}\\
\mbox{\tt \ \ \ \ \ \ \ \ enumFromThenTo\ n\ n'\ m\ =\ takeWhile}\\
\mbox{\tt \ \ \ \ \ \ \ \ \ \ \ \ \ \ \ \ \ \ \ \ \ \ \ \ \ \ \ \ \ \ \ \ \ \ ((if\ n'\ >=\ n\ then\ (>=)\ else\ (<=))\ m)}\\
\mbox{\tt \ \ \ \ \ \ \ \ \ \ \ \ \ \ \ \ \ \ \ \ \ \ \ \ \ \ \ \ \ \ \ \ \ \ (enumFromThen\ n\ n')}
}
\ecaption{Standard Classes and Associated Functions}
\label{standard-classes}
\indextt{Eq}\indextt{==}\indextt{/=}
\indextt{Ord}\indextt{<}\indextt{<=}\indextt{>}\indextt{>=}\indextt{max}\indextt{min}
\indextt{Text}\indextt{showsPrec}\indextt{readsPrec}\indextt{showList}\indextt{readList}
\indextt{Binary}\indextt{showBin}\indextt{readBin}
\indextt{Ix}\indextt{range}\indextt{index}\indextt{inRange}
\indextt{Enum}\indextt{enumFrom}\indextt{enumFromThen}
\indextt{enumFromTo}\indextt{enumFromThenTo}
\end{figure}

Figure~\ref{standard-classes} shows some standard \Haskell{}
classes, including the use of superclasses; note the class inclusion
diagram on the right.  For example, \mbox{\tt Eq} is a superclass of \mbox{\tt Ord}, and
thus in any context \mbox{\tt Ord\ a} is equivalent to \mbox{\tt (Eq\ a,\ Ord\ a)}.  

A \mbox{\tt class}
declaration with no \mbox{\tt where} part
\index{class declaration!with an empty \mbox{\tt where} part}
may be useful for combining a
collection of classes into a larger one that inherits all of the
operations in the original ones.  For example:
\bprog
\mbox{\tt class\ \ (Ord\ a,\ Text\ a,\ Binary\ a)\ =>\ Data\ a}
\eprog
In such a case, if a type is an instance of all superclasses,\index{superclass} it is
not {\em automatically} an instance of the subclass, even though the
subclass has no immediate operations.  The \mbox{\tt instance} declaration must be
given explicitly, and it must have an empty \mbox{\tt where} part as well.
\index{instance declaration!with an empty \mbox{\tt where} part}

The superclass relation must not be cyclic; i.e.~it must form a
directed acyclic graph.\index{superclass}

\subsubsection{Instance Declarations}
\label{instance-decls}
\index{instance declaration}

\begin{flushleft}\it\begin{tabbing}
\hspace{0.5in}\=\hspace{3.0in}\=\kill
$\it topdecl$\>\makebox[3.5em]{$\rightarrow$}$\it \makebox{\tt instance}\ [context\ \makebox{\tt =>}]\ tycls\ inst\ [\makebox{\tt where}\ \makebox{\tt {\char'173}}\ valdefs\ [\makebox{\tt ;}]\ \makebox{\tt {\char'175}}]$\\ 
$\it inst$\>\makebox[3.5em]{$\rightarrow$}$\it tycon$\>\makebox[3em]{}$\it (\arity{tycon}=0)$\\ 
$\it $\>\makebox[3.5em]{$|$}$\it \makebox{\tt (}\ tycon\ tyvar_1\ \ldots \ tyvar_k\ \makebox{\tt )}$\>\makebox[3em]{}$\it (k\geq 1,\ tyvars\ {\rm\ distinct})$\\ 
$\it $\>\makebox[3.5em]{$|$}$\it \makebox{\tt (}\ tyvar_1\ \makebox{\tt ,}\ \ldots \ \makebox{\tt ,}\ tyvar_k\ \makebox{\tt )}$\>\makebox[3em]{}$\it (k\geq 2,\ tyvars\ {\rm\ distinct})$\\ 
$\it $\>\makebox[3.5em]{$|$}$\it \makebox{\tt ()}$\\ 
$\it $\>\makebox[3.5em]{$|$}$\it \makebox{\tt [}\ tyvar\ \makebox{\tt ]}$\\ 
$\it $\>\makebox[3.5em]{$|$}$\it \makebox{\tt (}\ tyvar_1\ \makebox{\tt ->}\ tyvar_2\ \makebox{\tt )}$\>\makebox[3em]{}$\it tyvar_1\ {\rm\ and}\ tyvar_2\ {\rm\ distinct}$\\ 
$\it $\\ 
$\it valdefs$\>\makebox[3.5em]{$\rightarrow$}$\it valdef_1\ \makebox{\tt ;}\ \ldots \ \makebox{\tt ;}\ valdef_n$\>\makebox[3em]{}$\it (n\geq 0)$
\end{tabbing}\end{flushleft}
\index{topdecl@{\em topdecl} (\mbox{\tt instance})}
\indexsyn{inst}%
\indexsyn{valdefs}%
%\index{instance declaration@{\ptt instance} declaration}
\index{declaration!within an {\ptt instance} declaration}
%\ToDo{tycls left off above}
An {\em instance declaration} introduces an instance of a class.  Let
\[ \mbox{$\it \makebox{\tt class}\ c\ \makebox{\tt =>}\ C\ u\ \makebox{\tt where}\ \makebox{\tt {\char'173}}\ cbody\ \makebox{\tt {\char'175}}$} \]
be a \mbox{\tt class} declaration.  The general form of the corresponding
instance declaration is:
\[ \mbox{$\it \makebox{\tt instance}\ c'\ \makebox{\tt =>}\ C\ (T\ u_1\ \ldots \ u_k)\ \makebox{\tt where}\ \makebox{\tt {\char'173}}\ d\ \makebox{\tt {\char'175}}$} \]
where \mbox{$\it k\geq0$} and \mbox{$\it T$} is not a type synonym.
\index{type synonym}
The type being instanced, \mbox{$\it (T\ u_1\ \ldots \ u_k)$}, is
a type constructor applied to simple type variables \mbox{$\it u_1,\ \ldots \ u_k$},
which must be distinct.  This prohibits instance declarations
such as:
\bprog
\mbox{\tt instance\ C\ (a,a)\ where\ ...}\\
\mbox{\tt instance\ C\ (Int,a)\ where\ ...}\\
\mbox{\tt instance\ C\ [[a]]\ where\ ...}
\eprog

The context \mbox{$\it c'$} must
imply the context \mbox{$\it c[(T\ u_1\ \ldots \ u_k)/u]$}, and \mbox{$\it d$} may contain bindings
\index{class method}
only for the class methods of \mbox{$\it C$}.  No type signatures
\index{type signature} may appear in \mbox{$\it d$}, as the signatures for the
methods have already been given in the \mbox{\tt class} declaration.

%No contexts may appear
%in \mbox{$\it d$}, since they are implied: any signature declaration in \mbox{$\it d$} will
%have the form \mbox{$\it v\ \makebox{\tt ::}\ t$}, abbreviating \mbox{$\it v\ \makebox{\tt ::}\ c'\ \makebox{\tt =>}\ t$}.
%Each \mbox{$\it v_i$} has type:
%\[ \mbox{$\it v_i\ \makebox{\tt ::}\ c'\ \makebox{\tt =>}\ (t_i[(T\ u_1\ \ldots \ u_k)/u])$} \]

If no binding is given for some class method then the
corresponding default method
\index{default method}
in the \mbox{\tt class} declaration is used (if
present); if such a default does
not exist then the class method at this instance
is implicitly bound to the completely undefined
function (of the appropriate type) and no static error results.

An \mbox{\tt instance} declaration that makes the type \mbox{$\it T$} to be an instance
of class \mbox{$\it C$} is called a {\em C-T instance declaration}
\index{C-T instance declaration@$C$-$T$ instance declaration} and is
subject to these static restrictions:
\index{instance declaration!with respect to modules}
\begin{itemize}
\item A $C$-$T$ instance declaration may only appear either in the module
in which $C$ is declared or in the module in which $T$ is declared, and
only where both $C$ and $T$ are in scope.

\item A type may not be declared as an instance of a
particular class more than once in the same scope.
\end{itemize}

Examples of \mbox{\tt instance} declarations may be found in the next section on
derived instances.  

\subsubsection{Derived Instances}
\index{derived instance}
\label{derived-decls}

As mentioned in Section~\ref{datatype-decls}, \mbox{\tt data} declarations
contain an optional \mbox{\tt deriving} form.  If the form is included, then
{\em derived instance declarations} are automatically generated for
the datatype in each of the named classes.
If a derived instance of a subclass is asked
for, then each of the superclasses\index{superclass} must either be asked for or an
explicit instance declaration must be given for it.

Derived instances provide convenient commonly-used operations for
user-de\-fined da\-ta\-types.  For example, derived instances for datatypes
in the class \mbox{\tt Eq} define the operations \mbox{\tt ==} and \mbox{\tt /=}, freeing the
programmer from the need to define them.
%and taking advantage of
%\Haskell{}'s class mechanism to overload these operations.

The only classes for which derived instances are allowed are
\mbox{\tt Eq}\index{Eq@{\ptt Eq}!derived instance},
\mbox{\tt Ord}\index{Ord@{\ptt Ord}!derived instance}, 
\mbox{\tt Ix}\index{Ix@{\ptt Ix}!derived instance}, 
\mbox{\tt Enum}\index{Enum@{\ptt Enum}!derived instance}, 
\mbox{\tt Text}\index{Text@{\ptt Text}!derived instance}, and 
\mbox{\tt Binary}\index{Binary@{\ptt Binary}!derived instance},
all defined in Figure~\ref{standard-classes}, page~\pageref{standard-classes}.
The
precise details of how the derived instances are generated for each of
these classes are provided in Appendix~\ref{derived-appendix}, including
a specification of when such derived instances are possible. 
%(which is important for the following discussion).

If it is not possible to derive an \mbox{\tt instance} declaration over a class
named in a \mbox{\tt deriving} form, then a static error results.  For example,
not all datatypes can properly support operations in \mbox{\tt Enum}.\indextt{Enum}  It is
also a static error to give an explicit \mbox{\tt instance} declaration for
one that is also derived.
%These rules also apply to the superclasses
%of the class in question.

If the \mbox{\tt deriving} form is omitted from a \mbox{\tt data}
declaration, then {\em no} instance declarations will be derived for
that datatype; that is, omitting a \mbox{\tt deriving} form is equivalent to
including an empty deriving form: \mbox{\tt deriving\ ()}.

% OLD:
%On the other hand, if the \mbox{\tt deriving} form is omitted from a \mbox{\tt data}
%declaration, then \mbox{\tt instance} declarations are derived for the datatype
%in as many of the six classes mentioned above as is possible (see
%Appendix~\ref{derived-appendix}); that is, no
%static error will result if the \mbox{\tt instance} declarations cannot be generated.

%OLD:
%If {\em no} derived \mbox{\tt instance} declarations for a datatype
%are wanted, then the empty deriving form \mbox{\tt deriving\ ()} must be given
%in the \mbox{\tt data} declaration for that type.

\subsubsection{Defaults for Overloaded Operations}
\label{default-decls}
\index{default declaration@{\ptt default} declaration}
\index{overloading!defaults}

\begin{flushleft}\it\begin{tabbing}
\hspace{0.5in}\=\hspace{3.0in}\=\kill
$\it topdecl$\>\makebox[3.5em]{$\rightarrow$}$\it \makebox{\tt default}\ (type\ |\ \makebox{\tt (}type_1\ \makebox{\tt ,}\ \ldots \ \makebox{\tt ,}\ type_n\makebox{\tt )})$\>\makebox[3em]{}$\it \qquad\ (n\geq 0)$
\end{tabbing}\end{flushleft}
\index{topdecl@{\em topdecl} (\mbox{\tt default})}

\noindent
A problem inherent with overloading is the possibility of an ambiguous type.
\index{ambiguous type}
For example, using the
\mbox{\tt read} and \mbox{\tt show} functions defined in Appendix~\ref{derived-appendix},
and supposing that just \mbox{\tt Int} and \mbox{\tt Bool} are members of \mbox{\tt Text}, then
the expression
\bprog
\mbox{\tt let\ x\ =\ read\ "..."\ in\ show\ x\ \ \ \ --\ ILLEGAL}
\eprog
is ambiguous, because the types for \mbox{\tt show} and \mbox{\tt read},
\[\begin{array}{ll}
\mbox{\tt show} & \mbox{$\it ::\ \forall\ a.~\makebox{\tt Text}~\ a\ \Rightarrow\ a\ \rightarrow\ \makebox{\tt String}$} \\
\mbox{\tt read} & \mbox{$\it ::\ \forall\ a.~\makebox{\tt Text}~\ a\ \Rightarrow\ \makebox{\tt String}\ \rightarrow\ a$}
\end{array}\]
could be satisfied by instantiating \mbox{\tt a} as either \mbox{\tt Int}
in both cases, or \mbox{\tt Bool}.  Such expressions
are considered ill-typed, a static error.

We say that an expression \mbox{\tt e} is {\em ambiguously
overloaded}
\index{overloading!ambiguous}
if, in its type \mbox{$\it \forall\ \overline{u}.~c\ \Rightarrow\ t$}, 
there is a type variable $u$ in $\overline{u}$ which occurs in \mbox{$\it c$} 
but not in \mbox{$\it t$}.  Such types are illegal.

For example, the earlier expression involving \mbox{\tt show} and \mbox{\tt read} is
ambiguously overloaded since its type is 
$\forall a.~ \mbox{\tt Text}~ a \Rightarrow \mbox{\tt String}$.

Overloading ambiguity, although rare, can only be circumvented by
input from the user.  One way is through the use of {\em expression
type-signatures}
\index{expression type-signature}
as described in Section~\ref{expression-type-sigs}.
For example, for the ambiguous expression given earlier, one could
write:
\bprog
\mbox{\tt let\ x\ =\ read\ "..."\ in\ show\ (x::Bool)}
\eprog
which disambiguates the type.

Occasionally, an otherwise ambiguous expression needs to be made
the same type as some variable, rather than being given a fixed
type with an expression type-signature.  This is the purpose
of the function \mbox{\tt asTypeOf} (Appendix~\ref{stdprelude}):
\mbox{$\it x$} \mbox{\tt asTypeOf} \mbox{$\it y$} has the value of \mbox{$\it x$}, but \mbox{$\it x$} and \mbox{$\it y$} are
forced to have the same type.  For example,
\bprog
\mbox{\tt approxSqrt\ x\ =\ encodeFloat\ 1\ (exponent\ x\ `div`\ 2)\ `asTypeOf`\ x}
\eprog
(See Section~\ref{coercion}.)

Ambiguities in the class \mbox{\tt Num}\indextt{Num}
are most common, so \Haskell{}
provides another way to resolve them---with a {\em
default declaration}:
\[
\mbox{$\it \makebox{\tt default\ (}t_1\ \makebox{\tt ,}\ \ldots \ \makebox{\tt ,}\ t_n\makebox{\tt )}$}
\]
where \mbox{$\it n\geq0$} (the parentheses may be omitted when \mbox{$\it n=1$}), and each
\mbox{$\it t_i$} must be a monotype for which \mbox{$\it \makebox{\tt Num\ }t_i$} holds.
In situations where an ambiguous type is discovered, an
ambiguous type variable is defaultable if at least one
of its classes is a numeric class and if all of its classes
are either numeric classes or standard classes.
(Figures~\ref{basic-numeric-1}--\ref{basic-numeric-3},
pages~\pageref{basic-numeric-1}--\pageref{basic-numeric-3},
show the numeric classes, and
Figure~\ref{standard-classes}, page~\pageref{standard-classes},
shows the standard classes.)
Each defaultable variable is replaced by the first type in the
default list that is an instance of all the ambiguous variable's classes.
It is a static error if no such type is found.

Only one default declaration is permitted per module, and its effect
is limited to that module.  If no default declaration is given in a
module then it defaults to:
\bprog
\mbox{\tt default\ (Int,\ Double)}
\eprog
The empty default declaration \mbox{\tt default\ ()} must be given to turn off
all defaults in a module.

\subsection{Nested Declarations}
\label{nested}

The following declarations may be used in any declaration list,
including the top level of a module.

\subsubsection{Type Signatures}
\index{type signature}
\label{type-signatures}

\begin{flushleft}\it\begin{tabbing}
\hspace{0.5in}\=\hspace{3.0in}\=\kill
$\it decl$\>\makebox[3.5em]{$\rightarrow$}$\it vars\ \makebox{\tt ::}\ [context\ \makebox{\tt =>}]\ type$\\ 
$\it vars$\>\makebox[3.5em]{$\rightarrow$}$\it var_1\ \makebox{\tt ,}\ \ldots \makebox{\tt ,}\ var_n$\>\makebox[3em]{}$\it (n\geq 1)$
\end{tabbing}\end{flushleft}
\indexsyn{decl}%
\indexsyn{vars}%
A type signature specifies types for variables, possibly with respect
to a context.  A type signature has the form:
\[
\mbox{$\it x_1,\ \ldots ,\ x_n\ \makebox{\tt ::}\ c\ \makebox{\tt =>}\ t$}
\]
which is equivalent to asserting
\mbox{$\it x_i\ \makebox{\tt ::}\ c\ \makebox{\tt =>}\ t$}
for each \mbox{$\it i$} from \mbox{$\it 1$} to \mbox{$\it n$}.  Each \mbox{$\it x_i$} must have a value binding in
the same declaration list that contains the type signature; i.e.~it is
illegal to give a type signature for a variable bound in an
outer scope.
Moreover, it is illegal to give more than one type
signature for one variable.

As mentioned in Section~\ref{type-syntax},
every type variable appearing in a signature
is universally quantified over that signature, and hence
the scope of a type variable is limited to the type
signature that contains it.  For example, in the following
declarations
\bprog
\mbox{\tt f\ ::\ a\ ->\ a}\\
\mbox{\tt f\ x\ =\ x::a\ \ \ \ \ \ \ \ \ \ \ \ \ \ \ \ \ \ \ \ \ \ --\ ILLEGAL}
\eprog
the \mbox{\tt a}'s in the two type signatures are quite distinct.  Indeed,
these declarations contain a static error, since \mbox{\tt x} does not have
type $\forall a.~a$.

A type signature for \mbox{$\it x$} may be more specific than the principal
type derivable from the value binding of \mbox{$\it x$} (see
Section~\ref{type-semantics}), but it is an error to give a type
that is more
general than, or incomparable to, the principal type.
\index{principal type}
If a more specific type is given then all occurrences of the
variable must be used at the more specific type or at a more
specific type still.
%
For example, if we define\nopagebreak[4]
\bprog
\mbox{\tt sqr\ x\ \ =\ \ x*x}
\eprog
then the principal type is 
$\mbox{\tt sqr} :: \forall a.~ \mbox{\tt Num}~ a \Rightarrow a \rightarrow a$, 
which allows
applications such as \mbox{\tt sqr\ 5} or \mbox{\tt sqr\ 0.1}.  It is also legal to declare
a more specific type, such as
\bprog
\mbox{\tt sqr\ ::\ Int\ ->\ Int}
\eprog
but now applications such as \mbox{\tt sqr\ 0.1} are illegal.  Type signatures such as
\bprog
\mbox{\tt sqr\ ::\ (Num\ a,\ Num\ b)\ =>\ a\ ->\ b\ \ \ \ \ --\ ILLEGAL}\\
\mbox{\tt sqr\ ::\ a\ ->\ a\ \ \ \ \ \ \ \ \ \ \ \ \ \ \ \ \ \ \ \ \ \ \ --\ ILLEGAL}
\eprog
are illegal, as they are more general than the principal type of \mbox{\tt sqr}.

\subsubsection{Function and Pattern Bindings}
\label{function-bindings}\label{pattern-bindings}
\index{function binding}\index{pattern binding}

\begin{flushleft}\it\begin{tabbing}
\hspace{0.5in}\=\hspace{3.0in}\=\kill
$\it decl$\>\makebox[3.5em]{$\rightarrow$}$\it valdef$\\ 
$\it $\\ 
$\it valdef$\>\makebox[3.5em]{$\rightarrow$}$\it lhs\ \makebox{\tt =}\ exp\ [\makebox{\tt where}\ \makebox{\tt {\char'173}}\ decls\ [\makebox{\tt ;}]\ \makebox{\tt {\char'175}}]$\\ 
$\it $\>\makebox[3.5em]{$|$}$\it lhs\ gdrhs\ [\makebox{\tt where}\ \makebox{\tt {\char'173}}\ decls\ [\makebox{\tt ;}]\ \makebox{\tt {\char'175}}]$\\ 
$\it $\\ 
$\it lhs$\>\makebox[3.5em]{$\rightarrow$}$\it apat$\\ 
$\it $\>\makebox[3.5em]{$|$}$\it funlhs$\\ 
$\it funlhs$\>\makebox[3.5em]{$\rightarrow$}$\it afunlhs$\\ 
$\it $\>\makebox[3.5em]{$|$}$\it pat^{i+1}_1\ varop^{({\rm\ n},i)}\ pat^{i+1}_2$\>\makebox[3em]{}$\it (0\leq i\leq 9)$\\ 
$\it $\>\makebox[3.5em]{$|$}$\it lpat^i\ varop^{({\rm\ l},i)}\ pat^{i+1}$\>\makebox[3em]{}$\it (0\leq i\leq 9)$\\ 
$\it $\>\makebox[3.5em]{$|$}$\it pat^{i+1}\ varop^{({\rm\ r},i)}\ rpat^i$\>\makebox[3em]{}$\it (0\leq i\leq 9)$\\ 
$\it afunlhs$\>\makebox[3.5em]{$\rightarrow$}$\it var\ apat$\\ 
$\it $\>\makebox[3.5em]{$|$}$\it \makebox{\tt (}\ funlhs\ \makebox{\tt )}\ apat$\\ 
$\it $\>\makebox[3.5em]{$|$}$\it afunlhs\ apat$\\ 
$\it $\\ 
$\it gdrhs$\>\makebox[3.5em]{$\rightarrow$}$\it gd\ \makebox{\tt =}\ exp\ [gdrhs]$\\ 
$\it $\\ 
$\it gd$\>\makebox[3.5em]{$\rightarrow$}$\it \makebox{\tt |}\ exp$
\end{tabbing}\end{flushleft}
\indexsyn{decl}%
\indexsyn{valdef}%
\indexsyn{lhs}%
\indexsyn{funlhs}%
\indexsyn{afunlhs}%
\indexsyn{gdrhs}%
\indexsyn{gd}%
We distinguish two cases within this syntax: a {\em pattern binding}
occurs when \mbox{$\it lhs$} is \mbox{$\it apat$}; otherwise, the binding is called a {\em function
binding}.  Either binding may appear at the top-level of a module or
within a \mbox{\tt where} or \mbox{\tt let} construct.  The use of the nonterminal \mbox{$\it apat$}
(rather than \mbox{$\it pat$}) in the production for \mbox{$\it lhs$} 
disallows top level $n\mbox{\tt +}k$ pattern bindings;
\index{n+k pattern@\mbox{$\it n\makebox{\tt +}k$} pattern}
otherwise, programs such as \mbox{\tt x\ +\ 2\ =\ 3} could be parsed either as a
definition of \mbox{\tt +} or as a pattern binding.

\paragraph*{Function bindings.}
\index{function binding}
A function binding binds a variable to a function value.  The general
form of a function binding for variable \mbox{$\it x$} is:
\[\ba{lll}
\mbox{$\it x$} & \mbox{$\it p_{11}\ \ldots \ p_{1k}$} & \mbox{$\it match_1$}\\
\mbox{$\it \ldots $} \\
\mbox{$\it x$} & \mbox{$\it p_{n1}\ \ldots \ p_{nk}$} & \mbox{$\it match_n$}
\ea\]
where each \mbox{$\it p_{ij}$} is a pattern, and where each \mbox{$\it match_i$} is of the
general form:
\[\ba{l}
\mbox{$\it \makebox{\tt =}\ e\ \makebox{\tt where\ {\char'173}}\ decls\ \makebox{\tt {\char'175}}$}
\ea\]
or
\[\ba{lll}
\mbox{$\it \makebox{\tt |}\ g_{i1}$}   & \mbox{$\it \makebox{\tt =}\ e_{i1}$} \\
\mbox{$\it \ldots $} \\
\mbox{$\it \makebox{\tt |}\ g_{im_i}$} & \mbox{$\it \makebox{\tt =}\ e_{im_i}$} \\
               & \multicolumn{2}{l}{\mbox{$\it \makebox{\tt where\ {\char'173}}\ decls_i\ \makebox{\tt {\char'175}}$}}
\ea\]
and where \mbox{$\it n\geq 1$}, \mbox{$\it 1\leq i\leq n$}, \mbox{$\it m_i\geq 1$}.  The former is treated
as shorthand for a particular case of the latter, namely:
\[\ba{l}
\mbox{$\it \makebox{\tt |\ True\ =}\ e\ \makebox{\tt where\ {\char'173}}\ decls\ \makebox{\tt {\char'175}}$}
\ea\]

The set of patterns corresponding to each match must be {\em
linear}\index{linearity}\index{linear pattern}---no variable is allowed
to appear more than once in the entire set.

Alternative syntax is provided for binding functional values to infix
operators.  For example, these two function
definitions are equivalent:
\bprog
\mbox{\tt plus\ x\ y\ z\ =\ x+y+z}\\
\mbox{\tt (x\ }\bkqB\mbox{\tt plus}\bkqA\mbox{\tt \ y)\ z\ =\ x+y+z}
\eprogNoSkip

\outline{
\paragraph*{Translation:}
The general binding form for functions is semantically
equivalent to the equation (i.e.~simple pattern binding):
\[
x\ x_1\ x_2\ ...\ x_k\mbox{\tt \ =\ case\ (}x_1\mbox{\tt ,\ }...\mbox{\tt ,\ }x_k\mbox{\tt )\ of}
\ba[t]{lcl}
\mbox{$\it \makebox{\tt (}p_{11},\ \ldots ,\ p_{1k}\makebox{\tt )}\ match_1$}  \\
\mbox{$\it \ldots $} \\
\mbox{$\it \makebox{\tt (}p_{m1},\ \ldots ,\ p_{mk}\makebox{\tt )}\ match_m$}
\ea\]
where the \mbox{$\it x_i$} are new identifiers.
}

\paragraph*{Pattern bindings.}
\index{pattern binding}
A pattern binding binds variables to values.  A {\em simple} pattern
binding has form \mbox{$\it p\ =\ e$}.
\index{simple pattern binding}
In both a \mbox{\tt where} or \mbox{\tt let} clause
and at the top level of a module, the pattern \mbox{$\it p$} is
matched ``lazily'' as an irrefutable pattern
\index{irrefutable pattern}
by default (as if there
were an implicit \mbox{\tt {\char'176}} in front of it).  See the translation in
Section~\ref{let-expressions}.

The {\em general} form of a pattern binding is \mbox{$\it p\ match$}, where a
\mbox{$\it match$} is the same structure as for function bindings above; in other
words, a pattern binding is:
\[\ba{rcl}
\mbox{$\it p$} & \mbox{$\it \makebox{\tt |}\ g_{1}$}   & \mbox{$\it \makebox{\tt =}\ e_{1}$} \\
    & \mbox{$\it \makebox{\tt |}\ g_{2}$}   & \mbox{$\it \makebox{\tt =}\ e_{2}$} \\
    & \mbox{$\it \ldots $} \\
    & \mbox{$\it \makebox{\tt |}\ g_{m}$}   & \mbox{$\it \makebox{\tt =}\ e_{m}$} \\
    & \multicolumn{2}{l}{\mbox{$\it \makebox{\tt where\ {\char'173}}\ decls\ \makebox{\tt {\char'175}}$}}
\ea\]

%{\em Note}: the simple form
%\WeSay{Yes}
%\mbox{$\it p\ \makebox{\tt =}\ e$} is equivalent to \mbox{$\it p\ \makebox{\tt |\ True\ =}\ e$}.

\outline{
\paragraph*{Translation:}
The pattern binding above is semantically equivalent to this
simple pattern binding:
\[\ba{lcl}
\mbox{$\it p$} &\mbox{\tt =}& \mbox{$\it \makebox{\tt let}\ decls\ \makebox{\tt in}$} \\
    &   & \mbox{$\it \makebox{\tt if\ }g_1\makebox{\tt \ then\ }e_1\makebox{\tt \ else}$} \\
    &   & \mbox{$\it \makebox{\tt if\ }g_2\makebox{\tt \ then\ }e_2\makebox{\tt \ else}$} \\
    &   & ...                          \\
    &   & \mbox{$\it \makebox{\tt if\ }g_m\makebox{\tt \ then\ }e_m\makebox{\tt \ else\ error\ "Unmatched\ pattern"}$}
\ea\]
}

\subsection{Static semantics of function and pattern bindings}

The static semantics of the function and pattern bindings of
a \mbox{\tt let} expression or \mbox{\tt where} clause
% (including that of the top-level of
% a program that has been translated into a \mbox{\tt let} expression as
% described at the beginning of Section~\ref{modules})
is discussed in this section.

\subsubsection{Dependency analysis}

In general the static semantics is given by the
normal Hindley-Milner\index{Hindley-Milner type system} inference rules,
% as described in Appendix~\ref{static-semantics},
except that a {\em dependency
analysis\index{dependency analysis} transformation} is first performed
to enhance polymorphism, as follows.
Two variables bound by value declarations are in the
same {\em declaration group} if either
\index{declaration group}
\begin{enumerate}
\item
they are bound by the same pattern binding, or
\item
their bindings are mutually recursive (perhaps via some
other declarations which are also part of the group).
\end{enumerate}
Careful application of the following 
rules causes each \mbox{\tt let} or \mbox{\tt where} construct to bind only the
variables of a single declaration group, thus capturing the required
dependency analysis:\footnote{%
A similar transformation is described in 
Peyton Jones' book \cite{peyton-jones:book}.}
\begin{center}
\bt{l}
(1)~The order of declarations in \mbox{\tt where}/\mbox{\tt let} constructs is irrelevant. \\
(2)~\mbox{\tt let\ {\char'173}}$d_1$\mbox{\tt ;\ }$d_2$\mbox{\tt {\char'175}\ in\ }$e$ = \mbox{\tt let\ {\char'173}}$d_1$\mbox{\tt {\char'175}\ in\ (let\ {\char'173}}$d_2$\mbox{\tt {\char'175}\ in\ }$e$\mbox{\tt )} \\
\ \ \ \ (when no identifier bound in $d_2$ appears free in $d_1$)
\et
\end{center}

%----------------

\subsubsection{Generalisation}
\label{generalisation}

The Hindley-Milner type system assigns types to a \mbox{\tt let}-expression
in two stages.
First, the right-hand side of the declaration is typed, giving a type with
no universal quantification.  Second, all type variables which occur in this
type are universally quantified unless they are associated with
bound variables in the type environment;
this is called {\em generalisation}.\index{generalisation}
Finally, the body of the \mbox{\tt let}-expression is typed.

For example, consider the declaration
\bprog
\mbox{\tt \ \ \ \ \ \ \ \ f\ x\ =\ let\ g\ y\ =\ (y,y)}\\
\mbox{\tt \ \ \ \ \ \ \ \ \ \ \ \ \ \ in\ ...}\\
\mbox{\tt }
\eprog
The type of \mbox{\tt g}'s definition is 
$a \rightarrow (a,a)$.  The generalisation step
attributes to \mbox{\tt g} the polymorphic type 
$\forall a.~ a \rightarrow (a,a)$,
after which the typing of the ``\mbox{\tt ...}'' part can proceed.

When typing overloaded definitions, all the overloading 
constraints from a single declaration group are collected together, 
to form the context for the type of each variable declared in the group.
For example, in the definition:
\bprog
\mbox{\tt \ \ \ \ \ \ \ \ f\ x\ =\ let\ g1\ x\ y\ =\ if\ x>y\ then\ show\ x\ else\ g2\ y\ x}\\
\mbox{\tt \ \ \ \ \ \ \ \ \ \ \ \ \ \ \ \ \ \ g2\ p\ q\ =\ g1\ q\ p}\\
\mbox{\tt \ \ \ \ \ \ \ \ \ \ \ \ \ \ in\ ...}
\eprog
The types of the definitions of \mbox{\tt g1} and \mbox{\tt g2} are both
$a \rightarrow a \rightarrow \mbox{\tt String}$, and the accumulated constraints are
$\mbox{\tt Ord}~a$ (arising from the use of \mbox{\tt >}), and $\mbox{\tt Text}~a$ (arising from the
use of \mbox{\tt show}).
The type variables appearing in this collection of constraints are
called the {\em constrained type variables}.

The generalisation step attributes to both \mbox{\tt g1} and \mbox{\tt g2} the type
$\forall a.~(\mbox{\tt Ord}~a,~\mbox{\tt Text}~a) \Rightarrow 
a \rightarrow a \rightarrow \mbox{\tt String}$.
Notice that \mbox{\tt g2} is overloaded in the same way as \mbox{\tt g1} even though the
occurrences of \mbox{\tt >} and \mbox{\tt show} are in the definition of \mbox{\tt g1}.

If the programmer supplies explicit type signatures for more than one variable
in a declaration group, the contexts of these signatures must be 
identical up to renaming of the type variables.

\subsubsection{Monomorphism}

Sometimes it is not possible to generalise over all the type variables
used in the type of the definition.
For example, consider the declaration\nopagebreak[4]
\bprog
\mbox{\tt \ \ \ \ \ \ \ \ f\ x\ =\ let\ g\ y\ z\ =\ ([x,y],\ z)}\\
\mbox{\tt \ \ \ \ \ \ \ \ \ \ \ \ \ \ in\ ...}
\eprog
In an environment where \mbox{\tt x} has type $a$,
the type of \mbox{\tt g}'s definition is 
$a \rightarrow b \rightarrow \mbox{\tt ([}a\mbox{\tt ]},b\mbox{\tt )}$.
The generalisation step attributes to \mbox{\tt g} the type 
$\forall b.~ a \rightarrow b \rightarrow \mbox{\tt ([}a\mbox{\tt ]},b\mbox{\tt )}$;
only $b$ can be universally quantified because $a$ occurs in the
type environment.
We say that the type of \mbox{\tt g} is {\em monomorphic in the type variable $a$}.
\index{monomorphic type variable}

The effect of such monomorphism is that the first argument of all 
applications of \mbox{\tt g} must be of a single type.  
For example, it would be legal for
the ``\mbox{\tt ...}'' to be
\bprog
\mbox{\tt \ \ \ \ \ \ \ \ (g\ True,\ g\ False)}
\eprog
(which would, incidentally, force \mbox{\tt x} to have type \mbox{\tt Bool}) but illegal for it to be
\bprog
\mbox{\tt \ \ \ \ \ \ \ \ (g\ True,\ g\ 'c')}
\eprog
In general, a type $\forall \overline{u}.~c \Rightarrow t$
is said to be {\em monomorphic}
\index{monomorphic type variable}
in the type variable \mbox{$\it a$} if \mbox{$\it a$} is free in
$\forall \overline{u}.~c \Rightarrow t$.

It is worth noting that the explicit type signatures provided by \Haskell{}
are not powerful enough to express types which include monomorphic type
variables.  For example, we cannot write
\bprog
\mbox{\tt \ \ \ \ \ \ \ \ f\ x\ =\ let\ }\\
\mbox{\tt \ \ \ \ \ \ \ \ \ \ \ \ \ \ \ \ g\ ::\ a\ ->\ b\ ->\ ([a],b)}\\
\mbox{\tt \ \ \ \ \ \ \ \ \ \ \ \ \ \ \ \ g\ y\ z\ =\ ([x,y],\ z)}\\
\mbox{\tt \ \ \ \ \ \ \ \ \ \ \ \ \ \ in\ ...}
\eprog
because that would claim that \mbox{\tt g} was polymorphic in both \mbox{\tt a} and \mbox{\tt b}
(Section~\ref{type-signatures}).  In this program, \mbox{\tt g} can only be given
a type signature if its first argument is restricted to a type not involving
type variables; for example
\bprog
\mbox{\tt \ \ \ \ \ \ \ \ \ \ \ \ \ \ \ \ g\ ::\ Int\ ->\ b\ ->\ ([Int],b)}
\eprog
This signature would also cause \mbox{\tt x} to have type \mbox{\tt Int}.

\subsubsection{The monomorphism restriction}
\index{monomorphism restriction}
\label{sect:monomorphism-restriction}

\Haskell{} places certain extra restrictions on the generalisation
step, beyond the standard Hindley-Milner restriction described above,
which further reduce polymorphism in particular cases.

The monomorphism restriction uses the binding syntax of a
variable.  Recall that a variable is bound by either a {\em function
binding} or a {\em pattern binding}, and that a {\em simple} pattern
binding is a pattern binding in which the pattern consists of only a
single variable (Section~\ref{pattern-bindings}).

Two rules define the monomorphism restriction:
\begin{description}
\item[Rule 1.]
We say that a given declaration group is {\em unrestricted} if and only if:
\begin{description}
\item[(a):]
every variable in the group is bound by a function binding or a simple
pattern binding, {\em and}
\item[(b):]
an explicit type signature is given for every variable in the group
which is bound by simple pattern binding.
\end{description}
The usual Hindley-Milner restriction on polymorphism is that
only type variables free in the environment may be generalised.
In addition, {\em the constrained type variables of a
a restricted declaration group may not be generalised}.
(Recall that a type variable is constrained if it must belong
to some type class; see Section~\ref{generalisation}.)
% 
% \item[Rule 1.]
% The variables of a given declaration group are monomorphic in
% all their constrained type variables if and only if:
% \begin{description}
% \item[either (a):]
% one or more variables in the declaration group 
% is bound by a non-simple pattern binding.
% \item[or (b):]
% one or more variables in the declaration group is bound 
% by a simple pattern binding, and
% no type signature is given for any of the variables in the group.
% \end{description}

\item[Rule 2.]
The type of a variable exported from a module must be completely polymorphic;
that is, it must not have any free type variables.
It follows from Rule~1 that if all top-level declaration groups are
unrestricted, then Rule~2 is automatically satisfied.
\end{description}

% When all variables in a declaration group are declared using function
% binding the monomorphism restriction will not apply.  Any variable
% declared in a non-simple pattern binding will invoke monomorphism for
% the entire group containing it.  Simple pattern bindings will be
% monomorphic unless a type signature is supplied.
%
Rule 1 is required for two reasons, both of which are fairly subtle.
First, it prevents computations from being unexpectedly repeated.
For example, recall that \mbox{\tt genericLength} is a standard function whose
type is given by
\bprog
\mbox{\tt \ \ \ \ \ \ \ \ genericLength\ ::\ Num\ a\ =>\ [b]\ ->\ a}
\eprog
Now consider the following expression:
\bprog
\mbox{\tt \ \ \ \ \ \ \ \ let\ {\char'173}\ len\ =\ genericLength\ xs\ {\char'175}\ in\ (len,\ len)}
\eprog
It looks as if \mbox{\tt len} should be computed only once, but without Rule~1 it might
be computed twice, once at each of two different overloadings.  If the 
programmer does actually wish the computation to be repeated, an explicit
type signature may be added:
\bprog
\mbox{\tt \ \ \ \ \ \ \ \ let\ {\char'173}\ len\ ::\ Num\ a\ =>\ a;\ len\ =\ genericLength\ xs\ {\char'175}\ in\ (len,\ len)}
\eprog
When non-simple pattern bindings are used, the types inferred are 
always monomorphic in their constrained type variables, irrespective of whether
a type signature is provided.  For example, in
\bprog
\mbox{\tt \ \ \ (f,g)\ =\ ((+),(-))}
\eprog
both \mbox{\tt f} and \mbox{\tt g} will be monomorphic regardless of any type
signatures supplied for \mbox{\tt f} or \mbox{\tt g}.

Rule~1 also prevents ambiguity.  For example, consider the declaration
group
\bprog
\mbox{\tt \ \ \ \ \ \ \ \ [(n,s)]\ =\ reads\ t}
\eprog
Recall that \mbox{\tt reads} is a standard function whose type is given by the
signature
\bprog
\mbox{\tt \ \ \ \ \ \ \ \ reads\ ::\ (Text\ a)\ =>\ String\ ->\ [(a,String)]}
\eprog
Without Rule~1, \mbox{\tt n} would be assigned the 
type $\forall a.~\mbox{\tt Text}~a \Rightarrow a$ 
and \mbox{\tt s} the type $\forall a.~\mbox{\tt Text}~a \Rightarrow \mbox{\tt String}$.
The latter is an illegal type, because it is inherently ambiguous.
It is not possible to determine at what overloading to use \mbox{\tt s}.
Rule~1 makes \mbox{\tt n} and \mbox{\tt s} monomorphic in $a$.

Lastly, Rule~2 is required because there is no way to enforce monomorphic use
of an exported binding, except by performing type inference on the entire
program at once.

The monomorphism rule has a number of consequences for the programmer.
Anything defined with function syntax will usually
generalize as a function is expected to.  Thus in
\bprog
\mbox{\tt \ \ \ \ \ \ \ \ f\ x\ y\ =\ x+y}
\eprog
the function \mbox{\tt f} may be used at any overloading in class \mbox{\tt Num}.
There is no danger of recomputation here.  However, the same function
defined with pattern syntax
\bprog
\mbox{\tt \ \ \ \ \ \ \ \ f\ =\ {\char'134}x\ ->\ {\char'134}y\ ->\ x+y}
\eprog
requires a type signature if \mbox{\tt f} is to be fully overloaded.
Many functions are most naturally defined using simple pattern
bindings; the user must be careful to affix these with type signatures
to retain full overloading.  The standard prelude contains many
examples of this:
\bprog
\mbox{\tt \ \ \ \ \ \ \ \ indices\ ::\ (Ix\ a)\ =>\ Array\ a\ b\ ->\ [a]}\\
\mbox{\tt \ \ \ \ \ \ \ \ indices\ =\ \ range\ .\ bounds}
\eprog

% Even when a function is defined using a function binding, it may still
% be made monomorphic by another variable in the same declaration group.
% Since groups defined through mutually recursive functions need not be
% syntacticly adjacent, it may be difficult to see where overloading is
% being lost.  In this example \mbox{\tt fact'} is defined with a pattern binding
% and forces \mbox{\tt fact} to be monomorphic in the absence of a type signature
% for either \mbox{\tt fact} or \mbox{\tt fact'}.  This would in turn result in an error as
% per Rule~2.
% \bprog
% 
% module Mod1(fact)
% import Mod2
% fact 0 = 1
% fact n = n*fact'(n-1)
% 
% module Mod2(fact')
% import Mod1
% fact' = fact
% 
% \eprog

% Local Variables: 
% mode: latex
% End:
