%
% $Header$
%
\subsection{Optional Requests}
\label{io-options}
\index{input/output!optional request}

These optional I/O requests may be useful in a \Haskell{}
implementation.

\begin{itemize}
\item
\mbox{\tt ReadChannels\ [cname1,\ ...,\ cnamek]}\\
\mbox{\tt ReadBinChannels\ [cname1,\ ...,\ cnamek]}

Opens \mbox{\tt cname1} through \mbox{\tt cnamek} for input.  A successful response has
form \mbox{\tt Tag\ vals} [\mbox{\tt BinTag\ vals}] where \mbox{\tt vals} is a list of values
tagged with the name of the channel.  These responses require an
extension to the \mbox{\tt Response} datatype:
\bprog
\mbox{\tt data\ \ Response\ =\ ...}\\
\mbox{\tt \ \ \ \ \ \ \ \ \ \ \ \ \ \ \ |\ Tag\ \ \ \ [(String,Char)]}\\
\mbox{\tt \ \ \ \ \ \ \ \ \ \ \ \ \ \ \ |\ BinTag\ [(String,Bin)]}
\eprog
The tagged list of values is the non-deterministic merge of the values
read from the
individual channels.  If an element of this list has form
\mbox{\tt (cnamei,val)}, then it came from channel \mbox{\tt cnamei}.

If any \mbox{\tt cnamei} does not exist then the response 
\mbox{\tt Failure\ (SearchError\ string)} is induced; all other errors induce
\mbox{\tt Failure\ (ReadError\ string)}.

\item
\mbox{\tt CreateProcess\ prog}

Introduces a new program \mbox{\tt prog} into the operating
system.  \mbox{\tt prog} must have type \mbox{\tt [Response]\ ->\ [Request]}.  Either \mbox{\tt Success}
and \mbox{\tt Failure\ (OtherError\ string)} is induced.

\item
\mbox{\tt CreateDirectory\ name\ string}\\
\mbox{\tt DeleteDirectory\ name\ }

Create or delete directory \mbox{\tt name}.  The \mbox{\tt string} argument to
\mbox{\tt CreateDirectory} is an implementation-dependent specification of the
initial state of the directory.

\item
\mbox{\tt OpenFile\ \ \ \ \ name\ inout}\\
\mbox{\tt OpenBinFile\ \ name\ inout}\\
\mbox{\tt CloseFile\ \ \ \ file}\\
\mbox{\tt ReadVal\ \ \ \ \ \ file}\\
\mbox{\tt ReadBinVal\ \ \ file}\\
\mbox{\tt WriteVal\ \ \ \ \ file\ char}\\
\mbox{\tt WriteBinVal\ \ file\ bin}

These requests emulate traditional file I/O in which
characters are read and written one at a time.
\bprog
\mbox{\tt data\ \ Response\ =\ ...}\\
\mbox{\tt \ \ \ \ \ \ \ \ \ \ \ \ \ \ \ |\ Fil\ File}\\
\mbox{\tt data\ \ File}\\
\mbox{\tt type\ \ Bins\ \ \ \ \ =\ [Bin]}
\eprog
\mbox{\tt OpenFile\ name\ inout} [\mbox{\tt OpenBinFile\ name\ inout}]
opens the file \mbox{\tt name} in text [binary] mode with
direction \mbox{\tt inout} (\mbox{\tt True} for input, \mbox{\tt False} for output).
The response \mbox{\tt Fil\ file} is induced, where \mbox{\tt file} has type \mbox{\tt File}, a
primitive type that represents a handle to a file.
Subsequent use of that file by other requests is via this
handle.

\mbox{\tt CloseFile\ file} closes \mbox{\tt file}.  \mbox{\tt Failure\ (OtherError\ string)} is
induced if \mbox{\tt file} cannot be closed.

\mbox{\tt ReadVal} [\mbox{\tt ReadBinVal}] \mbox{\tt file} reads \mbox{\tt file}, inducing the response 
\mbox{\tt Str\ val} [\mbox{\tt Bins\ val}] or \mbox{\tt Failure\ (ReadError\ string)}.

\mbox{\tt WriteVal\ file\ char} [\mbox{\tt WriteBinVal\ file\ bin}] writes \mbox{\tt char} [\mbox{\tt bin}] to
\mbox{\tt file}.  The response \mbox{\tt Success} or \mbox{\tt Failure\ (WriteError\ string)} is
induced.

\mbox{\tt Failure\ (SearchError\ string)} is induced for \mbox{\tt ReadVal}, \mbox{\tt ReadBinVal},
\mbox{\tt WriteVal}, and \mbox{\tt WriteBinVal} if \mbox{\tt file} is not a text or
binary file, as appropriate.
\end{itemize}

% Local Variables: 
% mode: latex
% End:
