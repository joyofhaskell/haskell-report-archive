%
% $Header$
%
% NOTE:--------------------------------------------------------------
% The formatting of this report and the ``new font selection scheme''
% for LaTeX don't agree w/ each other.  Using an ``oldlfont'' style
% option may help.
% -------------------------------------------------------------------
%

% -------------------------------------------------------------------
% formatting for ONE-SIDED printing:
%  * De-comment the \documentstyle, etc., here; comment out the
%    two-sided ones below.
%  * Change the definition of \startnewstuff (below).
%  * Copy the pre-built index file for one-sided printing:
%       cp haskell.ind.one-sided haskell.ind
%  * Comment out the \mbox{$\it twosidefix$} stuff from Joe Fasel, just below.
%  * If you don't have \mbox{$\it makeindex$}, make the adjustments
%    listed in the README file.
%  * Run \mbox{$\it make\ haskell.dvi$} several times (three, at most) to be
%    sure that cross-references stabilise.  [For the 1.1 report,
%    one run should be enough.]
%\documentstyle[11pt,makeidx]{article}
%\oddsidemargin=.25in
%\evensidemargin=.25in

% formatting for double-sided
\documentstyle[twoside,11pt,makeidx]{article}
\evensidemargin=0in
\oddsidemargin=.5in
%---------------------------------------------------------------------
% Joe Fasel said this \mbox{$\it twosidefix$} is necessary if you really
% have a two-sided printer:
%       (note: double @'s for verbatim-ery)
\makeatletter
\def\titlepage{\@restonecolfalse\if@twocolumn\@restonecoltrue\onecolumn
 \else \newpage \fi \thispagestyle{empty}\c@page\m@ne}
\def\endtitlepage{\if@twoside\newpage\thispagestyle{empty}\hbox{}
                        \else \c@page\@z \fi
   \if@restonecol\twocolumn \else \newpage \fi}
\makeatother
%---------------------------------------------------------------------

% the major sections have \cleardoublepages between them
% if you want those between EVERY section, change the
% following defn:
\newcommand{\startnewsection}{\clearpage}
%
% if doing one-sided printing, change this defn to
% be just \mbox{$\it \clearpage$}:
\newcommand{\startnewstuff}{\cleardoublepage}
% keep some pages from looking unbelievably appalling
\raggedbottom

% table of contents: show only down to subsections
\setcounter{tocdepth}{2}

% general formatting
\textheight=8.5in
\textwidth=6.0in
\topmargin=0in
\pagestyle{headings}

\makeindex
% an extra thing for makeindex
\newcommand{\seealso}[2]{{\em see also\/} #1}

% NEWCOMMANDS

% general
\newcommand{\folks}[1]{\begin{quote}\sf#1\end{quote}}
\newcommand{\sectionpart}[1]{\vspace{2 ex}\noindent{\bf #1}}
\newcommand{\bq}{\begin{quote}}
\newcommand{\eq}{\end{quote}}
\newcommand{\bt}{\begin{tabular}}
\newcommand{\et}{\end{tabular}}
\newcommand{\bi}{\begin{itemize}}
\newcommand{\ei}{\end{itemize}}
\newcommand{\struthack}[1]{\rule{0pt}{#1}}

\newcommand{\ToDo}[1]{}
%\newcommand{\ToDo}[1]{({\bf $\spadesuit$ ToDo:} {\em #1})}

\newcommand{\WeSay}[1]{}
%\newcommand{\WeSay}[1]{({\bf $\clubsuit$ YaleSays:} {\em #1})}

% indexing
\newcommand{\indextt}[1]{\index{#1@{\tt #1}}}
\newcommand{\indexsyn}[1]{\index{#1@{\it #1}}}
\newcommand{\indexmodule}[1]{\index{#1@{\tt #1} (module)}}
\newcommand{\indextycon}[1]{\index{#1@{\tt #1} (datatype)}}
\newcommand{\indexsynonym}[1]{\index{#1@{\tt #1} (type synonym)}}
\newcommand{\indexnote}[1]{#1n}

\makeatletter
\def\theindex{\@restonecoltrue\if@twocolumn\@restonecolfalse\fi
\columnseprule \z@
\columnsep 35pt\twocolumn[\section*{Index}Index entries that refer to nonterminals in the \Haskell{} syntax are
shown in an \mbox{$\it italic$} font.  Code entities defined in the standard
prelude (Appendix~\ref{stdprelude}) are shown in \mbox{\tt typewriter} font.
Ordinary index entries are shown in a roman font.
\vskip 20pt]
 \@mkboth{INDEX}{INDEX}\thispagestyle{plain}\parindent\z@
 \parskip\z@ plus .3pt\relax\let\item\@idxitem}
\makeatother

% outlined figures
\newcommand{\ecaption}[1]{\vspace{-1 ex}\caption{#1}\vspace{1 ex}}
% partain fiddled here...
%   also had to change two lines in verbatim.lex from
%<SYNTAX>{nl}\mbox{$\it |$}{sp}    { printf (\mbox{$\it $\\\\\ \n$\\it$}); 
%                         printf (\mbox{$\it $\\>\\makebox[3em]{$|$}$\\it$}); }
% to
%
%<SYNTAX>{nl}\mbox{$\it |$}{sp}    { printf (\mbox{$\it $\\\\\ \n$\\it$}); 
%                         printf (\mbox{$\it $\\>\\makebox[3.5em]{$|$}$\\it$}); }
% so things would still line up.  Oh what a hack.
%
%\newcommand{\outline}{\outlinewidth{1.0}}
%\newcommand{\outlinewidth}[2]{
%\begin{center}
%\fbox{ \begin{minipage}{#1\textwidth}
%\vspace{1 ex}
%#2
%\end{minipage} }
%\vspace{1 ex}
%\end{center}
%}
% 6.0in (\textwidth) - 15pt (overfullness) ~=~ 415pt
\newcommand{\outline}[1]{%
\begin{center}
\fbox{ \begin{minipage}{415pt}
\vspace{1 ex}
#1
\end{minipage} }
\vspace{1 ex}
\end{center}
}

% haskell code
% partain fiddled here...
% \newcommand{\bprog}{\par \begin{tabular}{|l} 
%                   \mbox \bgroup \begin{minipage} {\textwidth} }
% 6.0in (\textwidth) - 17pt (\parindent) ~=~ 412pt
%\newcommand{\bprog}{\par \begin{tabular}{@{}l@{}} 
%                   \mbox \bgroup \begin{minipage} {412pt} }
%\newcommand{\eprog}{\end{minipage} 
%                    \egroup
%                    \end{tabular}\\[\parskip]}
% 17pt is \parindent
% this method gives a 17pt indent in _all_ situations
\newcommand{\bprog}{%
\par\noindent\begin{tabular}{@{\hspace*{17pt}}l@{}}}
\newcommand{\eprog}{%
\end{tabular}\\[\parskip]}
\newcommand{\eprogNoSkip}{%
\end{tabular}}
%
% variants for stdprelude; don't indent, and skip a little more
\newcommand{\bprogB}{%
\begin{tabular}{@{}l@{}}}
\newcommand{\eprogB}{%
\end{tabular}\\[0.6\baselineskip]}

%special characters
\newcommand{\bkq}{\mbox{\it \char'022}} % (syntax) backquote char
\newcommand{\bkqB}{\bkq} % (syntax) backquote char (Before)
\newcommand{\bkqA}{\hspace{-.2em}\mbox{\it \char'022}}% (syntax) backquote char (After)
\newcommand{\fwq}{\mbox{\it \char'023}} % (syntax) (forward) quote char
% math formatting
\newcommand{\ba}{\begin{array}}
\newcommand{\ea}{\end{array}}
\newcommand{\mc}{\multicolumn}
\newcommand{\pile}[1]{\ba[t]{@{}l@{}} #1 \ea}
\newcommand{\eqn}[1]{\ba[t]{@{}lcl@{}} #1 \ea}
\newcommand{\equate}[1]{\[\eqn{#1}\]}
\newcommand{\la}{\leftarrow}
\newcommand{\ra}{\rightarrow}
\newcommand{\sq}[1]{[\,#1\,]}
\newcommand{\ab}[1]{\langle#1\rangle}
\newcommand{\ablarge}[1]{\langle \pile{#1\,\rangle}}
\newcommand{\lb}{[\![}
\newcommand{\rb}{]\!]}
\newcommand{\db}[1]{\lb#1\rb}
\newcommand{\ti}[1]{\mbox{{\it #1}}}
\newcommand{\tr}[1]{\mbox{{\rm #1}}}
\newcommand{\tb}[1]{\mbox{{\bf #1}}}
\newcommand{\x}{\times}
\newcommand{\lam}{\lambda}
\newcommand{\kr}{\kappa_{\rho}}
\newcommand{\syneq}{\rightarrow}
% denotational semantics
\newcommand{\denote}[3]{\[\ba{c} {\cal #1} : #2 \\[1 ex] #3 \ea\]}
\newcommand{\den}[2]{{\cal #1}\db{#2}\,}

\newcommand{\A}{\den{A}}
\newcommand{\B}{\den{B}}
\newcommand{\D}{\den{D}}
\newcommand{\E}{\den{E}}
\newcommand{\F}{\den{F}}
\newcommand{\G}{\den{G}}
\newcommand{\I}{\den{I}}
%%% \renewcommand{\L}{\den{L}}
\newcommand{\LE}{\den{L_E}}
\newcommand{\LH}{\den{L_H}}
\newcommand{\M}{\den{M}}
%%% \renewcommand{\O}{\den{O}}
\renewcommand{\P}{\den{P}}
\newcommand{\Pbot}{\den{P_{\bot}}}
\newcommand{\Q}{\den{Q}}
\newcommand{\R}{\den{R}}
\renewcommand{\S}{\den{S}}
\newcommand{\V}{\den{V}}
\newcommand{\W}{\den{W}}
\newcommand{\T}[2]{\den{T}{#1}\,\db{#2}}
% meta language
\newcommand{\PP}{\den{P'}}
\newcommand{\PS}{\den{P_S}}
\newcommand{\otherwise}{\quad\tr{otherwise}}
\newcommand{\case}[2]{\pile{
 \tr{case}\ (#1)\ \tr{of} \\
 \ba{@{\quad}l@{\ \ra\ }l@{}} #2 \ea}}
\newcommand{\where}[2]{#1 \quad\tr{where}\quad #2}
\newcommand{\wherelarge}[2]{\pile{#1 \\ \tr{where} \\ \eqn{#2}}}
\newcommand{\cond}[3]{#1 \ra #2,\ #3}
\newcommand{\condlarge}[1]{\ba[t]{@{}l@{\ \ra\ }l@{}} #1 \ea}
\newcommand{\range}[2]{{}_{#1}^{#2}\,}
% semantic operators
\newcommand{\concat}{\frown}
\newcommand{\seq}[1]{\ti{List}\ #1}
\newcommand{\opt}[1]{\widetilde{#1}}
\newcommand{\ov}{\opt{v}}
\newcommand{\fail}{\ti{none}}
\newcommand{\nonfail}{\ti{proper}}
\newcommand{\sym}{\bigtriangledown}
\newcommand{\pri}{\mathbin{\vec{\sym}}}
\newcommand{\mrg}{\mathbin{\dot{\sym}}}
\newcommand{\Sym}{\mathbin{\nabla}}
\newcommand{\Pri}{\mathbin{\vec{\Sym}}}
\newcommand{\Mrg}{\mathbin{\dot{\Sym}}}
\newcommand{\optSym}{\mathbin{\opt{\Sym}}}
\newcommand{\optodot}{\mathbin{\opt{\odot}}}
\newcommand{\proj}{\mid}
\newcommand{\restrict}{\setminus}
\newcommand{\sel}[4]{\ti{sel}_{#3#2}\ #4}
\newcommand{\bindnone}{\ti{bindnone}}
\newcommand{\bindvar}[2]{\ti{bindvar}\ \db{#1}\ #2}
\newcommand{\bindcon}[2]{\ti{bindcon}\ \db{#1}\ #2}
\newcommand{\bindconlarge}[4]{
 #4\ \bindcon{#1}{\ablarge{ #2, \\ #3}}}
\newcommand{\bindmod}[2]{\ti{bindmod}\ \db{#1}\ #2}

\newcommand{\lookupval}[2]{\ti{lookupval}\ #1\ \db{#2}} %%% NEW
\newcommand{\lookupcon}[1]{\ti{lookupcon}\ \db{#1}} %%% NEW
\newcommand{\lookupdecon}[2]{\ti{lookupdecon}\ #1\ \db{#2}} %%% NEW

% used in static.verb
\newcommand{\TT}{\den{T_T}}
\newcommand{\TA}{\den{T_A}}
\newcommand{\TB}{\den{T_B}}
\newcommand{\TD}{\den{T_D}}
\newcommand{\TDA}{\den{T_{D_A}}}
\newcommand{\TDB}{\den{T_{D_B}}}
\newcommand{\TDP}{\den{T_{P_D}}}
\newcommand{\TE}{\den{T_E}}
\newcommand{\TLE}{\den{T_{L_E}}}
\newcommand{\TLH}{\den{T_{L_H}}}
%%% \newcommand{\TG}{\den{T_G}}
\newcommand{\TQ}{\den{T_Q}}
%%% \newcommand{\TR}{\den{T_R}}
\newcommand{\TF}{\den{T_F}}
\newcommand{\TFA}{\den{T_F'}}
\newcommand{\TP}{\den{T_P}}
\newcommand{\TPP}{\den{T_P'}}
\newcommand{\TPS}{\den{T_{PS}}}
\newcommand{\MGU}{\ti{MGU}}
\newcommand{\TI}{\den{T_I}}
\newcommand{\TL}{\den{T_L}}
\newcommand{\TM}{\den{T_M}}
%%% \newcommand{\TO}{\den{T_O}}
\newcommand{\TS}{\den{T_S}}
\newcommand{\TV}{\den{T_V}}
\newcommand{\tenvm}{\ddot{\nabla}}
\renewcommand{\tb}[1]{\triangleright#1\triangleleft}
\newcommand{\unbindvar}[2]{\ti{unbindvar}\ \db{#1}\ #2}
\newcommand{\unbindcon}[2]{\ti{unbindcon}\ \db{#1}\ #2}

%
% \newcommand{\bindnone}{\ab{[], []}}
% \newcommand{\bindvar}[2]{\ab{[\,#1 \mapsto #2\,], []}}
% \newcommand{\bindcon}[2]{\ab{[], [\,#1 \mapsto #2\,]}}
% \newcommand{\bindconlarge}[4]{
%  \langle [], [\,#1 \mapsto \langle \pile{#2 \\ #3 \rangle\,] #4 \rangle}}
% \newcommand{\bindmod}[2]{[\,#1 \mapsto #2\,]}
%
% Haskell syntax macros: math mode assumed
\newcommand{\system}[2]{#1\mbox{\tt ;;}\cdots\mbox{\tt ;;}#2}
\newcommand{\module}[4]{module\ #1\mbox{\tt :}\ #2\ #3\ #4}
%%% \newcommand{\exportnone}{\,}
%%% \newcommand{\export}[1]{\mbox{\tt export}\ #1\mbox{\tt ;}}
%%% \newcommand{\importnone}{\,}
%%% \newcommand{\importcomb}[2]{#1\ #2}
%%% \newcommand{\import}[1]{\mbox{\tt import}\ #1\mbox{\tt ;}}
%%% \newcommand{\importwith}[2]{\mbox{\tt import}\ #1\ #2\mbox{\tt ;}}
%%% \newcommand{\rename}[2]{#1\mbox{\tt <-}#2}
%%% \newcommand{\declcomb}[2]{#1\ \mbox{\tt ;;}\ #2}
\newcommand{\exposing}[1]{\mbox{\tt expose}\ #1}
\newcommand{\hiding}[1]{\mbox{\tt hide}\ #1}
\newcommand{\importnone}{\;}
\newcommand{\importcomb}[2]{#1\ \mbox{\tt ;;}\ #2}
\newcommand{\import}[2]{\mbox{\tt import}\ #1\ #2}
\newcommand{\rename}[2]{#1\ \mbox{\tt =}\ #2}
\newcommand{\declcomb}[2]{#1\ \mbox{\tt ;;}\ #2}

\newcommand{\type}[2]{\mbox{\tt type}\ #1\ \mbox{\tt =}\;#2}
\newcommand{\data}[2]{\mbox{\tt data}\ #1\ \mbox{\tt =}\;#2}
\newcommand{\tuple}[2]{\mbox{\tt tuple}\ #1\ \mbox{\tt =}\;#2} %%% NEW!
\newcommand{\view}[3]{\mbox{\tt view}\ #1\ \mbox{\tt =}\;#2\ \mbox{\tt where}\ \mbox{\tt {\char'173}}\ #3\ \mbox{\tt {\char'175}}}
\newcommand{\class}[2]{\mbox{\tt class}\ #1\ \mbox{\tt where}\ \mbox{\tt {\char'173}}\ #2\ \mbox{\tt {\char'175}}}
\newcommand{\instance}[2]{\mbox{\tt instance}\ #1\ \mbox{\tt where}\ \mbox{\tt {\char'173}}\ #2\ \mbox{\tt {\char'175}}}
\newcommand{\signature}[2]{#1\ \mbox{\tt ::}\ #2}
\newcommand{\binding}[2]{#1\ \mbox{\tt =}\ #2}
\newcommand{\lamexpr}[2]{\mbox{\tt {\char'134}} #1 \mbox{\tt ->} #2}
% While lambda defs. change...  if change, take care of preceding line MMG
\newcommand{\lamb}{\mbox{\tt {\char'134}\ }}
\newcommand{\whereexpr}[2]{#1\ \mbox{\tt where}\ \mbox{\tt {\char'173}}\ #2\ \mbox{\tt {\char'175}}}
\newcommand{\compexpr}[2]{\mbox{\tt [}#1\ \mbox{\tt |}\ #2\mbox{\tt ]}}
\newcommand{\genclause}[2]{#1\ \mbox{\tt <-}\ #2}
\newcommand{\qualcomb}[2]{#1\ \mbox{\tt ,}\ #2}
\newcommand{\genguard}[1]{\ #1\ }
\newcommand{\caseexpr}[2]{\mbox{\tt case}\ #1\ \mbox{\tt of}\ \mbox{\tt {\char'173}}\ #2\ \mbox{\tt {\char'175}}}
\newcommand{\simplecaseexpr}[5]{\mbox{\tt case}\ #1\ \mbox{\tt of}\ \mbox{\tt {\char'173}}\ #2\ \mbox{\tt ->}\ #3;\ #4\ \mbox{\tt ->}\ #5\ \mbox{\tt {\char'175}}} 
\newcommand{\iteexpr}[3]{\mbox{\tt if}\ #1\ \mbox{\tt then}\ #2\ \mbox{\tt else}\ #3}
\newcommand{\itexpr}[2]{\mbox{\tt if}\ #1\ \mbox{\tt then}\ #2}
\newcommand{\gpat}[2]{#1\ \mbox{\tt |}\ #2}
\newcommand{\aspat}[2]{#1 \mbox{\tt \ @\ } #2}
\newcommand{\fclause}[2]{#1\ \mbox{\tt =}\ #2}
\newcommand{\fsym}[2]{#1\ \mbox{\tt ;}\ #2}
\newcommand{\fpri}[2]{#1\ \mbox{\tt ;}\ \mbox{\tt else}\ \mbox{\tt ;}\ #2}
\newcommand{\aclause}[2]{#1\ \mbox{\tt ->}\ #2}
\newcommand{\saclause}[4]{#1\ \mbox{\tt ->}\ #2;\ #3\ \mbox{\tt ->}\ #4}
\newcommand{\asym}[2]{#1\ \mbox{\tt ;}\ #2}
\newcommand{\apri}[2]{#1\ \mbox{\tt ;}\ \mbox{\tt else}\ \mbox{\tt ;}\ #2}
\newcommand{\dotted}[3]{#1\ #2\ \ldots\ #3}
\newcommand{\functype}[2]{#1\ \mbox{\tt ->}\ #2}
\newcommand{\predtype}[2]{#1\ \mbox{\tt =>} #2}
\newcommand{\xp}{\dotted{x}{p_1}{p_n}}
\newcommand{\xpg}{\dotted{x}{p_1}{p_n\ \mbox{\tt {\char'173}}\ g\ \mbox{\tt {\char'175}}}}
\newcommand{\es}{e_1\ \ldots\ e_n}
\newcommand{\ps}{p_1\ \ldots\ p_n}
\newcommand{\vs}{v_1\ \ldots\ v_n} %%% NEW
\newcommand{\xs}{x_1\ \ldots\ x_n} %%% NEW
\newcommand{\cT}{\dotted{c}{T_1}{T_n}}
\newcommand{\cTm}{\dotted{c_i}{T_{i1}}{T_{in_i}}\, \mbox{\tt |}\range{i=1}{m}}
% syntax meta-language
\newcommand{\arity}[1]{\tr{arity}\ #1}
\newcommand{\infix}[1]{\tr{infix}\ #1}
\newcommand{\prefix}[1]{\tr{prefix}\ #1}
%
\newcommand{\tl}[1]{{\sc #1}}
%OLD: \newcommand{\Haskell}{{\sc Haskell}}
\newcommand{\Haskell}{Haskell}

%\sloppy

% a few hyphenation patterns, anyone?
\hyphenation{da-ta-type da-ta-types}
\hyphenation{Has-kell}

\begin{document}

\begin{titlepage}

\outline{
\vspace{.3in}
\begin{center}
{\LARGE\bf Report on the} \\[.1in]
{\LARGE\bf Programming Language} \\[.3in]
{\huge\bf Haskell} \\[.3in]
{\Large\bf A Non-strict, Purely Functional Language} \\[.3in]
{\Large\bf Version 1.1} \\[.1in]
% {\Large\bf - Final Draft -} \\[.1in]
{\large\bf August 1991}
\end{center}
\vspace{.15in}
\begin{center} \large
Paul Hudak$^1$ [editor] \\
Simon Peyton Jones$^2$ [editor] \\
Philip Wadler$^2$ [editor] \\
Brian Boutel$^3$ \\
Jon Fairbairn$^4$ \\
Joseph Fasel$^5$ \\
Mar{\'\i}a M. Guzm{\'a}n$^1$ \\
Kevin Hammond$^2$ \\
John Hughes$^2$ \\
Thomas Johnsson$^6$ \\
Dick Kieburtz$^7$ \\
Rishiyur Nikhil$^8$ \\
Will Partain$^2$ \\
John Peterson$^1$ 
\end{center}
\vspace{.15in}

\begin{quotation} \noindent
Authors' affiliations:
(1)~Yale University,
(2)~University of Glasgow,
(3)~Victoria University of Wellington,
(4)~Cambridge University,
(5)~Los Alamos National Laboratory,
(6)~Chalmers University of Technology,
(7)~Oregon Graduate Institute of Science and Technology,
(8)~Massachusetts Institute of Technology.
\end{quotation}
\vspace{.2in}
}

\end{titlepage}

\pagenumbering{roman}

\tableofcontents
\startnewstuff

\parskip=6pt plus2pt minus2pt

%
% $Header$
%
\begin{center}
\Large\bf Preface to Version~1.1
\end{center}

\vspace{.2in}

\noindent
Following the Version~1.0 \Haskell{}~report, several sites have
implemented \Haskell{} (or a subset thereof) and people have started
to use these implementations.  Based on this experience of
implementation and use, it became apparent that a modest revision of
the language would be desirable, in which some improvements in syntax
could be made and certain features generalised.  This Version~1.1
report is the result.

This revision was specifically {\em not} intended to add any
substantial new features to the language, but rather to ``tidy up''
the existing language.  Despite this narrow focus, a wide
debate ensued, conducted on the \Haskell{} mailing list
\index{Haskell mailing list@\Haskell{} mailing list}
(see page~\pageref{haskell-mailing-list}) rather than just among members of
the original committee.

In this minor revision, the tricky issues identified in the preface to
Version~1.0 remain, so that preface should be read in conjunction with
this one.

\subsection*{Summary of changes}
\label{preface-changes-11}

The main changes (other than concrete syntax) are as follows.

\begin{itemize}
\item
Class methods
\index{class method}
may be polymorphic and overloaded in type variables
other than the class variable (Section~\ref{class-decls}).

\item
The ``monomorphism restriction''
\index{monomorphism restriction}
has been made more precise, and
relaxed in the case where the programmer supplies a type signature
(Section~\ref{function-bindings}).

\item
The meaning of contexts in \mbox{\tt data} declarations has been clarified
\index{context!in data declaration@in {\ptt data} declaration}
(Section~\ref{datatype-decls}), and \mbox{\tt type} synonym declarations are no
longer permitted to have contexts (Section~\ref{type-synonym-decls}).
\index{type synonym}

\item
If the \mbox{\tt deriving} clause on a \mbox{\tt data} declaration is omitted, no
instances are automatically derived (Section~\ref{derived-decls}).
\index{derived instance}

\item
A module \mbox{$\it m$} may refer to all of its own local definitions in an export
list using \mbox{$\it m\makebox{\tt ..}$} (Section~\ref{export}).
\index{export list}
\end{itemize}

\noindent
The main syntactic changes are as follows:
\begin{itemize}
\item
A new form of expression, a \mbox{\tt let}-expression,
\index{let expression}
has been added, which replaces and has the same semantics as a \mbox{\tt where}
expression.  (In particular, the bindings it introduces are mutually
recursive; \Haskell{} has no non-recursive \mbox{\tt let} construct.)
Bindings may also be introduced by a \mbox{\tt where} clause, but such
\mbox{\tt where} clauses are now attached to a group of guarded right-hand
sides, and scope over the guards.  The previous inability to scope
definitions over guards was a significant shortcoming of the language.

\item
Sections
\index{section}
have been introduced for binary operators.  For example, the
expression \mbox{\tt (/\ 2)} is the function which divides its argument by 2,
and \mbox{\tt (2\ /)} is the function which divides 2 by its argument.

\item
The standard prelude has been debugged and revised.

%\item
%The syntax of interfaces has been changed, to make it clearer
%what is being {\em exported} and what is merely being {\em named}.

\end{itemize}

\noindent
A few other nontrivial changes to the syntax are listed in
Appendix~\ref{syntax-changes}.

\subsection*{Implementations}\label{implementors}

Several groups are working on implementations of \Haskell{}, including
those at Chalmers (contact: \mbox{\tt hbc@cs.chalmers.se}), Glasgow
(\mbox{\tt haskell-request@dcs.glasgow.ac.uk}), Syracuse (\mbox{\tt polar@top.cis.syr.edu}), and Yale
(\mbox{\tt haskell-request@cs.yale.edu}).  Official announcements about these
implementations will appear on the \Haskell{} technical mailing list
\index{Haskell mailing list@\Haskell{} mailing list}
(see page~\pageref{haskell-mailing-list}).

\subsection*{Formal Semantics}
\index{formal semantics}

Work has also been undertaken at Glasgow on a formal static and
dynamic semantics for \Haskell{}
\cite{dynamic-semantics,static-semantics}.  These efforts are well
advanced but as yet incomplete.

\subsection*{Acknowledgements}

Language design is an evolutionary process, and the group of people
involved undergoes evolution as well.  We wish to thank past members
of the \Haskell{} Committee---Arvind, Mike Reeve, David Wise, and
Jonathan Young---for their previous contributions and continued
support.  We also thank those who braved the storm of electronic mail
on the \Haskell{} mailing list, and responded with constructive
suggestions for the revised language.  The following were especially
helpful and active:
Lennart Augustsson,
Cordelia Hall,
Kent Karlsson,
Mark Jones,
Mark Lillibridge,
and Satish Thatte.  

%\ToDo{past authors sentence (Simon)}

Numerous others contributed to the debate, and we thank them also.

\begin{flushright}
The \Haskell{} Committee\\
19 August 1991
\end{flushright}

% Local Variables: 
% mode: latex
% End:


\startnewstuff
%
% $Header$
%
\begin{center}
\Large\bf Preface to Version~1.0 (revised)
\end{center}

\vspace{.2in}

\begin{quote}
``{\em Some half dozen persons have written technically on combinatory
logic, and most of these, including ourselves, have published
something erroneous.  Since some of our fellow sinners are among the
most careful and competent logicians on the contemporary scene, we
regard this as evidence that the subject is refractory.  Thus fullness
of exposition is necessary for accuracy; and excessive condensation
would be false economy here, even more than it is ordinarily.}''
\begin{flushright}
Haskell B.~Curry and Robert Feys \\
in the Preface to {\em Combinatory Logic} \cite{curry&feys:book}, May 31, 1956
\end{flushright}
\end{quote}

\vspace{.2in}

\noindent
In September of 1987 a meeting was held at the conference on
Functional Programming
Languages and Computer Architecture in
Portland, Oregon, to discuss an unfortunate situation
in the functional programming community: there had come into being
more than a dozen non-strict, purely functional programming languages,
all similar in expressive power and semantic underpinnings.  There
was a strong consensus at this meeting that more widespread use of
this class of functional languages\index{functional language} was
being hampered by the lack of a common language.  It was decided
that a committee should be formed to design such a language, providing
faster communication of new ideas, a stable foundation for real
applications development, and a vehicle through which others
would be encouraged to use functional languages.  This
document describes the result of that committee's efforts: a purely
functional programming language called \Haskell{},
\index{Haskell@\Haskell{}}
named after the logician Haskell B.~Curry\index{Curry, Haskell B.}
whose work provides the logical basis for much of ours.

\subsection*{Goals}

The committee's primary goal was to design a language that
satisfied these constraints:
\begin{enumerate}
\item It should be suitable for teaching, research, and applications,
      including building large systems.
\item It should be completely described via the publication of a formal
      syntax and semantics.
\item It should be freely available.  Anyone should be permitted to
      implement the language and distribute it to whomever they please.
\item It should be based on ideas that enjoy a wide consensus.  
\item It should reduce unnecessary diversity in functional 
      programming languages.
\end{enumerate}
The committee hopes that \Haskell{} can serve as a
basis for future research in language design.
We hope that extensions or
variants of the language may appear, incorporating experimental
features.

\subsection*{This Report}

This report is the official specification of the \Haskell{}
language and should be suitable for writing programs and building
implementations.  It is {\em not} a tutorial on programming in
\Haskell{}, so some familiarity with functional languages is assumed.
As this is the first edition of the specification, there may be some errors
and inconsistencies; beware.

\subsection*{The Next Stage}

\Haskell{} is a large and complex language, designed
for a wide spectrum of purposes.  It also introduces a major new 
technical innovation, namely using type classes to handle overloading in
a systematic way.  This innovation permeates every aspect of the language.

\Haskell{} is bound to contain infelicities and
errors of judgement.   We welcome your
comments, suggestions, and criticisms on the language or its presentation in
the report.  Together with your input and our own experience of using
the language, we plan to meet at some future time to resolve
difficulties and further stabilise the design.

A common mailing list for technical discussion of \Haskell{} can be
reached at either \mbox{\tt haskell@cs.yale.edu} or \mbox{\tt haskell@dcs.glasgow.ac.uk}.
\label{haskell-mailing-list}
\index{Haskell mailing list@\Haskell{} mailing list}
Errata sheets for this report will be posted there.
To subscribe, send a request to 
\mbox{\tt haskell-request@dcs.glasgow.ac.uk} (European residents) or
\mbox{\tt haskell-request@cs.yale.edu} (residents elsewhere).

We thought it would be helpful to identify the aspects of the language
design that seem to be most finely balanced, and hence are the
most likely candidates for change when we review the language.
The following list summarises these areas.  It will only be fully
comprehensible after you have read the report.

\paragraph*{Mutually recursive modules.}
Mutual recursion among modules is unrestricted at pre\-sent, which is
obviously desirable from the programmer's point of view, but which poses
significant challenges to the compilation system.  In particular, it is
{\em not} sufficient to start with trivial interfaces for each module and
iterate to a fixpoint, as this example shows:
\bprog
\mbox{\tt module\ F(\ f\ )\ where}\\
\mbox{\tt \ \ \ \ \ \ \ \ import\ G}\\
\mbox{\tt \ \ \ \ \ \ \ \ f\ [x]\ =\ g\ x}\\
\mbox{\tt }\\[-8pt]
\mbox{\tt module\ G(\ g\ )\ where}\\
\mbox{\tt \ \ \ \ \ \ \ \ import\ F}\\
\mbox{\tt \ \ \ \ \ \ \ \ g\ =\ f}
\eprog
If a compilation system starts off by giving \mbox{\tt F} and \mbox{\tt G} interfaces
that give the type signatures \mbox{\tt f::a} and \mbox{\tt g::b} respectively, then
compiling the two modules alternately will not reach a fixed point.
(This only happens if there is a type error, but it is obviously
undesirable behaviour.)  In general, a compiler may need to analyse a
set of mutually recursive modules as a whole, rather than separately.

\paragraph*{Generalising type classes.}  A number of restrictions are
placed on the class system in \Haskell{}.  Currently, instances
are attached to the top level type of an object and are exported
implicitly with classes and types.  A number of proposals for
generalising the class system have been discussed, among them attaching
instances to more complex types, parameterising classes over type
constructors, allowing redefinition of instances, and making instances
explicit in import and export lists.  Some of these proposals have
been implemented and are part of the available \Haskell{} systems.  As
we gain more experience with the class system we hope to improve it
in the future.

\paragraph*{Default methods.}
\index{default method}
Section~\ref{class-decls} describes how a class declaration may
include default methods for some of its operations.  We considered extending
this so that a class declaration could include default methods {\em for
operations of its superclasses}, which override the superclass's default
method.  This looks like an attractive idea, which will certainly
be considered for a future revision.

\paragraph*{Defaults for ambiguous types.}
Section~\ref{default-decls}
describes how ambiguous typings, which arise due to the type-class system,
are resolved.  Ideally, the choice made should not matter. For example,
consider the expression \mbox{\tt if\ round\ x\ >\ 0\ then\ E1\ else\ E2}.  It should
not matter whether \mbox{\tt round} returns \mbox{\tt Int} or \mbox{\tt Integer};
a bad choice could result in overflow, or a less efficient
program, but if a result is produced it will be correct.  

Our resolution rules strive only to resolve ambiguous
types where the type chosen does not ``matter'' in this sense, but we have
not been entirely successful, for example where floating point is concerned.
Further research and practical experience may suggest a better set of rules.

\paragraph*{Static semantics of \mbox{\tt let} and \mbox{\tt where} bindings.}
\label{next-stage-monomorphic}

The rules at the end of Section~\ref{pattern-bindings} comprise the
``monomorphism restriction''
\index{monomorphism restriction}
in \Haskell{}.  The
restriction solves two problems, which are summarised below, but at
the cost of restricting expressiveness.  Only experience will tell how
much of a problem this is for the programmer.

These are the two problems.  First, the expression
\mbox{\tt let\ x\ =\ factorial\ 1000\ in\ (x,x)} looks as though 
\mbox{\tt x} should only be computed once.  If \mbox{\tt x} were used
at different overloadings, however, \mbox{\tt factorial\ 1000} would be computed 
twice, once at each type.  We have found examples where the loss of efficiency
is exponential in the size of the program.  
Modest compiler optimisations can often eliminate
the problem, but we have found no simple scheme that can {\em guarantee} to do so.
The restriction solves the problem by ensuring that all uses of \mbox{\tt x} are
at the same overloading, and hence that its evaluation can be shared as usual.

Second, a rather subtle form of type 
ambiguity (Section~\ref{default-decls})
is eliminated by the restriction to non-overloaded pattern bindings.
An example is:
\bprog
\mbox{\tt readNum\ s\ r\ =\ (n*r,s')\ where\ [(n,s')]\ =\ reads\ s}
\eprog
Here \mbox{\tt n::(Num\ a,\ Text\ a)\ =>\ a}, \mbox{\tt s'::Text\ a\ =>\ String}.  If the
definition of \mbox{\tt [(n,s')]} is polymorphic, the \mbox{\tt a}'s may be resolved as
different types.

(Note: As of the version~1.1 report, the monomorphism restriction is
relaxed, provided that the programmer gives an explicit type
signature.  See Section~\ref{sect:monomorphism-restriction} for
precise details.)


\paragraph*{Overloaded constants.}
Overloaded constants (e.g.~\mbox{\tt 1}, which has type \mbox{\tt Num\ a\ =>\ a}) are
extraordinarily convenient when programming, but are the source of
several serious technical problems, including both of those mentioned
in the two preceding items.  One could eliminate overloaded
constants altogether; we considered this at length, and we are sure to
reconsider it when we review the language.

\paragraph*{Polymorphism in \mbox{\tt case} expressions.}

The type of a variable bound by a Standard~ML case-expression is monomorphic;
\index{monomorphic type variable}
we have made the same decision in \Haskell{}
(Section \ref{case-semantics}).
The question of whether such types can be made polymorphic interacts
with the restrictions on polymorphism for pattern-bound variables,
mentioned above.  For the present, we have erred on the side of conservatism,
but this decision should be reviewed.

%old:
%There is no technical reason why
%the type of such a variable should not be polymorphic; in such a case,
%the translation between \mbox{\tt let} expressions and
%\mbox{\tt case} expressions would preserve the static semantics.

%We have erred on the side of conservatism, but this
%decision will be reviewed.  If implemented, such a change would be
%upward-compatible.

\subsection*{Acknowledgements}

We heartily thank these people for their useful contributions
to this report:
Lennart Augustsson,
Richard Bird,
Stephen Blott,
Tom Blenko,
Duke Briscoe,
Chris Clack,
Guy Cousineau,
Tony Davie,
Chris Fasel,
Pat Fasel, 
Bob Hiromoto,
Nic Holt,
Simon B.~Jones, 
Stef Joosten, 
Mike Joy,
Richard Kelsey,
Siau-Cheng Khoo, 
Amir Kishon, 
John Launchbury,
Olaf Lubeck, 
Randy Michelsen, 
Rick Mohr,
Arthur Norman,
Paul Otto, 
Larne Pekowsky,
John Peterson,
Rinus Plasmeijer,  
John Robson, 
Colin Runciman, 
Lauren Smith, 
Raman Sundaresh,
Tom Thomson,
Pradeep Varma,
Tony Warnock,
Stuart Wray,
and Bonnie Yantis.
We also thank those who participated in the lively discussions
about \Haskell{} on the FP mailing list during an interim period of
the design.

%We owe a particular debt to Mar{\'\i}a Guzm{\'a}n at Yale and Will Partain
%at Glasgow, who have spent many hours working on the details
%and typography of the report.

Finally, aside from the important foundational work laid by Church,
Rosser, Curry, and others on the lambda calculus, we wish to
acknowledge the influence of many noteworthy programming languages
developed over the years.  Although it is difficult to pinpoint the
origin of many ideas, we particularly wish to acknowledge the
influence of McCarthy's Lisp \cite{mcca60} (and its modern-day
incarnation, Scheme \cite{RRRRS}); Landin's ISWIM \cite{landin66};
Backus's FP \cite{back78}; Gordon, Milner, and Wadsworth's ML
\cite{gordonetal78}; Burstall, MacQueen, and Sannella's Hope
\cite{burs80}; and Turner's series of languages culminating in
Miranda \cite{turn85}.\footnote{{\rm Miranda} is a trademark of
Research Software Ltd.} Without these forerunners \Haskell{} would
not have been possible.

\begin{flushright}
The \Haskell{} Committee\\
1 April 1990\\
(revised) 19 August 1991
\end{flushright}

% Local Variables: 
% mode: latex
% End:


\startnewstuff

\pagenumbering{arabic}

%
% $Header$
%
\section{Introduction}
\label{introduction}

\Haskell{}\index{Haskell@\Haskell{}} is a general purpose, purely functional
programming language incorporating many recent innovations in
programming language research,
including higher-order functions,
non-strict semantics, static polymorphic typing, user-defined
algebraic datatypes, pattern-matching, list comprehensions, a module
system, and a rich set of primitive datatypes, including lists,
arrays, arbitrary and fixed precision integers, and floating-point
numbers.  \Haskell{} is both the culmination
and solidification of many years of research on functional
languages---the design has been influenced by languages as old as
ISWIM and as new as Miranda.

Although the initial emphasis was on standardisation, \Haskell{}
also has several new features that both simplify and
generalise the design.  For example,
\begin{enumerate}
\item Rather than using {\em ad hoc} techniques for
overloading,\index{overloading}
\Haskell{} provides an explicit overloading facility, integrated with
the polymorphic type system\index{type system}, that allows the
precise definition of overloading behaviour for any operator or function.

%For example, this facility provides a simple way to handle
%``equality types'' (types whose elements may be tested for generic
%equality), a problem that arises with Hindley-Milner style
%polymorphism, and that is usually treated in an {\em ad hoc} manner.

\item The conventional notion of ``abstract data
type''\index{abstract datatype}
has been unbundled
into two orthogonal components:
data abstraction\index{data abstraction}
and information hiding.\index{information hiding}

\item \Haskell{} has a flexible I/O facility that unifies two
popular styles of purely functional I/O---the {\em stream} model and
the {\em continuation} model---and both styles can be mixed within the same
program.  The system supports most of the standard operations provided by
conventional operating systems while retaining referential
transparency within a program.

\item Recognising the importance of arrays, \Haskell{} has a
family of multidimensional non-strict immutable arrays\index{array}
whose special interaction with list comprehensions provides a
convenient ``array comprehension'' syntax for defining arrays
monolithically.
\end{enumerate}

This report defines the syntax for \Haskell{} programs and an
informal abstract semantics for the meaning of such
programs; the formal abstract semantics is in preparation.%
\index{formal semantics}
We leave as implementation
dependent the ways in which \Haskell{} programs are to be
manipulated, interpreted, compiled, etc.  This includes such issues as
the nature of batch versus interactive programming environments, and
the nature of error messages returned for undefined programs
(i.e.~programs that formally evaluate to $\bot$).

\subsection{Program Structure}\index{program structure}
\label{programs}

In this section, we describe the abstract syntactic and semantic structure of
\Haskell{}, as well as how it relates to the organisation of the
rest of the report.
\begin{enumerate}
\item At the topmost level a \Haskell{} program is a set
of {\em modules} (described in Section~\ref{modules}).  Modules provide
a way to control namespaces
and to re-use software in large programs.

\item The top level of a module consists of a collection of
{\em declarations}, of which there are several kinds, all described
in Section~\ref{declarations}.  Declarations
define things such as ordinary values, datatypes, type
classes, and fixity information.

\item At the next lower level are {\em expressions}, described
in Section~\ref{expressions}.  An expression denotes a {\em value}
and has a {\em static type}; expressions are at the heart of
\Haskell{} programming ``in the small.''

\item At the bottom level is \Haskell{}'s {\em
lexical structure}, defined in Section~\ref{lexical-structure}.  The
lexical structure captures the concrete
representation of \Haskell{} programs in text files.

\end{enumerate}
This report proceeds bottom-up with respect
to \Haskell{}'s syntactic structure.

The sections not mentioned above are Section~\ref{basic-types}, which
describes the standard built-in datatypes in \Haskell{}, and
Section~\ref{io}, which discusses the I/O facility in \Haskell{}
(i.e.~how \Haskell{} programs communicate with the outside world).
Also, there are several appendices describing the
standard prelude, the
concrete syntax,
%% static and dynamic formal semantics,
the semantics of I/O,
and the specification of derived instances.

Examples of \Haskell{} program fragments in running text are given
in typewriter font:
% highlighted with a vertical line to the left of the text:
% as in:
\bprog
\mbox{\tt \ let\ x\ =\ 1}\\
\mbox{\tt \ \ \ \ \ z\ =\ x+y}\\
\mbox{\tt \ in\ \ z+1}
\eprog
``Holes'' in program fragments representing arbitrary
pieces of \Haskell{} code are written in italics, as in 
\mbox{$\it \makebox{\tt if}\ e_1\ \makebox{\tt then}\ e_2\ \makebox{\tt else}\ e_3$}.  Generally the italicised names will
be mnemonic, such as \mbox{$\it e$} for expressions, \mbox{$\it d$} for
declarations, \mbox{$\it t$} for types, etc.

\subsection{The \Haskell{} Kernel}
\index{Haskell kernel@\Haskell{} kernel}
\label{intro-kernel}

\Haskell{} has adopted many of the convenient syntactic structures
that have become popular
in functional programming.  In all cases, their formal
semantics can be given via translation into a proper subset of
\Haskell{} called the \Haskell{} {\em kernel}.  It is
essentially a slightly sugared variant of the lambda calculus with
a straightforward denotational semantics.  The translation of each
syntactic structure into the kernel is given as the syntax is
introduced.
% and the formal semantics of the kernel is given in
% Appendix~\ref{formal-semantics}.
This modular design facilitates
reasoning about \Haskell{} programs and provides useful guidelines
for implementors of the language.

% In specifying the translation of special syntax, named entities are
% often referred to ``as defined in the standard prelude.''  This means
% that even if these names are rebound (i.e.~the standard prelude name
% is not currently in scope), the translation still takes on the meaning
% as defined in the standard prelude.  In other words, the meaning of
% \Haskell{}'s syntax is invariant.

\subsection{Values and Types}
\index{value}\index{type}
\label{errors}\index{error}

An expression\index{expression} evaluates to a {\em value} and has a
static {\em type}.  Values and types are not mixed in
\Haskell{}.
However, the type system
allows user-defined datatypes of various sorts, and permits not only
parametric polymorphism\index{polymorphism} (using a
traditional Hindley-Milner\index{Hindley-Milner type system} type structure) but
also {\em ad hoc} polymorphism, or {\em overloading} (using
{\em type classes}).

Errors in \Haskell{} are semantically equivalent to
$\bot$.  Technically, they are not distinguishable
from nontermination, so the language includes no mechanism
for detecting or acting upon errors.  Of course, implementations
will probably try to provide useful information about
errors.
% A more elaborate treatment of errors is left as a future extension to
% \Haskell{}.

\subsection{Namespaces}
\index{namespaces}
\label{namespaces}

There are six kinds of names in \Haskell{}: those for {\em variables} and
{\em constructors} denote values; those for {\em type variables}, {\em
type constructors}, and {\em type classes} refer to entities related
to the type system; and {\em module names} refer to modules.
There are three constraints on naming:\nopagebreak[4]
\begin{enumerate}
\item Names for variables and type variables are identifiers beginning
      with small letters; the other four kinds of names are
      identifiers beginning with capitals.
\item Constructor operators are operators beginning with ``\mbox{\tt :}'';
      variable operators are operators not beginning with ``\mbox{\tt :}''.
\item An identifier must not be used as the name of a type constructor
      and a class in the same scope.
\end{enumerate}
These are the only constraints; for example,
\mbox{\tt Int} may simultaneously be the name of a module, class, and constructor
within a single scope.

\Haskell{} provides a lexical syntax for infix {\em
operators} (either functions or constructors).  To emphasise
that operators are bound to the same things as identifiers, and to
allow the two to be used interchangeably, there is a simple way to
convert between the two: any function or constructor identifier may be
converted into an operator by enclosing it in grave accents, and any
operator may be converted into an identifier by enclosing it in
parentheses.  For example, \mbox{\tt x\ +\ y} is equivalent to
\mbox{\tt (+)\ x\ y}, and \mbox{\tt f\ x\ y} is the same as
\mbox{\mbox{\tt x} \bkqB\mbox{\tt f}\bkqA\mbox{\tt \ y}}.
These lexical matters are discussed further in
Section~\ref{lexical-structure}.

\subsection{Layout}\index{layout}
\label{lexemes-layout}

In the syntax given in the rest of the report, {\em declaration
lists} are always preceded by one of the keywords \mbox{\tt let}, \mbox{\tt where},
or \mbox{\tt of}, and are enclosed within curly braces (\mbox{\tt {\char'173}\ {\char'175}}) with
the individual declarations separated (or terminated) by semicolons
(\mbox{\tt ;}). For example, the syntax of a \mbox{\tt let} expression is:
\[
\mbox{$\it \makebox{\tt let\ {\char'173}}\ decl_1\ \makebox{\tt ;}\ decl_2\ \makebox{\tt ;}\ \ldots \ \makebox{\tt ;}\ decl_n\ \makebox{\tt {\char'175}\ in\ }\ exp$}
\]

%
% $Header$
%
% partain:
% in a separate file, because it is included twice (for now);
%  in intro.verb and syntax.verb

\Haskell{} permits the omission of the braces and semicolons by
using {\em layout} to convey the same information.  This allows both
layout-sensitive and -insensitive styles of coding, which
can be freely mixed within one program.  Because layout is
not required, \Haskell{} programs can be straightforwardly
produced by other programs.
% without worrying about deeply nested layout difficulties.

The layout (or ``off-side'') rule\index{off-side rule} takes effect whenever the
open brace is omitted after the keyword \mbox{\tt where}, \mbox{\tt let} or \mbox{\tt of}.
When this happens, the indentation of the next lexeme (whether or not
on a new line) is remembered and the omitted open brace is inserted
(the whitespace preceding the lexeme may include comments).
For each subsequent line, if it contains only whitespace or is
indented more, then the previous item is continued (nothing is
inserted); if it is indented the same amount, then a new item begins
(a semicolon is inserted); and if it is indented less, then the
declaration list ends (a close brace is inserted).  A close brace is
also inserted whenever the syntactic category containing the
declaration list ends; that is, if an illegal lexeme is encountered at a
point where a close brace would be legal, a close brace is inserted.
The layout rule will match only those open braces
that it has inserted; an
open brace that the user has inserted must be
matched by a close brace inserted by the user.

Given these rules, a single newline may actually terminate several
declaration lists.  Also, these rules permit:
\bprog
\mbox{\tt f\ x\ =\ let\ a\ =\ 1;\ b\ =\ 2\ }\\
\mbox{\tt \ \ \ \ \ \ \ \ \ \ g\ y\ =\ exp2\ in\ exp1}
\eprog
making \mbox{\tt a}, \mbox{\tt b} and \mbox{\tt g} all part of the same declaration
list.

To facilitate the use of layout at the top level of a module
(several modules may reside in one file), the keywords
\mbox{\tt module} and \mbox{\tt interface} and the end-of-file token are assumed to occur in column
0 (whereas normally the first column is 1).  Otherwise, all
top-level declarations would have to be indented.


As an example, Figure~\ref{layout-before} shows a (somewhat contrived)
module and Figure~\ref{layout-after} shows the result of applying the
layout rule to it.  Note in particular: (a)~the line beginning \mbox{\tt {\char'175}{\char'175};pop},
where the termination of the previous line invokes three applications
of the layout rule, corresponding to the depth (3) of the nested
\mbox{\tt where} clauses, (b)~the close braces in the \mbox{\tt where} clause nested
within the tuple and \mbox{\tt case} expression, inserted because the end of the
tuple was detected, and (c)~the close brace at the very end, inserted
because of the column 0 indentation of the end-of-file token.

\begin{figure}
\outline{
\mbox{\tt module\ AStack(\ Stack,\ push,\ pop,\ top,\ size\ )\ where}\\
\mbox{\tt data\ Stack\ a\ =\ Empty\ }\\
\mbox{\tt \ \ \ \ \ \ \ \ \ \ \ \ \ |\ MkStack\ a\ (Stack\ a)}\\
\mbox{\tt }\\[-8pt]
\mbox{\tt push\ ::\ a\ ->\ Stack\ a\ ->\ Stack\ a}\\
\mbox{\tt push\ x\ s\ =\ MkStack\ x\ s}\\
\mbox{\tt }\\[-8pt]
\mbox{\tt size\ ::\ Stack\ a\ ->\ Integer}\\
\mbox{\tt size\ s\ =\ length\ (stkToLst\ s)\ \ where}\\
\mbox{\tt \ \ \ \ \ \ \ \ \ \ \ stkToLst\ \ Empty\ \ \ \ \ \ \ \ \ =\ []}\\
\mbox{\tt \ \ \ \ \ \ \ \ \ \ \ stkToLst\ (MkStack\ x\ s)\ \ =\ x:xs\ where\ xs\ =\ stkToLst\ s}\\
\mbox{\tt }\\[-8pt]
\mbox{\tt pop\ ::\ Stack\ a\ ->\ (a,\ Stack\ a)}\\
\mbox{\tt pop\ (MkStack\ x\ s)}\\
\mbox{\tt \ \ =\ (x,\ case\ s\ of\ r\ ->\ i\ r\ where\ i\ x\ =\ x)\ --\ (pop\ Empty)\ is\ an\ error}\\
\mbox{\tt }\\[-8pt]
\mbox{\tt top\ ::\ Stack\ a\ ->\ a}\\
\mbox{\tt top\ (MkStack\ x\ s)\ =\ x\ \ \ \ \ \ \ \ \ \ \ \ \ \ \ \ \ \ \ \ \ --\ (top\ Empty)\ is\ an\ error}
}
\ecaption{A sample program}
\label{layout-before}
\outline{
\mbox{\tt module\ AStack(\ Stack,\ push,\ pop,\ top,\ size\ )\ where}\\
\mbox{\tt {\char'173}data\ Stack\ a\ =\ Empty\ }\\
\mbox{\tt \ \ \ \ \ \ \ \ \ \ \ \ \ |\ MkStack\ a\ (Stack\ a)}\\
\mbox{\tt }\\[-8pt]
\mbox{\tt ;push\ ::\ a\ ->\ Stack\ a\ ->\ Stack\ a}\\
\mbox{\tt ;push\ x\ s\ =\ MkStack\ x\ s}\\
\mbox{\tt }\\[-8pt]
\mbox{\tt ;size\ ::\ Stack\ a\ ->\ Integer}\\
\mbox{\tt ;size\ s\ =\ length\ (stkToLst\ s)\ \ where}\\
\mbox{\tt \ \ \ \ \ \ \ \ \ \ \ {\char'173}stkToLst\ \ Empty\ \ \ \ \ \ \ \ \ =\ []}\\
\mbox{\tt \ \ \ \ \ \ \ \ \ \ \ ;stkToLst\ (MkStack\ x\ s)\ \ =\ x:xs\ where\ {\char'173}xs\ =\ stkToLst\ s}\\
\mbox{\tt }\\[-8pt]
\mbox{\tt {\char'175}{\char'175};pop\ ::\ Stack\ a\ ->\ (a,\ Stack\ a)}\\
\mbox{\tt ;pop\ (MkStack\ x\ s)}\\
\mbox{\tt \ \ =\ (x,\ case\ s\ of\ {\char'173}r\ ->\ i\ r\ where\ {\char'173}i\ x\ =\ x{\char'175}{\char'175})\ --\ (pop\ Empty)\ is\ an\ error}\\
\mbox{\tt }\\[-8pt]
\mbox{\tt ;top\ ::\ Stack\ a\ ->\ a}\\
\mbox{\tt ;top\ (MkStack\ x\ s)\ =\ x\ \ \ \ \ \ \ \ \ \ \ \ \ \ \ \ \ \ \ \ \ \ \ \ --\ (top\ Empty)\ is\ an\ error}\\
\mbox{\tt {\char'175}}
}
\ecaption{Sample program with layout expanded}
\label{layout-after}

\end{figure}

When comparing indentations for standard \Haskell{} programs, a
fixed-width font with this tab convention is assumed: tab
stops are 8 characters apart (with the first tab stop in column 9),
and a tab character causes the insertion of enough spaces (always
$\geq 1$) to align the current position with the next tab stop.
Particular implementations may alter this rule to accommodate
variable-width fonts and alternate tab conventions, but standard
\Haskell{} (i.e., portable) programs must observe this rule.

% Local Variables: 
% mode: latex
% End:
\startnewsection
%
% $Header$
%
\section{Lexical Structure}\index{lexical structure}
\label{lexical-structure}

% this quote is not in because: it is supposed to show Curry as an
% early advocate of abstract syntax.  However, as far as I know (which
% I can't be sure w/out going to the library), this quote is from 1972
% (when Vol II was published ?), which isn't particularly early.  This
% doesn't seem the sort of thing to get wrong!  If someone wants to do
% the checking and put it in (and re-BibTeX and re-check the page
% breaks :-), please feel free.

%% \begin{quote}
%% ``{\em The various systems of combinatory logic are presented in this
%% book as abstract obs systems.  It was mentioned in \S{}~1 that any
%% such system can be represented as a concrete syntactical system, and
%% that this can be done mechanically.  Nevertheless it seems expedient
%% to give such a representation explicitly for two reasons: in the first
%% place there is at present an almost universal insistence on such a
%% representation; in the second place the discussion of certain
%% philosophical questions is made somewhat easier by having a specific
%% representation before us.}''
%% \begin{flushright}
%% Haskell B.~Curry {\em et al.}\\
%% in {\em Combinatory Logic}, Vol.~II, page 8 \cite{curry-etal:volII}.
%% \end{flushright}
%% \end{quote}

\noindent
In this section, 
we describe the low-level lexical structure of \Haskell{}.
Most of the details may be skipped in a first reading of
the report.

\subsection{Notational Conventions}
\label{notational-conventions}

These notational conventions are used for presenting syntax:

\[\ba{cl}
\mbox{$\it [pattern]$}             & \tr{optional} \\
\mbox{$\it \{pattern\}$}           & \tr{zero or more repetitions} \\
\mbox{$\it (pattern)$}             & \tr{grouping} \\
\mbox{$\it pat_1\ |\ pat_2$}         & \tr{choice} \\
\mbox{$\it pat_{\{pat'\}}$}        & \tr{difference---elements generated by \mbox{$\it pat$}} \\
                        & \tr{except those generated by \mbox{$\it pat'$}} \\
\mbox{$\it \makebox{\tt fibonacci}$}           & \tr{terminal syntax in typewriter font}
\ea\]

Because the syntax in this section describes {\em lexical} syntax, all
whitespace is expressed explicitly; there is no
implicit space between juxtaposed symbols.  BNF-like syntax is used
throughout, with productions having the form:
\begin{flushleft}\it\begin{tabbing}
\hspace{0.5in}\=\hspace{3.0in}\=\kill
$\it nonterm$\>\makebox[3.5em]{$\rightarrow$}$\it alt_1\ |\ alt_2\ |\ \ldots \ |\ alt_n$
\end{tabbing}\end{flushleft}

There are some families of nonterminals indexed by
precedence levels (written as a superscript).  Similarly, the
lexeme classes \mbox{$\it op$}, \mbox{$\it varop$}, and \mbox{$\it conop$} have a double index:  a letter \mbox{$\it l$},
\mbox{$\it r$}, or \mbox{$\it n$} for left-, right- or nonassociativity and a precedence
level.  So, for example
\begin{flushleft}\it\begin{tabbing}
\hspace{0.5in}\=\hspace{3.0in}\=\kill
$\it exp^i$\>\makebox[3.5em]{$\rightarrow$}$\it exp^{i+1}\ [op^{({\rm\ n},i)}\ exp^{i+1}]$\>\makebox[3em]{}$\it (0\leq i\leq 9)$
\end{tabbing}\end{flushleft}
actually stands for 10~productions where \mbox{$\it op$} is non-associative.
Refer to Section~\ref{fixity} for information on fixity declarations.

Care must be taken in distinguishing metalogical syntax such as \mbox{$\it |$}
and \mbox{$\it [\ldots ]$} from concrete terminal syntax (given in typewriter font)
such as \mbox{\tt |} and \mbox{\tt [...]}, although usually the context makes the
distinction clear.

\Haskell{} source programs are currently biased toward the ASCII
\index{ASCII character set}
character set, although future \Haskell{} standardisation efforts will
likely address broader character standards.

\subsection{Lexical Program Structure}
\label{lexemes}
\label{whitespace}

\begin{flushleft}\it\begin{tabbing}
\hspace{0.5in}\=\hspace{3.0in}\=\kill
$\it program$\>\makebox[3.5em]{$\rightarrow$}$\it \{\ lexeme\ |\ whitespace\ \}$\\ 
$\it lexeme$\>\makebox[3.5em]{$\rightarrow$}$\it varid\ |\ conid\ |\ varop\ |\ conop\ |\ literal\ |\ special\ |\ reservedop\ |\ reservedid$\\ 
$\it literal$\>\makebox[3.5em]{$\rightarrow$}$\it integer\ |\ float\ |\ char\ |\ string$\\ 
$\it special$\>\makebox[3.5em]{$\rightarrow$}$\it \makebox{\tt (}\ |\ \makebox{\tt )}\ |\ \makebox{\tt ,}\ |\ \makebox{\tt ;}\ |\ \makebox{\tt [}\ |\ \makebox{\tt ]}\ |\ \makebox{\tt {\char'137}}\ |\ \makebox{\tt {\char'173}}\ |\ \makebox{\tt {\char'175}}$\\ 
$\it $\\ 
$\it whitespace$\>\makebox[3.5em]{$\rightarrow$}$\it whitestuff\ \{whitestuff\}$\\ 
$\it whitestuff$\>\makebox[3.5em]{$\rightarrow$}$\it whitechar\ |\ comment\ |\ ncomment$\\ 
$\it whitechar$\>\makebox[3.5em]{$\rightarrow$}$\it newline\ |\ space\ |\ tab\ |\ vertab\ |\ formfeed$\\ 
$\it newline$\>\makebox[3.5em]{$\rightarrow$}$\it \tr{a\ newline\ (system\ dependent)}$\\ 
$\it space$\>\makebox[3.5em]{$\rightarrow$}$\it \tr{a\ space}$\\ 
$\it tab$\>\makebox[3.5em]{$\rightarrow$}$\it \tr{a\ horizontal\ tab}$\\ 
$\it vertab$\>\makebox[3.5em]{$\rightarrow$}$\it \tr{a\ vertical\ tab}$\\ 
$\it formfeed$\>\makebox[3.5em]{$\rightarrow$}$\it \tr{a\ form\ feed}$\\ 
$\it comment$\>\makebox[3.5em]{$\rightarrow$}$\it \makebox{\tt --}\ \{any\}\ newline$\\ 
$\it ncomment$\>\makebox[3.5em]{$\rightarrow$}$\it \makebox{\tt {\char'173}-}\ ANYseq\ \{ncomment\ ANYseq\}\ \makebox{\tt -{\char'175}}$\\ 
$\it ANYseq$\>\makebox[3.5em]{$\rightarrow$}$\it \{ANY\}_{\{ANY\}\ (\ \makebox{\tt {\char'173}-}\ |\ \makebox{\tt -{\char'175}}\ )\ \{ANY\}}$\\ 
$\it ANY$\>\makebox[3.5em]{$\rightarrow$}$\it any\ |\ newline\ |\ vertab\ |\ formfeed$\\ 
$\it any$\>\makebox[3.5em]{$\rightarrow$}$\it graphic\ |\ space\ |\ tab$\\ 
$\it graphic$\>\makebox[3.5em]{$\rightarrow$}$\it large\ |\ small\ |\ digit$\\ 
$\it $\>\makebox[3.5em]{$|$}$\it \makebox{\tt !}\ |\ \makebox{\tt "}\ |\ \makebox{\tt {\char'43}}\ |\ \makebox{\tt {\char'44}}\ |\ \makebox{\tt {\char'45}}\ |\ \makebox{\tt {\char'46}}\ |\ \fwq\ |\ \makebox{\tt (}\ |\ \makebox{\tt )}\ |\ \makebox{\tt *}\ |\ \makebox{\tt +}$\\ 
$\it $\>\makebox[3.5em]{$|$}$\it \makebox{\tt ,}\ |\ \makebox{\tt -}\ |\ \makebox{\tt .}\ |\ \makebox{\tt /}\ |\ \makebox{\tt :}\ |\ \makebox{\tt ;}\ |\ \makebox{\tt <}\ |\ \makebox{\tt =}\ |\ \makebox{\tt >}\ |\ \makebox{\tt ?}\ |\ @$\\ 
$\it $\>\makebox[3.5em]{$|$}$\it \makebox{\tt [}\ |\ \makebox{\tt {\char'134}}\ |\ \makebox{\tt ]}\ |\ \makebox{\tt {\char'136}}\ |\ \makebox{\tt {\char'137}}\ |\ \bkq\ |\ \makebox{\tt {\char'173}}\ |\ \makebox{\tt |}\ |\ \makebox{\tt {\char'175}}\ |\ \makebox{\tt {\char'176}}$\\ 
$\it $\\ 
$\it small$\>\makebox[3.5em]{$\rightarrow$}$\it \makebox{\tt a}\ |\ \makebox{\tt b}\ |\ \ldots \ |\ \makebox{\tt z}$\\ 
$\it large$\>\makebox[3.5em]{$\rightarrow$}$\it \makebox{\tt A}\ |\ \makebox{\tt B}\ |\ \ldots \ |\ \makebox{\tt Z}$\\ 
$\it digit$\>\makebox[3.5em]{$\rightarrow$}$\it \makebox{\tt 0}\ |\ \makebox{\tt 1}\ |\ \ldots \ |\ \makebox{\tt 9}$
\end{tabbing}\end{flushleft}
\indexsyn{program}%
\indexsyn{lexeme}%
\indexsyn{literal}%
\indexsyn{special}%
\indexsyn{whitespace}%
\indexsyn{whitestuff}%
\indexsyn{whitechar}%
\indexsyn{newline}%
\indexsyn{space}%
\indexsyn{tab}%
\indexsyn{vertab}%
\indexsyn{formfeed}%
\indexsyn{comment}%
\indexsyn{ncomment}%
\indexsyn{ANYseq}%
\indexsyn{ANY}%
\indexsyn{any}%
\indexsyn{graphic}%
\indexsyn{small}%
\indexsyn{large}%
\indexsyn{digit}%

Characters not in the category \mbox{$\it graphic$} or \mbox{$\it whitestuff$} are not valid
in \Haskell{} programs and should result in a lexing error.

Comments\index{comment} are valid \mbox{$\it whitespace$}.  An ordinary
comment\index{comment!end-of-line} begins with two consecutive
dashes (\mbox{\tt --}) and extends to the following newline.
A nested
comment\index{comment!nested} begins with \mbox{\tt {\char'173}-} and ends with
\mbox{\tt -{\char'175}}; it can be between any two lexemes.
All character sequences not containing \mbox{\tt {\char'173}-} nor \mbox{\tt -{\char'175}}
are ignored within a nested comment.
Nested comments may be
nested to any depth: any occurrence of \mbox{\tt {\char'173}-} within the nested
comment starts a new nested comment, terminated by \mbox{\tt -{\char'175}}.  Within
a nested comment, each \mbox{\tt {\char'173}-} is matched by a corresponding
occurrence of \mbox{\tt -{\char'175}}.  In an ordinary comment, the character
sequences \mbox{\tt {\char'173}-} and \mbox{\tt -{\char'175}} have no special significance, and, in a
nested comment, the sequence \mbox{\tt --} has no special significance.

If some code is commented out using a nested comment, then any
occurrence of \mbox{\tt {\char'173}-} or \mbox{\tt -{\char'175}} within a string or within an end-of-line
comment in that code will interfere with the nesting of the nested
comments.

\subsection{Identifiers and Operators}\index{identifier}\index{operator}
\label{ids}

\begin{flushleft}\it\begin{tabbing}
\hspace{0.5in}\=\hspace{3.0in}\=\kill
$\it avarid$\>\makebox[3.5em]{$\rightarrow$}$\it (small\ \{small\ |\ large\ |\ digit\ |\ \fwq\ |\ \makebox{\tt {\char'137}}\})_{\{reservedid\}}$\\ 
$\it varid$\>\makebox[3.5em]{$\rightarrow$}$\it avarid\ |\ \makebox{\tt (}avarop\makebox{\tt )}$\\ 
$\it aconid$\>\makebox[3.5em]{$\rightarrow$}$\it large\ \{small\ |\ large\ |\ digit\ |\ \fwq\ |\ \makebox{\tt {\char'137}}\}$\\ 
$\it conid$\>\makebox[3.5em]{$\rightarrow$}$\it aconid\ |\ \makebox{\tt (}aconop\makebox{\tt )}$\\ 
$\it reservedid$\>\makebox[3.5em]{$\rightarrow$}$\it \makebox{\tt case}\ |\ \makebox{\tt class}\ |\ \makebox{\tt data}\ |\ \makebox{\tt default}\ |\ \makebox{\tt deriving}\ |\ \makebox{\tt else}\ |\ \makebox{\tt hiding}$\\ 
$\it $\>\makebox[3.5em]{$|$}$\it \makebox{\tt if}\ |\ \makebox{\tt import}\ |\ \makebox{\tt in}\ |\ \makebox{\tt infix}\ |\ \makebox{\tt infixl}\ |\ \makebox{\tt infixr}\ |\ \makebox{\tt instance}\ |\ \makebox{\tt interface}$\\ 
$\it $\>\makebox[3.5em]{$|$}$\it \makebox{\tt let}\ |\ \makebox{\tt module}\ |\ \makebox{\tt of}\ |\ \makebox{\tt renaming}\ |\ \makebox{\tt then}\ |\ \makebox{\tt to}\ |\ \makebox{\tt type}\ |\ \makebox{\tt where}$
\end{tabbing}\end{flushleft}
\indexsyn{avarid}%
\indexsyn{varid}%
\indexsyn{aconid}%
\indexsyn{conid}%
\indexsyn{reservedid}%
An identifier consists of a letter followed by zero or more letters,
digits, underscores, and acute accents.  Identifiers are lexically
distinguished into two classes: those that begin with a small letter
(variable identifiers) and those that begin with a capital
(constructor identifiers).  Identifiers are case sensitive: \mbox{\tt name},
\mbox{\tt naMe}, and \mbox{\tt Name} are three distinct identifiers (the first two are
variable identifiers, the last is a constructor identifier).

\begin{flushleft}\it\begin{tabbing}
\hspace{0.5in}\=\hspace{3.0in}\=\kill
$\it avarop$\>\makebox[3.5em]{$\rightarrow$}$\it (\ (\ symbol\ |\ presymbol\ )\ \{symbol\ |\ \makebox{\tt :}\}\ )_{\{reservedop\}}$\\ 
$\it varop$\>\makebox[3.5em]{$\rightarrow$}$\it avarop\ |\ \bkqB{avarid}\bkqA$\\ 
$\it aconop$\>\makebox[3.5em]{$\rightarrow$}$\it (\makebox{\tt :}\ \{symbol\ |\ \makebox{\tt :}\})_{\{reservedop\}}$\\ 
$\it conop$\>\makebox[3.5em]{$\rightarrow$}$\it aconop\ |\ \bkqB{aconid}\bkqA$\\ 
$\it presymbol$\>\makebox[3.5em]{$\rightarrow$}$\it \makebox{\tt -}\ |\ \makebox{\tt {\char'176}}$\\ 
$\it symbol$\>\makebox[3.5em]{$\rightarrow$}$\it \makebox{\tt !}\ |\ \makebox{\tt {\char'43}}\ |\ \makebox{\tt {\char'44}}\ |\ \makebox{\tt {\char'45}}\ |\ \makebox{\tt {\char'46}}\ |\ \makebox{\tt *}\ |\ \makebox{\tt +}\ |\ \makebox{\tt .}\ |\ \makebox{\tt /}\ |\ \makebox{\tt <}\ |\ \makebox{\tt =}\ |\ \makebox{\tt >}\ |\ \makebox{\tt ?}\ |\ @\ |\ \makebox{\tt {\char'134}}\ |\ \makebox{\tt {\char'136}}\ |\ \makebox{\tt |}$\\ 
$\it reservedop$\>\makebox[3.5em]{$\rightarrow$}$\it \makebox{\tt ..}\ |\ \makebox{\tt ::}\ |\ \makebox{\tt =>}\ |\ \makebox{\tt =}\ |\ @\ |\ \makebox{\tt {\char'134}}\ |\ \makebox{\tt |}\ |\ \makebox{\tt {\char'176}}\ |\ \makebox{\tt <-}\ |\ \makebox{\tt ->}$
\end{tabbing}\end{flushleft}
\indexsyn{avarop}%
\indexsyn{varop}%
\indexsyn{aconop}%
\indexsyn{conop}%
\indexsyn{presymbol}%
\indexsyn{symbol}%
\indexsyn{reservedop}%
An operator is either symbolic or alphanumeric.  Symbolic operators
are formed from one or more symbols, as defined above, and are
lexically distinguished into two classes: those that start with a
colon (constructors) and those that do not (functions).

Alphanumeric operators are formed by enclosing an identifier between
grave accents (backquotes).  Any variable or constructor may be used as
an operator in this way.  If \mbox{$\it fun$} is an identifier (either variable
or constructor), then an expression of the form \mbox{$\it fun\ x\ y$} is
equivalent to \mbox{$\it x\ \bkqB{fun}\bkqA\ y$}.  If no fixity\index{fixity}
declaration is given for \mbox{$\it \bkqB{fun}\bkqA$} then it defaults
to highest precedence and left associativity
(see Section~\ref{fixity}).

Similarly, any symbolic operator may be used as a (curried) variable
or constructor by enclosing it in parentheses.  If \mbox{$\it op$} is an infix
operator, then an expression or pattern of the form \mbox{\mbox{$\it x\ op\ y$}} is
equivalent to {\mbox{$\it \makebox{\tt (}op\makebox{\tt )}\ x\ y$}}.

No white space is permitted in names such as \mbox{$\it \bkqB{fun}\bkqA$} and
\mbox{$\it \makebox{\tt (}op\makebox{\tt )}$}.

Other than the special syntax for prefix negation, all operators are
infix, although each infix operator can be used in a {\em
section}\index{section} to yield partially applied operators (see
Section~\ref{sections}).
All of the standard infix operators are just
predefined symbols and may be rebound.

Although \mbox{\tt case} is a reserved word, \mbox{\tt cases} is not.
Similarly, although \mbox{\tt =} is reserved, \mbox{\tt ==} and \mbox{\tt {\char'176}=} are
not.  At each point, the longest possible lexeme
\index{longest lexeme rule}
is read, using a context-independent deterministic lexical analysis
(i.e.~no lookahead beyond the current character is required).
Any kind of \mbox{$\it whitespace$} is also a proper delimiter for lexemes.

In the remainder of the report six different kinds of 
names\index{namespaces} will be used:
\begin{flushleft}\it\begin{tabbing}
\hspace{0.5in}\=\hspace{3.0in}\=\kill
$\it var$\>\makebox[3.5em]{$\rightarrow$}$\it varid$\>\makebox[3em]{}$\it (variables)$\\ 
$\it con$\>\makebox[3.5em]{$\rightarrow$}$\it conid$\>\makebox[3em]{}$\it (constructors)$\\ 
$\it tyvar$\>\makebox[3.5em]{$\rightarrow$}$\it avarid$\>\makebox[3em]{}$\it (type\ variables)$\\ 
$\it tycon$\>\makebox[3.5em]{$\rightarrow$}$\it aconid$\>\makebox[3em]{}$\it (type\ constructors)$\\ 
$\it tycls$\>\makebox[3.5em]{$\rightarrow$}$\it aconid$\>\makebox[3em]{}$\it (type\ classes)$\\ 
$\it modid$\>\makebox[3.5em]{$\rightarrow$}$\it aconid$\>\makebox[3em]{}$\it (modules)$
\end{tabbing}\end{flushleft}
\indexsyn{var}%
\indexsyn{con}%
\indexsyn{tyvar}%
\indexsyn{tycon}%
\indexsyn{tycls}%
\indexsyn{modid}%
Variables and type variables are represented by identifiers beginning
with small letters, and the other four by identifiers beginning with
capitals; also, variables and constructors have infix forms, the other
four do not.  Namespaces are also discussed in
Section~\ref{namespaces}.

\subsection{Numeric Literals}\index{number!literal syntax}
\label{lexemes-numeric}

\begin{flushleft}\it\begin{tabbing}
\hspace{0.5in}\=\hspace{3.0in}\=\kill
$\it integer$\>\makebox[3.5em]{$\rightarrow$}$\it digit\{digit\}$\\ 
$\it float$\>\makebox[3.5em]{$\rightarrow$}$\it integer\makebox{\tt .}integer[(\makebox{\tt e}\ |\ \makebox{\tt E})[\makebox{\tt -}\ |\ \makebox{\tt +}]integer]$
\end{tabbing}\end{flushleft}
\indexsyn{integer}%
\indexsyn{float}%
There are two distinct kinds of numeric literals: integer and
floating.  A floating literal must contain digits both before and
after the decimal point; this ensures that a decimal point cannot be
mistaken for another use of the dot character.  Negative numeric
literals are discussed in Section~\ref{operators}.  The typing of
numeric literals is discussed in Section~\ref{numeric-literals}.

\subsection{Character and String Literals}
\index{character!literal syntax}
\index{string!literal syntax}
\label{lexemes-char}

\begin{flushleft}\it\begin{tabbing}
\hspace{0.5in}\=\hspace{3.0in}\=\kill
$\it char$\>\makebox[3.5em]{$\rightarrow$}$\it \fwq\ (graphic_{\{\fwq\ |\ \makebox{\tt {\char'134}}\}}\ |\ space\ |\ escape_{\{\makebox{\tt {\char'134}{\char'46}}\}})\ \fwq$\\ 
$\it string$\>\makebox[3.5em]{$\rightarrow$}$\it \makebox{\tt "}\ \{graphic_{\{\makebox{\tt "}\ |\ \makebox{\tt {\char'134}}\}}\ |\ space\ |\ escape\ |\ gap\}\ \makebox{\tt "}$\\ 
$\it escape$\>\makebox[3.5em]{$\rightarrow$}$\it \makebox{\tt {\char'134}}\ (\ charesc\ |\ ascii\ |\ integer\ |\ \makebox{\tt o}\ octit\{octit\}\ |\ \makebox{\tt x}\ hexit\{hexit\}\ )$\\ 
$\it charesc$\>\makebox[3.5em]{$\rightarrow$}$\it \makebox{\tt a}\ |\ \makebox{\tt b}\ |\ \makebox{\tt f}\ |\ \makebox{\tt n}\ |\ \makebox{\tt r}\ |\ \makebox{\tt t}\ |\ \makebox{\tt v}\ |\ \makebox{\tt {\char'134}}\ |\ \makebox{\tt "}\ |\ \fwq\ |\ \makebox{\tt {\char'46}}$\\ 
$\it ascii$\>\makebox[3.5em]{$\rightarrow$}$\it \makebox{\tt {\char'136}}cntrl\ |\ \makebox{\tt NUL}\ |\ \makebox{\tt SOH}\ |\ \makebox{\tt STX}\ |\ \makebox{\tt ETX}\ |\ \makebox{\tt EOT}\ |\ \makebox{\tt ENQ}\ |\ \makebox{\tt ACK}$\\ 
$\it $\>\makebox[3.5em]{$|$}$\it \makebox{\tt BEL}\ |\ \makebox{\tt BS}\ |\ \makebox{\tt HT}\ |\ \makebox{\tt LF}\ |\ \makebox{\tt VT}\ |\ \makebox{\tt FF}\ |\ \makebox{\tt CR}\ |\ \makebox{\tt SO}\ |\ \makebox{\tt SI}\ |\ \makebox{\tt DLE}$\\ 
$\it $\>\makebox[3.5em]{$|$}$\it \makebox{\tt DC1}\ |\ \makebox{\tt DC2}\ |\ \makebox{\tt DC3}\ |\ \makebox{\tt DC4}\ |\ \makebox{\tt NAK}\ |\ \makebox{\tt SYN}\ |\ \makebox{\tt ETB}\ |\ \makebox{\tt CAN}$\\ 
$\it $\>\makebox[3.5em]{$|$}$\it \makebox{\tt EM}\ |\ \makebox{\tt SUB}\ |\ \makebox{\tt ESC}\ |\ \makebox{\tt FS}\ |\ \makebox{\tt GS}\ |\ \makebox{\tt RS}\ |\ \makebox{\tt US}\ |\ \makebox{\tt SP}\ |\ \makebox{\tt DEL}$\\ 
$\it cntrl$\>\makebox[3.5em]{$\rightarrow$}$\it large\ |\ @\ |\ \makebox{\tt [}\ |\ \makebox{\tt {\char'134}}\ |\ \makebox{\tt ]}\ |\ \makebox{\tt {\char'136}}\ |\ \makebox{\tt {\char'137}}$\\ 
$\it gap$\>\makebox[3.5em]{$\rightarrow$}$\it \makebox{\tt {\char'134}}\ whitechar\ \{whitechar\}\ \makebox{\tt {\char'134}}$\\ 
$\it hexit$\>\makebox[3.5em]{$\rightarrow$}$\it digit\ |\ \makebox{\tt A}\ |\ \makebox{\tt B}\ |\ \makebox{\tt C}\ |\ \makebox{\tt D}\ |\ \makebox{\tt E}\ |\ \makebox{\tt F}\ |\ \makebox{\tt a}\ |\ \makebox{\tt b}\ |\ \makebox{\tt c}\ |\ \makebox{\tt d}\ |\ \makebox{\tt e}\ |\ \makebox{\tt f}$\\ 
$\it octit$\>\makebox[3.5em]{$\rightarrow$}$\it \makebox{\tt 0}\ |\ \makebox{\tt 1}\ |\ \makebox{\tt 2}\ |\ \makebox{\tt 3}\ |\ \makebox{\tt 4}\ |\ \makebox{\tt 5}\ |\ \makebox{\tt 6}\ |\ \makebox{\tt 7}$
\end{tabbing}\end{flushleft}
\indexsyn{char}%
\indexsyn{string}%
\indexsyn{escape}%
\indexsyn{charesc}%
\indexsyn{ascii}%
\indexsyn{cntrl}%
\indexsyn{gap}%
\indexsyn{hexit}%
\indexsyn{octit}%

Character literals are written between acute accents, as in
\fwq\mbox{\tt a}\fwq, and strings between double quotes, as in \mbox{\tt "Hello"}.

Escape codes may be used in characters and strings to represent
special characters.  Note that \fwq\ may be used in a string, but
must be escaped in a character; similarly, \mbox{\tt "} may be used in a
character, but must be escaped in a string.  \mbox{\tt {\char'134}} must always be
escaped.  The category \mbox{$\it charesc$} also includes portable
representations for the characters ``alert'' (\mbox{\tt {\char'134}a}), ``backspace''
(\mbox{\tt {\char'134}b}), ``form feed'' (\mbox{\tt {\char'134}f}), ``new line'' (\mbox{\tt {\char'134}n}), ``carriage return''
(\mbox{\tt {\char'134}r}), ``horizontal tab'' (\mbox{\tt {\char'134}t}), and ``vertical tab'' (\mbox{\tt {\char'134}v}).

Escape characters for the ASCII\index{ASCII character set} character
set, including
control characters such as \mbox{\tt {\char'134}{\char'136}X}, are also provided.
Numeric escapes such as \mbox{\tt {\char'134}137} are used to designate the character
with (implementation dependent) decimal representation 137; octal
(e.g.~\mbox{\tt {\char'134}o137}) and hexadecimal (e.g.~\mbox{\tt {\char'134}x137}) representations are also
allowed.  Numeric escapes that are out-of-range of the ASCII standard
are undefined and thus non-portable.

Consistent with the ``consume longest lexeme'' rule,
\index{longest lexeme rule}
numeric escape
characters in strings consist of all consecutive digits and may
be of arbitrary length.  Similarly, the one ambiguous ASCII escape
code, \mbox{\tt "{\char'134}SOH"}, is parsed as a string of length 1.  The escape
character \mbox{\tt {\char'134}{\char'46}} is provided as a ``null character'' to allow strings
such as \mbox{\tt "{\char'134}137{\char'134}{\char'46}9"} and \mbox{\tt "{\char'134}SO{\char'134}{\char'46}H"} to be constructed (both of length
two).  Thus \mbox{\tt "{\char'134}{\char'46}"} is equivalent to \mbox{\tt ""} and the character
\fwq\mbox{\tt {\char'134}{\char'46}}\fwq\ is disallowed.  Further equivalences of characters
are defined in Section~\ref{characters}.

A string may include a ``gap''---two backslants enclosing
white characters---which is ignored.
This allows one to write long strings on more than one line by writing
a backslant at the end of one line and at the start of the next.  For
example,
\bprog
\mbox{\tt "Here\ is\ a\ backslant\ {\char'134}{\char'134}\ as\ well\ as\ {\char'134}137,\ {\char'134}}\\
\mbox{\tt \ \ \ \ {\char'134}a\ numeric\ escape\ character,\ and\ {\char'134}{\char'136}X,\ a\ control\ character."}
\eprogNoSkip

String literals are actually abbreviations for lists of characters
(see Section~\ref{lists}).

% Local Variables: 
% mode: latex
% End:
\startnewsection
%
% $Header$
%
\section{Expressions}\index{expression}
\label{expressions}

In this section, we describe the syntax and informal semantics of \Haskell{}
{\em expressions}, including their translations into the
\Haskell{} kernel, where appropriate.
% (see Appendix~\ref{formal-semantics} for a formal syntax and
% semantics of the kernel).

\begin{flushleft}\it\begin{tabbing}
\hspace{0.5in}\=\hspace{3.0in}\=\kill
$\it exp$\>\makebox[3.5em]{$\rightarrow$}$\it \makebox{\tt {\char'134}}\ apat_1\ \ldots \ apat_n\ \makebox{\tt ->}\ exp$\>\makebox[3em]{}$\it (\tr{lambda\ abstraction},\ n\geq 1)$\\ 
$\it $\>\makebox[3.5em]{$|$}$\it \makebox{\tt let}\ \makebox{\tt {\char'173}}\ decls\ [\makebox{\tt ;}]\ \makebox{\tt {\char'175}}\ \makebox{\tt in}\ exp$\>\makebox[3em]{}$\it ({\tr{let\ expression}})$\\ 
$\it $\>\makebox[3.5em]{$|$}$\it \makebox{\tt if}\ exp\ \makebox{\tt then}\ exp\ \makebox{\tt else}\ exp$\>\makebox[3em]{}$\it (\tr{conditional})$\\ 
$\it $\>\makebox[3.5em]{$|$}$\it \makebox{\tt case}\ exp\ \makebox{\tt of}\ \makebox{\tt {\char'173}}\ alts\ [\makebox{\tt ;}]\ \makebox{\tt {\char'175}}$\>\makebox[3em]{}$\it (\tr{case\ expression})$\\ 
$\it $\>\makebox[3.5em]{$|$}$\it exp^0\ \makebox{\tt ::}\ [context\ \makebox{\tt =>}]\ atype$\>\makebox[3em]{}$\it (\tr{expression\ type\ signature})$\\ 
$\it $\>\makebox[3.5em]{$|$}$\it exp^0$\\ 
$\it exp^i$\>\makebox[3.5em]{$\rightarrow$}$\it exp^{i+1}\ [op^{({\rm\ n},i)}\ exp^{i+1}]$\>\makebox[3em]{}$\it (0\leq i\leq 9)$\\ 
$\it $\>\makebox[3.5em]{$|$}$\it lexp^i\ op^{({\rm\ l},i)}\ exp^{i+1}$\\ 
$\it $\>\makebox[3.5em]{$|$}$\it exp^{i+1}\ op^{({\rm\ r},i)}\ rexp^i$\\ 
$\it lexp^i$\>\makebox[3.5em]{$\rightarrow$}$\it [lexp^i\ op^{({\rm\ l},i)}]\ exp^{i+1}$\>\makebox[3em]{}$\it (0\leq i\leq 9)$\\ 
$\it lexp^6$\>\makebox[3.5em]{$\rightarrow$}$\it \makebox{\tt -}\ exp^7$\\ 
$\it rexp^i$\>\makebox[3.5em]{$\rightarrow$}$\it exp^{i+1}\ [op^{({\rm\ r},i)}\ rexp^i]$\>\makebox[3em]{}$\it (0\leq i\leq 9)$\\ 
$\it exp^{10}$\>\makebox[3.5em]{$\rightarrow$}$\it exp^{10}\ aexp$\>\makebox[3em]{}$\it (\tr{function\ application})$\\ 
$\it $\>\makebox[3.5em]{$|$}$\it aexp$\\ 
$\it $\\ 
$\it aexp$\>\makebox[3.5em]{$\rightarrow$}$\it var$\>\makebox[3em]{}$\it (\tr{variable})$\\ 
$\it $\>\makebox[3.5em]{$|$}$\it con$\>\makebox[3em]{}$\it (\tr{constructor})$\\ 
$\it $\>\makebox[3.5em]{$|$}$\it literal$\\ 
$\it $\>\makebox[3.5em]{$|$}$\it \makebox{\tt ()}$\>\makebox[3em]{}$\it (\tr{unit})$\\ 
$\it $\>\makebox[3.5em]{$|$}$\it \makebox{\tt (}\ exp\ \makebox{\tt )}$\>\makebox[3em]{}$\it (\tr{parenthesised\ expression})$\\ 
$\it $\>\makebox[3.5em]{$|$}$\it \makebox{\tt (}\ exp_1\ \makebox{\tt ,}\ \ldots \ \makebox{\tt ,}\ exp_k\ \makebox{\tt )}$\>\makebox[3em]{}$\it (\tr{tuple},\ k\geq 2)$\\ 
$\it $\>\makebox[3.5em]{$|$}$\it \makebox{\tt [}\ exp_1\ \makebox{\tt ,}\ \ldots \ \makebox{\tt ,}\ exp_k\ \makebox{\tt ]}$\>\makebox[3em]{}$\it (\tr{list},\ k\geq 0)$\\ 
$\it $\>\makebox[3.5em]{$|$}$\it \makebox{\tt [}\ exp_1\ [\makebox{\tt ,}\ exp_2]\ \makebox{\tt ..}\ [exp_3]\ \makebox{\tt ]}$\>\makebox[3em]{}$\it (\tr{arithmetic\ sequence})$\\ 
$\it $\>\makebox[3.5em]{$|$}$\it \makebox{\tt [}\ exp\ \makebox{\tt |}\ qual_1\ \makebox{\tt ,}\ \ldots \ \makebox{\tt ,}\ qual_n\ \makebox{\tt ]}$\>\makebox[3em]{}$\it (\tr{list\ comprehension},\ n\geq 1)$\\ 
$\it $\>\makebox[3.5em]{$|$}$\it \makebox{\tt (}\ exp^{i+1}\ op^{(a,i)}\ \makebox{\tt )}$\>\makebox[3em]{}$\it (0\ \leq i\leq \ 9,\ a\in\{l,r,n\})$\\ 
$\it $\>\makebox[3.5em]{$|$}$\it \makebox{\tt (}\ op^{(a,i)}\ exp^{i+1}\ \makebox{\tt )}$\>\makebox[3em]{}$\it (0\ \leq i\leq \ 9,\ a\in\{l,r,n\})$\\ 
$\it %\ |\ \makebox{\tt (}\ lexp^i\ op^{({\rm\ l},i)}\ \makebox{\tt )}$\>\makebox[3em]{}$\it (0\ \leq \ i\ \leq \ 9)$\\ 
$\it %\ |\ \makebox{\tt (}\ op^{({\rm\ r},i)}\ rexp^i\ \makebox{\tt )}$\>\makebox[3em]{}$\it (0\ \leq \ i\ \leq \ 9)$\\ 
$\it $\\ 
$\it op$\>\makebox[3.5em]{$\rightarrow$}$\it varop\ |\ conop$
\end{tabbing}\end{flushleft}
\indexsyn{exp}%
\index{exp@\mbox{$\it exp^i$}}%
\index{lexp@\mbox{$\it lexp^i$}}%
\index{rexp@\mbox{$\it rexp^i$}}%
\indexsyn{aexp}%
\indexsyn{op}%

The grammar above embodies the following precedence hierarchy for
expressions, using productions with superscripts as described in
Section~\ref{notational-conventions}:
\begin{itemize}
\item
Function application binds most tightly of all.
\item
Expressions involving infix operators are disambiguated by the
operator's fixity (see Section~\ref{fixity}).
Consecutive unparenthesised operators with the same
precedence must both be either left or right associative
to avoid a syntax error.
Given an unparenthesised expression ``\mbox{$\it x\ op^{(a,i)}\ y\ op^{(b,j)}\ z$}'', parentheses
must be added around either ``\mbox{$\it x\ op^{(a,i)}\ y$}'' or ``\mbox{$\it y\ op^{(b,j)}\\
\it z$}'' when \mbox{$\it i=j$} unless \mbox{$\it a=b={\rm\ l}$} or \mbox{$\it a=b={\rm\ r}$}.
\item
Negation\index{negation} is the only prefix operator in
\Haskell{}; it has the same precedence as the infix \mbox{\tt -} operator
defined in the standard prelude (see Figure~\ref{prelude-fixities}).
\item
Expression type signatures are parsed as if \mbox{\tt ::} were a
left-associative infix operator with precedence lower than any other
operator.
\end{itemize}
Sample parses using this grammar are shown below.
\[\bt{|l|l|}\hline
This                                & Parses as                             \\
\hline
\mbox{\tt f\ x\ +\ g\ y}                         & \mbox{\tt (f\ x)\ +\ (g\ y)}                       \\
\mbox{\tt -\ f\ x\ +\ y}                         & \mbox{\tt (-\ (f\ x))\ +\ y}                       \\
\mbox{\tt let\ {\char'173}\ ...\ {\char'175}\ in\ x\ +\ y}              & \mbox{\tt let\ {\char'173}\ ...\ {\char'175}\ in\ (x\ +\ y)}              \\
\mbox{\tt f\ x\ y\ ::\ Int}                      & \mbox{\tt (f\ x\ y)\ ::\ Int}                      \\
\mbox{\tt {\char'134}\ x\ ->\ a+b\ ::\ Int}                 & \mbox{\tt {\char'134}\ x\ ->\ ((a+b)\ ::\ Int})               \\
\hline\et\]

For the sake of clarity, the rest of this section shows the syntax of
expressions without their precedences.

\subsection{Curried Applications and Lambda Abstractions}
\label{applications}
\label{lambda-abstractions}
\index{lambda abstraction}
\index{\ pats -> expr@\mbox{$\it \makebox{\tt {\char'134}}\ pats\ \makebox{\tt ->}\ expr$}}
\index{application}
%\index{function application|see{application}}
%
\begin{flushleft}\it\begin{tabbing}
\hspace{0.5in}\=\hspace{3.0in}\=\kill
$\it exp$\>\makebox[3.5em]{$\rightarrow$}$\it exp\ aexp$\\ 
$\it exp$\>\makebox[3.5em]{$\rightarrow$}$\it \makebox{\tt {\char'134}}\ apat_1\ \ldots \ apat_n\ \makebox{\tt ->}\ exp$
\end{tabbing}\end{flushleft}
\indexsyn{exp}%
%
\noindent
{\em Function application}\index{application} is written 
\mbox{$\it e_1\ e_2$}.  Application associates to the left, so the
parentheses may be omitted in \mbox{\tt (f\ x)\ y}, for example.  Because \mbox{$\it e_1$} could
be a constructor, partial applications of constructors are allowed.

{\em Lambda abstractions} are written 
\mbox{$\it \makebox{\tt {\char'134}}\ p_1\ \ldots \ p_n\ \makebox{\tt ->}\ e$}, where the \mbox{$\it p_i$} are {\em patterns}.
An expression such as \mbox{\tt {\char'134}x:xs->x} is syntactically incorrect,
and must be rewritten as \mbox{\tt {\char'134}(x:xs)->x}.

The set of patterns must be {\em linear}\index{linearity}
\index{linear pattern}---no variable may appear more than once in the set.

\outline{
\paragraph*{Translation:}
The lambda abstraction \mbox{$\it \makebox{\tt {\char'134}}\ p_1\ \ldots \ p_n\ \makebox{\tt ->}\ e$} is equivalent to
\[
\mbox{$\it \makebox{\tt {\char'134}}\ x_1\ \ldots \ x_n\ \makebox{\tt ->\ case\ (}x_1\makebox{\tt ,}\ \ldots \makebox{\tt ,}\ x_n\makebox{\tt )\ of\ (}p_1\makebox{\tt ,}\ \ldots \makebox{\tt ,}\ p_n\makebox{\tt )\ ->}\ e$}
\]
where the \mbox{$\it x_i$} are new identifiers.
Given this translation combined with the semantics of case
expressions and pattern-matching described in
Section~\ref{case-semantics}, if the
pattern fails to match, then the result is \mbox{$\it \bot$}.
}
             
The type of a variable bound by a lambda abstraction is monomorphic,
\index{monomorphic type variable}
as is always the case in the Hindley-Milner type system.

\subsection{Operator Applications}
\index{operator application}
%\index{operator application|seealso{application}}
\label{operators}
%
\begin{flushleft}\it\begin{tabbing}
\hspace{0.5in}\=\hspace{3.0in}\=\kill
$\it exp$\>\makebox[3.5em]{$\rightarrow$}$\it exp_1\ op\ exp_2$\\ 
$\it $\>\makebox[3.5em]{$|$}$\it \makebox{\tt -}\ exp$\>\makebox[3em]{}$\it (prefix\ negation)$
\end{tabbing}\end{flushleft}
\indexsyn{exp}%
\noindent
The form \mbox{$\it e_1\ op\ e_2$} is the infix application of binary
operator\index{operator} \mbox{$\it op$} to expressions \mbox{$\it e_1$} and \mbox{$\it e_2$}.  

The special
form \mbox{$\it \makebox{\tt -}e$} denotes prefix negation\index{negation}, the one and only
prefix operator in \Haskell{}, and is simply
syntax for \mbox{$\it \makebox{\tt negate\ }(e)$},\indextt{negate} where \mbox{\tt negate} is as
defined in the standard prelude (see
Figure~\ref{basic-numeric-1},
page~\pageref{basic-numeric-1}).
Because \mbox{\tt e1-e2} parses as an
infix application of the binary operator \mbox{\tt -}, one must write \mbox{\tt e1(-e2)} for
the alternative parsing.  Similarly, \mbox{\tt (-)} is syntax for 
\mbox{\tt ({\char'134}\ x\ y\ ->\ x-y)}, as with any infix operator, and does not denote 
\mbox{\tt ({\char'134}\ x\ ->\ -x)}---one must use \mbox{\tt negate} for that.

\outline{
\paragraph*{Translation:}
\mbox{$\it e_1\ op\ e_2$} is equivalent to \mbox{$\it \makebox{\tt (}op\makebox{\tt )}\ e_1\ e_2$}.  \mbox{$\it \makebox{\tt -}e$} is
equivalent to \mbox{$\it \makebox{\tt \ negate}\ (e)$} where \mbox{\tt negate}, an operator in the class
\mbox{\tt Num}, is as defined in the standard prelude.
}

\subsection{Sections}
\index{section}
%\index{section|seealso{operator application}}
\label{sections}
%
\begin{flushleft}\it\begin{tabbing}
\hspace{0.5in}\=\hspace{3.0in}\=\kill
$\it aexp$\>\makebox[3.5em]{$\rightarrow$}$\it \makebox{\tt (}\ exp\ op\ \makebox{\tt )}$\\ 
$\it $\>\makebox[3.5em]{$|$}$\it \makebox{\tt (}\ op\ exp\ \makebox{\tt )}$
\end{tabbing}\end{flushleft}
\indexsyn{aexp}%
\noindent
{\em Sections} are written as \mbox{$\it \makebox{\tt (}\ op\ e\ \makebox{\tt )}$} or \mbox{$\it \makebox{\tt (}\ e\ op\ \makebox{\tt )}$}, where
\mbox{$\it op$} is a binary operator and \mbox{$\it e$} is an expression.  Sections are a
convenient syntax for partial application of binary operators.

The normal rules of syntactic precedence apply to sections; for
example, \mbox{\tt (*a+b)} is syntactically invalid, but \mbox{\tt (+a*b)} and
\mbox{\tt (*(a+b))} are valid.  Syntactic associativity, however, is not
taken into account in sections; thus, \mbox{\tt (a+b+)} must be written
\mbox{\tt ((a+b)+)}.

Because \mbox{\tt -} is treated specially in the grammar,
\mbox{$\it \makebox{\tt (-}\ exp\makebox{\tt )}$} is not a section, but an application of prefix
negation,\index{negation} as
described in the preceding section.  However, there is a \mbox{\tt subtract}
function defined in the standard prelude such that
\mbox{$\it \makebox{\tt (subtract}\ exp\makebox{\tt )}$} is equivalent to the disallowed section.
The expression \mbox{$\it \makebox{\tt (+\ (-}\ exp\makebox{\tt ))}$} can serve the same purpose.

\outline{
\paragraph*{Translation:}
For binary operator \mbox{$\it op$} and expression \mbox{$\it e$}, if \mbox{$\it x$} is a variable
that does not occur free in \mbox{$\it e$}, the section
\mbox{$\it \makebox{\tt (}op\ e\makebox{\tt )}$} is equivalent to \mbox{$\it \makebox{\tt {\char'134}}\ x\ \makebox{\tt ->}\ x\ op\ e$}, and the section 
\mbox{$\it \makebox{\tt (}e\ op\makebox{\tt )}$} is equivalent to \mbox{$\it \makebox{\tt {\char'134}}\ x\ \makebox{\tt ->}\ e\ op\ x$}.
}

\subsection{Conditionals}
\label{conditionals}\index{conditional expression}
%
\begin{flushleft}\it\begin{tabbing}
\hspace{0.5in}\=\hspace{3.0in}\=\kill
$\it exp$\>\makebox[3.5em]{$\rightarrow$}$\it \makebox{\tt if}\ exp_1\ \makebox{\tt then}\ exp_2\ \makebox{\tt else}\ exp_3$
\end{tabbing}\end{flushleft}
\indexsyn{exp}%
%\indextt{if ... then ... else ...}
A {\em conditional expression}
\index{conditional expression}
has the form 
\mbox{$\it \makebox{\tt if}\ e_1\ \makebox{\tt then}\ e_2\ \makebox{\tt else}\ e_3$} and returns the value of \mbox{$\it e_2$} if the
value of \mbox{$\it e_1$} is \mbox{\tt True}, \mbox{$\it e_3$} if \mbox{$\it e_1$} is \mbox{\tt False}, and \mbox{$\it \bot$}
otherwise.

\outline{
\paragraph*{Translation:}
\mbox{$\it \makebox{\tt if}\ e_1\ \makebox{\tt then}\ e_2\ \makebox{\tt else}\ e_3$} is equivalent to:
\[
\mbox{$\it \makebox{\tt case}\ e_1\ \makebox{\tt of\ {\char'173}\ True\ ->}\ e_2\ \makebox{\tt ;\ False\ ->}\ e_3\ \makebox{\tt {\char'175}}$}
\]
where \mbox{\tt True} and \mbox{\tt False} are the two nullary constructors from the
type \mbox{\tt Bool}, as defined in the standard prelude.
}

\subsection{Lists}
\label{lists}
%
\begin{flushleft}\it\begin{tabbing}
\hspace{0.5in}\=\hspace{3.0in}\=\kill
$\it aexp$\>\makebox[3.5em]{$\rightarrow$}$\it \makebox{\tt [}\ exp_1\ \makebox{\tt ,}\ \ldots \ \makebox{\tt ,}\ exp_k\ \makebox{\tt ]}$\>\makebox[3em]{}$\it (k\geq 0)$
\end{tabbing}\end{flushleft}
\indexsyn{aexp}%
\index{[e1,...en]@\mbox{$\it \makebox{\tt [}e_1,\ldots ,e_n\makebox{\tt ]}$} (list)}
%
{\em Lists}\index{list} are written \mbox{$\it \makebox{\tt [}e_1\makebox{\tt ,}\ \ldots \makebox{\tt ,}\ e_k\makebox{\tt ]}$}, where
\mbox{$\it k\geq 0$}; the empty list is written \mbox{\tt []}.  Standard operations on
lists are given in the standard prelude (see
Appendix~\ref{stdprelude}, notably Section~\ref{preludelist}).

\outline{
\paragraph*{Translation:}  
\mbox{$\it \makebox{\tt [}e_1\makebox{\tt ,}\ \ldots \makebox{\tt ,}\ e_k\makebox{\tt ]}$} is equivalent to
\[
\mbox{$\it e_1\ \makebox{\tt :\ (}e_2\ \makebox{\tt :\ (}\ \ldots \ \makebox{\tt (}e_k\ \makebox{\tt :\ [])))}$}
\]
where \mbox{\tt :} and \mbox{\tt []} are constructors for lists, as defined in
the standard prelude (see Section~\ref{basic-lists}).  The types
of \mbox{$\it e_1$} through \mbox{$\it e_k$} must all be the same (call it \mbox{$\it t\/$}), and the
type of the overall expression is \mbox{\tt [}\mbox{$\it t$}\mbox{\tt ]} (see Section~\ref{type-syntax}).
}

\subsection{Tuples}
\label{tuples}
%
\begin{flushleft}\it\begin{tabbing}
\hspace{0.5in}\=\hspace{3.0in}\=\kill
$\it aexp$\>\makebox[3.5em]{$\rightarrow$}$\it \makebox{\tt (}\ exp_1\ \makebox{\tt ,}\ \ldots \ \makebox{\tt ,}\ exp_k\ \makebox{\tt )}$\>\makebox[3em]{}$\it (k\geq 2)$
\end{tabbing}\end{flushleft}
\indexsyn{aexp}%
\index{(e1,...,en)@\mbox{$\it \makebox{\tt (}e_1,\ldots ,e_n\makebox{\tt )}$} (tuple)}
%
{\em Tuples}\index{tuple} are written \mbox{$\it \makebox{\tt (}e_1\makebox{\tt ,}\ \ldots \makebox{\tt ,}\ e_k\makebox{\tt )}$}, and may be
of arbitrary length \mbox{$\it k\geq 2$}.  Standard operations on tuples are given
in the standard prelude (see Appendix~\ref{stdprelude}).

\outline{
\paragraph*{Translation:}  
\mbox{$\it \makebox{\tt (}e_1\makebox{\tt ,}\ \ldots \makebox{\tt ,}\ e_k\makebox{\tt )}$} for \mbox{$\it k\geq2$} is an instance of a \mbox{$\it k$}-tuple as
defined in the standard prelude, and requires no translation.  If
\mbox{$\it t_1$} through \mbox{$\it t_k$} are the types of \mbox{$\it e_1$} through \mbox{$\it e_k$},
respectively, then the type of the resulting tuple is 
\mbox{$\it \makebox{\tt (}t_1\makebox{\tt ,}\ \ldots \makebox{\tt ,}\ t_k\makebox{\tt )}$} (see Section~\ref{type-syntax}).
}

\subsection{Unit Expressions and Parenthesised Expressions}
\label{unit-expression}
\index{unit expression}
%
\begin{flushleft}\it\begin{tabbing}
\hspace{0.5in}\=\hspace{3.0in}\=\kill
$\it aexp$\>\makebox[3.5em]{$\rightarrow$}$\it \makebox{\tt ()}$\\ 
$\it $\>\makebox[3.5em]{$|$}$\it \makebox{\tt (}\ exp\ \makebox{\tt )}$
\end{tabbing}\end{flushleft}
\indexsyn{aexp}%
%
\noindent
The form \mbox{$\it \makebox{\tt (}e\makebox{\tt )}$} is simply a {\em parenthesised expression}, and is
equivalent to \mbox{$\it e$}.  The {\em unit expression} \mbox{\tt ()} has type
\mbox{\tt ()}\index{trivial type} (see
Section~\ref{type-syntax}); it is the only member of that type (it can
be thought of as the ``nullary tuple'')---see Section~\ref{basic-trivial}.
\nopagebreak[4]
\outline{
\paragraph{Translation:}  
\mbox{$\it \makebox{\tt (}e\makebox{\tt )}$} is equivalent to \mbox{$\it e$}.
}

\subsection{Arithmetic Sequences}
\label{arithmetic-sequences}
%
\begin{flushleft}\it\begin{tabbing}
\hspace{0.5in}\=\hspace{3.0in}\=\kill
$\it aexp$\>\makebox[3.5em]{$\rightarrow$}$\it \makebox{\tt [}\ exp_1\ [\makebox{\tt ,}\ exp_2]\ \makebox{\tt ..}\ [exp_3]\ \makebox{\tt ]}$
\end{tabbing}\end{flushleft}
\indexsyn{aexp}%
\noindent
The form \mbox{$\it \makebox{\tt [}e_1\makebox{\tt ,}\ e_2\ \makebox{\tt ..}\ e_3\makebox{\tt ]}$} denotes an {\em arithmetic
sequence}\index{arithmetic sequence} from \mbox{$\it e_1$} in increments of
\mbox{$\it e_2-e_1$} up to \mbox{$\it e_3$} (if the increment is positive) or down to \mbox{$\it e_3$}
(if the increment is negative).  An infinite list of \mbox{$\it e_1$}'s results
if the increment is zero, and the empty list results if \mbox{$\it e_3$} is less
than \mbox{$\it e_1$} and the increment is positive, or if \mbox{$\it e_3$} is greater than
\mbox{$\it e_1$} and the increment is negative.  If the comma and \mbox{$\it e_2$} are
omitted, then the increment is 1; if \mbox{$\it e_3$} is omitted, then
the sequence is includes all elements of the enumeration, and is thus
infinite for infinite enumerations.

Arithmetic sequences may be defined over any type in class \mbox{\tt Enum},
including \mbox{\tt Char}, \mbox{\tt Int}, and \mbox{\tt Integer} (see
Figure~\ref{standard-classes} and Section~\ref{derived-decls}).
For example, \mbox{\tt ['a'..'z']} denotes
the list of lower-case letters in alphabetical order.

\outline{
\paragraph{Translation:}
Arithmetic sequences satisfy these identities:
\begin{center}
\begin{tabular}{lcl}%
\struthack{17pt}%
\mbox{\tt [\ }$e_1$\mbox{\tt ..\ ]}         & $=$ 
                        & \mbox{\tt enumFrom} $e_1$ \\
\mbox{\tt [\ }$e_1$\mbox{\tt ,}$e_2$\mbox{\tt ..\ ]} & $=$ 
                        & \mbox{\tt enumFromThen} $e_1$ $e_2$ \\
\mbox{\tt [\ }$e_1$\mbox{\tt ..}$e_3$\mbox{\tt \ ]}  & $=$ 
                        & \mbox{\tt enumFromTo} $e_1$ $e_3$ \\
\mbox{\tt [\ }$e_1$\mbox{\tt ,}$e_2$\mbox{\tt ..}$e_3$\mbox{\tt \ ]} 
                        & $=$ 
                        & \mbox{\tt enumFromThenTo} $e_1$ $e_2$ $e_3$
\end{tabular}
\end{center}
where \mbox{\tt enumFrom}, \mbox{\tt enumFromThen}, \mbox{\tt enumFromTo}, and \mbox{\tt enumFromThenTo}
are operations in the class \mbox{\tt Enum} as defined in the standard prelude
(see Figure~\ref{standard-classes}).
}

\subsection{List Comprehensions}
\index{list comprehension}
\label{list-comprehensions}
%
\begin{flushleft}\it\begin{tabbing}
\hspace{0.5in}\=\hspace{3.0in}\=\kill
$\it aexp$\>\makebox[3.5em]{$\rightarrow$}$\it \makebox{\tt [}\ exp\ \makebox{\tt |}\ qual_1\ \makebox{\tt ,}\ \ldots \ \makebox{\tt ,}\ qual_n\ \makebox{\tt ]}$\>\makebox[3em]{}$\it (\tr{list\ comprehension},\ n\geq 1)$\\ 
$\it qual$\>\makebox[3.5em]{$\rightarrow$}$\it pat\ \makebox{\tt <-}\ exp$\\ 
$\it $\>\makebox[3.5em]{$|$}$\it exp$
\end{tabbing}\end{flushleft}
\indexsyn{aexp}
\indexsyn{qual}

\noindent
A {\em list comprehension} has the form \mbox{$\it \makebox{\tt [}\ e\ \makebox{\tt |}\ q_1\makebox{\tt ,}\ \ldots \makebox{\tt ,}\ q_n\ \makebox{\tt ]},\\
\it n\geq 1,$} where the \mbox{$\it q_i$} qualifiers\index{qualifier} are either {\em
generators}\index{generator} of the form \mbox{$\it p\ \makebox{\tt <-}\ e$}, where \mbox{$\it p$} is a
pattern (see Section~\ref{pattern-matching}) of type \mbox{$\it t$} and \mbox{$\it e$} is an
expression of type \mbox{$\it \makebox{\tt [}t\makebox{\tt ]}$}; or {\em guards},\index{guard} which are
arbitrary expressions of type \mbox{\tt Bool}.

Such a list comprehension returns the list of elements
produced by evaluating \mbox{$\it e$} in the successive environments
created by the nested, depth-first evaluation of the generators in the
qualifier list.  Binding of variables occurs according to the normal
pattern-matching rules (see Section~\ref{pattern-matching}), and if a
match fails then that element of the list is simply skipped over.  Thus:\nopagebreak[4]
\bprog
\mbox{\tt [\ x\ |\ \ xs\ \ \ <-\ [\ [(1,2),(3,4)],\ [(5,4),(3,2)]\ ],\ }\\
\mbox{\tt \ \ \ \ \ \ (3,x)\ <-\ xs\ ]}
\eprog
yields the list \mbox{\tt [4,2]}.  If a qualifier is a guard, it must evaluate
to \mbox{\tt True} for the previous pattern-match to succeed.  
As usual, bindings in list comprehensions can shadow those in outer
scopes; for example:
\[\ba{lll}
\mbox{\tt [\ x\ |\ x\ <-\ x,\ x\ <-\ x\ ]} & = & \mbox{\tt [\ z\ |\ y\ <-\ x,\ z\ <-\ y]} \\
\ea\]
\outline{
\paragraph{Translation:}
List comprehensions satisfy these identities, which may be
used as a translation into the kernel:
\begin{center}
\bt{lcl}%
\struthack{17pt}%
\mbox{\tt [\ }$e$\mbox{\tt \ |\ }$b$\mbox{\tt \ ]}            &=& \mbox{\tt if\ }$b$\mbox{\tt \ then\ [}$e$\mbox{\tt ]\ else\ []} \\
\mbox{\tt [\ }$e$\mbox{\tt \ |\ }$q_1$\mbox{\tt ,\ }$q_2$\mbox{\tt \ ]} &=& \mbox{\tt concat\ [\ [\ }$e$\mbox{\tt \ |\ }$q_2$\mbox{\tt \ ]\ |\ }$q_1$\mbox{\tt \ ]} \\
\mbox{\tt [\ }$e$\mbox{\tt \ |\ }$p$\mbox{\tt \ <-\ }$l$\mbox{\tt \ ]}   &=& \mbox{\tt let\ ok\ }$p$\mbox{\tt \ \ =\ \ True} \\
                               & & \mbox{\tt \ \ \ \ ok\ {\char'137}\ \ =\ \ False} \\
                               & & \mbox{\tt in} \\
                               & & \mbox{\tt \ map\ ({\char'134}}$p$\mbox{\tt \ ->\ }$e$\mbox{\tt )\ (filter\ ok\ }$l$\mbox{\tt )} \\
\et
\end{center}
where \mbox{$\it e$} ranges over expressions, \mbox{$\it p$} ranges over
patterns, \mbox{$\it l$} ranges over list-valued expressions, \mbox{$\it b$} ranges over
boolean expressions, \mbox{$\it q_1$} and \mbox{$\it q_2$} range over non-empty lists of
qualifiers, and \mbox{\tt ok} is a new identifier not appearing in \mbox{$\it e$}, \mbox{$\it p$}, or
\mbox{$\it l$}.  These three equations uniquely define list comprehensions.
\mbox{\tt True}, \mbox{\tt False}, \mbox{\tt map}, \mbox{\tt concat} and \mbox{\tt filter} are all as defined in the
standard prelude.
}

\subsection{Let Expressions}
\index{let expression}
\label{let-expressions}
%
% Including this syntax blurb does REALLY bad things to page breaking
% in the 1.1 report (sigh); ToDo: hope it goes away.
%@@
%exp    ->  \mbox{\tt let\ {\char'173}} decls \mbox{\tt {\char'175}\ in} exp
%@@
\indexsyn{exp}
\index{declaration!within a {\ptt let} expression}
\noindent
{\em Let expressions} have the general form
\mbox{$\it \makebox{\tt let\ {\char'173}}\ d_1\ \makebox{\tt ;}\ \ldots \ \makebox{\tt ;}\ d_n\ \makebox{\tt {\char'175}\ in}\ e$},
and introduce a
nested, lexically-scoped, {\em
mutually-recursive} list of declarations (\mbox{\tt let} is often called \mbox{\tt letrec} in
other languages).  The scope of the declarations is the expression \mbox{$\it e$}
and the right hand side of the declarations.  Declarations are
described in Section~\ref{declarations}.  Pattern bindings are matched
lazily as irrefutable patterns.\index{irrefutable pattern}

\outline{
\paragraph*{Translation:} The dynamic semantics of the expression 
\mbox{$\it \makebox{\tt let\ {\char'173}}\ d_1\ \makebox{\tt ;}\ \ldots \ \makebox{\tt ;}\ d_n\ \makebox{\tt {\char'175}\ in}\ e_0$} is captured by this
translation: After removing all type signatures, each
declaration \mbox{$\it d_i$} is translated into an equation of the form 
\mbox{$\it p_i\ \makebox{\tt =}\ e_i$}, where \mbox{$\it p_i$} and \mbox{$\it e_i$} are patterns and expressions
respectively, using the translation in
Section~\ref{function-bindings}.  Once done, these identities
hold, which may be used as a translation into the kernel:
\begin{center}
\bt{lcl}%
\struthack{17pt}%
\mbox{\tt let\ {\char'173}}$p_1$\mbox{\tt \ =\ }$e_1$\mbox{\tt ;\ }...\mbox{\tt ;\ }$p_n$\mbox{\tt \ =\ }$e_n$\mbox{\tt {\char'175}\ in} $e_0$
      &=& \mbox{\tt let\ ({\char'176}}$p_1$\mbox{\tt ,}...\mbox{\tt ,{\char'176}}$p_n$\mbox{\tt )\ =\ (}$e_1$\mbox{\tt ,}...\mbox{\tt ,}$e_n$\mbox{\tt )\ in} $e_0$ \\
\mbox{\tt let\ }$p$\mbox{\tt \ =\ }$e_1$ \mbox{\tt \ in\ } $e_0$
        &=& \mbox{\tt case\ }$e_1$\mbox{\tt \ of\ {\char'176}}$p$\mbox{\tt \ ->\ }$e_0$   \\
        & & {\rm where no variable in $p$ appears free in $e_1$} \\
\mbox{\tt let\ }$p$\mbox{\tt \ =\ }$e_1$ \mbox{\tt \ in\ } $e_0$
      &=& \mbox{\tt let\ }$p$\mbox{\tt \ =\ fix\ (\ {\char'134}\ {\char'176}}$p$\mbox{\tt \ ->\ }$e_1$\mbox{\tt )\ in} $e_0$
\et
\end{center}
where \mbox{\tt fix} is the least fixpoint operator.  Note the use of the
irrefutable patterns in the second and third rules.  
% This same semantics applies to the top-level of
%a program that has been translated into a \mbox{\tt let} expression,
% as described at the beginning of Section~\ref{modules}.
The static semantics of the bindings in a \mbox{\tt let} expression
is described in 
Section~\ref{pattern-bindings}.
}

\subsection{Case Expressions}
\label{case}
%
\begin{flushleft}\it\begin{tabbing}
\hspace{0.5in}\=\hspace{3.0in}\=\kill
$\it exp$\>\makebox[3.5em]{$\rightarrow$}$\it \makebox{\tt case}\ exp\ \makebox{\tt of}\ \makebox{\tt {\char'173}}\ alts\ [\makebox{\tt ;}]\ \makebox{\tt {\char'175}}$\\ 
$\it alts$\>\makebox[3.5em]{$\rightarrow$}$\it alt_1\ \makebox{\tt ;}\ \ldots \ \makebox{\tt ;}\ alt_n$\>\makebox[3em]{}$\it (n\geq 0)$\\ 
$\it alt$\>\makebox[3.5em]{$\rightarrow$}$\it pat\ \makebox{\tt ->}\ exp\ [\makebox{\tt where}\ \makebox{\tt {\char'173}}\ decls\ [\makebox{\tt ;}]\ \makebox{\tt {\char'175}}]$\\ 
$\it $\>\makebox[3.5em]{$|$}$\it pat\ gdpat\ [\makebox{\tt where}\ \makebox{\tt {\char'173}}\ decls\ [\makebox{\tt ;}]\ \makebox{\tt {\char'175}}]$\\ 
$\it $\\ 
$\it gdpat$\>\makebox[3.5em]{$\rightarrow$}$\it gd\ \makebox{\tt ->}\ exp\ [\ gdpat\ ]$\\ 
$\it gd$\>\makebox[3.5em]{$\rightarrow$}$\it \makebox{\tt |}\ exp$
\end{tabbing}\end{flushleft}
\indexsyn{exp}%
\indexsyn{alts}%
\indexsyn{alt}%
\indexsyn{gdpat}%
\indexsyn{gd}%
%\indextt{case ... of ...}
A {\em case expression}\index{case expression} has the general form
\[
\mbox{$\it \makebox{\tt case}\ e\ \makebox{\tt of\ {\char'173}\ }p_1\ match_1\ \makebox{\tt ;}\ \ldots \ \makebox{\tt ;}\ p_n\ match_n\ \makebox{\tt {\char'175}}$}
\]
where each \mbox{$\it match_i$} is of the general form
\[\ba{lll}
 & \mbox{$\it \makebox{\tt |}\ g_{i1}$}   & \mbox{$\it \makebox{\tt ->}\ e_{i1}\ \makebox{\tt ;}$} \\
 & \mbox{$\it \ldots $} \\
 & \mbox{$\it \makebox{\tt |}\ g_{im_i}$} & \mbox{$\it \makebox{\tt ->}\ e_{im_i}$} \\
 & \multicolumn{2}{l}{\mbox{$\it \makebox{\tt where\ {\char'173}}\ decls_i\ \makebox{\tt {\char'175}}$}}
\ea\]
where each clause \mbox{$\it p_i\ matches_i$} consists of a a {\em
pattern}\index{pattern} \mbox{$\it p_i$} and its \mbox{$\it matches_i$}, which
consists of pairs of optional {\em guards}\index{guard}
\mbox{$\it g_{ij}$} and {\em bodies} \mbox{$\it e_{ij}$} (expressions), as well as
optional bindings (\mbox{$\it decls_i$}) that scope over all of the guards and
expressions of the clause.  An alternative of the form
\[
\mbox{$\it pat\ \makebox{\tt ->}\ expr\ \makebox{\tt where\ {\char'173}}\ decls\ \makebox{\tt {\char'175}}$}
\]
is treated as shorthand for:
\[\ba{lll}
 & \mbox{$\it pat\ \makebox{\tt |\ True}$}   & \mbox{$\it \makebox{\tt ->}\ expr$} \\
 & \multicolumn{2}{l}{\mbox{$\it \makebox{\tt where\ {\char'173}}\ decls\ \makebox{\tt {\char'175}}$}}
\ea\]

A case expression must have at least one clause and each clause must
have at least one body.  Each body must have the same type, and the
type of the whole expression is that type.

A case expression is evaluated by pattern-matching the expression \mbox{$\it e$}
against the individual clauses.  The matches are tried sequentially,
from top to bottom.  The first successful match causes evaluation of
the corresponding clause body, in the environment of the case
expression extended by the bindings created during the matching of
that clause and by the \mbox{$\it decls_i$} associated with that clause.  If no
match succeeds, the result is \mbox{$\it \bot$}.  Pattern matching is described
in Section~\ref{pattern-matching}, with the formal semantics of case
expressions in Section~\ref{case-semantics}.

\subsection{Expression Type-Signatures}
\index{expression type-signature}
\label{expression-type-sigs}
%
\begin{flushleft}\it\begin{tabbing}
\hspace{0.5in}\=\hspace{3.0in}\=\kill
$\it exp$\>\makebox[3.5em]{$\rightarrow$}$\it exp\ \makebox{\tt ::}\ [context\ \makebox{\tt =>}]\ atype$
\end{tabbing}\end{flushleft}
\indexsyn{exp}
\indextt{::}
\nopagebreak[4]
{\em Expression type-signatures} have the form \mbox{$\it e\ \makebox{\tt ::}\ t$}, where \mbox{$\it e$}
is an expression and \mbox{$\it t$} is a type (Section~\ref{type-syntax}); they
are used to type an expression explicitly
and may be used to resolve ambiguous typings due to overloading (see
Section~\ref{default-decls}).  The value of the expression is just that of
\mbox{$\it exp$}.  As with normal type signatures (see
Section~\ref{type-signatures}), the declared type may be more specific than 
the principal type derivable from \mbox{$\it exp$}, but it is an error to give
a type that is more general than, or not comparable to, the
principal type.


\subsection{Pattern-Matching}
\index{pattern-matching}
\label{pattern-matching}
\label{patterns}

{\em Patterns} appear in lambda abstractions, function definitions, pattern
bindings, list comprehensions, and case expressions.  However, the
first four of these ultimately translate into case expressions, so
defining the semantics of pattern-matching for case expressions is sufficient.
%it suffices to restrict the definition of the semantics of
%pattern-matching to case expressions.

\subsubsection{Patterns}

Patterns\index{pattern} have this syntax:
\begin{flushleft}\it\begin{tabbing}
\hspace{0.5in}\=\hspace{3.0in}\=\kill
$\it pat$\>\makebox[3.5em]{$\rightarrow$}$\it pat^0$\\ 
$\it pat^i$\>\makebox[3.5em]{$\rightarrow$}$\it pat^{i+1}_1\ [conop^{({\rm\ n},i)}\ pat^{i+1}_2]$\>\makebox[3em]{}$\it (0\leq i\leq 9)$\\ 
$\it $\>\makebox[3.5em]{$|$}$\it lpat^i\ conop^{({\rm\ l},i)}\ pat^{i+1}$\\ 
$\it $\>\makebox[3.5em]{$|$}$\it pat^{i+1}\ conop^{({\rm\ r},i)}\ rpat^i$\\ 
$\it lpat^i$\>\makebox[3.5em]{$\rightarrow$}$\it [lpat^i\ conop^{({\rm\ l},i)}]\ pat^{i+1}$\>\makebox[3em]{}$\it (0\leq i\leq 9)$\\ 
$\it lpat^6$\>\makebox[3.5em]{$\rightarrow$}$\it lpat^6\ \makebox{\tt +}\ integer$\>\makebox[3em]{}$\it (\tr{successor\ pattern})$\\ 
$\it $\>\makebox[3.5em]{$|$}$\it \makebox{\tt -}\ \{integer\ |\ float\}$\>\makebox[3em]{}$\it (\tr{negative\ literal})$\\ 
$\it rpat^i$\>\makebox[3.5em]{$\rightarrow$}$\it pat^{i+1}\ [conop^{({\rm\ r},i)}\ rpat^i]$\>\makebox[3em]{}$\it (0\leq i\leq 9)$\\ 
$\it pat^{10}$\>\makebox[3.5em]{$\rightarrow$}$\it apat$\\ 
$\it $\>\makebox[3.5em]{$|$}$\it con\ apat_1\ \ldots \ apat_k$\>\makebox[3em]{}$\it (\arity{con}=k\geq 1)$\\ 
$\it $\\ 
$\it apat$\>\makebox[3.5em]{$\rightarrow$}$\it var\ [{\tt\ @}\ apat]$\>\makebox[3em]{}$\it (\tr{as\ pattern})$\\ 
$\it $\>\makebox[3.5em]{$|$}$\it con$\>\makebox[3em]{}$\it (\arity{con}=0)$\\ 
$\it $\>\makebox[3.5em]{$|$}$\it literal$\\ 
$\it $\>\makebox[3.5em]{$|$}$\it \makebox{\tt {\char'137}}$\>\makebox[3em]{}$\it (\tr{wildcard})$\\ 
$\it $\>\makebox[3.5em]{$|$}$\it \makebox{\tt ()}$\>\makebox[3em]{}$\it (\tr{unit\ pattern})$\\ 
$\it $\>\makebox[3.5em]{$|$}$\it \makebox{\tt (}\ pat\ \makebox{\tt )}$\>\makebox[3em]{}$\it (\tr{parenthesised\ pattern})$\\ 
$\it $\>\makebox[3.5em]{$|$}$\it \makebox{\tt (}\ pat_1\ \makebox{\tt ,}\ \ldots \ \makebox{\tt ,}\ pat_k\ \makebox{\tt )}$\>\makebox[3em]{}$\it (\tr{tuple\ pattern},\ k\geq 2)$\\ 
$\it $\>\makebox[3.5em]{$|$}$\it \makebox{\tt [}\ pat_1\ \makebox{\tt ,}\ \ldots \ \makebox{\tt ,}\ pat_k\ \makebox{\tt ]}$\>\makebox[3em]{}$\it (\tr{list\ pattern},\ k\geq 0)$\\ 
$\it $\>\makebox[3.5em]{$|$}$\it \makebox{\tt {\char'176}}\ apat$\>\makebox[3em]{}$\it (\tr{irrefutable\ pattern})$
\end{tabbing}\end{flushleft}
\indexsyn{pat}%
\index{pat@\mbox{$\it pat^i$}}%
\index{lpat@\mbox{$\it lpat^i$}}%
\index{rpat@\mbox{$\it rpat^i$}}%
\indexsyn{apat}%
The arity of a constructor must match the number of
sub-patterns associated with it; one cannot match against a
partially-applied constructor.

All patterns must be {\em linear}\index{linearity}\index{linear
pattern}---no variable may appear more than once.

Patterns of the form \mbox{$\it var$}{\tt @}\mbox{$\it pat$} are called {\em as-patterns},
\index{as-pattern ({\ptt {\char'100}})}
and allow one to use \mbox{$\it var$}
as a name for the value being matched by \mbox{$\it pat$}.  For example,\nopagebreak[4]
\bprog
\mbox{\tt case\ e\ of\ {\char'173}\ xs@(x:rest)\ ->\ if\ x==0\ then\ rest\ else\ xs\ {\char'175}}
\eprog
is equivalent to:
\bprog
\mbox{\tt let\ {\char'173}\ xs\ =\ e\ {\char'175}\ in}\\
\mbox{\tt \ \ case\ xs\ of\ {\char'173}\ (x:rest)\ ->\ if\ x\ ==\ 0\ then\ rest\ else\ xs\ {\char'175}}
\eprog
%This transformation of a case expression is always valid, and
%is assumed done prior to the pattern-matching semantics given below.

Patterns of the form \mbox{\tt {\char'137}} are {\em
wildcards}\index{wildcard pattern ({\ptt {\char'137}})} and are useful when some part of a pattern
is not referenced on the right-hand-side.  It is as if an
identifier not used elsewhere were put in its place.  For example,
\bprog
\mbox{\tt case\ e\ of\ {\char'173}\ [x,{\char'137},{\char'137}]\ \ ->\ \ if\ x==0\ then\ True\ else\ False\ {\char'175}}
\eprog
is equivalent to:
\bprog
\mbox{\tt case\ e\ of\ {\char'173}\ [x,y,z]\ \ ->\ \ if\ x==0\ then\ True\ else\ False\ {\char'175}}
\eprog
% where \mbox{\tt y} and \mbox{\tt z} are identifiers not used elsewhere.

%old:
%This translation is also 
%assumed prior to the semantics given below.

In the pattern-matching rules given below we distinguish two kinds of
patterns: an {\em irrefutable pattern}
\index{irrefutable pattern}
is either of the form \mbox{$\it \makebox{\tt {\char'176}}apat$}, a variable, or a wildcard; all
other patterns are {\em refutable}.
\index{refutable pattern}

\subsubsection{Informal semantics of pattern-matching}

Patterns are matched against values.  Attempting to match a pattern
can have one of three results: it may {\em fail\/}; it may {\em
succeed}, returning a binding for each variable in the pattern; or it
may {\em diverge} (i.e.~return \mbox{$\it \bot$}).  Pattern-matching proceeds
from left to right, and outside in, according to these rules:
\begin{enumerate}
\item Matching a value \mbox{$\it v$} against the irrefutable pattern
\index{irrefutable pattern}
\mbox{$\it var$} always succeeds and binds \mbox{$\it var$} to \mbox{$\it v$}.  Similarly, matching \mbox{$\it v$}
against the irrefutable pattern \mbox{$\it \makebox{\tt {\char'176}}apat$} always succeeds.  The free
variables in \mbox{$\it apat$} are bound to the appropriate values if matching
\mbox{$\it v$} against \mbox{$\it apat$} would otherwise succeed, and to \mbox{$\it \bot$} if matching
\mbox{$\it v$} against \mbox{$\it apat$} fails or diverges.  (Binding does {\em
not} imply evaluation.)

Operationally, this means that no matching is done on an
irrefutable pattern until one of the variables in the pattern is used.
At that point the entire pattern is matched against the value, and if
the match fails or diverges, so does the overall computation.

\item Matching \mbox{$\it \bot$} against a refutable pattern always diverges.
\index{refutable pattern}

\item Matching a non-\mbox{$\it \bot$} value can occur against two kinds of
refutable patterns:
\begin{enumerate}
\item Matching a non-\mbox{$\it \bot$} value against a constructed pattern
\index{constructed pattern}
fails if the outermost constructors are different.  If the constructors are
the same, the result of the match is the result of matching the
sub-patterns left-to-right: if all matches succeed, the overall match
succeeds; the first to fail or diverge causes the overall match to
fail or diverge, respectively.  

Constructed values consist of those created by prefix or infix
constructors, tuple or list patterns, and strings (which are
lists of characters).  Characters and \mbox{\tt ()} are treated as
nullary constructors.  Numeric literals are matched using the
overloaded \mbox{\tt ==} function.

%Also, literals (characters, positive and
%negative integers, and the unit value \mbox{\tt ()}) are treated as nullary
%constructors.

\item Matching a non-\mbox{$\it \bot$} value \mbox{$\it n$} against a pattern of the form
\mbox{$\it x\makebox{\tt +}k$}
\index{n+k pattern@\mbox{$\it n\makebox{\tt +}k$} pattern}
(where \mbox{$\it x$} is a variable and \mbox{$\it k$} is a positive integer
literal) succeeds if \mbox{$\it n\geq k$}, resulting in the binding of \mbox{$\it x$} to \mbox{$\it n-k$},
and fails if \mbox{$\it n<k$}.  For example, the Fibonacci function may be
defined as follows:

\bprog
\mbox{\tt fib\ n\ =\ case\ n\ of\ {\char'173}}\\
\mbox{\tt \ \ \ \ \ \ \ \ \ \ 0\ \ \ ->\ 1\ ;}\\
\mbox{\tt \ \ \ \ \ \ \ \ \ \ 1\ \ \ ->\ 1\ ;}\\
\mbox{\tt \ \ \ \ \ \ \ \ \ \ n+2\ ->\ fib\ n\ +\ fib\ (n+1)\ {\char'175}}
\eprog
Since \mbox{\tt n} must be bound to a positive value, \mbox{\tt fib} diverges for a
negative argument, and exactly one of the equations matches any
non-negative argument.
\end{enumerate}

\item
The result of matching a value \mbox{$\it v$} against an as-pattern \mbox{$\it var$}{\tt @}\mbox{$\it pat$} is
\index{as-pattern ({\ptt {\char'100}})}
the result of matching \mbox{$\it v$} against \mbox{$\it pat$} augmented with the binding of
\mbox{$\it var$} to \mbox{$\it v$}.  If the match of \mbox{$\it v$} against \mbox{$\it pat$} fails or diverges,
then so does the overall match.
\end{enumerate}

Aside from the obvious static type constraints (for
example, it is a static error to match a character against a
boolean), these static class constraints hold: an integer
literal pattern
\index{integer literal pattern}
can only be matched against a value in the class
\mbox{\tt Num}; a floating literal pattern
\index{floating literal pattern}
can only be matched against a value
in the class \mbox{\tt Fractional}; and a \mbox{$\it n\makebox{\tt +}k$} pattern
\index{n+k pattern@\mbox{$\it n\makebox{\tt +}k$} pattern}
can only be matched
against a value in the class \mbox{\tt Integral}.

Here are some examples:
\begin{enumerate}
\item If the pattern \mbox{\tt [1,2]} is matched against \mbox{$\it \makebox{\tt [0,}\bot\makebox{\tt ]}$}, then \mbox{\tt 1}
\mbox{$\it fails$} to match against \mbox{\tt 0}, and the result is a failed match.  But
if \mbox{\tt [1,2]} is matched against \mbox{$\it \makebox{\tt [}\bot\makebox{\tt ,0]}$}, then attempting to match
\mbox{\tt 1} against \mbox{$\it \bot$} causes the match to \mbox{$\it diverge$}.

\item These examples demonstrate refutable vs.~irrefutable
matching:
\nopagebreak[4]
\bprog
\mbox{\tt ({\char'134}\ {\char'176}(x,y)\ ->\ 0)\ }\mbox{$\it \bot$}\mbox{\tt \ \ \ \ }\mbox{$\it \Rightarrow$}\mbox{\tt \ \ \ \ 0}\\
\mbox{\tt ({\char'134}\ \ (x,y)\ ->\ 0)\ }\mbox{$\it \bot$}\mbox{\tt \ \ \ \ }\mbox{$\it \Rightarrow$}\mbox{\tt \ \ \ \ }\mbox{$\it \bot$}\mbox{\tt }
\eprog
\bprog
\mbox{\tt ({\char'134}\ {\char'176}[x]\ ->\ 0)\ []\ \ \ \ }\mbox{$\it \Rightarrow$}\mbox{\tt \ \ \ \ 0}\\
\mbox{\tt ({\char'134}\ {\char'176}[x]\ ->\ x)\ []\ \ \ \ }\mbox{$\it \Rightarrow$}\mbox{\tt \ \ \ \ }\mbox{$\it \bot$}\mbox{\tt }
\eprog
\bprog
\mbox{\tt ({\char'134}\ {\char'176}[x,{\char'176}(a,b)]\ ->\ x)\ [(0,1),}\mbox{$\it \bot$}\mbox{\tt ]\ \ \ \ }\mbox{$\it \Rightarrow$}\mbox{\tt \ \ \ \ (0,1)}\\
\mbox{\tt ({\char'134}\ {\char'176}[x,\ (a,b)]\ ->\ x)\ [(0,1),}\mbox{$\it \bot$}\mbox{\tt ]\ \ \ \ }\mbox{$\it \Rightarrow$}\mbox{\tt \ \ \ \ }\mbox{$\it \bot$}\mbox{\tt }
\eprog
\bprog
\mbox{\tt ({\char'134}\ \ (x:xs)\ ->\ x:x:xs)\ }\mbox{$\it \bot$}\mbox{\tt \ \ \ }\mbox{$\it \Rightarrow$}\mbox{\tt \ \ \ }\mbox{$\it \bot$}\mbox{\tt }\\
\mbox{\tt ({\char'134}\ {\char'176}(x:xs)\ ->\ x:x:xs)\ }\mbox{$\it \bot$}\mbox{\tt \ \ \ }\mbox{$\it \Rightarrow$}\mbox{\tt \ \ \ }\mbox{$\it \bot$}\mbox{\tt :}\mbox{$\it \bot$}\mbox{\tt :}\mbox{$\it \bot$}\mbox{\tt }
\eprogNoSkip
\end{enumerate}

Top level patterns in case
expressions, and the set of top level patterns in function or pattern
bindings, may have zero or more associated {\em guards}\index{guard}.  A guard is
a boolean expression that is evaluated only after all of the
arguments have been successfully matched, and it must be true for the
overall pattern-match to succeed.  The environment of the guard is the same
as the right-hand-side of the case expression
clause, function definition, or pattern binding to which it is attached.

The guard semantics has an obvious influence on the
strictness characteristics of a function or case expression.  In
particular, an otherwise irrefutable pattern
\index{irrefutable pattern}
may be evaluated because of a guard.  For example, in
\bprog
\mbox{\tt f\ {\char'176}(x,y,z)\ [a]\ |\ a==y\ =\ 1}
\eprog
both \mbox{\tt a} and \mbox{\tt y} will be evaluated.


\subsubsection{Formal semantics of pattern-matching}
\label{case-semantics}

The semantics of all other constructs that use
pattern-matching is defined by giving identities that relate those constructs to
\mbox{\tt case} expressions.

The semantics of \mbox{\tt case} expressions are given as a series of
identities. Figure~\ref{simple-case-expr} shows the
identities:
$e$, $e'$ and $e_i$ are arbitrary expressions; 
$g$ and $g_i$ are boolean-valued expressions; 
$p$ and $p_i$ are patterns; 
$x$ and $x_i$ are variables; 
$K$ and $K'$ are constructors (including tuple constructors); 
a $match_i$ is a form as shown in rule~(a);
and $k$ is a character, string, or numeric literal.

For clarity, several rules are expressed using
\mbox{\tt let} (used only in a non-recursive way); their usual purpose is to
prevent name capture
(e.g., in rule~(b)).  The rules may be re-expressed entirely with
\mbox{\tt cases} by applying this identity:
\[
\mbox{$\it \makebox{\tt let\ }x\makebox{\tt \ =\ }y\makebox{\tt \ in\ }e\makebox{\tt \ }\ =\makebox{\tt \ \ case\ }y\makebox{\tt \ of\ {\char'173}\ }x\makebox{\tt \ ->\ }e\makebox{\tt \ {\char'175}}$}
\]

\begin{figure}
\outline{
\begin{tabular}{@{}cl}
(a)&\mbox{\tt case\ }$e_0$\mbox{\tt \ of\ {\char'173}\ }$p_1\ \ match_1$\mbox{\tt ;\ \ }$\ldots{}$\mbox{\tt \ ;\ }$p_n\ \ match_n$\mbox{\tt \ {\char'175}}\\
   &$=$\mbox{\tt \ \ case\ }$e_0$\mbox{\tt \ of\ {\char'173}\ }$p_1\ \ match_1$\mbox{\tt \ ;}\\
   &\mbox{\tt \ \ \ \ \ \ \ \ \ \ \ \ \ \ \ \ {\char'137}\ \ ->\ }$\ldots{}$\mbox{\tt \ case\ }$e_0$\mbox{\tt \ of\ {\char'173}}\\
   &\mbox{\tt \ \ \ \ \ \ \ \ \ \ \ \ \ \ \ \ \ \ \ \ \ \ \ \ \ \ \ }$p_n\ \ match_n$\\
   &\mbox{\tt \ \ \ \ \ \ \ \ \ \ \ \ \ \ \ \ \ \ \ \ \ \ \ \ \ \ \ {\char'137}\ \ ->\ error\ "No\ match"\ {\char'175}}$\ldots$\mbox{\tt {\char'175}}\\
   &\mbox{\tt \ }{\rm where each $match_i$ has the form:}\\
   &\mbox{\tt \ \ |\ }$g_{i,1}$  \mbox{\tt \ ->\ }$e_{i,1}$\mbox{\tt \ ;\ }$\ldots{}$\mbox{\tt \ ;\ |\ }$g_{i,m_i}$\mbox{\tt \ ->\ }$e_{i,m_i}$\mbox{\tt \ where\ {\char'173}\ }$decls_i$\mbox{\tt \ {\char'175}}\\[4pt]
%\\
(b)&\mbox{\tt case\ }$e_0$\mbox{\tt \ of\ {\char'173}\ }$p$\mbox{\tt \ |\ }$g_1$\mbox{\tt \ ->\ }$e_1$\mbox{\tt \ ;\ }$\ldots{}$\\
   &\hspace*{4pt}\mbox{\tt \ \ \ \ \ \ \ \ \ \ \ \ \ \ |\ }$g_n$\mbox{\tt \ ->\ }$e_n$\mbox{\tt \ where\ {\char'173}\ }$decls$\mbox{\tt \ {\char'175}}\\
   &\hspace*{2pt}\mbox{\tt \ \ \ \ \ \ \ \ \ \ \ \ {\char'137}\ \ \ \ \ \ ->\ }$e'$\mbox{\tt \ {\char'175}}\\
   &$=$\mbox{\tt \ let\ {\char'173}\ }$y$\mbox{\tt \ =\ }$e'$\mbox{\tt \ {\char'175}\ \ \ }(where $y$ is a completely new variable)\\
   &\mbox{\tt \ \ \ in\ \ case\ }$e_0$\mbox{\tt \ of\ {\char'173}}\\
   &\mbox{\tt \ \ \ \ \ \ \ \ \ }$p$\mbox{\tt \ ->\ let\ {\char'173}\ }$decls$\mbox{\tt \ {\char'175}\ in}\\
   &\mbox{\tt \ \ \ \ \ \ \ \ \ \ \ \ \ \ \ \ if\ }$g_1$\mbox{\tt \ then\ }$e_1$\mbox{\tt \ }$\ldots{}$\mbox{\tt \ else\ if\ }$g_n$\mbox{\tt \ then\ }$e_n$\mbox{\tt \ else\ }$y$\\
   &\mbox{\tt \ \ \ \ \ \ \ \ \ {\char'137}\ ->\ }$y$\mbox{\tt \ {\char'175}}\\[4pt]
%\\
(c)&\mbox{\tt case\ }$e_0$\mbox{\tt \ of\ {\char'173}\ {\char'176}}$p$\mbox{\tt \ ->\ }$e$\mbox{\tt ;\ {\char'137}\ ->\ }$e'$\mbox{\tt \ {\char'175}}\\
&$=$\mbox{\tt \ let\ {\char'173}\ }$y$\mbox{\tt \ =\ }$e_0$\mbox{\tt \ {\char'175}\ in}\\
&\mbox{\tt \ \ \ let\ {\char'173}\ }$x'_1$\mbox{\tt \ =\ case\ }$y$\mbox{\tt \ of\ {\char'173}\ }$p$\mbox{\tt \ ->\ }$x_1$\mbox{\tt \ {\char'175}{\char'175}\ in\ }$\ldots$\\
&\mbox{\tt \ \ \ let\ {\char'173}\ }$x'_n$\mbox{\tt \ =\ case\ }$y$\mbox{\tt \ of\ {\char'173}\ }$p$\mbox{\tt \ ->\ }$x_n$\mbox{\tt \ {\char'175}{\char'175}\ in\ }$e\,[x'_1/x_1, \ldots, x'_n/x_n]$\\[2pt]
&{\rm $x_1, \ldots, x_n$ are all the variables in $p\/$; $y, x'_1, \ldots, x'_n$ are completely new variables}\\[4pt]
%\\
(d)&\mbox{\tt case\ }$e_0$\mbox{\tt \ of\ {\char'173}\ }$x${\tt @}$p$\mbox{\tt \ ->\ }$e$\mbox{\tt ;\ {\char'137}\ ->\ }$e'$\mbox{\tt \ {\char'175}}\\
&$=$\mbox{\tt \ let\ {\char'173}\ }$y$\mbox{\tt \ =\ }$e_0$\mbox{\tt \ {\char'175}\ \ \ }(where $y$ is a completely new variable)\\
&\mbox{\tt \ \ \ in\ \ case\ }$y$\mbox{\tt \ of\ {\char'173}\ }$p$\mbox{\tt \ ->\ (\ {\char'134}\ }$x$\mbox{\tt \ ->\ }$e$\mbox{\tt \ )\ }$y$\mbox{\tt \ ;\ {\char'137}\ ->\ }$e'$\mbox{\tt \ {\char'175}}\\[4pt]
%\\
(e)&\mbox{\tt case\ }$e_0$\mbox{\tt \ of\ {\char'173}\ {\char'137}\ ->\ }$e$\mbox{\tt ;\ {\char'137}\ ->\ }$e'$\mbox{\tt \ {\char'175}\ }$=$\mbox{\tt \ }$e$\\[4pt]
%\\
(f)&\mbox{\tt case\ }$e_0$\mbox{\tt \ of\ {\char'173}\ }$K\ p_1 \ldots p_n$\mbox{\tt \ ->\ }$e$\mbox{\tt ;\ {\char'137}\ ->\ }$e'$\mbox{\tt \ {\char'175}}\\
&$=$\mbox{\tt \ let\ {\char'173}\ }$y$\mbox{\tt \ =\ }$e'$\mbox{\tt \ {\char'175}}\\
&\mbox{\tt \ \ \ in\ \ case\ }$e_0$\mbox{\tt \ of\ {\char'173}}\\
&\mbox{\tt \ \ \ \ \ }$K\ x_1 \ldots x_n$\mbox{\tt \ ->\ case\ }$x_1$\mbox{\tt \ of\ {\char'173}}\\
&\mbox{\tt \ \ \ \ \ \ \ \ \ \ \ \ \ \ \ \ \ \ \ \ }$p_1$\mbox{\tt \ ->\ }$\ldots$\mbox{\tt \ case\ }$x_n$\mbox{\tt \ of\ {\char'173}\ }$p_n$\mbox{\tt \ ->\ }$e$\mbox{\tt \ ;\ {\char'137}\ ->\ }$y$\mbox{\tt \ {\char'175}\ }$\ldots$\\
&\mbox{\tt \ \ \ \ \ \ \ \ \ \ \ \ \ \ \ \ \ \ \ \ {\char'137}\ \ ->\ }$y$\mbox{\tt \ {\char'175}}\\
&\mbox{\tt \ \ \ \ \ {\char'137}\ ->\ }$y$\mbox{\tt \ {\char'175}\ }{\rm (where $y, x_1, \ldots, x_n$ are completely new variables)}\\[4pt]
%\\
(g)&\mbox{\tt case\ }$e_0$\mbox{\tt \ of\ {\char'173}\ }$k$\mbox{\tt \ ->\ }$e$\mbox{\tt ;\ {\char'137}\ ->\ }$e'$\mbox{\tt \ {\char'175}\ }$=$\mbox{\tt \ if\ (}$k$\mbox{\tt \ ==\ }$e_0$\mbox{\tt )\ then\ }$e$\mbox{\tt \ else\ }$e'$\\[4pt]
%\\
(h)&\mbox{\tt case\ }$e_0$\mbox{\tt \ of\ {\char'173}\ }$x$\mbox{\tt +}$k$\mbox{\tt \ ->\ }$e$\mbox{\tt ;\ {\char'137}\ ->\ }$e'$\mbox{\tt \ {\char'175}}\\
&$=$\mbox{\tt \ \ if\ (}$e_0$\mbox{\tt \ >=\ }$k$\mbox{\tt )\ then\ let\ {\char'173}\ }$x$\mbox{\tt \ =\ (}$e_0$\mbox{\tt -}$k$\mbox{\tt )\ {\char'175}\ in\ }$e$\mbox{\tt \ else\ }$e'$\\[4pt]
%\\
(i)&\mbox{\tt case\ }$e_0$\mbox{\tt \ of\ {\char'173}\ }$x$\mbox{\tt \ ->\ }$e$\mbox{\tt ;\ {\char'137}\ ->\ }$e'$\mbox{\tt \ {\char'175}\ }$=$\mbox{\tt \ case\ }$e_0$\mbox{\tt \ of\ {\char'173}\ }$x$\mbox{\tt \ ->\ }$e$\mbox{\tt \ {\char'175}}\\[4pt]
%\\
(j)&\mbox{\tt case\ }$e_0$\mbox{\tt \ of\ {\char'173}\ }$x$\mbox{\tt \ ->\ }$e$\mbox{\tt \ {\char'175}\ }$=$\mbox{\tt \ (\ {\char'134}\ }$x$\mbox{\tt \ ->\ }$e$\mbox{\tt \ )\ }$e_0$\\[4pt]
%\\
(k)&\mbox{\tt case\ (}$K'$\mbox{\tt \ }$e_1$\mbox{\tt \ }$\ldots$\mbox{\tt \ }$e_m$\mbox{\tt )\ of\ {\char'173}\ }$K$\mbox{\tt \ }$x_1$\mbox{\tt \ }$\ldots$\mbox{\tt \ }$x_n$\mbox{\tt \ ->\ }$e$\mbox{\tt ;\ {\char'137}\ ->\ }$e'$\mbox{\tt \ {\char'175}\ }$=$\mbox{\tt \ }$e'$\\
&{\rm where $K$ and $K'$ are distinct constructors of arity $n$ and $m$, respectively}\\[4pt]
%\\
(l)&\mbox{\tt case\ (}$K$\mbox{\tt \ }$e_1$\mbox{\tt \ }$\ldots$\mbox{\tt \ }$e_n$\mbox{\tt )\ of\ {\char'173}\ }$K$\mbox{\tt \ }$x_1$\mbox{\tt \ }$\ldots$\mbox{\tt \ }$x_n$\mbox{\tt \ ->\ }$e$\mbox{\tt ;\ {\char'137}\ ->\ }$e'$\mbox{\tt \ {\char'175}}\\
&$=$\mbox{\tt \ \ case\ }$e_1$\mbox{\tt \ of\ {\char'173}\ }$x_1$\mbox{\tt \ ->\ }$\ldots$\mbox{\tt \ \ case\ }$e_n$\mbox{\tt \ of\ {\char'173}\ }$x_n$\mbox{\tt \ ->\ }$e$\mbox{\tt \ {\char'175}}$\ldots$\mbox{\tt {\char'175}}\\
&{\rm where $K$ is a constructor of arity $n$}
\end{tabular}
}
\ecaption{Semantics of Case Expressions}
\label{simple-case-expr}
\end{figure}

Using all but the last two identities (rules~(k) and~(l)) in Figure~\ref{simple-case-expr}
in a left-to-right manner yields a translation into a
subset of general \mbox{\tt case} expressions called {\em simple case expressions}.%
\index{simple case expression}
Rule~(a) matches a general source-language
\mbox{\tt case} expression, regardless of whether it actually includes
guards---if no guards are written, then \mbox{\tt True} is substituted for the guards $g_{i,j}$
in the $match_i$ forms.
Subsequent identities manipulate the resulting \mbox{\tt case} expression into simpler
and simpler forms.
The semantics of simple \mbox{\tt case} expressions is 
given by the last two identities ((k) and~(l)).

Rules~(g) and~(h) in Figure~\ref{simple-case-expr} involve the
overloaded operators \mbox{\tt ==} and \mbox{\tt <=}; it is these rules that define the
meaning of pattern-matching against overloaded constants.
\index{pattern-matching!overloaded constant}

When used as a translation, the identities in
Figure~\ref{simple-case-expr} will generate a very inefficient
program.  This can be fixed by using further \mbox{\tt case} or \mbox{\tt let} 
expressions, but doing so 
would clutter the identities, which are intended only to convey the semantics.

These identities all preserve the static semantics.  Rules~(d) and~(j)
use a lambda rather than a \mbox{\tt let}; this indicates that variables bound
by \mbox{\tt case} are monomorphically typed (Section~\ref{type-semantics}).
\index{monomorphic type variable}

% Local Variables: 
% mode: latex
% End:
\startnewsection
%
% $Header$
%
\section{Declarations and Bindings}
\index{declaration}
\index{binding}
\label{declarations}

In this section, we describe the syntax and informal semantics of \Haskell{}
{\em declarations}.
% including their translations into
% the \Haskell{} kernel where appropriate.
% (see Appendix~\ref{formal-semantics} for a complete formal semantics).

\begin{flushleft}\it\begin{tabbing}
\hspace{0.5in}\=\hspace{3.0in}\=\kill
$\it module$\>\makebox[3.5em]{$\rightarrow$}$\it \makebox{\tt module}\ modid\ [exports]\ \makebox{\tt where}\ body$\\ 
$\it $\>\makebox[3.5em]{$|$}$\it body$\\ 
$\it body$\>\makebox[3.5em]{$\rightarrow$}$\it \makebox{\tt {\char'173}}\ [impdecls\ \makebox{\tt ;}]\ [[fixdecls\ \makebox{\tt ;}]\ topdecls\ [\makebox{\tt ;}]]\ \makebox{\tt {\char'175}}$\\ 
$\it $\>\makebox[3.5em]{$|$}$\it \makebox{\tt {\char'173}}\ impdecls\ [\makebox{\tt ;}]\ \makebox{\tt {\char'175}}$\\ 
$\it $\\ 
$\it topdecls$\>\makebox[3.5em]{$\rightarrow$}$\it topdecl_1\ \makebox{\tt ;}\ \ldots \ \makebox{\tt ;}\ topdecl_n$\>\makebox[3em]{}$\it \qquad\ (n\geq 1)$\\ 
$\it topdecl$\>\makebox[3.5em]{$\rightarrow$}$\it \makebox{\tt type}\ simple\ \makebox{\tt =}\ type$\\ 
$\it $\>\makebox[3.5em]{$|$}$\it \makebox{\tt data}\ [context\ \makebox{\tt =>}]\ simple\ \makebox{\tt =}\ constrs\ [\makebox{\tt deriving}\ (tycls\ |\ \makebox{\tt (}tyclses\makebox{\tt )})]$\\ 
$\it $\>\makebox[3.5em]{$|$}$\it \makebox{\tt class}\ [context\ \makebox{\tt =>}]\ class\ [\makebox{\tt where}\ \makebox{\tt {\char'173}}\ cbody\ [\makebox{\tt ;}]\ \makebox{\tt {\char'175}}]$\\ 
$\it $\>\makebox[3.5em]{$|$}$\it \makebox{\tt instance}\ [context\ \makebox{\tt =>}]\ tycls\ inst\ [\makebox{\tt where}\ \makebox{\tt {\char'173}}\ valdefs\ [\makebox{\tt ;}]\ \makebox{\tt {\char'175}}]$\\ 
$\it $\>\makebox[3.5em]{$|$}$\it \makebox{\tt default}\ (type\ |\ \makebox{\tt (}type_1\ \makebox{\tt ,}\ \ldots \ \makebox{\tt ,}\ type_n\makebox{\tt )})$\>\makebox[3em]{}$\it \qquad\ (n\geq 0)$\\ 
$\it $\>\makebox[3.5em]{$|$}$\it decl$\\ 
$\it $\\ 
$\it decls$\>\makebox[3.5em]{$\rightarrow$}$\it decl_1\ \makebox{\tt ;}\ \ldots \ \makebox{\tt ;}\ decl_n$\>\makebox[3em]{}$\it \qquad\ (n\geq 0)$\\ 
$\it decl$\>\makebox[3.5em]{$\rightarrow$}$\it vars\ \makebox{\tt ::}\ [context\ \makebox{\tt =>}]\ type$\\ 
$\it $\>\makebox[3.5em]{$|$}$\it valdef$
\end{tabbing}\end{flushleft}
\indexsyn{module}%
\indexsyn{body}%
\indexsyn{topdecls}%
\indexsyn{topdecl}%
\indexsyn{decls}%
\indexsyn{decl}%

The declarations in the syntactic category \mbox{$\it topdecls$} are only allowed
at the top level of a \Haskell{} module (see
Section~\ref{modules}), whereas \mbox{$\it decls$} may be used either at the top level or
in nested scopes (i.e.~those within a \mbox{\tt let} or \mbox{\tt where} construct).

For exposition, we divide the declarations into
three groups: user-defined datatypes, consisting of \mbox{\tt type} and \mbox{\tt data}
declarations (Section~\ref{user-defined-datatypes}); type classes and
overloading, consisting of \mbox{\tt class}, \mbox{\tt instance}, and \mbox{\tt default}
declarations (Section~\ref{overloading}); and nested declarations,
consisting of value bindings and type signatures
(Section~\ref{nested}).

%The \mbox{\tt module} declaration, along with \mbox{\tt import} and
%infix declarations, is described in Section~\ref{modules}.

\Haskell{} has several primitive datatypes that are ``hard-wired''
(such as integers and arrays), but most ``built-in'' datatypes are
defined in the standard prelude with normal \Haskell{} code, using
\mbox{\tt type} and \mbox{\tt data} declarations. % (see Section~\ref{user-defined-datatypes}).
These ``built-in'' datatypes are described in 
detail in Section~\ref{basic-types}.

\subsection{Overview of Types and Classes}
\label{types-overview}

\Haskell{} uses a traditional
Hindley-Milner\index{Hindley-Milner type system}
polymorphic type system to provide a static type semantics
\cite{damas-milner82,hindley69}, but the type system has been extended with
{\em type classes} (or just {\em classes}\index{class}) that provide
a structured way to introduce {\em overloaded} functions.
This is the major technical innovation in \Haskell{}.

A \mbox{\tt class} declaration (Section~\ref{class-decls}) introduces a new
{\em type class} and the overloaded {\em operations} that must be
supported by any type that is an instance of that class.  An
\mbox{\tt instance} declaration (Section~\ref{instance-decls}) declares that a
type is an {\em instance} of a class and includes
the definitions of the overloaded operations---called {\em
methods}---instantiated on the named type.
\index{class method}

For example, suppose we wish to overload the operations \mbox{\tt (+)} and
\mbox{\tt negate} on types \mbox{\tt Int} and \mbox{\tt Float}.  We introduce a new
type class called \mbox{\tt Num}:\nopagebreak[4]
\bprog
\mbox{\tt class\ Num\ a\ \ where\ \ \ \ \ \ \ \ \ \ --\ simplified\ class\ declaration\ for\ Num}\\
\mbox{\tt \ \ (+)\ \ \ \ ::\ a\ ->\ a\ ->\ a}\\
\mbox{\tt \ \ negate\ ::\ a\ ->\ a}
\eprog
This declaration may be read ``a type \mbox{\tt a} is an instance of the class
\mbox{\tt Num} if there are (overloaded) operations \mbox{\tt (+)} and \mbox{\tt negate}, of the
appropriate types, defined on it.''

We may then declare \mbox{\tt Int} and \mbox{\tt Float} to be instances of this class:
\bprog
\mbox{\tt instance\ Num\ Int\ \ where\ \ \ \ \ --\ simplified\ instance\ of\ Num\ Int}\\
\mbox{\tt \ \ x\ +\ y\ \ \ \ \ \ \ =\ \ addInt\ x\ y}\\
\mbox{\tt \ \ negate\ x\ \ \ \ =\ \ negateInt\ x}\\
\mbox{\tt }\\[-8pt]
\mbox{\tt instance\ Num\ Float\ \ where\ \ \ --\ simplified\ instance\ of\ Num\ Float}\\
\mbox{\tt \ \ x\ +\ y\ \ \ \ \ \ \ =\ \ addFloat\ x\ y}\\
\mbox{\tt \ \ negate\ x\ \ \ \ =\ \ negateFloat\ x}
\eprog
where \mbox{\tt addInt}, \mbox{\tt negateInt}, \mbox{\tt addFloat}, and \mbox{\tt negateFloat} are assumed
in this case to be primitive functions, but in general could be any
user-defined function.  The first declaration above may be read
``\mbox{\tt Int} is an instance of the class \mbox{\tt Num} as witnessed by these
definitions (i.e.~methods)\index{class method} for \mbox{\tt (+)} and \mbox{\tt negate}.''

More examples can be found in Wadler and Blott's paper
\cite{wadler:classes}.

\subsubsection{Syntax of Types}
\label{type-syntax}
\index{type}
\label{types}

\begin{flushleft}\it\begin{tabbing}
\hspace{0.5in}\=\hspace{3.0in}\=\kill
$\it type$\>\makebox[3.5em]{$\rightarrow$}$\it atype$\\ 
$\it $\>\makebox[3.5em]{$|$}$\it type_1\ \makebox{\tt ->}\ type_2$\\ 
$\it $\>\makebox[3.5em]{$|$}$\it tycon\ atype_1\ \ldots \ atype_k$\>\makebox[3em]{}$\it (\arity{tycon}=k\geq 1)$\\ 
$\it $\\ 
$\it atype$\>\makebox[3.5em]{$\rightarrow$}$\it tyvar$\\ 
$\it $\>\makebox[3.5em]{$|$}$\it tycon$\>\makebox[3em]{}$\it (\arity{tycon}=0)$\\ 
$\it $\>\makebox[3.5em]{$|$}$\it \makebox{\tt ()}$\>\makebox[3em]{}$\it (\tr{unit\ type})$\\ 
$\it $\>\makebox[3.5em]{$|$}$\it \makebox{\tt (}\ type\ \makebox{\tt )}$\>\makebox[3em]{}$\it (\tr{parenthesised\ type})$\\ 
$\it $\>\makebox[3.5em]{$|$}$\it \makebox{\tt (}\ type_1\ \makebox{\tt ,}\ \ldots \ \makebox{\tt ,}\ type_k\ \makebox{\tt )}$\>\makebox[3em]{}$\it (\tr{tuple\ type},\ k\geq 2)$\\ 
$\it $\>\makebox[3.5em]{$|$}$\it \makebox{\tt [}\ type\ \makebox{\tt ]}$
\end{tabbing}\end{flushleft}
\indexsyn{type}%
\indexsyn{atype}%
%\ToDo{I left off \mbox{$\it tyvar$} and \mbox{$\it tycon$}}

\noindent
The syntax for \Haskell{}
{\em type expressions} is given above.\index{type expression} 
They are built in the usual way
from type variables, function types, type constructors, tuple types,
and list types.  Type variables are identifiers beginning with a lower-case
letter and type constructors are identifiers beginning with an upper-case
letter.  A type is one of:\nopagebreak[4]
\begin{enumerate}
\item A {\em function type}\index{function type} having form 
\mbox{$\it t_1\ \makebox{\tt ->}\ t_2$}.  Function arrows associate to the right.

\item A {\em constructed type}\index{constructed type} having form 
\mbox{$\it T\ t_1\ \ldots \ t_k$}, where \mbox{$\it T$} is a type constructor of arity \mbox{$\it k$}.

\item A {\em tuple type}\index{tuple type} having form 
\mbox{$\it \makebox{\tt (}t_1\makebox{\tt ,}\ \ldots \makebox{\tt ,}\ t_k\makebox{\tt )}$} where \mbox{$\it k\geq 2$}.  It denotes the type of
\mbox{$\it k$}-tuples with the first component of type \mbox{$\it t_1$}, the second
component of type \mbox{$\it t_2$}, and so on (see Sections~\ref{tuples}
and \ref{basic-tuples}).

\item A {\em list type}\index{list type} has the form \mbox{$\it \makebox{\tt [}t\makebox{\tt ]}$}.  
It denotes the type of lists with elements of type \mbox{$\it t$} (see
Sections~\ref{lists} and \ref{basic-lists}).

\item The {\em trivial type}\index{trivial type} having form \mbox{\tt ()}.
% was ``degenerate tuple'' (mismatched exps.verb)
It denotes the ``nullary tuple'' type, and has exactly one value,
also written \mbox{\tt ()} (see Sections~\ref{unit-expression}
and~\ref{basic-trivial}).

\item A {\em parenthesised type}, having form \mbox{$\it \makebox{\tt (}t\makebox{\tt )}$}, is identical to
the type \mbox{$\it t$}.
\end{enumerate}

Although the tuple, list, and trivial types have special syntax, they
are not different from user-defined types with equivalent
functionality.

% \outline{
% \paragraph{Translation:}  
% The list type \mbox{$\it \makebox{\tt [}t\makebox{\tt ]}$} is equivalent to the constructed type 
% \mbox{$\it \makebox{\tt List}\ t$}, where \mbox{\tt List} is a datatype defined in the standard prelude
% (see Section~\ref{lists}).  Similarly, the tuple type
% \mbox{$\it \makebox{\tt (}t_1\makebox{\tt ,}\ldots \makebox{\tt ,}t_k\makebox{\tt )}$} is syntax for \mbox{$\it \makebox{\tt Tuple}k\ t_1\ \ldots \ t_k$}, where
% \mbox{$\it \makebox{\tt Tuple}k$} is a type constructor implicitly defined for k-tuples in
% the standard prelude (see Section~\ref{tuples}).  Finally, the
% trivial type \mbox{\tt ()} is syntax for the type \mbox{\tt Triv} defined in the
% standard prelude (see Section~\ref{basic-trivial}).
% }

Expressions and types have a consistent syntax.
If \mbox{$\it t_i$} is the type of
expression or pattern \mbox{$\it e_i$}, then the expressions \mbox{$\it \makebox{\tt ({\char'134}}\ e_1\ \makebox{\tt ->}\ e_2\makebox{\tt )}$},
\mbox{$\it \makebox{\tt [}e_1\makebox{\tt ]}$}, and \mbox{$\it \makebox{\tt (}e_1,e_2\makebox{\tt )}$} have the types \mbox{$\it \makebox{\tt (}t_1\ \makebox{\tt ->}\ t_2\makebox{\tt )}$},
\mbox{$\it \makebox{\tt [}t_1\makebox{\tt ]}$}, and \mbox{$\it \makebox{\tt (}t_1,t_2\makebox{\tt )}$}, respectively.

With one exception, the type variables in a \Haskell{} type expression
are all assumed to be universally quantified; there is no explicit
syntax for universal quantification \cite{damas-milner82,reynolds90}.
For example, the type expression
\mbox{\tt a\ ->\ a} denotes the type $\forall a.~a \rightarrow a$.
For clarity, however, we will often write quantification explicitly
when discussing the types of \Haskell{} programs.

The exception referred to is that of the distinguished type variable
in a class declaration (Section~\ref{class-decls}).

%Every type variable appearing in a signature
%is universally quantified over that signature.  This last
%constraint implies that signatures such as:
%\bprog
%@
%       \ x -> ([x] :: [a])
%@
%\eprog
%are not valid, because this declares \mbox{\tt [x]} to be of type 
%\mbox{$\it \forall\ a.\makebox{\tt [}a\makebox{\tt ]}$}.  In contrast, this {\em is} a valid signature:
%\mbox{\tt ({\char'134}\ x\ ->\ [x])\ ::\ a\ ->\ [a]}; it declares that \mbox{\tt ({\char'134}\ x\ ->\ [x])} has type
%\mbox{$\it \forall\ a.a\ \makebox{\tt ->}\ \makebox{\tt [}a\makebox{\tt ]}$}.

\subsubsection{Syntax of Class Assertions and Contexts}
\index{class assertion}
\index{context}
\label{classes&contexts}

\begin{flushleft}\it\begin{tabbing}
\hspace{0.5in}\=\hspace{3.0in}\=\kill
$\it context$\>\makebox[3.5em]{$\rightarrow$}$\it class$\\ 
$\it $\>\makebox[3.5em]{$|$}$\it \makebox{\tt (}\ class_1\ \makebox{\tt ,}\ \ldots \ \makebox{\tt ,}\ class_n\ \makebox{\tt )}$\>\makebox[3em]{}$\it (n\geq 1)$\\ 
$\it class$\>\makebox[3.5em]{$\rightarrow$}$\it tycls\ tyvar$\\ 
$\it tycls$\>\makebox[3.5em]{$\rightarrow$}$\it aconid$\\ 
$\it tyvar$\>\makebox[3.5em]{$\rightarrow$}$\it avarid$
\end{tabbing}\end{flushleft}
\indexsyn{context}%
\indexsyn{class}%
\indexsyn{tycls}%
\indexsyn{tyvar}%
A {\em class assertion} has form \mbox{$\it tycls\ tyvar$}, and
indicates the membership of the parameterised type \mbox{$\it tyvar$} in the class
\mbox{$\it tycls$}.  A class identifier begins with a capital
letter.

A {\em context} consists of one or more class assertions,
and has the general form
\[
\mbox{$\it \makebox{\tt (}\ C_1\ u_1,\ \ldots ,\ C_n\ u_n\ \makebox{\tt )}$}
\]
where \mbox{$\it C_1,\ \ldots ,\ C_n$} are class identifiers, and \mbox{$\it u_1,\ \ldots ,\ u_n$} are
type variables; the parentheses may be omitted when \mbox{$\it n=1$}.  In
general, we use \mbox{$\it c$} to denote a context and we write \mbox{$\it c\ \makebox{\tt =>}\ t$} to
indicate the type \mbox{$\it t$} restricted by the context \mbox{$\it c$}.
The context \mbox{$\it c$} must only contain type variables referenced in \mbox{$\it t$}.
For convenience,
we write \mbox{$\it c\ \makebox{\tt =>}\ t$} even if the context \mbox{$\it c$} is empty, although in this
case the concrete syntax contains no \mbox{\tt =>}.

\subsubsection{Semantics of Types and Classes}
\label{type-semantics}

In this subsection, we provide informal details of the type system.
% the formal semantics is described in Appendix~\ref{static-semantics}
(Wadler and Blott \cite{wadler:classes} discuss type
classes further.)

%A type is a {\em monotype\/}\index{monotype} if it contains no type
%variables, and is {\em monomorphic\/}
%\index{monomorphic type}
%if it contains type variables
%but is not polymorphic (in Milner's original terminology,
%it is monomorphic if it contains no generic type variables).

The \Haskell{} type system attributes a {\em type} to each
\index{type}
expression in the program.  In general, a type is of the form
$\forall \overline{u}.~c \Rightarrow t$,
where $\overline{u}$ is a set of type variables \mbox{$\it u_1,\ \ldots ,\ u_n$}.
In any such type, any of the universally-quantified type variables \mbox{$\it u_i$}
which are free in \mbox{$\it c$} must also be free in \mbox{$\it t$}.

The type of an expression $e$
depends on a {\em type environment}
\index{type environment}
that gives types for the free variables in \mbox{$\it e$}, and a
{\em class environment}
\index{class environment}
that declares which types are instances of which classes (a type becomes
an instance of a class only via the presence of an
\mbox{\tt instance} declaration or a \mbox{\tt deriving} clause).

Types are related by a generalisation order
\index{generalisation order}
(specified below);
the most general type that can be assigned to a particular
expression (in a given environment) is called its {\em
principal type}.
\index{principal type}
\Haskell{}'s extended Hindley-Milner type system can infer the principal
type of all expressions, including the proper use of overloaded
operations (although certain ambiguous overloadings could arise, as
described in Section~\ref{default-decls}).  Therefore, explicit typings (called
{\em type signatures})
\index{type signature}
are optional (see
Sections~\ref{expression-type-sigs} and~\ref{type-signatures}).

The type $\forall \overline{u}.~c_1 \Rightarrow t_1$ is
{\em more general than}
the type $\forall \overline{w}.~c_2 \Rightarrow t_2$ if and only if there is 
a substitution \mbox{$\it S$} whose domain is $\overline{u}$ such that:
\begin{itemize}
\item \mbox{$\it t_2$} is identical to \mbox{$\it S(t_1)$}.
\item Whenever \mbox{$\it c_2$} holds in the class environment, \mbox{$\it S(c_1)$} also holds.
\end{itemize}

The main point about contexts above is that, given the type
$\forall \overline{u}.~c \Rightarrow t$,
the presence of \mbox{$\it C\ u_i$} in the context \mbox{$\it c$} expresses the
constraint that the type variable \mbox{$\it u_i$} may be instantiated as \mbox{$\it t'$}
within the type expression \mbox{$\it t$} only if \mbox{$\it t'$} is a member of the class
\mbox{$\it C$}.  For example, consider the function \mbox{\tt double}:
\bprog
\mbox{\tt \ \ \ \ \ \ \ \ double\ x\ =\ x\ +\ x}
\eprog
The most general type of \mbox{\tt double} is
$\forall a.~\mbox{\tt Num}~a \Rightarrow a \rightarrow a$.
\mbox{\tt double} may be applied to values of type \mbox{\tt Int} (instantiating \mbox{$\it a$} to
\mbox{\tt Int}), since \mbox{\tt Int} is an instance of the class \mbox{\tt Num}.  However,
\mbox{\tt double} may not be applies to values of type \mbox{\tt Char}, because \mbox{\tt Char}
is not an instance of class \mbox{\tt Num}.


\subsection{User-Defined Datatypes}
\index{datatype}
\label{user-defined-datatypes}

In this section, we describe algebraic datatypes (\mbox{\tt data} declarations)
and type synonyms (\mbox{\tt type} declarations).  These declarations
may only appear at the top level of a module.

\subsubsection{Algebraic Datatype Declarations}
\index{algebraic datatype}
\label{datatype-decls}

\begin{flushleft}\it\begin{tabbing}
\hspace{0.5in}\=\hspace{3.0in}\=\kill
$\it topdecl$\>\makebox[3.5em]{$\rightarrow$}$\it \makebox{\tt data}\ [context\ \makebox{\tt =>}]\ simple\ \makebox{\tt =}\ constrs\ [\makebox{\tt deriving}\ (tycls\ |\ \makebox{\tt (}tyclses\makebox{\tt )})]$\\ 
$\it simple$\>\makebox[3.5em]{$\rightarrow$}$\it tycon\ tyvar_1\ \ldots \ tyvar_k$\>\makebox[3em]{}$\it (\arity{tycon}=k\geq 0)$\\ 
$\it constrs$\>\makebox[3.5em]{$\rightarrow$}$\it constr_1\ \makebox{\tt |}\ \ldots \ \makebox{\tt |}\ constr_n$\>\makebox[3em]{}$\it (n\geq 1)$\\ 
$\it constr$\>\makebox[3.5em]{$\rightarrow$}$\it con\ atype_1\ \ldots \ atype_k$\>\makebox[3em]{}$\it (\arity{con}=k\geq 0)$\\ 
$\it $\>\makebox[3.5em]{$|$}$\it type_1\ conop\ type_2$\>\makebox[3em]{}$\it (\infix{conop})$\\ 
$\it tyclses$\>\makebox[3.5em]{$\rightarrow$}$\it tycls_1\makebox{\tt ,}\ \ldots \makebox{\tt ,}\ tycls_n$\>\makebox[3em]{}$\it (n\geq 0)$
\end{tabbing}\end{flushleft}
\index{topdecl@{\em topdecl} (\mbox{\tt data})}%
\indexsyn{simple}%
\indexsyn{constrs}%
\indexsyn{constr}%
\indexsyn{tyclses}%
\index{data declaration@{\ptt data} declaration}
The precedence\index{precedence} for \mbox{$\it constr$} is the same as that for
expressions---normal constructor application has higher precedence
than infix constructor application (thus \mbox{\tt a\ :\ Foo\ a} parses as 
\mbox{\tt a\ :\ (Foo\ a)}).

An algebraic datatype declaration introduces a new type
and constructors over that type and has the form:
\[
\mbox{$\it \makebox{\tt data}\ c\ \makebox{\tt =>}\ T\ u_1\ \ldots \ u_k\ \makebox{\tt =}\ K_1\ t_{11}\ \ldots \ t_{1k_1}\ \makebox{\tt |}\ \cdots\ \makebox{\tt |}\ K_n\ t_{n1}\ \ldots \ t_{nk_n}$}
\]
where \mbox{$\it c$} is a context.
\index{context!in data declaration@in {\ptt data} declaration}
This declaration
introduces a new type constructor \mbox{$\it T$} with constituent data
constructors \mbox{$\it K_1,\ \ldots ,\ K_n$} whose types are given by:
\[
\mbox{$\it K_i\ ::\ \forall\ u_1\ \ldots \ u_k.~\ c_i\ \Rightarrow\ t_{i1}\ \rightarrow\ \cdots\ \rightarrow\ t_{ik_i}\ \rightarrow\ (T\ u_1\ \ldots \ u_k)$}
\]
where \mbox{$\it c_i$} is the largest subset of \mbox{$\it c$} that constrains only those type
variables free in the types \mbox{$\it t_{i1},\ \ldots ,\ t_{ik_i}$}.
The type variables \mbox{$\it u_1$} through \mbox{$\it u_k$} must be distinct and may appear
in \mbox{$\it c$} and the $t_{ij}$; it is a static error
for any other type variable to appear in \mbox{$\it c$} or on the right-hand-side.

For example, the declaration
\bprog
\mbox{\tt \ \ \ \ \ \ \ \ data\ Eq\ a\ =>\ Set\ a\ =\ NilSet\ |\ ConsSet\ a\ (Set\ a)}
\eprog
introduces a type constructor \mbox{\tt Set}, and constructors \mbox{\tt NilSet} and
\mbox{\tt ConsSet} with types
\[\begin{array}{ll}
\mbox{\tt NilSet}  & \mbox{$\it ::\ \forall\ a.~\ \makebox{\tt Set}~\ a$} \\
\mbox{\tt ConsSet} & \mbox{$\it ::\ \forall\ a.~\ \makebox{\tt Eq}~\ a\ \Rightarrow\ a\ \rightarrow\ \makebox{\tt Set}~\ a\ \rightarrow\ \makebox{\tt Set}~\ a$}
\end{array}\]
In the example given, the overloaded
type for \mbox{\tt ConsSet} ensures that \mbox{\tt ConsSet} can only be applied to values whose
type is an instance of the class \mbox{\tt Eq}.  The context in the \mbox{\tt data}
declaration has no other effect whatsoever.  In particular, pattern
matching is unaffected.

The visibility of a datatype's constructors (i.e.~the ``abstractness''
of the datatype) outside of the module in which the datatype is
defined is controlled by the form of the datatype's name in the export
list as described in Section~\ref{abstract-types}.

The optional \mbox{$\it \makebox{\tt deriving}$} part of a \mbox{\tt data} declaration has to do
with {\em derived instances}, and is described in
Section~\ref{derived-decls}.

\subsubsection{Type Synonym Declarations}
\index{type synonym}
\label{type-synonym-decls}

\begin{flushleft}\it\begin{tabbing}
\hspace{0.5in}\=\hspace{3.0in}\=\kill
$\it topdecl$\>\makebox[3.5em]{$\rightarrow$}$\it \makebox{\tt type}\ simple\ \makebox{\tt =}\ type$\\ 
$\it simple$\>\makebox[3.5em]{$\rightarrow$}$\it tycon\ tyvar_1\ \ldots \ tyvar_k$\>\makebox[3em]{}$\it (\arity{tycon}=k\geq 0)$
\end{tabbing}\end{flushleft}
\index{topdecl@{\em topdecl} (\mbox{\tt type})}%
\indexsyn{simple}%
A type synonym declaration introduces a new type that
is equivalent to an old type and has the form
\[
\mbox{$\it \makebox{\tt type}\ T\ u_1\ \ldots \ u_k\ \makebox{\tt =}\ t$}
\]
which introduces a new type constructor, \mbox{$\it T$}.  The type \mbox{$\it (T\ t_1\ \ldots \\
\it t_k)$} is equivalent to the type \mbox{$\it t[t_1/u_1,\ \ldots ,\ t_k/u_k]$}.  The type
variables \mbox{$\it u_1$} through \mbox{$\it u_k$} must be distinct and are scoped only
over \mbox{$\it t$}; it is a static error for any other type variable to appear
in \mbox{$\it t$}.

Although recursive and mutually recursive datatypes are allowed,
\index{recursive datatype}
\index{type synonym!recursive}
this is not so for type synonyms, {\em unless an algebraic datatype
intervenes}.  For example,
\bprog
\mbox{\tt type\ Rec\ a\ \ \ =\ \ [Circ\ a]}\\
\mbox{\tt data\ Circ\ a\ \ =\ \ Tag\ [Rec\ a]}
\eprog
is allowed, whereas
\bprog
\mbox{\tt type\ Rec\ a\ \ \ =\ \ [Circ\ a]\ \ \ \ \ \ \ \ --\ ILLEGAL}\\
\mbox{\tt type\ Circ\ a\ \ =\ \ [Rec\ a]\ \ \ \ \ \ \ \ \ --}
\eprog
is not. Similarly, \mbox{\tt type\ Rec\ a\ =\ [Rec\ a]} is not allowed.

\subsection{Type Classes and Overloading}
\index{class}
\index{overloading}
\label{overloading}
\label{classes}

\subsubsection{Class Declarations}
\index{class declaration}
%\index{class declaration@{\ptt class} declaration}
\label{class-decls}

\begin{flushleft}\it\begin{tabbing}
\hspace{0.5in}\=\hspace{3.0in}\=\kill
$\it topdecl$\>\makebox[3.5em]{$\rightarrow$}$\it \makebox{\tt class}\ [context\ \makebox{\tt =>}]\ class\ [\makebox{\tt where}\ \makebox{\tt {\char'173}}\ cbody\ [\makebox{\tt ;}]\ \makebox{\tt {\char'175}}]$\\ 
$\it cbody$\>\makebox[3.5em]{$\rightarrow$}$\it [csigns\ \makebox{\tt ;}]\ [valdefs]$\\ 
$\it csigns$\>\makebox[3.5em]{$\rightarrow$}$\it csign_1\ \makebox{\tt ;}\ \ldots \ \makebox{\tt ;}\ csign_n$\>\makebox[3em]{}$\it (n\geq 1)$\\ 
$\it csign$\>\makebox[3.5em]{$\rightarrow$}$\it vars\ \makebox{\tt ::}\ [context\ \makebox{\tt =>}]\ type$\\ 
$\it vars$\>\makebox[3.5em]{$\rightarrow$}$\it var_1\ \makebox{\tt ,}\ \ldots \makebox{\tt ,}\ var_n$\>\makebox[3em]{}$\it (n\geq 1)$
\end{tabbing}\end{flushleft}
\index{topdecl@{\em topdecl} (\mbox{\tt class})}%
\indexsyn{cbody}%
\indexsyn{csigns}%
\indexsyn{csign}%
\indexsyn{vars}%
\index{declaration!within a {\ptt class} declaration}
%\ToDo{tycls and tyvar left off above}

\noindent
A {\em class declaration} introduces a new class and the operations on it.
A class declaration has the general form:
\[\begin{array}{rl}
\mbox{$\it \makebox{\tt class}\ c\ \makebox{\tt =>}\ C\ u\ \makebox{\tt where}\ \makebox{\tt {\char'173}}$}&\mbox{$\it v_1\ \makebox{\tt ::}\ c_1\ \makebox{\tt =>}\ t_1\ \makebox{\tt ;}\ \ldots \ \makebox{\tt ;}\ v_n\ \makebox{\tt ::}\ c_n\ \makebox{\tt =>}\ t_n\ \makebox{\tt ;}$}\\
                                &\mbox{$\it valdef_1\ \makebox{\tt ;}\ \ldots \ \makebox{\tt ;}\ valdef_m\ \makebox{\tt {\char'175}}$}
\end{array}\]
This introduces a new class name \mbox{$\it C$}; the type variable \mbox{$\it u$} is
scoped only over the method signatures in the class body.
The context \mbox{$\it c$} specifies the superclasses\index{superclass} of \mbox{$\it C$}, as
described below; the only type variable that may be referred to in \mbox{$\it c$}
is \mbox{$\it u$}.
The class declaration introduces new {\em class methods}
\index{class method}
\mbox{$\it v_1,\ \ldots ,\ v_n$}, whose scope extends outside the \mbox{\tt class} declaration,
with types:
\[
v_i :: \forall u,\overline{w}.~(C u, c_i) \Rightarrow t_i
\]
%old:
%Note the implicit context in the types for each \mbox{$\it v_i$}.  
The \mbox{$\it t_i$} must mention \mbox{$\it u$}; they may mention type variables
$\overline{w}$ other than \mbox{$\it u$}, and the type of \mbox{$\it v_i$} is
polymorphic in both \mbox{$\it u$} and $\overline{w}\/$.
The \mbox{$\it c_i$} may constrain only $\overline{w}$; in particular,
the \mbox{$\it c_i$} may not constrain \mbox{$\it u$}.
For example:
\bprog
\mbox{\tt \ \ \ \ \ \ \ \ class\ Foo\ a\ where}\\
\mbox{\tt \ \ \ \ \ \ \ \ \ \ \ \ \ \ \ \ op\ ::\ Num\ b\ =>\ a\ ->\ b\ ->\ a}
\eprog
Here the type of \mbox{\tt op} is
$\forall a, b.~(\mbox{\tt Foo}~a,~\mbox{\tt Num}~b)~ \Rightarrow a \rightarrow b \rightarrow a$.

{\em Default methods}
\index{default method}
for any of the \mbox{$\it v_i$} may be included in the
\mbox{\tt class} declaration as a normal \mbox{$\it valdef$}; no other definitions are
permitted.  The default method for \mbox{$\it v_i$} is used if no binding for it
is given in a particular \mbox{\tt instance} declaration (see
Section~\ref{instance-decls}).

Two classes in scope at the same time may not share any of the same
methods.

\begin{figure}
\outline{
\mbox{\tt class\ \ Eq\ a\ \ where\ \ \ \ \ \ \ \ \ \ \ \ \ \ \ \ \ \ \ \ \ \ \ \ \ \ \ \ \ \ \ --\ \ \ Eq}\\
\mbox{\tt \ \ \ \ \ \ \ \ (==),\ (/=)\ \ ::\ \ a\ ->\ a\ ->\ Bool\ \ \ \ \ \ \ \ \ \ \ --\ \ \ \ |}\\
\mbox{\tt \ \ \ \ \ \ \ \ \ \ \ \ \ \ \ \ \ \ \ \ \ \ \ \ \ \ \ \ \ \ \ \ \ \ \ \ \ \ \ \ \ \ \ \ \ \ \ \ \ --\ \ \ Ord}\\
\mbox{\tt \ \ \ \ \ \ \ \ x\ /=\ y\ \ \ =\ \ not\ (x\ ==\ y)\ \ \ \ \ \ \ \ \ \ \ \ \ \ \ \ \ --\ \ \ /\ {\char'134}}\\
\mbox{\tt \ \ \ \ \ \ \ \ \ \ \ \ \ \ \ \ \ \ \ \ \ \ \ \ \ \ \ \ \ \ \ \ \ \ \ \ \ \ \ \ \ \ \ \ \ \ \ \ \ --\ \ Ix\ Enum}\\
\mbox{\tt class\ \ (Eq\ a)\ =>\ Ord\ a\ \ where}\\
\mbox{\tt \ \ \ \ \ \ \ \ (<),\ (<=),\ (>=),\ (>)\ ::\ \ a\ ->\ a\ ->\ Bool}\\
\mbox{\tt \ \ \ \ \ \ \ \ max,\ min\ \ \ \ \ \ \ \ \ \ \ \ \ ::\ \ a\ ->\ a\ ->\ a}\\
\mbox{\tt }\\[-8pt]
\mbox{\tt \ \ \ \ \ \ \ \ x\ <\ \ y\ \ \ \ \ \ \ \ \ \ \ \ \ \ \ \ =\ \ x\ <=\ y\ {\char'46}{\char'46}\ x\ /=\ y}\\
\mbox{\tt \ \ \ \ \ \ \ \ x\ >=\ y\ \ \ \ \ \ \ \ \ \ \ \ \ \ \ \ =\ \ y\ <=\ x}\\
\mbox{\tt \ \ \ \ \ \ \ \ x\ >\ \ y\ \ \ \ \ \ \ \ \ \ \ \ \ \ \ \ =\ \ y\ <\ \ x}\\
\mbox{\tt \ \ \ \ \ \ \ \ max\ x\ y\ |\ x\ >=\ y\ \ \ \ \ \ =\ \ x}\\
\mbox{\tt \ \ \ \ \ \ \ \ \ \ \ \ \ \ \ \ |\ y\ >=\ x\ \ \ \ \ \ =\ \ y}\\
\mbox{\tt \ \ \ \ \ \ \ \ min\ x\ y\ |\ x\ <=\ y\ \ \ \ \ \ =\ \ x}\\
\mbox{\tt \ \ \ \ \ \ \ \ \ \ \ \ \ \ \ \ |\ y\ <=\ x\ \ \ \ \ \ =\ \ y}\\
\mbox{\tt }\\[-8pt]
\mbox{\tt class\ \ Text\ a\ \ where}\\
\mbox{\tt \ \ \ \ \ \ \ \ showsPrec\ ::\ Int\ ->\ a\ ->\ String\ ->\ String}\\
\mbox{\tt \ \ \ \ \ \ \ \ readsPrec\ ::\ Int\ ->\ String\ ->\ [(a,String)]}\\
\mbox{\tt \ \ \ \ \ \ \ \ showList\ \ ::\ [a]\ ->\ String\ ->\ String}\\
\mbox{\tt \ \ \ \ \ \ \ \ readList\ \ ::\ String\ ->\ [([a],String)]}\\
\mbox{\tt }\\[-8pt]
\mbox{\tt \ \ \ \ \ \ \ \ showList\ =\ ...\ --\ see\ Appendix\ A\ \ }\\
\mbox{\tt \ \ \ \ \ \ \ \ readList\ =\ ...\ --\ see\ Appendix\ A\ \ }\\
\mbox{\tt }\\[-8pt]
\mbox{\tt class\ \ Binary\ a\ \ where\ \ \ \ \ \ \ \ \ \ \ \ \ \ \ \ \ \ \ \ \ \ }\\
\mbox{\tt \ \ \ \ \ \ \ \ showBin\ ::\ a\ ->\ Bin\ ->\ Bin\ }\\
\mbox{\tt \ \ \ \ \ \ \ \ readBin\ ::\ Bin\ ->\ (a,Bin)\ \ }\\
\mbox{\tt }\\[-8pt]
\mbox{\tt class\ \ (Ord\ a)\ =>\ Ix\ a\ \ where}\\
\mbox{\tt \ \ \ \ \ \ \ \ range\ \ \ ::\ (a,a)\ ->\ [a]}\\
\mbox{\tt \ \ \ \ \ \ \ \ index\ \ \ ::\ (a,a)\ ->\ a\ ->\ Int}\\
\mbox{\tt \ \ \ \ \ \ \ \ inRange\ ::\ (a,a)\ ->\ a\ ->\ Bool}\\
\mbox{\tt }\\[-8pt]
\mbox{\tt class\ \ (Ord\ a)\ =>\ Enum\ a\ \ where}\\
\mbox{\tt \ \ \ \ \ \ \ \ enumFrom\ \ \ \ \ \ \ ::\ a\ ->\ [a]\ \ \ \ \ \ \ \ \ \ \ \ \ --\ [n..]}\\
\mbox{\tt \ \ \ \ \ \ \ \ enumFromThen\ \ \ ::\ a\ ->\ a\ ->\ [a]\ \ \ \ \ \ \ \ --\ [n,n'..]}\\
\mbox{\tt \ \ \ \ \ \ \ \ enumFromTo\ \ \ \ \ ::\ a\ ->\ a\ ->\ [a]\ \ \ \ \ \ \ \ --\ [n..m]}\\
\mbox{\tt \ \ \ \ \ \ \ \ enumFromThenTo\ ::\ a\ ->\ a\ ->\ a\ ->\ [a]\ \ \ --\ [n,n'..m]}\\
\mbox{\tt }\\[-8pt]
\mbox{\tt \ \ \ \ \ \ \ \ enumFromTo\ n\ m\ \ \ \ \ \ \ \ =\ takeWhile\ ((>=)\ m)\ (enumFrom\ n)}\\
\mbox{\tt \ \ \ \ \ \ \ \ enumFromThenTo\ n\ n'\ m\ =\ takeWhile}\\
\mbox{\tt \ \ \ \ \ \ \ \ \ \ \ \ \ \ \ \ \ \ \ \ \ \ \ \ \ \ \ \ \ \ \ \ \ \ ((if\ n'\ >=\ n\ then\ (>=)\ else\ (<=))\ m)}\\
\mbox{\tt \ \ \ \ \ \ \ \ \ \ \ \ \ \ \ \ \ \ \ \ \ \ \ \ \ \ \ \ \ \ \ \ \ \ (enumFromThen\ n\ n')}
}
\ecaption{Standard Classes and Associated Functions}
\label{standard-classes}
\indextt{Eq}\indextt{==}\indextt{/=}
\indextt{Ord}\indextt{<}\indextt{<=}\indextt{>}\indextt{>=}\indextt{max}\indextt{min}
\indextt{Text}\indextt{showsPrec}\indextt{readsPrec}\indextt{showList}\indextt{readList}
\indextt{Binary}\indextt{showBin}\indextt{readBin}
\indextt{Ix}\indextt{range}\indextt{index}\indextt{inRange}
\indextt{Enum}\indextt{enumFrom}\indextt{enumFromThen}
\indextt{enumFromTo}\indextt{enumFromThenTo}
\end{figure}

Figure~\ref{standard-classes} shows some standard \Haskell{}
classes, including the use of superclasses; note the class inclusion
diagram on the right.  For example, \mbox{\tt Eq} is a superclass of \mbox{\tt Ord}, and
thus in any context \mbox{\tt Ord\ a} is equivalent to \mbox{\tt (Eq\ a,\ Ord\ a)}.  

A \mbox{\tt class}
declaration with no \mbox{\tt where} part
\index{class declaration!with an empty \mbox{\tt where} part}
may be useful for combining a
collection of classes into a larger one that inherits all of the
operations in the original ones.  For example:
\bprog
\mbox{\tt class\ \ (Ord\ a,\ Text\ a,\ Binary\ a)\ =>\ Data\ a}
\eprog
In such a case, if a type is an instance of all superclasses,\index{superclass} it is
not {\em automatically} an instance of the subclass, even though the
subclass has no immediate operations.  The \mbox{\tt instance} declaration must be
given explicitly, and it must have an empty \mbox{\tt where} part as well.
\index{instance declaration!with an empty \mbox{\tt where} part}

The superclass relation must not be cyclic; i.e.~it must form a
directed acyclic graph.\index{superclass}

\subsubsection{Instance Declarations}
\label{instance-decls}
\index{instance declaration}

\begin{flushleft}\it\begin{tabbing}
\hspace{0.5in}\=\hspace{3.0in}\=\kill
$\it topdecl$\>\makebox[3.5em]{$\rightarrow$}$\it \makebox{\tt instance}\ [context\ \makebox{\tt =>}]\ tycls\ inst\ [\makebox{\tt where}\ \makebox{\tt {\char'173}}\ valdefs\ [\makebox{\tt ;}]\ \makebox{\tt {\char'175}}]$\\ 
$\it inst$\>\makebox[3.5em]{$\rightarrow$}$\it tycon$\>\makebox[3em]{}$\it (\arity{tycon}=0)$\\ 
$\it $\>\makebox[3.5em]{$|$}$\it \makebox{\tt (}\ tycon\ tyvar_1\ \ldots \ tyvar_k\ \makebox{\tt )}$\>\makebox[3em]{}$\it (k\geq 1,\ tyvars\ {\rm\ distinct})$\\ 
$\it $\>\makebox[3.5em]{$|$}$\it \makebox{\tt (}\ tyvar_1\ \makebox{\tt ,}\ \ldots \ \makebox{\tt ,}\ tyvar_k\ \makebox{\tt )}$\>\makebox[3em]{}$\it (k\geq 2,\ tyvars\ {\rm\ distinct})$\\ 
$\it $\>\makebox[3.5em]{$|$}$\it \makebox{\tt ()}$\\ 
$\it $\>\makebox[3.5em]{$|$}$\it \makebox{\tt [}\ tyvar\ \makebox{\tt ]}$\\ 
$\it $\>\makebox[3.5em]{$|$}$\it \makebox{\tt (}\ tyvar_1\ \makebox{\tt ->}\ tyvar_2\ \makebox{\tt )}$\>\makebox[3em]{}$\it tyvar_1\ {\rm\ and}\ tyvar_2\ {\rm\ distinct}$\\ 
$\it $\\ 
$\it valdefs$\>\makebox[3.5em]{$\rightarrow$}$\it valdef_1\ \makebox{\tt ;}\ \ldots \ \makebox{\tt ;}\ valdef_n$\>\makebox[3em]{}$\it (n\geq 0)$
\end{tabbing}\end{flushleft}
\index{topdecl@{\em topdecl} (\mbox{\tt instance})}
\indexsyn{inst}%
\indexsyn{valdefs}%
%\index{instance declaration@{\ptt instance} declaration}
\index{declaration!within an {\ptt instance} declaration}
%\ToDo{tycls left off above}
An {\em instance declaration} introduces an instance of a class.  Let
\[ \mbox{$\it \makebox{\tt class}\ c\ \makebox{\tt =>}\ C\ u\ \makebox{\tt where}\ \makebox{\tt {\char'173}}\ cbody\ \makebox{\tt {\char'175}}$} \]
be a \mbox{\tt class} declaration.  The general form of the corresponding
instance declaration is:
\[ \mbox{$\it \makebox{\tt instance}\ c'\ \makebox{\tt =>}\ C\ (T\ u_1\ \ldots \ u_k)\ \makebox{\tt where}\ \makebox{\tt {\char'173}}\ d\ \makebox{\tt {\char'175}}$} \]
where \mbox{$\it k\geq0$} and \mbox{$\it T$} is not a type synonym.
\index{type synonym}
The type being instanced, \mbox{$\it (T\ u_1\ \ldots \ u_k)$}, is
a type constructor applied to simple type variables \mbox{$\it u_1,\ \ldots \ u_k$},
which must be distinct.  This prohibits instance declarations
such as:
\bprog
\mbox{\tt instance\ C\ (a,a)\ where\ ...}\\
\mbox{\tt instance\ C\ (Int,a)\ where\ ...}\\
\mbox{\tt instance\ C\ [[a]]\ where\ ...}
\eprog

The context \mbox{$\it c'$} must
imply the context \mbox{$\it c[(T\ u_1\ \ldots \ u_k)/u]$}, and \mbox{$\it d$} may contain bindings
\index{class method}
only for the class methods of \mbox{$\it C$}.  No type signatures
\index{type signature} may appear in \mbox{$\it d$}, as the signatures for the
methods have already been given in the \mbox{\tt class} declaration.

%No contexts may appear
%in \mbox{$\it d$}, since they are implied: any signature declaration in \mbox{$\it d$} will
%have the form \mbox{$\it v\ \makebox{\tt ::}\ t$}, abbreviating \mbox{$\it v\ \makebox{\tt ::}\ c'\ \makebox{\tt =>}\ t$}.
%Each \mbox{$\it v_i$} has type:
%\[ \mbox{$\it v_i\ \makebox{\tt ::}\ c'\ \makebox{\tt =>}\ (t_i[(T\ u_1\ \ldots \ u_k)/u])$} \]

If no binding is given for some class method then the
corresponding default method
\index{default method}
in the \mbox{\tt class} declaration is used (if
present); if such a default does
not exist then the class method at this instance
is implicitly bound to the completely undefined
function (of the appropriate type) and no static error results.

An \mbox{\tt instance} declaration that makes the type \mbox{$\it T$} to be an instance
of class \mbox{$\it C$} is called a {\em C-T instance declaration}
\index{C-T instance declaration@$C$-$T$ instance declaration} and is
subject to these static restrictions:
\index{instance declaration!with respect to modules}
\begin{itemize}
\item A $C$-$T$ instance declaration may only appear either in the module
in which $C$ is declared or in the module in which $T$ is declared, and
only where both $C$ and $T$ are in scope.

\item A type may not be declared as an instance of a
particular class more than once in the same scope.
\end{itemize}

Examples of \mbox{\tt instance} declarations may be found in the next section on
derived instances.  

\subsubsection{Derived Instances}
\index{derived instance}
\label{derived-decls}

As mentioned in Section~\ref{datatype-decls}, \mbox{\tt data} declarations
contain an optional \mbox{\tt deriving} form.  If the form is included, then
{\em derived instance declarations} are automatically generated for
the datatype in each of the named classes.
If a derived instance of a subclass is asked
for, then each of the superclasses\index{superclass} must either be asked for or an
explicit instance declaration must be given for it.

Derived instances provide convenient commonly-used operations for
user-de\-fined da\-ta\-types.  For example, derived instances for datatypes
in the class \mbox{\tt Eq} define the operations \mbox{\tt ==} and \mbox{\tt /=}, freeing the
programmer from the need to define them.
%and taking advantage of
%\Haskell{}'s class mechanism to overload these operations.

The only classes for which derived instances are allowed are
\mbox{\tt Eq}\index{Eq@{\ptt Eq}!derived instance},
\mbox{\tt Ord}\index{Ord@{\ptt Ord}!derived instance}, 
\mbox{\tt Ix}\index{Ix@{\ptt Ix}!derived instance}, 
\mbox{\tt Enum}\index{Enum@{\ptt Enum}!derived instance}, 
\mbox{\tt Text}\index{Text@{\ptt Text}!derived instance}, and 
\mbox{\tt Binary}\index{Binary@{\ptt Binary}!derived instance},
all defined in Figure~\ref{standard-classes}, page~\pageref{standard-classes}.
The
precise details of how the derived instances are generated for each of
these classes are provided in Appendix~\ref{derived-appendix}, including
a specification of when such derived instances are possible. 
%(which is important for the following discussion).

If it is not possible to derive an \mbox{\tt instance} declaration over a class
named in a \mbox{\tt deriving} form, then a static error results.  For example,
not all datatypes can properly support operations in \mbox{\tt Enum}.\indextt{Enum}  It is
also a static error to give an explicit \mbox{\tt instance} declaration for
one that is also derived.
%These rules also apply to the superclasses
%of the class in question.

If the \mbox{\tt deriving} form is omitted from a \mbox{\tt data}
declaration, then {\em no} instance declarations will be derived for
that datatype; that is, omitting a \mbox{\tt deriving} form is equivalent to
including an empty deriving form: \mbox{\tt deriving\ ()}.

% OLD:
%On the other hand, if the \mbox{\tt deriving} form is omitted from a \mbox{\tt data}
%declaration, then \mbox{\tt instance} declarations are derived for the datatype
%in as many of the six classes mentioned above as is possible (see
%Appendix~\ref{derived-appendix}); that is, no
%static error will result if the \mbox{\tt instance} declarations cannot be generated.

%OLD:
%If {\em no} derived \mbox{\tt instance} declarations for a datatype
%are wanted, then the empty deriving form \mbox{\tt deriving\ ()} must be given
%in the \mbox{\tt data} declaration for that type.

\subsubsection{Defaults for Overloaded Operations}
\label{default-decls}
\index{default declaration@{\ptt default} declaration}
\index{overloading!defaults}

\begin{flushleft}\it\begin{tabbing}
\hspace{0.5in}\=\hspace{3.0in}\=\kill
$\it topdecl$\>\makebox[3.5em]{$\rightarrow$}$\it \makebox{\tt default}\ (type\ |\ \makebox{\tt (}type_1\ \makebox{\tt ,}\ \ldots \ \makebox{\tt ,}\ type_n\makebox{\tt )})$\>\makebox[3em]{}$\it \qquad\ (n\geq 0)$
\end{tabbing}\end{flushleft}
\index{topdecl@{\em topdecl} (\mbox{\tt default})}

\noindent
A problem inherent with overloading is the possibility of an ambiguous type.
\index{ambiguous type}
For example, using the
\mbox{\tt read} and \mbox{\tt show} functions defined in Appendix~\ref{derived-appendix},
and supposing that just \mbox{\tt Int} and \mbox{\tt Bool} are members of \mbox{\tt Text}, then
the expression
\bprog
\mbox{\tt let\ x\ =\ read\ "..."\ in\ show\ x\ \ \ \ --\ ILLEGAL}
\eprog
is ambiguous, because the types for \mbox{\tt show} and \mbox{\tt read},
\[\begin{array}{ll}
\mbox{\tt show} & \mbox{$\it ::\ \forall\ a.~\makebox{\tt Text}~\ a\ \Rightarrow\ a\ \rightarrow\ \makebox{\tt String}$} \\
\mbox{\tt read} & \mbox{$\it ::\ \forall\ a.~\makebox{\tt Text}~\ a\ \Rightarrow\ \makebox{\tt String}\ \rightarrow\ a$}
\end{array}\]
could be satisfied by instantiating \mbox{\tt a} as either \mbox{\tt Int}
in both cases, or \mbox{\tt Bool}.  Such expressions
are considered ill-typed, a static error.

We say that an expression \mbox{\tt e} is {\em ambiguously
overloaded}
\index{overloading!ambiguous}
if, in its type \mbox{$\it \forall\ \overline{u}.~c\ \Rightarrow\ t$}, 
there is a type variable $u$ in $\overline{u}$ which occurs in \mbox{$\it c$} 
but not in \mbox{$\it t$}.  Such types are illegal.

For example, the earlier expression involving \mbox{\tt show} and \mbox{\tt read} is
ambiguously overloaded since its type is 
$\forall a.~ \mbox{\tt Text}~ a \Rightarrow \mbox{\tt String}$.

Overloading ambiguity, although rare, can only be circumvented by
input from the user.  One way is through the use of {\em expression
type-signatures}
\index{expression type-signature}
as described in Section~\ref{expression-type-sigs}.
For example, for the ambiguous expression given earlier, one could
write:
\bprog
\mbox{\tt let\ x\ =\ read\ "..."\ in\ show\ (x::Bool)}
\eprog
which disambiguates the type.

Occasionally, an otherwise ambiguous expression needs to be made
the same type as some variable, rather than being given a fixed
type with an expression type-signature.  This is the purpose
of the function \mbox{\tt asTypeOf} (Appendix~\ref{stdprelude}):
\mbox{$\it x$} \mbox{\tt asTypeOf} \mbox{$\it y$} has the value of \mbox{$\it x$}, but \mbox{$\it x$} and \mbox{$\it y$} are
forced to have the same type.  For example,
\bprog
\mbox{\tt approxSqrt\ x\ =\ encodeFloat\ 1\ (exponent\ x\ `div`\ 2)\ `asTypeOf`\ x}
\eprog
(See Section~\ref{coercion}.)

Ambiguities in the class \mbox{\tt Num}\indextt{Num}
are most common, so \Haskell{}
provides another way to resolve them---with a {\em
default declaration}:
\[
\mbox{$\it \makebox{\tt default\ (}t_1\ \makebox{\tt ,}\ \ldots \ \makebox{\tt ,}\ t_n\makebox{\tt )}$}
\]
where \mbox{$\it n\geq0$} (the parentheses may be omitted when \mbox{$\it n=1$}), and each
\mbox{$\it t_i$} must be a monotype for which \mbox{$\it \makebox{\tt Num\ }t_i$} holds.
In situations where an ambiguous type is discovered, an
ambiguous type variable is defaultable if at least one
of its classes is a numeric class and if all of its classes
are either numeric classes or standard classes.
(Figures~\ref{basic-numeric-1}--\ref{basic-numeric-3},
pages~\pageref{basic-numeric-1}--\pageref{basic-numeric-3},
show the numeric classes, and
Figure~\ref{standard-classes}, page~\pageref{standard-classes},
shows the standard classes.)
Each defaultable variable is replaced by the first type in the
default list that is an instance of all the ambiguous variable's classes.
It is a static error if no such type is found.

Only one default declaration is permitted per module, and its effect
is limited to that module.  If no default declaration is given in a
module then it defaults to:
\bprog
\mbox{\tt default\ (Int,\ Double)}
\eprog
The empty default declaration \mbox{\tt default\ ()} must be given to turn off
all defaults in a module.

\subsection{Nested Declarations}
\label{nested}

The following declarations may be used in any declaration list,
including the top level of a module.

\subsubsection{Type Signatures}
\index{type signature}
\label{type-signatures}

\begin{flushleft}\it\begin{tabbing}
\hspace{0.5in}\=\hspace{3.0in}\=\kill
$\it decl$\>\makebox[3.5em]{$\rightarrow$}$\it vars\ \makebox{\tt ::}\ [context\ \makebox{\tt =>}]\ type$\\ 
$\it vars$\>\makebox[3.5em]{$\rightarrow$}$\it var_1\ \makebox{\tt ,}\ \ldots \makebox{\tt ,}\ var_n$\>\makebox[3em]{}$\it (n\geq 1)$
\end{tabbing}\end{flushleft}
\indexsyn{decl}%
\indexsyn{vars}%
A type signature specifies types for variables, possibly with respect
to a context.  A type signature has the form:
\[
\mbox{$\it x_1,\ \ldots ,\ x_n\ \makebox{\tt ::}\ c\ \makebox{\tt =>}\ t$}
\]
which is equivalent to asserting
\mbox{$\it x_i\ \makebox{\tt ::}\ c\ \makebox{\tt =>}\ t$}
for each \mbox{$\it i$} from \mbox{$\it 1$} to \mbox{$\it n$}.  Each \mbox{$\it x_i$} must have a value binding in
the same declaration list that contains the type signature; i.e.~it is
illegal to give a type signature for a variable bound in an
outer scope.
Moreover, it is illegal to give more than one type
signature for one variable.

As mentioned in Section~\ref{type-syntax},
every type variable appearing in a signature
is universally quantified over that signature, and hence
the scope of a type variable is limited to the type
signature that contains it.  For example, in the following
declarations
\bprog
\mbox{\tt f\ ::\ a\ ->\ a}\\
\mbox{\tt f\ x\ =\ x::a\ \ \ \ \ \ \ \ \ \ \ \ \ \ \ \ \ \ \ \ \ \ --\ ILLEGAL}
\eprog
the \mbox{\tt a}'s in the two type signatures are quite distinct.  Indeed,
these declarations contain a static error, since \mbox{\tt x} does not have
type $\forall a.~a$.

A type signature for \mbox{$\it x$} may be more specific than the principal
type derivable from the value binding of \mbox{$\it x$} (see
Section~\ref{type-semantics}), but it is an error to give a type
that is more
general than, or incomparable to, the principal type.
\index{principal type}
If a more specific type is given then all occurrences of the
variable must be used at the more specific type or at a more
specific type still.
%
For example, if we define\nopagebreak[4]
\bprog
\mbox{\tt sqr\ x\ \ =\ \ x*x}
\eprog
then the principal type is 
$\mbox{\tt sqr} :: \forall a.~ \mbox{\tt Num}~ a \Rightarrow a \rightarrow a$, 
which allows
applications such as \mbox{\tt sqr\ 5} or \mbox{\tt sqr\ 0.1}.  It is also legal to declare
a more specific type, such as
\bprog
\mbox{\tt sqr\ ::\ Int\ ->\ Int}
\eprog
but now applications such as \mbox{\tt sqr\ 0.1} are illegal.  Type signatures such as
\bprog
\mbox{\tt sqr\ ::\ (Num\ a,\ Num\ b)\ =>\ a\ ->\ b\ \ \ \ \ --\ ILLEGAL}\\
\mbox{\tt sqr\ ::\ a\ ->\ a\ \ \ \ \ \ \ \ \ \ \ \ \ \ \ \ \ \ \ \ \ \ \ --\ ILLEGAL}
\eprog
are illegal, as they are more general than the principal type of \mbox{\tt sqr}.

\subsubsection{Function and Pattern Bindings}
\label{function-bindings}\label{pattern-bindings}
\index{function binding}\index{pattern binding}

\begin{flushleft}\it\begin{tabbing}
\hspace{0.5in}\=\hspace{3.0in}\=\kill
$\it decl$\>\makebox[3.5em]{$\rightarrow$}$\it valdef$\\ 
$\it $\\ 
$\it valdef$\>\makebox[3.5em]{$\rightarrow$}$\it lhs\ \makebox{\tt =}\ exp\ [\makebox{\tt where}\ \makebox{\tt {\char'173}}\ decls\ [\makebox{\tt ;}]\ \makebox{\tt {\char'175}}]$\\ 
$\it $\>\makebox[3.5em]{$|$}$\it lhs\ gdrhs\ [\makebox{\tt where}\ \makebox{\tt {\char'173}}\ decls\ [\makebox{\tt ;}]\ \makebox{\tt {\char'175}}]$\\ 
$\it $\\ 
$\it lhs$\>\makebox[3.5em]{$\rightarrow$}$\it apat$\\ 
$\it $\>\makebox[3.5em]{$|$}$\it funlhs$\\ 
$\it funlhs$\>\makebox[3.5em]{$\rightarrow$}$\it afunlhs$\\ 
$\it $\>\makebox[3.5em]{$|$}$\it pat^{i+1}_1\ varop^{({\rm\ n},i)}\ pat^{i+1}_2$\>\makebox[3em]{}$\it (0\leq i\leq 9)$\\ 
$\it $\>\makebox[3.5em]{$|$}$\it lpat^i\ varop^{({\rm\ l},i)}\ pat^{i+1}$\>\makebox[3em]{}$\it (0\leq i\leq 9)$\\ 
$\it $\>\makebox[3.5em]{$|$}$\it pat^{i+1}\ varop^{({\rm\ r},i)}\ rpat^i$\>\makebox[3em]{}$\it (0\leq i\leq 9)$\\ 
$\it afunlhs$\>\makebox[3.5em]{$\rightarrow$}$\it var\ apat$\\ 
$\it $\>\makebox[3.5em]{$|$}$\it \makebox{\tt (}\ funlhs\ \makebox{\tt )}\ apat$\\ 
$\it $\>\makebox[3.5em]{$|$}$\it afunlhs\ apat$\\ 
$\it $\\ 
$\it gdrhs$\>\makebox[3.5em]{$\rightarrow$}$\it gd\ \makebox{\tt =}\ exp\ [gdrhs]$\\ 
$\it $\\ 
$\it gd$\>\makebox[3.5em]{$\rightarrow$}$\it \makebox{\tt |}\ exp$
\end{tabbing}\end{flushleft}
\indexsyn{decl}%
\indexsyn{valdef}%
\indexsyn{lhs}%
\indexsyn{funlhs}%
\indexsyn{afunlhs}%
\indexsyn{gdrhs}%
\indexsyn{gd}%
We distinguish two cases within this syntax: a {\em pattern binding}
occurs when \mbox{$\it lhs$} is \mbox{$\it apat$}; otherwise, the binding is called a {\em function
binding}.  Either binding may appear at the top-level of a module or
within a \mbox{\tt where} or \mbox{\tt let} construct.  The use of the nonterminal \mbox{$\it apat$}
(rather than \mbox{$\it pat$}) in the production for \mbox{$\it lhs$} 
disallows top level $n\mbox{\tt +}k$ pattern bindings;
\index{n+k pattern@\mbox{$\it n\makebox{\tt +}k$} pattern}
otherwise, programs such as \mbox{\tt x\ +\ 2\ =\ 3} could be parsed either as a
definition of \mbox{\tt +} or as a pattern binding.

\paragraph*{Function bindings.}
\index{function binding}
A function binding binds a variable to a function value.  The general
form of a function binding for variable \mbox{$\it x$} is:
\[\ba{lll}
\mbox{$\it x$} & \mbox{$\it p_{11}\ \ldots \ p_{1k}$} & \mbox{$\it match_1$}\\
\mbox{$\it \ldots $} \\
\mbox{$\it x$} & \mbox{$\it p_{n1}\ \ldots \ p_{nk}$} & \mbox{$\it match_n$}
\ea\]
where each \mbox{$\it p_{ij}$} is a pattern, and where each \mbox{$\it match_i$} is of the
general form:
\[\ba{l}
\mbox{$\it \makebox{\tt =}\ e\ \makebox{\tt where\ {\char'173}}\ decls\ \makebox{\tt {\char'175}}$}
\ea\]
or
\[\ba{lll}
\mbox{$\it \makebox{\tt |}\ g_{i1}$}   & \mbox{$\it \makebox{\tt =}\ e_{i1}$} \\
\mbox{$\it \ldots $} \\
\mbox{$\it \makebox{\tt |}\ g_{im_i}$} & \mbox{$\it \makebox{\tt =}\ e_{im_i}$} \\
               & \multicolumn{2}{l}{\mbox{$\it \makebox{\tt where\ {\char'173}}\ decls_i\ \makebox{\tt {\char'175}}$}}
\ea\]
and where \mbox{$\it n\geq 1$}, \mbox{$\it 1\leq i\leq n$}, \mbox{$\it m_i\geq 1$}.  The former is treated
as shorthand for a particular case of the latter, namely:
\[\ba{l}
\mbox{$\it \makebox{\tt |\ True\ =}\ e\ \makebox{\tt where\ {\char'173}}\ decls\ \makebox{\tt {\char'175}}$}
\ea\]

The set of patterns corresponding to each match must be {\em
linear}\index{linearity}\index{linear pattern}---no variable is allowed
to appear more than once in the entire set.

Alternative syntax is provided for binding functional values to infix
operators.  For example, these two function
definitions are equivalent:
\bprog
\mbox{\tt plus\ x\ y\ z\ =\ x+y+z}\\
\mbox{\tt (x\ }\bkqB\mbox{\tt plus}\bkqA\mbox{\tt \ y)\ z\ =\ x+y+z}
\eprogNoSkip

\outline{
\paragraph*{Translation:}
The general binding form for functions is semantically
equivalent to the equation (i.e.~simple pattern binding):
\[
x\ x_1\ x_2\ ...\ x_k\mbox{\tt \ =\ case\ (}x_1\mbox{\tt ,\ }...\mbox{\tt ,\ }x_k\mbox{\tt )\ of}
\ba[t]{lcl}
\mbox{$\it \makebox{\tt (}p_{11},\ \ldots ,\ p_{1k}\makebox{\tt )}\ match_1$}  \\
\mbox{$\it \ldots $} \\
\mbox{$\it \makebox{\tt (}p_{m1},\ \ldots ,\ p_{mk}\makebox{\tt )}\ match_m$}
\ea\]
where the \mbox{$\it x_i$} are new identifiers.
}

\paragraph*{Pattern bindings.}
\index{pattern binding}
A pattern binding binds variables to values.  A {\em simple} pattern
binding has form \mbox{$\it p\ =\ e$}.
\index{simple pattern binding}
In both a \mbox{\tt where} or \mbox{\tt let} clause
and at the top level of a module, the pattern \mbox{$\it p$} is
matched ``lazily'' as an irrefutable pattern
\index{irrefutable pattern}
by default (as if there
were an implicit \mbox{\tt {\char'176}} in front of it).  See the translation in
Section~\ref{let-expressions}.

The {\em general} form of a pattern binding is \mbox{$\it p\ match$}, where a
\mbox{$\it match$} is the same structure as for function bindings above; in other
words, a pattern binding is:
\[\ba{rcl}
\mbox{$\it p$} & \mbox{$\it \makebox{\tt |}\ g_{1}$}   & \mbox{$\it \makebox{\tt =}\ e_{1}$} \\
    & \mbox{$\it \makebox{\tt |}\ g_{2}$}   & \mbox{$\it \makebox{\tt =}\ e_{2}$} \\
    & \mbox{$\it \ldots $} \\
    & \mbox{$\it \makebox{\tt |}\ g_{m}$}   & \mbox{$\it \makebox{\tt =}\ e_{m}$} \\
    & \multicolumn{2}{l}{\mbox{$\it \makebox{\tt where\ {\char'173}}\ decls\ \makebox{\tt {\char'175}}$}}
\ea\]

%{\em Note}: the simple form
%\WeSay{Yes}
%\mbox{$\it p\ \makebox{\tt =}\ e$} is equivalent to \mbox{$\it p\ \makebox{\tt |\ True\ =}\ e$}.

\outline{
\paragraph*{Translation:}
The pattern binding above is semantically equivalent to this
simple pattern binding:
\[\ba{lcl}
\mbox{$\it p$} &\mbox{\tt =}& \mbox{$\it \makebox{\tt let}\ decls\ \makebox{\tt in}$} \\
    &   & \mbox{$\it \makebox{\tt if\ }g_1\makebox{\tt \ then\ }e_1\makebox{\tt \ else}$} \\
    &   & \mbox{$\it \makebox{\tt if\ }g_2\makebox{\tt \ then\ }e_2\makebox{\tt \ else}$} \\
    &   & ...                          \\
    &   & \mbox{$\it \makebox{\tt if\ }g_m\makebox{\tt \ then\ }e_m\makebox{\tt \ else\ error\ "Unmatched\ pattern"}$}
\ea\]
}

\subsection{Static semantics of function and pattern bindings}

The static semantics of the function and pattern bindings of
a \mbox{\tt let} expression or \mbox{\tt where} clause
% (including that of the top-level of
% a program that has been translated into a \mbox{\tt let} expression as
% described at the beginning of Section~\ref{modules})
is discussed in this section.

\subsubsection{Dependency analysis}

In general the static semantics is given by the
normal Hindley-Milner\index{Hindley-Milner type system} inference rules,
% as described in Appendix~\ref{static-semantics},
except that a {\em dependency
analysis\index{dependency analysis} transformation} is first performed
to enhance polymorphism, as follows.
Two variables bound by value declarations are in the
same {\em declaration group} if either
\index{declaration group}
\begin{enumerate}
\item
they are bound by the same pattern binding, or
\item
their bindings are mutually recursive (perhaps via some
other declarations which are also part of the group).
\end{enumerate}
Careful application of the following 
rules causes each \mbox{\tt let} or \mbox{\tt where} construct to bind only the
variables of a single declaration group, thus capturing the required
dependency analysis:\footnote{%
A similar transformation is described in 
Peyton Jones' book \cite{peyton-jones:book}.}
\begin{center}
\bt{l}
(1)~The order of declarations in \mbox{\tt where}/\mbox{\tt let} constructs is irrelevant. \\
(2)~\mbox{\tt let\ {\char'173}}$d_1$\mbox{\tt ;\ }$d_2$\mbox{\tt {\char'175}\ in\ }$e$ = \mbox{\tt let\ {\char'173}}$d_1$\mbox{\tt {\char'175}\ in\ (let\ {\char'173}}$d_2$\mbox{\tt {\char'175}\ in\ }$e$\mbox{\tt )} \\
\ \ \ \ (when no identifier bound in $d_2$ appears free in $d_1$)
\et
\end{center}

%----------------

\subsubsection{Generalisation}
\label{generalisation}

The Hindley-Milner type system assigns types to a \mbox{\tt let}-expression
in two stages.
First, the right-hand side of the declaration is typed, giving a type with
no universal quantification.  Second, all type variables which occur in this
type are universally quantified unless they are associated with
bound variables in the type environment;
this is called {\em generalisation}.\index{generalisation}
Finally, the body of the \mbox{\tt let}-expression is typed.

For example, consider the declaration
\bprog
\mbox{\tt \ \ \ \ \ \ \ \ f\ x\ =\ let\ g\ y\ =\ (y,y)}\\
\mbox{\tt \ \ \ \ \ \ \ \ \ \ \ \ \ \ in\ ...}\\
\mbox{\tt }
\eprog
The type of \mbox{\tt g}'s definition is 
$a \rightarrow (a,a)$.  The generalisation step
attributes to \mbox{\tt g} the polymorphic type 
$\forall a.~ a \rightarrow (a,a)$,
after which the typing of the ``\mbox{\tt ...}'' part can proceed.

When typing overloaded definitions, all the overloading 
constraints from a single declaration group are collected together, 
to form the context for the type of each variable declared in the group.
For example, in the definition:
\bprog
\mbox{\tt \ \ \ \ \ \ \ \ f\ x\ =\ let\ g1\ x\ y\ =\ if\ x>y\ then\ show\ x\ else\ g2\ y\ x}\\
\mbox{\tt \ \ \ \ \ \ \ \ \ \ \ \ \ \ \ \ \ \ g2\ p\ q\ =\ g1\ q\ p}\\
\mbox{\tt \ \ \ \ \ \ \ \ \ \ \ \ \ \ in\ ...}
\eprog
The types of the definitions of \mbox{\tt g1} and \mbox{\tt g2} are both
$a \rightarrow a \rightarrow \mbox{\tt String}$, and the accumulated constraints are
$\mbox{\tt Ord}~a$ (arising from the use of \mbox{\tt >}), and $\mbox{\tt Text}~a$ (arising from the
use of \mbox{\tt show}).
The type variables appearing in this collection of constraints are
called the {\em constrained type variables}.

The generalisation step attributes to both \mbox{\tt g1} and \mbox{\tt g2} the type
$\forall a.~(\mbox{\tt Ord}~a,~\mbox{\tt Text}~a) \Rightarrow 
a \rightarrow a \rightarrow \mbox{\tt String}$.
Notice that \mbox{\tt g2} is overloaded in the same way as \mbox{\tt g1} even though the
occurrences of \mbox{\tt >} and \mbox{\tt show} are in the definition of \mbox{\tt g1}.

If the programmer supplies explicit type signatures for more than one variable
in a declaration group, the contexts of these signatures must be 
identical up to renaming of the type variables.

\subsubsection{Monomorphism}

Sometimes it is not possible to generalise over all the type variables
used in the type of the definition.
For example, consider the declaration\nopagebreak[4]
\bprog
\mbox{\tt \ \ \ \ \ \ \ \ f\ x\ =\ let\ g\ y\ z\ =\ ([x,y],\ z)}\\
\mbox{\tt \ \ \ \ \ \ \ \ \ \ \ \ \ \ in\ ...}
\eprog
In an environment where \mbox{\tt x} has type $a$,
the type of \mbox{\tt g}'s definition is 
$a \rightarrow b \rightarrow \mbox{\tt ([}a\mbox{\tt ]},b\mbox{\tt )}$.
The generalisation step attributes to \mbox{\tt g} the type 
$\forall b.~ a \rightarrow b \rightarrow \mbox{\tt ([}a\mbox{\tt ]},b\mbox{\tt )}$;
only $b$ can be universally quantified because $a$ occurs in the
type environment.
We say that the type of \mbox{\tt g} is {\em monomorphic in the type variable $a$}.
\index{monomorphic type variable}

The effect of such monomorphism is that the first argument of all 
applications of \mbox{\tt g} must be of a single type.  
For example, it would be legal for
the ``\mbox{\tt ...}'' to be
\bprog
\mbox{\tt \ \ \ \ \ \ \ \ (g\ True,\ g\ False)}
\eprog
(which would, incidentally, force \mbox{\tt x} to have type \mbox{\tt Bool}) but illegal for it to be
\bprog
\mbox{\tt \ \ \ \ \ \ \ \ (g\ True,\ g\ 'c')}
\eprog
In general, a type $\forall \overline{u}.~c \Rightarrow t$
is said to be {\em monomorphic}
\index{monomorphic type variable}
in the type variable \mbox{$\it a$} if \mbox{$\it a$} is free in
$\forall \overline{u}.~c \Rightarrow t$.

It is worth noting that the explicit type signatures provided by \Haskell{}
are not powerful enough to express types which include monomorphic type
variables.  For example, we cannot write
\bprog
\mbox{\tt \ \ \ \ \ \ \ \ f\ x\ =\ let\ }\\
\mbox{\tt \ \ \ \ \ \ \ \ \ \ \ \ \ \ \ \ g\ ::\ a\ ->\ b\ ->\ ([a],b)}\\
\mbox{\tt \ \ \ \ \ \ \ \ \ \ \ \ \ \ \ \ g\ y\ z\ =\ ([x,y],\ z)}\\
\mbox{\tt \ \ \ \ \ \ \ \ \ \ \ \ \ \ in\ ...}
\eprog
because that would claim that \mbox{\tt g} was polymorphic in both \mbox{\tt a} and \mbox{\tt b}
(Section~\ref{type-signatures}).  In this program, \mbox{\tt g} can only be given
a type signature if its first argument is restricted to a type not involving
type variables; for example
\bprog
\mbox{\tt \ \ \ \ \ \ \ \ \ \ \ \ \ \ \ \ g\ ::\ Int\ ->\ b\ ->\ ([Int],b)}
\eprog
This signature would also cause \mbox{\tt x} to have type \mbox{\tt Int}.

\subsubsection{The monomorphism restriction}
\index{monomorphism restriction}
\label{sect:monomorphism-restriction}

\Haskell{} places certain extra restrictions on the generalisation
step, beyond the standard Hindley-Milner restriction described above,
which further reduce polymorphism in particular cases.

The monomorphism restriction uses the binding syntax of a
variable.  Recall that a variable is bound by either a {\em function
binding} or a {\em pattern binding}, and that a {\em simple} pattern
binding is a pattern binding in which the pattern consists of only a
single variable (Section~\ref{pattern-bindings}).

Two rules define the monomorphism restriction:
\begin{description}
\item[Rule 1.]
We say that a given declaration group is {\em unrestricted} if and only if:
\begin{description}
\item[(a):]
every variable in the group is bound by a function binding or a simple
pattern binding, {\em and}
\item[(b):]
an explicit type signature is given for every variable in the group
which is bound by simple pattern binding.
\end{description}
The usual Hindley-Milner restriction on polymorphism is that
only type variables free in the environment may be generalised.
In addition, {\em the constrained type variables of a
a restricted declaration group may not be generalised}.
(Recall that a type variable is constrained if it must belong
to some type class; see Section~\ref{generalisation}.)
% 
% \item[Rule 1.]
% The variables of a given declaration group are monomorphic in
% all their constrained type variables if and only if:
% \begin{description}
% \item[either (a):]
% one or more variables in the declaration group 
% is bound by a non-simple pattern binding.
% \item[or (b):]
% one or more variables in the declaration group is bound 
% by a simple pattern binding, and
% no type signature is given for any of the variables in the group.
% \end{description}

\item[Rule 2.]
The type of a variable exported from a module must be completely polymorphic;
that is, it must not have any free type variables.
It follows from Rule~1 that if all top-level declaration groups are
unrestricted, then Rule~2 is automatically satisfied.
\end{description}

% When all variables in a declaration group are declared using function
% binding the monomorphism restriction will not apply.  Any variable
% declared in a non-simple pattern binding will invoke monomorphism for
% the entire group containing it.  Simple pattern bindings will be
% monomorphic unless a type signature is supplied.
%
Rule 1 is required for two reasons, both of which are fairly subtle.
First, it prevents computations from being unexpectedly repeated.
For example, recall that \mbox{\tt genericLength} is a standard function whose
type is given by
\bprog
\mbox{\tt \ \ \ \ \ \ \ \ genericLength\ ::\ Num\ a\ =>\ [b]\ ->\ a}
\eprog
Now consider the following expression:
\bprog
\mbox{\tt \ \ \ \ \ \ \ \ let\ {\char'173}\ len\ =\ genericLength\ xs\ {\char'175}\ in\ (len,\ len)}
\eprog
It looks as if \mbox{\tt len} should be computed only once, but without Rule~1 it might
be computed twice, once at each of two different overloadings.  If the 
programmer does actually wish the computation to be repeated, an explicit
type signature may be added:
\bprog
\mbox{\tt \ \ \ \ \ \ \ \ let\ {\char'173}\ len\ ::\ Num\ a\ =>\ a;\ len\ =\ genericLength\ xs\ {\char'175}\ in\ (len,\ len)}
\eprog
When non-simple pattern bindings are used, the types inferred are 
always monomorphic in their constrained type variables, irrespective of whether
a type signature is provided.  For example, in
\bprog
\mbox{\tt \ \ \ (f,g)\ =\ ((+),(-))}
\eprog
both \mbox{\tt f} and \mbox{\tt g} will be monomorphic regardless of any type
signatures supplied for \mbox{\tt f} or \mbox{\tt g}.

Rule~1 also prevents ambiguity.  For example, consider the declaration
group
\bprog
\mbox{\tt \ \ \ \ \ \ \ \ [(n,s)]\ =\ reads\ t}
\eprog
Recall that \mbox{\tt reads} is a standard function whose type is given by the
signature
\bprog
\mbox{\tt \ \ \ \ \ \ \ \ reads\ ::\ (Text\ a)\ =>\ String\ ->\ [(a,String)]}
\eprog
Without Rule~1, \mbox{\tt n} would be assigned the 
type $\forall a.~\mbox{\tt Text}~a \Rightarrow a$ 
and \mbox{\tt s} the type $\forall a.~\mbox{\tt Text}~a \Rightarrow \mbox{\tt String}$.
The latter is an illegal type, because it is inherently ambiguous.
It is not possible to determine at what overloading to use \mbox{\tt s}.
Rule~1 makes \mbox{\tt n} and \mbox{\tt s} monomorphic in $a$.

Lastly, Rule~2 is required because there is no way to enforce monomorphic use
of an exported binding, except by performing type inference on the entire
program at once.

The monomorphism rule has a number of consequences for the programmer.
Anything defined with function syntax will usually
generalize as a function is expected to.  Thus in
\bprog
\mbox{\tt \ \ \ \ \ \ \ \ f\ x\ y\ =\ x+y}
\eprog
the function \mbox{\tt f} may be used at any overloading in class \mbox{\tt Num}.
There is no danger of recomputation here.  However, the same function
defined with pattern syntax
\bprog
\mbox{\tt \ \ \ \ \ \ \ \ f\ =\ {\char'134}x\ ->\ {\char'134}y\ ->\ x+y}
\eprog
requires a type signature if \mbox{\tt f} is to be fully overloaded.
Many functions are most naturally defined using simple pattern
bindings; the user must be careful to affix these with type signatures
to retain full overloading.  The standard prelude contains many
examples of this:
\bprog
\mbox{\tt \ \ \ \ \ \ \ \ indices\ ::\ (Ix\ a)\ =>\ Array\ a\ b\ ->\ [a]}\\
\mbox{\tt \ \ \ \ \ \ \ \ indices\ =\ \ range\ .\ bounds}
\eprog

% Even when a function is defined using a function binding, it may still
% be made monomorphic by another variable in the same declaration group.
% Since groups defined through mutually recursive functions need not be
% syntacticly adjacent, it may be difficult to see where overloading is
% being lost.  In this example \mbox{\tt fact'} is defined with a pattern binding
% and forces \mbox{\tt fact} to be monomorphic in the absence of a type signature
% for either \mbox{\tt fact} or \mbox{\tt fact'}.  This would in turn result in an error as
% per Rule~2.
% \bprog
% 
% module Mod1(fact)
% import Mod2
% fact 0 = 1
% fact n = n*fact'(n-1)
% 
% module Mod2(fact')
% import Mod1
% fact' = fact
% 
% \eprog

% Local Variables: 
% mode: latex
% End:
\startnewsection
%
% $Header$
%
\section{Modules} 
\label{modules} 
\index{module}

A module defines a collection of values, datatypes, type synonyms,
classes, etc.~(see Section~\ref{declarations}),
and {\em exports} some of these resources, making them available to
other modules.  We use the term {\em entity}\index{entity} to refer to
the values, types, and classes defined in and perhaps exported from a
module.

A \Haskell{} {\em program} is a collection of modules, one of
which, by convention, must be called \mbox{\tt Main}\indexmodule{Main} and must
export the value \mbox{\tt main}\indextt{main}.  The {\em value} of the program
is the value of the identifier \mbox{\tt main} in module \mbox{\tt Main}, and \mbox{\tt main}
must have type \mbox{\tt Dialogue} (see Section~\ref{io}).

% More precisely, if all
% importations, renamings, etc. as described in this section are
% eliminated, and the original declarations (with suitable renamings 
% to prevent name clashes) are collected together into one set of
% declarations called \mbox{$\it decls$}, then the value of the program is 
% \mbox{$\it \makebox{\tt main\ where\ }decls$}.  The semantics of a \mbox{\tt where} clause is defined
% in Section~\ref{let-expressions}.

Modules may reference other modules via explicit
\mbox{\tt import} declarations, each giving the name of a module to be
imported, specifying its entities to be imported, and
optionally renaming some or all of them.  Modules may be mutually
recursive.

The name-space for modules is flat, with each module being associated
with a unique module name (which are \Haskell{} identifiers
beginning with a capital letter; i.e.~$aconid$).  There are two
distinguished modules, \mbox{\tt PreludeCore} and \mbox{\tt Prelude}, both
discussed in Section~\ref{standard-prelude}.

\subsection{Overview}
\label{module-structure}

\subsubsection{Interfaces and Implementations}

A module consists of an {\em interface}\index{interface} and an
{\em implementation}\index{implementation} of that interface.

The interface of a module provides complete information about the
static semantics of that module, including type signatures, class
definitions, and type declarations for the various entities made
available by the module.  This information is complete in this
sense:  If a module $M$ imports modules $M_1, \ldots, M_n$,
then only the interfaces of $M_1, \ldots, M_n$ need be examined in
order to perform static checking on the implementation of {\it M.}  No
implementations of $M_1, \ldots, M_n$ need to exist, nor need any
further interfaces be consulted.  Interfaces are discussed
in Section~\ref{module-interfaces}.

An implementation ``fills in'' the information about a module missing
from the interface.  For example, for each value given a type
signature in the interface the implementation either imports a module
that defines the value or defines the value itself.  Implementations
are discussed in Section~\ref{module-implementations}.

\subsubsection{Original Names}
\label{original-names}
\index{original name}
\index{renaming}

It may be that a particular entity is imported into a module by more
than one route---for example, because it is exported by two modules
both of which are imported by a third module.  It is important that
benign name-clashes of this form are allowed, but that accidental
name-clashes are detected and reported as errors.  This is done as
follows:

Each entity (class, type constructor, value, etc.) has an {\em
original name} that is a pair consisting of the name of the module in
which it was originally declared, and the name it was given in that
declaration.  The original name is carried with the entity wherever it
is exported.  Two entities are the same if and only if they have the
same original name.

\index{renaming!with respect to original name}
Renaming does {\em not} affect the original name; it is a purely
syntactic operation that affects only the name by which the entity is
currently known.  For example, if a class is renamed and a type is
declared to be an instance of the newly-named class, then it is also
an instance of the original class---there is just one class, which
happens to be known by different names in different parts of the
program.  Also, fixity is a property of the original name of an
identifier or operator and is not affected by renaming; the new
name has the same fixity as the old one.

A given entity may be known by at most one name in any scope.
So, for example, a module may not import an entity twice and rename it
differently on each occasion.  Either it must be renamed in the same
way on each import or else not imported twice (for example, by using
a \mbox{\tt hiding} clause).

As there are several name spaces, a single name may identify 
more than one entity. In a \mbox{\tt renaming} clause, such as 
\mbox{$\it \makebox{\tt renaming(}\ldots ,n_1\ \makebox{\tt to}\ n_2,\ldots \makebox{\tt )}$}, {\em all} the entities to
which $n_1$ refers are renamed to $n_2$.

\subsubsection{Closure}
\label{closure}
\index{module!closure}

The implementation together with the interfaces of
the modules it imports must be {\em statically closed} according to
this rule: {\em every value, type, or class referred to
in the text of 
an implementation together with the
entities that it imports, must be 
declared in the implementation or in one of the
imported interfaces}.

It is an error for a module to export a collection of entities that
cannot possibly become closed.  For example, if a module \mbox{\tt A} declares
both the type \mbox{\tt T} and a value \mbox{\tt t} of type \mbox{\tt T}, it may not export \mbox{\tt t}
without also exporting \mbox{\tt T}.  But if
%However, the closure condition applies on {\em import},
%not on {\em export}.  For example, if
another module \mbox{\tt B} imported \mbox{\tt T}
from module \mbox{\tt A}, and declared another value \mbox{\tt s} of type \mbox{\tt T}, it may
export \mbox{\tt s} without exporting \mbox{\tt T}---but any module importing \mbox{\tt B} must
also import the type \mbox{\tt T} by some other route, for example by also
importing \mbox{\tt A}.

%There is one important exception to the closure rule: {\em instance
%declarations in imported interfaces are not subject to it, and are
%ignored if they refer to a class or type constructor which is not in
%scope}.  This allows an interface to export a C-T instance declaration
%(see Section~\ref{instance-decls}) which takes effect when both C and T
%are in scope, but does no harm if one or the other (or both) is not.
%
\subsubsection{The Compilation System}

The task of checking consistency between interfaces and
implementations must be done by the {\em compilation
system}\index{compilation system}.

\Haskell{} does not specify any particular association between
implementations and interfaces on the one hand, and {\em files} on the
other; nor does it specify how implementations and interfaces are
produced.  These matters are determined by the compilation system, and
many variations are possible, depending on the programming
environment.  For example, a compilation system could insist that each
implementation and each interface reside alone in a file, and that the
module name is the same as that of the file, with the implementation
and interface distinguished by a suffix.
%Alternatively, if no such
%restrictions are made, then the compilation system has
%to map module names onto file names.

Similarly, a compilation system may require the programmer to write
the interface, or it may derive the interface from examination of the
implementation, or some hybrid of the two.  \Haskell{} is defined so
that, given the interfaces of all imported modules, it is always
possible to perform a complete static check on the implementation,
and, if it is well-typed, to derive its unique
interface automatically.  However, given a set of mutually recursive
implementations, the compilation system may have to examine several
modules at once to derive the interfaces, which will still be unique
with one exception: because of the shorthand for exporting all
entities from an imported module, the set of exports may not be
unique.  Any set satisfying the consistency constraints is a valid
solution for a well-typed \Haskell{} program, but if an
implementation automatically derives the interface it must derive the
smallest set of exports.

For optimisation across module boundaries, a compilation system may
need more information (e.g., information about strictness, inlining,
uncurrying, etc.) than is provided by the standard interface as
defined in this report.  Draft proposals exist for including such
information as comments in interfaces; for details, contact the
implementors listed in the preface (page~\pageref{implementors}).

\subsection{Module Implementations} 
\label{module-implementations}
\index{module!implementation}

A module implementation\index{implementation} defines a mutually
recursive scope containing declarations for value bindings, data
types, type synonyms, classes, etc. (see Section~\ref{declarations}).

\begin{flushleft}\it\begin{tabbing}
\hspace{0.5in}\=\hspace{3.0in}\=\kill
$\it module$\>\makebox[3.5em]{$\rightarrow$}$\it \makebox{\tt module}\ modid\ [exports]\ \makebox{\tt where}\ body$\\ 
$\it $\>\makebox[3.5em]{$|$}$\it body$\\ 
$\it body$\>\makebox[3.5em]{$\rightarrow$}$\it \makebox{\tt {\char'173}}\ [impdecls\ \makebox{\tt ;}]\ [[fixdecls\ \makebox{\tt ;}]\ topdecls\ [\makebox{\tt ;}]]\ \makebox{\tt {\char'175}}$\\ 
$\it $\>\makebox[3.5em]{$|$}$\it \makebox{\tt {\char'173}}\ impdecls\ [\makebox{\tt ;}]\ \makebox{\tt {\char'175}}$\\ 
$\it $\\ 
$\it modid$\>\makebox[3.5em]{$\rightarrow$}$\it aconid$\\ 
$\it impdecls$\>\makebox[3.5em]{$\rightarrow$}$\it impdecl_1\ \makebox{\tt ;}\ \ldots \ \makebox{\tt ;}\ impdecl_n$\>\makebox[3em]{}$\it \qquad\ (n\geq 1)$\\ 
$\it topdecls$\>\makebox[3.5em]{$\rightarrow$}$\it topdecl_1\ \makebox{\tt ;}\ \ldots \ \makebox{\tt ;}\ topdecl_n$\>\makebox[3em]{}$\it \qquad\ (n\geq 0)$
\end{tabbing}\end{flushleft}
\indexsyn{module}%
\indexsyn{body}%
\indexsyn{modid}%
\indexsyn{impdecls}%
\indexsyn{topdecls}%

% [:: [context =>] type]                TYPE SIGS NOT ALLOWED NOW
% EXPOSE IS OUT
% topdecl        -> expose [modid] entities where {topdecls }
%         |  \ldots \tr{(see Section~\ref{declarations})}

%\indextt{module}
A module implementation begins with a header: the keyword
\mbox{\tt module}, the module name, and a list of entities (enclosed in round
parentheses) to be exported.  The header is followed by an optional
list of \mbox{\tt import} declarations that specify modules to be imported,
optionally restricting and renaming the imported bindings.  This is
followed by an optional list of fixity declarations and the module
body.  The module body is simply a list of top-level declarations
($topdecls$), as described in Section~\ref{declarations}.  

An abbreviated form of module is permitted, which consists only of
the module body.  If this is used, the header is assumed to be
\mbox{\tt module\ Main\ where}.
If the first lexeme in the
abbreviated module is not a \mbox{\tt {\char'173}}, then the layout rule applies
for the top level of the module.
It is inadvisable
for a compilation system to permit 
an abbreviated module to appear in the same file as some
unabbreviated modules.

\subsubsection{Export Lists}
\label{export}
\index{export list}

\begin{flushleft}\it\begin{tabbing}
\hspace{0.5in}\=\hspace{3.0in}\=\kill
$\it exports$\>\makebox[3.5em]{$\rightarrow$}$\it \makebox{\tt (}\ export_1\ \makebox{\tt ,}\ \ldots \ \makebox{\tt ,}\ export_n\ \makebox{\tt )}$\>\makebox[3em]{}$\it \qquad\ (n\geq 1)$\\ 
$\it $\\ 
$\it export$\>\makebox[3.5em]{$\rightarrow$}$\it entity$\\ 
$\it $\>\makebox[3.5em]{$|$}$\it modid\ \makebox{\tt ..}$\\ 
$\it $\\ 
$\it entity$\>\makebox[3.5em]{$\rightarrow$}$\it varid$\\ 
$\it $\>\makebox[3.5em]{$|$}$\it tycon$\\ 
$\it $\>\makebox[3.5em]{$|$}$\it tycon\ \makebox{\tt (..)}$\\ 
$\it $\>\makebox[3.5em]{$|$}$\it tycon\ \makebox{\tt (}\ conid_1\ \makebox{\tt ,}\ \ldots \ \makebox{\tt ,}\ conid_n\ \makebox{\tt )}$\>\makebox[3em]{}$\it \qquad\ (n\geq 1)$\\ 
$\it $\>\makebox[3.5em]{$|$}$\it tycls\ \makebox{\tt (..)}$\\ 
$\it $\>\makebox[3.5em]{$|$}$\it tycls\ \makebox{\tt (}\ varid_1\ \makebox{\tt ,}\ \ldots \ \makebox{\tt ,}\ varid_n\ \makebox{\tt )}$\>\makebox[3em]{}$\it \qquad\ (n\geq 0)$
\end{tabbing}\end{flushleft}
\indexsyn{exports}%
\indexsyn{export}%
\indexsyn{entity}%
% [\mbox{\tt ::} [context \mbox{\tt =>}] type]            TYPE SIGS NOT ALLOWED NOW

An {\em export list} identifies the entities to be exported by a
module declaration.  A module implementation may only export an entity
that it declares, or that it imports from some other module.  If the
export list is omitted, all values, types and classes defined in the
module are exported, {\em but not those that are imported}.

Entities in an export list may be named as follows:
\begin{enumerate}
\item
Ordinary values, whether declared in the implementation
body or imported,
% (including within a named \mbox{\tt expose} declaration) 
may be named by giving the name of the value as a $varid$.
Operators should be enclosed in parentheses to turn them into
$varid$'s.
% Such a name may be annotated with a type signature if it designates an
% ordinary value.
\item
A type synonym $T$ declared by a \mbox{\tt type} declaration
may be named by simply giving the name of the type.
\index{type synonym}
\item
An algebraic datatype $T$ with constructors $K_1,\ldots,K_n$
\index{algebraic datatype}
declared by a \mbox{\tt data} declaration may be named in one of three ways:
\begin{itemize}
\item
The form $T$ names the type {\em but not the constructors}.
The ability to export a type without its constructors allows the
construction of abstract datatypes (see Section~\ref{abstract-types}).
\item
The form $T\mbox{\tt (}K_1\mbox{\tt ,}\ldots\mbox{\tt ,}K_n\mbox{\tt )}$, where
{\em all} and only the constructors are listed without duplications,
names the type and {\em all} its constructors.  
\item
The abbreviated form $T\mbox{\tt (..)}$ also names the type 
and all its constructors.
\end{itemize}
Data constructors may not be named in export lists in any other way.
\item
A class $C$ with operations $f_1,\ldots,f_n$
declared in a \mbox{\tt class} declaration may be named in one of two ways, both of which
name the class together with all its operations:
\index{class declaration}
\begin{itemize}
\item
The form $C\mbox{\tt (}f_1\mbox{\tt ,}\ldots\mbox{\tt ,}f_n\mbox{\tt )}$, where
all and only the operations in that class are listed without duplications.
\item
The abbreviated form $C\mbox{\tt (..)}$.
\end{itemize}
Operators in a class may not be named in export lists in any other way.
\item
The set of all entities brought into scope (after renaming) from a
module $m$ by one or more \mbox{\tt import} declarations may be named by the
form $m\mbox{\tt ..}$, which is equivalent to listing all of the entities
imported from the module.  For example,
\bprog
\mbox{\tt \ \ \ \ \ \ module\ Queue(\ Stack..,\ enqueue,\ dequeue\ )\ where}\\
\mbox{\tt \ \ \ \ \ \ \ \ \ \ import\ Stack}\\
\mbox{\tt \ \ \ \ \ \ \ \ \ \ ...}
\eprog
Here the module \mbox{\tt Queue} uses the module name \mbox{\tt Stack} in its export
list to abbreviate all the entities imported from \mbox{\tt Stack}.

It is a static error to have circular dependencies between
imports/exports using this naming convention.  For example, the
following is not allowed:
\bprog
\mbox{\tt module\ X(\ Y..\ )\ \ \ \ \ --\ ILLEGAL}\\
\mbox{\tt import\ Y\ \ \ \ \ \ \ \ \ \ \ \ --}\\
\mbox{\tt x\ =\ 1\ \ \ \ \ \ \ \ \ \ \ \ \ \ \ --}\\
\mbox{\tt }\\[-8pt]
\mbox{\tt module\ Y(\ X..\ )\ \ \ \ \ --}\\
\mbox{\tt import\ X\ \ \ \ \ \ \ \ \ \ \ \ --}\\
\mbox{\tt y\ =\ 1\ \ \ \ \ \ \ \ \ \ \ \ \ \ \ --}
\eprogNoSkip
\item
A module can name its own local definitions in its export
list using its own name in the \mbox{$\it m\makebox{\tt ..}$} syntax.  For example,
\bprog
\mbox{\tt \ \ \ \ \ \ \ module\ Mod1(Mod1..,\ Mod2..)}\\
\mbox{\tt \ \ \ \ \ \ \ import\ Mod2}\\
\mbox{\tt \ \ \ \ \ \ \ import\ Mod3}
\eprog
Here module \mbox{\tt Mod1} exports all local definitions as well as those
from \mbox{\tt Mod2} but not \mbox{\tt Mod3}.
\end{enumerate}

\subsubsection{Import Declarations}
\label{import}
\index{import declaration}

\begin{flushleft}\it\begin{tabbing}
\hspace{0.5in}\=\hspace{3.0in}\=\kill
$\it impdecl$\>\makebox[3.5em]{$\rightarrow$}$\it \makebox{\tt import}\ modid\ [impspec]\ [\makebox{\tt renaming}\ renamings]$\\ 
$\it impspec$\>\makebox[3.5em]{$\rightarrow$}$\it \makebox{\tt (}\ import_1\ \makebox{\tt ,}\ \ldots \ \makebox{\tt ,}\ import_n\ \makebox{\tt )}$\>\makebox[3em]{}$\it \qquad\ (n\geq 0)$\\ 
$\it $\>\makebox[3.5em]{$|$}$\it \makebox{\tt hiding}\ \makebox{\tt (}\ import_1\ \makebox{\tt ,}\ \ldots \ \makebox{\tt ,}\ import_n\ \makebox{\tt )}$\>\makebox[3em]{}$\it \qquad\ (n\geq 1)$\\ 
$\it import$\>\makebox[3.5em]{$\rightarrow$}$\it entity$\\ 
$\it renamings$\>\makebox[3.5em]{$\rightarrow$}$\it \makebox{\tt (}\ renaming_1\ \makebox{\tt ,}\ \ldots \ \makebox{\tt ,}\ renaming_n\ \makebox{\tt )}$\>\makebox[3em]{}$\it \qquad\ (n\geq 1)$\\ 
$\it renaming$\>\makebox[3.5em]{$\rightarrow$}$\it varid_1\ \makebox{\tt to}\ varid_2$\\ 
$\it $\>\makebox[3.5em]{$|$}$\it conid_1\ \makebox{\tt to}\ conid_2$
\end{tabbing}\end{flushleft}
\indexsyn{impdecl}%
\indexsyn{impspec}%
\indexsyn{import}%
\indexsyn{renamings}%
\indexsyn{renaming}%
% [\mbox{\tt ::} [context \mbox{\tt =>}] type]            TYPE SIGS NOT ALLOWED NOW

The entities exported by a module may be brought into scope in
another module with
an \mbox{\tt import}
declaration at the beginning
of the module.  
The \mbox{\tt import} declaration names the module to be
imported, optionally specifies the entities to be imported,
and optionally provides renamings for imported entities.
A single module may be imported by more
than one \mbox{\tt import} declaration.

Exactly which entities are to be imported can be specified in one
of three ways:\nopagebreak[4]
\begin{enumerate}
\item
The set of entities to be imported can be specified explicitly
by listing them in parentheses.
Items in the list have the same form as those in export lists, except
that the $modid$ abbreviation is not permitted.

The list must name a subset of the 
entities exported by the imported module.
The list may be empty, in which case nothing is imported;
this is only useful in the case of the module \mbox{\tt Prelude} (see
Section~\ref{std-prel-shadowing}).
\item
Specific entities can be excluded by 
using the form \mbox{\tt hiding(} $import_1 \mbox{\tt ,} ... \mbox{\tt ,} import_n$ \mbox{\tt )}, which
specifies that all entities exported by the named module should
be imported apart from those named in the list.
\item
Finally, if $impspec$ is omitted then 
all the entities exported by the specified module are imported.
\end{enumerate}

As instance declarations do not have names, their import cannot be
controlled by the \mbox{$\it impspec$} list.  Instead, the following rule is
used: {\em A $C$-$T$ instance declaration is imported from an
interface if and only if $C$ is imported or $T$ is imported
from that interface.}

Some or all of the imported entities may be renamed,\index{renaming}
thus allowing them to be known by a new name in the importing scope
(see Section~\ref{original-names}).  This is done using the
\mbox{\tt renaming} keyword, with a renaming of the form $oldname\mbox{\tt \ to\ }newname$.
%All renaming is subject to the constraint that each name in a scope
%must refer to exactly one entity; however, a single entity may be given
%more than one name.

\subsection{Module Interfaces}
\label{module-interfaces}
\index{module!interface}

Every module has an {\em interface}\index{interface}
containing all the information needed to do static checks on any
importing module.  All static checks on a module implementation can be
done by inspecting its text and the interfaces of the modules
it imports.

%
%interface-> \mbox{\tt interface} modid \mbox{\tt where} ibody
%
%ibody    -> \mbox{\tt {\char'173}} [ [fixdecls \mbox{\tt ;}] itopdecls] \mbox{\tt {\char'175}}
%itopdecls -> itopdecl_1 \mbox{\tt ;} ... \mbox{\tt ;} itopdecl_n  & \qquad (n>=1) 
%itopdecl  -> \mbox{\tt type} simple \mbox{\tt =} type
%          | \mbox{\tt data} [context \mbox{\tt =>}] simple [\mbox{\tt =} constrs [\mbox{\tt deriving} (tycls | \mbox{\tt (}tyclses\mbox{\tt )})]]
%          | \mbox{\tt class} [context \mbox{\tt =>}] class [\mbox{\tt where} \mbox{\tt {\char'173}} icdecls [\mbox{\tt ;}] \mbox{\tt {\char'175}}]
%          | \mbox{\tt instance} [context \mbox{\tt =>}] tycls inst
%          | vars \mbox{\tt ::} [context \mbox{\tt =>}] type
%icdecls           -> icdecl_1 \mbox{\tt ;} ... \mbox{\tt ;} icdecl_n  & (n>=1)
%icdecl    -> vars \mbox{\tt ::} type
%

\begin{flushleft}\it\begin{tabbing}
\hspace{0.5in}\=\hspace{3.0in}\=\kill
$\it interface$\>\makebox[3.5em]{$\rightarrow$}$\it \makebox{\tt interface}\ modid\ \makebox{\tt where}\ ibody$\\ 
$\it $\\ 
$\it ibody$\>\makebox[3.5em]{$\rightarrow$}$\it \makebox{\tt {\char'173}}\ [iimpdecls\ \makebox{\tt ;}]\ [fixdecls\ \makebox{\tt ;}]\ itopdecls\ [\makebox{\tt ;}]\ \makebox{\tt {\char'175}}$\\ 
$\it $\>\makebox[3.5em]{$|$}$\it \makebox{\tt {\char'173}}\ iimpdecls\ [\makebox{\tt ;}]\ \makebox{\tt {\char'175}}$\\ 
$\it iimpdecls$\>\makebox[3.5em]{$\rightarrow$}$\it iimpdecl_1\ \makebox{\tt ;}\ \ldots \ \makebox{\tt ;}\ iimpdecl_n$\>\makebox[3em]{}$\it \qquad\ (n\geq 1)$\\ 
$\it iimpdecl$\>\makebox[3.5em]{$\rightarrow$}$\it \makebox{\tt import}\ modid\ \makebox{\tt (}\ import_1\ \makebox{\tt ,}\ \ldots \ \makebox{\tt ,}\ import_n\ \makebox{\tt )}$\\ 
$\it $\>\makebox[3em]{}$\it [\makebox{\tt renaming}\ renamings]$\>\makebox[3em]{}$\it \qquad\ (n\geq 1)$\\ 
$\it itopdecls$\>\makebox[3.5em]{$\rightarrow$}$\it itopdecl_1\ \makebox{\tt ;}\ \ldots \ \makebox{\tt ;}\ itopdecl_n$\>\makebox[3em]{}$\it \qquad\ (n\geq 1)$\\ 
$\it itopdecl$\>\makebox[3.5em]{$\rightarrow$}$\it \makebox{\tt type}\ simple\ \makebox{\tt =}\ type$\\ 
$\it $\>\makebox[3.5em]{$|$}$\it \makebox{\tt data}\ [context\ \makebox{\tt =>}]\ simple\ [\makebox{\tt =}\ constrs]\ [\makebox{\tt deriving}\ (tycls\ |\ \makebox{\tt (}tyclses\makebox{\tt )})]$\\ 
$\it $\>\makebox[3.5em]{$|$}$\it \makebox{\tt class}\ [context\ \makebox{\tt =>}]\ class\ [\makebox{\tt where}\ \makebox{\tt {\char'173}}\ icdecls\ [\makebox{\tt ;}]\ \makebox{\tt {\char'175}}]$\\ 
$\it $\>\makebox[3.5em]{$|$}$\it \makebox{\tt instance}\ [context\ \makebox{\tt =>}]\ tycls\ inst$\\ 
$\it $\>\makebox[3.5em]{$|$}$\it vars\ \makebox{\tt ::}\ [context\ \makebox{\tt =>}]\ type$\\ 
$\it icdecls$\>\makebox[3.5em]{$\rightarrow$}$\it icdecl_1\ \makebox{\tt ;}\ \ldots \ \makebox{\tt ;}\ icdecl_n$\>\makebox[3em]{}$\it (n\geq 1)$\\ 
$\it icdecl$\>\makebox[3.5em]{$\rightarrow$}$\it vars\ \makebox{\tt ::}\ type$
\end{tabbing}\end{flushleft}
\indexsyn{interface}%
\indexsyn{ibody}%
\indexsyn{iimpdecls}%
\indexsyn{iimpdecl}%
\indexsyn{itopdecls}%
\indexsyn{itopdecl}%
\indexsyn{icdecls}%
\indexsyn{icdecl}%

The syntax of \mbox{\tt interface} is similar to that of \mbox{\tt module}, except:
\begin{itemize}
\item
There is no export list: everything in the interface is exported.
\item
\mbox{\tt import} declarations have a
slightly different purpose from those
in implementations (see Section~\ref{interface-imports}).
The list of entities to be imported is always specified explicitly.
\item
\mbox{\tt data} declarations appear without their constructors if these
are not exported.
\item
There is no implementation part to \mbox{\tt instance} declarations.
\item
Value declarations do not appear at all; for exported values, type signatures
take their place.
\end{itemize}


%The syntax of \mbox{\tt interface} is similar to that of \mbox{\tt module}, except:
%\begin{itemize}
%\item
%In interface files, the syntax for \mbox{$\it var$}, \mbox{$\it tycon$}, and \mbox{$\it tycls$} is expanded 
%to include full names, and thus the productions in Section~\ref{ids} should
%be replaced by:
%
%var    ->  varid [\mbox{\tt {\char'173}}modid [varid]\mbox{\tt {\char'175}}]         & (variables)
%tycon  ->  aconid [\mbox{\tt {\char'173}}modid [aconid]\mbox{\tt {\char'175}}]       & (type constructors)
%tycls  ->  aconid [\mbox{\tt {\char'173}}modid [aconid]\mbox{\tt {\char'175}}]       & (type classes)
%
%which are BNF rules (i.e.~not part of the lexical syntax) and thus
%whitespace is allowed between lexemes.
%\item
%There is no export list: everything in the interface is exported.
%\item
%There are no \mbox{\tt import} declarations.
%\item
%\mbox{\tt data} declarations appear without their constructors if these
%are not exported.
%\item
%There is no implementation part to \mbox{\tt instance} declarations.
%\item
%Value declarations do not appear at all; for exported values, type signatures
%take their place.
%\end{itemize}

\subsubsection{Consistency}
\label{consistency}

The interface and implementation of a module must obey certain
constraints.
(In the following,
the phrase ``in the implementation'' refers to something
either declared within the implementation or imported by it.)
\begin{enumerate}
%\item
%The names of entities exported by a module, and types or classes not
%exported but 
%referred to in an interface must appear as full names. (See
%Section~\ref{interface-imports}).
\item
Every entity given a declaration in an interface must either have an
import declaration for the entity in the interface (the import
specifies the module that defines it) or have a definition of the
entity in the implementation.  Furthermore, if an interface A imports
an entity X from module B (perhaps renaming it), then the interface
for B must define X but not import it.

\item
A class, type synonym, algebraic datatype, or value appears in the
\index{algebraic datatype}
\index{type synonym}
interface exactly when its name appears in the implementation's export
list or, if the export list is omitted, when it is {\em declared} in
the implementation.

\item
A type signature appears in the interface for every value that the
implementation exports.  
%Apart from any required use of full names, 
This
type signature must be the same as that 
in the implementation (see Section~\ref{type-semantics}), where
the latter is obtained from the explicit type signature in the
implementation (when present) or is the most general type
inferred from the declaration of the value.

\item
%Apart from any required use of full names, 
A \mbox{\tt type} declaration in an
interface must be identical to that in the implementation.

\item
%Apart from any required use of full names,  
A \mbox{\tt class} declaration in an
interface must be identical to that in the implementation, except that
default-method declarations are omitted.
\index{default method}

\item
If the constructors of a \mbox{\tt data} declaration are not exported, 
%then, apart from any required use of full names,
then the \mbox{\tt data} declaration in the interface differs from that in the
implementation by omitting everything after (and including) the \mbox{\tt =}
sign.  If the \mbox{\tt data} declaration in the implementation uses the
\mbox{\tt deriving} mechanism to derive instance declarations for the type, a
separate \mbox{\tt instance} declaration must appear in the interface for each
class of which the type is made an instance of.  Hence, the
information that certain instances are derived is hidden when the
constructors are hidden, since in this case the type is abstract (see
Section~\ref{abstract-types}).
% The interface 
% should not change in any way if the implementation were to change an
% instance from being derived to being explicit, or vice versa.

\item
If the constructors of a \mbox{\tt data} declaration are to be exported, then
the \mbox{\tt data} declaration in the interface is identical to that in the
implementation including the \mbox{\tt deriving} part.\footnote{ It is important
to retain the information about which instances are derived and which
are not, because the importing module ``knows'' more about derived
instances.}

\item
If a $C$-$T$ instance is declared in a module or imported by it, then the
instance declaration appears in the interface (omitting the \mbox{\tt where}
part) if {\em either} $C$ is exported {\em or} $T$ is exported.
Instance declarations are not named explicitly in export or
import lists.  This rule ensures that, if $C$ and $T$ are both in
scope, then the (unique) $C$-$T$ instance declaration will also be in
scope.\footnote{The reverse also applies.  For example,
suppose that a new type $T$ is declared and made an instance of an
imported class $C$.  The instance declaration will be exported along
with $T$, and so the closure rule (Section~\ref{closure}) will require
that $C$ is also in scope in every importing scope.}

No explicit instance declaration should appear in the interface for
instances that are specified by the \mbox{\tt deriving} part of a \mbox{\tt data}
declaration in the interface.

\item
A fixity declaration for a value or constructor appears in an
interface exactly when (a)~the value or constructor is declared by the
interface, and (b)~the identical fixity declaration appears either in
the implementation or in an imported interface.
\end{enumerate}

%\subsubsection{Full Names and Original Names in Interfaces}
%\label{interface-imports}
%\index{original name!in an interface}
%\index{full name!in an interface}
%
%An entity defined in module \mbox{\tt A} as \mbox{\tt f} has an {\em original name} which is
%a pair of \mbox{\tt A} and \mbox{\tt f}, but is never written except as part of a {\em full
%name}. 
%
%Full names, or permitted abbreviations of full names are used in
%interfaces for the names of type synonyms, datatypes, classes and values.
%
%The {\em full name} of an entity defines its original name and its
%current renaming, if any. The {\em full name} of an entity defined in
%module \mbox{\tt A} as \mbox{\tt f}, and subsequently renamed as \mbox{\tt g}, is \mbox{\tt g\ {\char'173}A\ f{\char'175}}. The
%{\em full name} of an entity defined in module \mbox{\tt A} as
%\mbox{\tt f}, and not subsequently renamed, is \mbox{\tt f\ {\char'173}A\ f{\char'175}}. The {\em local name}
%\index{local name}
%of an entity is the first component of its full name, i.e. excluding
%the part in braces.
%
%In the following, a {\em defining occurrence} of a name is an occurence in
%one of the following positions in an {\em itopdecl}, indicating that the
%entity associated with the name is exported.
%\begin{enumerate}
%\item
%Immediately following the keyword {\it type}, or {\em data} or {\em class},
%or immediately following the \mbox{\tt =>} terminating a context immediately
%following one of these keywords.
%\item
%As a {\em var} before the \mbox{\tt ::} in a type signature corresponding to a value
%declaration in the module implementation.
%\end{enumerate}
%
%Full names may be abbreviated in some circumstances:
%\begin{enumerate}
%\item
%For an entity that has not been renamed, \mbox{\tt f\ {\char'173}A\ f{\char'175}} may be written as \mbox{\tt f\ {\char'173}A{\char'175}}.
%\item
%For an entity defined in the current module, no module name need be given,
%so in the interface for module \mbox{\tt A}, \mbox{\tt f\ {\char'173}A\ f{\char'175}} may be written as \mbox{\tt f}.
%\item
%If the full name \mbox{\tt g\ {\char'173}A\ f{\char'175}} or any abbreviation of it appears in a defining
%occurence in the current interface, all other occurrences of that name in the
%interface may use the local name \mbox{\tt g}.
%\item
%The names of data constructors (in {\em constrs}) and class
%operations (in {\em icdels}) may be abbreviated to the local name.
%
%\item
%Entities imported from the standard prelude without renaming may always use
%the local name.
%\end{enumerate}

\subsubsection{Imports and Original Names}
\label{interface-imports}
\index{original name!in an interface}

The original-name information is carried in the interface file using
\mbox{\tt import} declarations in a special way.

Suppose that a module \mbox{\tt A} exports an entity \mbox{\tt x}; the
interface for \mbox{\tt A} will contain static information about \mbox{\tt x}.  
If \mbox{\tt x} was originally defined in \mbox{\tt A}, then this is
all that appears.
But, suppose that \mbox{\tt x} was imported by \mbox{\tt A} from some other module \mbox{\tt B}
and that \mbox{\tt x} was originally defined in module \mbox{\tt C}
with name \mbox{\tt y}; this declaration must appear in the interface for \mbox{\tt A}:
\bprog
\mbox{\tt import\ C(y)\ renaming\ (\ y\ to\ x\ )}
\eprog
No reference to \mbox{\tt B} remains in the interface.  {\em The
\mbox{\tt import} declaration in the interface serves only to convey to the
importing module the original name of \mbox{\tt x}}, and does {\em not} imply
that module \mbox{\tt C}'s interface must be consulted when reading module \mbox{\tt A}'s
interface.  Multiple imports from a single original module may
optionally be grouped in a single import declaration in the interface.

A module may export a value whose typing involves a type and/or class
that is not exported.
(Any importing module would have to import the type or class by some
other route.)
{\em Nevertheless, it is still required that the interface contain the
import declaration required to give the original name of the type or class.}

In summary, for every entity \mbox{\tt e1} mentioned in the interface 
of a module \mbox{\tt M} whose original name is \mbox{\tt e2} in module \mbox{\tt N},
\mbox{\tt M}'s interface must
contain the \mbox{\tt import} declaration
\bprog
\mbox{\tt \ \ \ \ \ \ import\ N(e2)\ renaming\ (\ e2\ to\ e1\ )}
\eprog
The word ``mentioned'' includes mention in the type signature
of an exported value, as discussed above.

This example illustrates most of these constraints; first, the
interface:
\bprog
\mbox{\tt interface\ A\ where}\\
\mbox{\tt infixr\ 4\ }\bkqB\mbox{\tt sameShape}\bkqA\mbox{\tt \ \ \ \ \ \ \ \ }\\
\mbox{\tt import\ PreludeList(sum)\ renaming\ (\ sum\ to\ oldSum\ )}\\
\mbox{\tt data\ \ BinTree\ a\ =\ Empty\ |\ Branch\ a\ (BinTree\ a)\ (BinTree\ a)\ \ \ \ \ \ }\\
\mbox{\tt class\ Tree\ a\ where\ }\\
\mbox{\tt \ \ \ \ \ \ \ \ sameShape\ ::\ a\ ->\ a\ ->\ Bool}\\
\mbox{\tt instance\ Tree\ (BinTree\ a)}\\
\mbox{\tt sum\ ::\ Num\ a\ =>\ BinTree\ a\ ->\ a}\\
\mbox{\tt oldSum\ ::\ Num\ a\ =>\ [a]\ ->\ a}
\eprog
Now the implementation:
\bprog
\mbox{\tt module\ A(\ BinTree(..),\ Tree(..),\ sum,\ oldSum\ )\ where}\\
\mbox{\tt import\ Prelude\ renaming\ (\ sum\ to\ oldSum\ )}\\
\mbox{\tt infixr\ 4\ }\bkqB\mbox{\tt sameShape}\bkqA\mbox{\tt }\\
\mbox{\tt \ \ \ \ \ \ \ \ --\ }\bkqB\mbox{\tt sameShape}\bkqA\mbox{\tt \ is\ an\ operation\ of\ class\ C\ below}\\
\mbox{\tt }\\[-8pt]
\mbox{\tt data\ BinTree\ a\ =\ Empty\ |\ Branch\ a\ (BinTree\ a)\ (BinTree\ a)}\\
\mbox{\tt }\\[-8pt]
\mbox{\tt class\ Tree\ a\ where\ }\\
\mbox{\tt \ \ \ \ \ \ sameShape\ ::\ a\ ->\ a\ ->\ Bool}\\
\mbox{\tt \ \ \ \ \ \ t1\ }\bkqB\mbox{\tt sameShape}\bkqA\mbox{\tt \ t2\ =\ False\ \ \ \ \ --\ Default\ method}\\
\mbox{\tt }\\[-8pt]
\mbox{\tt instance\ Tree\ (BinTree\ a)\ where\ }\\
\mbox{\tt \ \ \ \ \ \ \ \ \ Empty\ }\bkqB\mbox{\tt sameShape}\bkqA\mbox{\tt \ Empty\ \ =\ \ True}\\
\mbox{\tt \ \ \ \ \ \ \ \ \ (Branch\ {\char'137}\ t1\ t2)\ }\bkqB\mbox{\tt sameShape}\bkqA\mbox{\tt \ (Branch\ {\char'137}\ t1'\ t2')\ \ }\\
\mbox{\tt \ \ \ \ \ \ \ \ \ \ \ \ =\ \ (t1\ }\bkqB\mbox{\tt sameShape}\bkqA\mbox{\tt \ t1')\ {\char'46}{\char'46}\ (t2\ }\bkqB\mbox{\tt sameShape}\bkqA\mbox{\tt \ t2')}\\
\mbox{\tt \ \ \ \ \ \ \ \ \ t1\ }\bkqB\mbox{\tt sameShape}\bkqA\mbox{\tt \ t2\ =\ False}\\
\mbox{\tt }\\[-8pt]
\mbox{\tt sum\ \ Empty\ \ \ \ \ \ \ \ \ \ \ =\ 0}\\
\mbox{\tt sum\ (Branch\ n\ t1\ t2)\ =\ n\ +\ sum\ t1\ +\ sum\ t2}
\eprogNoSkip

\subsection{Standard Prelude}
\label{standard-prelude}
\index{standard prelude}

Many of the features of \Haskell{} are defined in \Haskell{}
itself, as a library of standard datatypes, classes and
functions, called the ``standard prelude.''  In
\Haskell{}, the standard prelude is specified as two distinct modules
(in the technical sense of this chapter), \mbox{\tt PreludeCore} and
\mbox{\tt Prelude}.\indexmodule{Prelude}
\indexmodule{PreludeCore}

\mbox{\tt PreludeCore} and \mbox{\tt Prelude} differ from other modules in that
{\em their interfaces,
and the semantics of the entities defined by those interfaces, are
part of the \Haskell{} language definition.} This means, for example, that a
compiler may optimise calls to functions in the
standard prelude, because it knows their semantics as well as their
interface.

Each of these modules is structured into submodules.
To avoid name-clashes with these sub-modules,
user-defined module names must not begin with the prefix \mbox{\tt Prelude}.

\subsubsection{The \mbox{\tt PreludeCore} Module}
\indexmodule{PreludeCore}

The \mbox{\tt PreludeCore} module contains {\em all the algebraic datatypes,
type synonyms, classes and instance declarations} specified by the
standard prelude.
\index{algebraic datatype}
\index{type synonym}
\index{class declaration}
\index{instance declaration}

\mbox{\tt PreludeCore} is {\em always implicitly imported}, so it
is not possible to import only part of it or to rename any of the
entities that it defines.

The semantics of the entities defined by \mbox{\tt PreludeCore} is specified by
an implementation written in \Haskell{}, in
Appendix~\ref{preludecore}.
A \Haskell{} system need not implement \mbox{\tt PreludeCore} in this way.
The interface for \mbox{\tt PreludeCore} may
be inferred from the implementation in
Appendix~\ref{preludecore}.

Some datatypes (such as \mbox{\tt Int}) and functions (such as addition of
\mbox{\tt Int}s) cannot be specified directly in \Haskell{}.  This is
expressed in the \mbox{\tt PreludeCore} implementation by importing these
built-in types and values from
\mbox{\tt PreludeBuiltin}.\indexmodule{PreludeBuiltin}
The semantics of the built-in datatypes and
functions is given as English text in
Appendix~\ref{preludebuiltin}.

The implementation for \mbox{\tt PreludeCore} is incomplete in its
treatment of tuples: there should be an infinite family of instance
declarations for tuples, but the implementation only gives a scheme.

The alert reader may notice that the implementation of
\mbox{\tt PreludeCore} given in Appendix~\ref{preludecore} uses
some functions defined in \mbox{\tt Prelude} (see next section).  There is no
conflict; \mbox{\tt PreludeCore} and \mbox{\tt Prelude} are mutually
recursive.

\subsubsection{The \mbox{\tt Prelude} Module}
\indexmodule{Prelude}

The \mbox{\tt Prelude} module contains all the {\em value} declarations
in the standard prelude.

The \mbox{\tt Prelude} module is imported automatically if and only if it is
not imported with an explicit \mbox{\tt import} declaration.  This provision
for explicit import allows values defined in the standard prelude to
be renamed or not imported at all.

The semantics of the entities in \mbox{\tt Prelude} is specified by
an implementation of \mbox{\tt Prelude} written in \Haskell{}, given in
Appendix~\ref{stdprelude}.  As for \mbox{\tt PreludeCore},
a \Haskell{} system may implement the \mbox{\tt Prelude} module
as it pleases, provided it maintains the semantics in
Appendix~\ref{stdprelude}.  The interface can be inferred
from this implementation.

\subsubsection{Shadowing Prelude Names and Non-Standard Preludes}
\label{std-prel-shadowing}

The rules about the standard prelude have been cast so that it is
possible to use standard prelude names for nonstandard purposes; however,
every module that does so will have an \mbox{\tt import} declaration
that makes this nonstandard usage explicit.  For example:
\bprog
\mbox{\tt \ \ \ \ \ module\ A\ where}\\
\mbox{\tt \ \ \ \ \ import\ Prelude\ hiding\ (map)}\\
\mbox{\tt \ \ \ \ \ map\ f\ x\ =\ x\ f}
\eprog
Module \mbox{\tt A} redefines \mbox{\tt map}, but it must indicate this by
importing \mbox{\tt Prelude} without \mbox{\tt map}.  Furthermore, \mbox{\tt A} exports \mbox{\tt map},
but every module that imports \mbox{\tt map} from \mbox{\tt A} must also hide
\mbox{\tt map} from \mbox{\tt Prelude} just as \mbox{\tt A} does.  Thus there is little danger of accidentally
shadowing standard prelude names.

It is possible to construct and use a different \mbox{\tt Prelude} module:
\bprog
\mbox{\tt \ \ \ \ \ \ module\ B\ where}\\
\mbox{\tt \ \ \ \ \ \ import\ Prelude()}\\
\mbox{\tt \ \ \ \ \ \ import\ MyPrelude}\\
\mbox{\tt \ \ \ \ \ \ ...}
\eprog
\mbox{\tt B} imports nothing from \mbox{\tt Prelude}, but the
explicit \mbox{\tt import\ Prelude} declaration prevents the automatic import of
\mbox{\tt Prelude}.  \mbox{\tt import\ MyPrelude} brings the
non-standard prelude into scope.  As before, the
standard prelude names are hidden explicitly.

%It is not possible to rename or hide the entities imported
%from \mbox{\tt PreludeCore}.

\subsection{Example}

As an example, here are two small modules:\nopagebreak
\bprog
\mbox{\tt module\ A(\ Tree(..),\ depth\ )\ where}\\
\mbox{\tt data\ Tree\ a\ =\ Leaf\ a\ |\ Branch\ (Tree\ a)\ (Tree\ a)}\\
\mbox{\tt depth\ (Leaf\ a)\ \ \ \ \ \ \ \ =\ \ 0}\\
\mbox{\tt depth\ (Branch\ xt\ yt)\ \ =\ \ (depth\ xt\ }\bkqB\mbox{\tt max}\bkqA\mbox{\tt \ depth\ yt)\ +\ 1}\\
\mbox{\tt }\\[-8pt]
\mbox{\tt module\ B(\ leaves\ )\ where}\\
\mbox{\tt import\ A}\\
\mbox{\tt leaves\ (Leaf\ a)\ \ \ \ \ \ \ \ =\ \ [a]}\\
\mbox{\tt leaves\ (Branch\ xt\ yt)\ \ =\ \ leaves\ xt\ ++\ leaves\ yt}
\eprog
Module \mbox{\tt A} must export \mbox{\tt Tree} because it exports \mbox{\tt depth}, and \mbox{\tt Tree}
could not be made visible in any other way.  However, \mbox{\tt B} is not
required to export \mbox{\tt Tree}, since a module importing \mbox{\tt B} could import
\mbox{\tt A} in order to satisfy the closure constraints.

Modules may be used to combine the resources of other modules.  For
example, one might use renaming to make trees available to
French speakers:
\bprog
\mbox{\tt module\ C(\ Arbre(..),\ fond,\ feuilles\ )\ where}\\
\mbox{\tt import\ A\ renaming\ (\ Tree\ to\ Arbre,\ Leaf\ to\ Feuille,\ Branch\ to\ Branche,}\\
\mbox{\tt \ \ \ \ \ \ \ \ \ \ \ \ \ \ \ \ \ \ \ \ depth\ to\ fond\ )}\\
\mbox{\tt import\ B\ renaming\ (\ leaves\ to\ feuilles\ )}
\eprogNoSkip

\subsection{Abstract Datatypes}
\label{abstract-types}

\index{abstract datatype}
The ability to export a datatype without its constructors
allows the construction of abstract datatypes (ADTs).  For example,
an ADT for stacks could be defined as:
\bprog
\mbox{\tt module\ Stack(\ StkType,\ push,\ pop,\ empty\ )\ where}\\
\mbox{\tt \ \ \ \ \ \ \ \ data\ StkType\ a\ =\ EmptyStk\ |\ Stk\ a\ (StkType\ a)}\\
\mbox{\tt \ \ \ \ \ \ \ \ push\ x\ s\ =\ Stk\ x\ s}\\
\mbox{\tt \ \ \ \ \ \ \ \ pop\ (Stk\ {\char'137}\ s)\ =\ s}\\
\mbox{\tt \ \ \ \ \ \ \ \ empty\ =\ EmptyStk}
\eprog
Modules importing \mbox{\tt Stack} cannot construct values of type \mbox{\tt StkType}
because they do not have access to the constructors of the type.

It is also possible to build an ADT on top of an existing type by
using a \mbox{\tt data} declaration with a single constructor with only one
field.  For example, stacks can be defined with lists:
\bprog
\mbox{\tt module\ Stack(\ StkType,\ push,\ pop,\ empty\ )\ where}\\
\mbox{\tt \ \ \ \ \ \ \ \ data\ StkType\ a\ =\ Stk\ [a]}\\
\mbox{\tt \ \ \ \ \ \ \ \ push\ x\ (Stk\ s)\ =\ Stk\ (x:s)}\\
\mbox{\tt \ \ \ \ \ \ \ \ pop\ (Stk\ (x:s))\ =\ Stk\ s}\\
\mbox{\tt \ \ \ \ \ \ \ \ empty\ =\ Stk\ []}
\eprogNoSkip

{\em Note 1.} Every ADT must be a module (but a
\Haskell{} compilation system may allow multiple modules in a single file).

%Arguably there should be a separate scope-limiting construct to allow
%constructors to be hidden within modules.  This was considered but
%finally discarded on grounds of simplicity.  Remember that

{\em Note 2.} Using a single-constructor single-field \mbox{\tt data}
declaration to create an isomorphic type introduces an unwanted extra
element to the new type, namely \mbox{\mbox{\tt (Stk\ }$\perp$\mbox{\tt )}}, with the
risk of an accompanying small inefficiency in the implementation.  

%It would be possible to introduce a variant of \mbox{\tt data} to avoid this
%problem, but this was omitted on grounds of simplicity.

\subsection{Fixity Declarations}
\index{fixity declaration}
\label{fixity}

\begin{flushleft}\it\begin{tabbing}
\hspace{0.5in}\=\hspace{3.0in}\=\kill
$\it fixdecls$\>\makebox[3.5em]{$\rightarrow$}$\it fix_1\ \makebox{\tt ;}\ \ldots \ \makebox{\tt ;}\ fix_n$\>\makebox[3em]{}$\it \qquad\ (n\geq 1)$\\ 
$\it fix$\>\makebox[3.5em]{$\rightarrow$}$\it \makebox{\tt infixl}\ [digit]\ ops$\\ 
$\it $\>\makebox[3.5em]{$|$}$\it \makebox{\tt infixr}\ [digit]\ ops$\\ 
$\it $\>\makebox[3.5em]{$|$}$\it \makebox{\tt infix\ }\ [digit]\ ops$\\ 
$\it ops$\>\makebox[3.5em]{$\rightarrow$}$\it op_1\ \makebox{\tt ,}\ \ldots \ \makebox{\tt ,}\ op_n$\>\makebox[3em]{}$\it \qquad\ (n\geq 1)$\\ 
$\it op$\>\makebox[3.5em]{$\rightarrow$}$\it varop\ |\ conop$
\end{tabbing}\end{flushleft}
\indexsyn{fixdecls}%
\indexsyn{fix}%
\indexsyn{ops}%
\indexsyn{op}%
%\indextt{infixl}\indextt{infixr}\indextt{infix}
A fixity declaration gives the fixity and binding
precedence of a set of operators.  Fixity declarations must appear only
at the start of a
module
%\footnote{This is to avoid parsing problems that arise when
%fixity declarations appear lexically after the operators to which they
%refer.}
and may only be given for identifiers defined in that module.
Fixity declarations cannot subsequently be overridden, and an
identifier can only have one fixity definition.

There are three kinds of fixity, non-, left- and right-associativity
(\mbox{\tt infix}, \mbox{\tt infixl}, and \mbox{\tt infixr}, respectively), and ten precedence
levels, 0 through 9 (level 0 binds least tightly, and level 9
binds most tightly).  If the \mbox{$\it digit$} is omitted, level 9 is assumed.
Any operator lacking a fixity declaration
is assumed to be \mbox{\tt infixl\ 9} (See Section~\ref{expressions} for more on
the use of fixities).
Figure~\ref{prelude-fixities} lists the fixities and precedences of
the operators defined in the standard prelude.
\begin{figure}
\centerline{
\begin{tabular}{|c|l|l|}\hline
Precedence & Fixity & Operators \\ \hline
9 & \mbox{\tt infixl} & \mbox{\tt !}, \mbox{\tt !!}, \mbox{\tt //} \\
  & \mbox{\tt infixr} & \mbox{\tt .} \\
8 & \mbox{\tt infixr} & \mbox{\tt **}, \mbox{\tt {\char'136}}, \mbox{\tt {\char'136}{\char'136}} \\
7 & \mbox{\tt infixl} & \mbox{\tt {\char'45}}, \mbox{\tt *}, \mbox{\tt :{\char'45}} \\
  & \mbox{\tt infix}  & \mbox{\tt /}, \mbox{\tt `div`}, \mbox{\tt `mod`}, \mbox{\tt `rem`} \\
6 & \mbox{\tt infixl} & \mbox{\tt +}, \mbox{\tt -} \\
  & \mbox{\tt infix}  & \mbox{\tt :+} \\
5 & \mbox{\tt infixr} & \mbox{\tt :}, \mbox{\tt ++} \\
  & \mbox{\tt infix}  & \mbox{\tt {\char'134}{\char'134}} \\
4 & \mbox{\tt infix}  & \mbox{\tt /=}, \mbox{\tt <}, \mbox{\tt <=}, \mbox{\tt ==}, \mbox{\tt >}, \mbox{\tt >=}, \mbox{\tt `elem`}, \mbox{\tt `notElem`} \\
3 & \mbox{\tt infixr} & \mbox{\tt {\char'46}{\char'46}} \\
2 & \mbox{\tt infixr} & \mbox{\tt ||} \\
1 & \mbox{\tt infix}  & \mbox{\tt :=} \\ \hline
\end{tabular}}
\ecaption{Precedences and fixities of prelude-defined operators}
\label{prelude-fixities}
\end{figure}

Fixity is a property of the original name of an identifier or operator
(see Section~\ref{original-names}).  Fixity is not affected by
renaming; the new name has the same fixity as the old one.  The same
fixity attaches to every occurrence of an operator name in a module,
whether at the top level or rebound at an inner level.  For example:
\bprog
\mbox{\tt module\ Foo}\\
\mbox{\tt import\ Bar}\\
\mbox{\tt infix\ 3\ `op`}\\
\mbox{\tt }\\[-8pt]
\mbox{\tt f\ x\ =\ ...\ where\ p\ `op`\ q\ =\ ...}
\eprog
Here \mbox{\tt `op`} has fixity 3 wherever it is in scope, provided \mbox{\tt Bar} does not
export the identifier \mbox{\tt op}.  If \mbox{\tt Bar} does export \mbox{\tt op}, then the example
becomes illegal, because the fixity (or lack thereof) of \mbox{\tt op} is defined 
in \mbox{\tt Bar} (or wherever \mbox{\tt Bar} imported \mbox{\tt op} from).

% Local Variables: 
% mode: latex
% End:
\startnewsection
%
% $Header$
%
\section{Basic Types}
\label{basic-types}

\subsection{Booleans}
\label{booleans}
\index{boolean}

The boolean type \mbox{\tt Bool} is an enumeration; Figure~\ref{prelude-bool}
shows its definition and standard functions \mbox{\tt {\char'46}{\char'46}}, \mbox{\tt ||}, \mbox{\tt not}, and \mbox{\tt otherwise}.

\begin{figure}
\outline{
\mbox{\tt data\ \ Bool\ \ =\ \ False\ |\ True}\\
\mbox{\tt }\\[-8pt]
\mbox{\tt ({\char'46}{\char'46}),\ (||)\ \ \ \ \ \ \ \ \ \ \ \ \ \ ::\ Bool\ ->\ Bool\ ->\ Bool}\\
\mbox{\tt True\ \ {\char'46}{\char'46}\ x\ \ \ \ \ \ \ \ \ \ \ \ \ \ =\ \ x}\\
\mbox{\tt False\ {\char'46}{\char'46}\ x\ \ \ \ \ \ \ \ \ \ \ \ \ \ =\ \ False}\\
\mbox{\tt True\ \ ||\ x\ \ \ \ \ \ \ \ \ \ \ \ \ \ =\ \ True}\\
\mbox{\tt False\ ||\ x\ \ \ \ \ \ \ \ \ \ \ \ \ \ =\ \ x}\\
\mbox{\tt }\\[-8pt]
\mbox{\tt not\ \ \ \ \ \ \ \ \ \ \ \ \ \ \ \ \ \ \ \ \ ::\ Bool\ ->\ Bool}\\
\mbox{\tt not\ True\ \ \ \ \ \ \ \ \ \ \ \ \ \ \ \ =\ \ False}\\
\mbox{\tt not\ False\ \ \ \ \ \ \ \ \ \ \ \ \ \ \ =\ \ True}\\
\mbox{\tt }\\[-8pt]
\mbox{\tt otherwise\ \ \ \ \ \ \ \ \ \ \ \ \ \ \ ::\ Bool}\\
\mbox{\tt otherwise\ \ \ \ \ \ \ \ \ \ \ \ \ \ \ =\ True}
}
\ecaption{Standard functions on booleans}
\label{prelude-bool}
\indextycon{Bool}
\indextt{False}\indextt{True}
\index{||@{\ptt {\char'174}{\char'174}}}
\index{&&@{\ptt \&\&}}
\indextt{not}
\indextt{otherwise}
\end{figure}

\subsection{Characters and Strings}
\label{characters}
\index{character}\index{string}

The character type \mbox{\tt Char}\indextycon{Char}
is an enumeration, and consists of 256 values, of which the first 128
are the ASCII\index{ASCII character set} character set.  The lexical syntax for
characters is defined in Section~\ref{lexemes-char}; character
literals are nullary constructors in the datatype \mbox{\tt Char}.  The
standard prelude provides an instance declaration for \mbox{\tt Char} in classes
\mbox{\tt Enum} and \mbox{\tt Ix} and two functions relating characters to
\mbox{\tt Int}s in the range $[ 0 , 255 ]$:
\bprog
\mbox{\tt ord\ ::\ Char\ ->\ Int}\\
\mbox{\tt chr\ ::\ Int\ \ ->\ Char}
\eprogNoSkip\indextt{chr}\indextt{ord}

Note that ASCII control characters each have several representations
in character literals: numeric escapes, ASCII mnemonic escapes,
and the \mbox{\tt {\char'134}{\char'136}}$X$ notation.
In addition, there are the following equivalences:
\mbox{\tt {\char'134}a} and \mbox{\tt {\char'134}BEL}, \mbox{\tt {\char'134}b} and \mbox{\tt {\char'134}BS}, \mbox{\tt {\char'134}f} and \mbox{\tt {\char'134}FF}, \mbox{\tt {\char'134}r} and \mbox{\tt {\char'134}CR},
\mbox{\tt {\char'134}t} and \mbox{\tt {\char'134}HT}, \mbox{\tt {\char'134}v} and \mbox{\tt {\char'134}VT}, and \mbox{\tt {\char'134}n} and \mbox{\tt {\char'134}LF}.

A {\em string} is a list of characters:\nopagebreak[4]
\bprog
\mbox{\tt type\ \ String\ \ =\ \ [Char]}
\eprog\indexsynonym{String}
Strings may be abbreviated using the lexical syntax described in
Section~\ref{lexemes-char}.  For example, \mbox{\tt "A\ string"} abbreviates
\[
\mbox{\tt [}\fwq\mbox{\tt A}\fwq\mbox{\tt ,}\fwq\mbox{\tt \ }\fwq\mbox{\tt ,}\fwq\mbox{\tt s}\fwq\mbox{\tt ,}\fwq\mbox{\tt t}\fwq\mbox{\tt ,}\fwq\mbox{\tt r}\fwq\mbox{\tt ,} \
\fwq\mbox{\tt i}\fwq\mbox{\tt ,}\fwq\mbox{\tt n}\fwq\mbox{\tt ,}\fwq\mbox{\tt g}\fwq\mbox{\tt ]}
\]

\subsection{Functions}
\label{basic-functions}
\index{function}

Functions are defined via lambda abstractions and function
definitions.  Besides application, an infix composition operator is
defined:
\bprog
\mbox{\tt (.)\ ::\ (b\ ->\ c)\ ->\ (a\ ->\ b)\ ->\ a\ ->\ c}\\
\mbox{\tt (f\ .\ g)\ x\ =\ f\ (g\ x)}
\eprog\index{.@{\ptt .}}
The function \mbox{\tt until}\indextt{until} applies a
function to an initial value zero or more times until the result
satisfies a given predicate:
\bprog
\mbox{\tt until\ ::\ (a\ ->\ Bool)\ ->\ (a\ ->\ a)\ ->\ a\ ->\ a}\\
\mbox{\tt until\ p\ f\ x\ |\ p\ x\ \ \ \ \ \ \ \ =\ \ x}\\
\mbox{\tt \ \ \ \ \ \ \ \ \ \ \ \ |\ otherwise\ \ =\ \ until\ p\ f\ (f\ x)}
\eprog
The function \mbox{\tt flip},\indextt{flip} applied to a binary function, reverses
the order of the arguments:
\bprog
\mbox{\tt flip\ ::\ (a\ ->\ b\ ->\ c)\ ->\ b\ ->\ a\ ->\ c}\\
\mbox{\tt flip\ f\ x\ y\ \ =\ \ f\ y\ x}
\eprogNoSkip

\subsection{Lists}
\label{basic-lists}
\index{list}

Lists are an algebraic datatype of two constructors, although
with special syntax, as described in Section~\ref{lists}.
The first constructor is the null list, written \mbox{\tt []},
\index{[]@{\ptt []} (nil)}%
and the second is \mbox{\tt :} (``cons'').
\index{:@{\ptt :}}
See the standard prelude (Appendix~\ref{preludelist}) for the
definitions of the standard list functions.  {\em
Arithmetic sequences}
\index{arithmetic sequence}
and {\em list comprehensions},
\index{list comprehension}
two convenient
syntaxes for special kinds of lists, are described in
Sections~\ref{arithmetic-sequences} and \ref{list-comprehensions},
respectively.

\subsection{Tuples}
\label{basic-tuples}
\index{tuple}

Tuples are also algebraic datatypes with special syntax, as defined
in Section~\ref{tuples}.  Each tuple type has a single constructor.
Six functions, named \mbox{\tt zip}, \mbox{\tt zip3}, \mbox{$\it \ldots $}, \mbox{\tt zip7}, are provided
\indextt{zip}%
\indextt{zip3}%
\indextt{zip4}%
\indextt{zip5}%
\indextt{zip6}%
\indextt{zip7}%
by the standard prelude (Appendix~\ref{preludelist}).  These produce
lists of $n$-tuples from $n$ lists,
for $2 \leq n \leq 7\/$.  The resulting lists are as long as the
shortest argument list; excess elements of other argument lists are
ignored.

\subsection{Unit Datatype}
\label{basic-trivial}

The unit datatype \mbox{\tt ()} has one
member, the nullary constructor \mbox{\tt ()}
(and thus an overloading of syntax)---see also Section~\ref{unit-expression}.
\index{trivial type}

\subsection{Binary Datatype}
\label{bin-type}

The \mbox{\tt Bin}\indextycon{Bin} datatype is a
primitive abstract datatype
including the value \mbox{\tt nullBin}\indextt{nullBin} (the empty or nullary
binary value), the function \mbox{\tt appendBin}\indextt{appendBin}, and the
predicate \mbox{\tt isNullBin}\indextt{isNullBin}
(which returns \mbox{\tt True} when applied to \mbox{\tt nullBin} and \mbox{\tt False} when
applied to all other values of
type \mbox{\tt Bin}).  Also, derived instances of the \mbox{\tt Binary} class
\index{Binary@{\ptt Binary}!derived instance}
generate definitions for \mbox{\tt showBin} and \mbox{\tt readBin}, as described in
Section~\ref{derived-decls} and Appendix~\ref{derived-appendix}.  The \mbox{\tt Bin}
datatype is used primarily for efficient and transparent I/O, as
described in Section~\ref{io}.

\subsection{Numbers}
\label{numbers}
\index{number}

\subsubsection{Introduction}

\Haskell{} provides several kinds of numbers; the numeric
types and the operations upon them have been heavily influenced by
Common Lisp \cite{steele:common-lisp} and Scheme \cite{RRRRS}.  Numeric
function names and operators are usually overloaded, using
several type classes with an inclusion relation shown
in Figure~\ref{numeric-inclusions}
(cf.~Figure~\ref{standard-classes}, page~\pageref{standard-classes}).
(Some classes are immediate subclasses of two other classes;
there are pairs of classes with a nontrivial
intersection.) The class \mbox{\tt Num}\indextt{Num} of numeric types is a subclass of
\mbox{\tt Eq}\indextt{Eq}, since all numbers may be compared for equality;
its subclass \mbox{\tt Real}\indextt{Real} is also a subclass of
\mbox{\tt Ord}\indextt{Ord}, since the other comparison operations apply to
all but complex numbers.  The class \mbox{\tt Integral}\indextt{Integral}
contains both fixed- and arbitrary-precision integers; the class
\mbox{\tt Fractional}\indextt{Fractional} contains all nonintegral types;
and the class \mbox{\tt Floating}\indextt{Floating} contains all floating-point
types, both real and complex.

%\WeSay{John and Joe changed this figure so that now Text is above Num
%(this is to insure you can print numbers if an error occurs while in
%one of these classes.  Take a look at the commented text.}
\begin{figure}
\outline{
%@
%                            Eq    Text
%                           / \    /
%                          /   \  /
%                         Ord  Num
%                          \   / \
%                           \ /   \
%                           Real Fractional
%                           / \   / \
%                          /   \ /   \
%                   Integral RealFrac Floating
%                                \   /
%                                 \ /
%                               RealFloat
%
%@
\begin{center}
\setlength{\unitlength}{0.0125in}%
\begin{picture}(225,176)(80,625)
\thicklines
\put(155,700){\line(-5,-4){ 25}}
\put(195,740){\line(-5,-4){ 25}}	% Real up to Num
\put(230,780){\line(-5,-4){ 25}}	% Num up to Text
\put(235,700){\line(-5,-4){ 25}}
\put(260,660){\line(-5,-4){ 25}}
\put(160,780){\line(-3,-4){ 15}}
\put(150,740){\line( 3,-4){ 15}}
\put(180,780){\line( 3,-4){ 15}}
\put(205,660){\line( 3,-4){ 15}}
\put(210,740){\line( 3,-4){ 15}}
\put(240,700){\line( 3,-4){ 15}}
\put(175,700){\line( 3,-4){ 15}}
\put(205,625){\makebox(0,0)[lb]{\raisebox{0pt}[0pt][0pt]{\tt RealFloat}}}
\put( 90,665){\makebox(0,0)[lb]{\raisebox{0pt}[0pt][0pt]{\tt Integral}}}
\put(245,665){\makebox(0,0)[lb]{\raisebox{0pt}[0pt][0pt]{\tt Floating}}}
\put(170,665){\makebox(0,0)[lb]{\raisebox{0pt}[0pt][0pt]{\tt RealFrac}}}
\put(205,705){\makebox(0,0)[lb]{\raisebox{0pt}[0pt][0pt]{\tt Fractional}}}
\put(150,705){\makebox(0,0)[lb]{\raisebox{0pt}[0pt][0pt]{\tt Real}}}
\put(190,745){\makebox(0,0)[lb]{\raisebox{0pt}[0pt][0pt]{\tt Num}}}
\put(135,745){\makebox(0,0)[lb]{\raisebox{0pt}[0pt][0pt]{\tt Ord}}}
\put(160,785){\makebox(0,0)[lb]{\raisebox{0pt}[0pt][0pt]{\tt Eq}}}
\put(225,785){\makebox(0,0)[lb]{\raisebox{0pt}[0pt][0pt]{\tt Text}}}
\end{picture}
\end{center}

\struthack{11pt}
}
\ecaption{Numeric class inclusions (cf.~Figure~\protect\ref{standard-classes}, page~\protect\pageref{standard-classes})}
\label{numeric-inclusions}
\end{figure}

Table~\ref{standard-numeric-types} lists the standard numeric types.
The type \mbox{\tt Int}\indextycon{Int} is a fixed-precision type, covering at least the range
$[ - 2^{29} + 1, 2^{29} - 1]\/$. The range chosen by an implementation must either be
symmetric about zero or contain one more negative value than positive
(to accommodate twos-complement representation) and should be large
enough to serve as array indices.  The constants
\mbox{\tt minInt}\indextt{minInt} and \mbox{\tt maxInt}\indextt{maxInt}
(Figure~\ref{basic-numeric-2}, page~\pageref{basic-numeric-2}) define the limits of
\mbox{\tt Int}\indextycon{Int} in each implementation.
\mbox{\tt Float}\indextycon{Float} is a
floating-point type, also implementation-defined; it is desirable that
this type be at least equal in range and precision to the IEEE
single-precision type.  Similarly, \mbox{\tt Double}\indextycon{Double} should
cover IEEE double-precision.  An implementation may provide other
numeric types, such as additional
precisions of integer and
floating-point.  The results of exceptional
conditions (such as overflow or underflow) on the fixed-precision
numeric types are undefined; an implementation may choose error
(\mbox{$\it \bot$}, semantically), a truncated value, or a special value such as
infinity, indefinite, etc.

\begin{table}
\[
\begin{tabular}{|l|l|l|}
\hline
\mc{1}{|c|}{Type} & 
        \mc{1}{c|}{Class} &
        \mc{1}{c|}{Description} \\ \hline
\mbox{\tt Integer} & \mbox{\tt Integral} & Arbitrary-precision integers \\
\mbox{\tt Int} & \mbox{\tt Integral} & Fixed-precision integers \\
\mbox{\tt (Integral\ a)\ =>\ Ratio\ a} & \mbox{\tt RealFrac} & Rational numbers \\
\mbox{\tt Float} & \mbox{\tt RealFloat} & Real floating-point, single precision \\
\mbox{\tt Double} & \mbox{\tt RealFloat} & Real floating-point, double precision \\
\mbox{\tt (RealFloat\ a)\ =>\ Complex\ a} & \mbox{\tt Floating} & Complex floating-point \\
\hline
\end{tabular}
\]
\ecaption{Standard numeric types}
\label{standard-numeric-types}
\index{numeric type}
\end{table}

The interface text (Section~\ref{module-interfaces}) associated with
the standard numeric classes, types, and operations is shown
in Figures~\ref{basic-numeric-1}--\ref{basic-numeric-3}.

\begin{figure}
\outline{
\mbox{\tt class\ \ (Eq\ a,\ Text\ a)\ =>\ Num\ a\ \ where}\\
\mbox{\tt \ \ \ \ (+),\ (-),\ (*)\ \ \ \ \ \ \ ::\ a\ ->\ a\ ->\ a}\\
\mbox{\tt \ \ \ \ negate\ \ \ \ \ \ \ \ \ \ \ \ \ \ ::\ a\ ->\ a}\\
\mbox{\tt \ \ \ \ abs,\ signum\ \ \ \ \ \ \ \ \ ::\ a\ ->\ a}\\
\mbox{\tt \ \ \ \ fromInteger\ \ \ \ \ \ \ \ \ ::\ Integer\ ->\ a}\\
\mbox{\tt \ \ \ \ x\ -\ y\ \ \ \ \ \ \ \ \ \ \ \ \ \ \ =\ \ x\ +\ negate\ y}\\
\mbox{\tt }\\[-8pt]
\mbox{\tt class\ \ (Num\ a,\ Ord\ a)\ =>\ Real\ a\ where}\\
\mbox{\tt \ \ \ \ toRational\ \ \ \ \ \ \ \ \ \ ::\ \ a\ ->\ Rational}\\
\mbox{\tt }\\[-8pt]
\mbox{\tt class\ \ (Real\ a)\ =>\ Integral\ a\ \ where}\\
\mbox{\tt \ \ \ \ div,\ rem,\ mod\ \ \ \ \ \ \ ::\ a\ ->\ a\ ->\ a}\\
\mbox{\tt \ \ \ \ divRem\ \ \ \ \ \ \ \ \ \ \ \ \ \ ::\ a\ ->\ a\ ->\ (a,a)}\\
\mbox{\tt \ \ \ \ even,\ odd\ \ \ \ \ \ \ \ \ \ \ ::\ a\ ->\ Bool}\\
\mbox{\tt \ \ \ \ toInteger\ \ \ \ \ \ \ \ \ \ \ ::\ a\ ->\ Integer}\\
\mbox{\tt \ \ \ \ x\ `div`\ y\ \ \ \ \ \ \ \ \ \ \ =\ \ q\ \ where\ (q,r)\ =\ divRem\ x\ y}\\
\mbox{\tt \ \ \ \ x\ `rem`\ y\ \ \ \ \ \ \ \ \ \ \ =\ \ r\ \ where\ (q,r)\ =\ divRem\ x\ y}\\
\mbox{\tt \ \ \ \ x\ `mod`\ y\ \ \ \ \ \ \ \ \ \ \ =\ \ if\ signum\ x\ ==\ -\ (signum\ y)\ then\ r\ +\ y\ else\ r}\\
\mbox{\tt \ \ \ \ \ \ \ \ \ \ \ \ \ \ \ \ \ \ \ \ \ \ \ \ \ \ \ where\ r\ =\ x\ `rem`\ y}\\
\mbox{\tt \ \ \ \ even\ x\ \ \ \ \ \ \ \ \ \ \ \ \ \ =\ \ x\ `rem`\ 2\ ==\ 0}\\
\mbox{\tt \ \ \ \ odd\ \ \ \ \ \ \ \ \ \ \ \ \ \ \ \ \ =\ \ not\ .\ even}\\
\mbox{\tt }\\[-8pt]
\mbox{\tt class\ \ (Num\ a)\ =>\ Fractional\ a\ \ where}\\
\mbox{\tt \ \ \ \ (/)\ \ \ \ \ \ \ \ \ \ \ \ \ \ \ \ \ ::\ a\ ->\ a\ ->\ a}\\
\mbox{\tt \ \ \ \ fromRational\ \ \ \ \ \ \ \ ::\ Rational\ ->\ a}\\
\mbox{\tt }\\[-8pt]
\mbox{\tt class\ \ (Fractional\ a)\ =>\ Floating\ a\ \ where}\\
\mbox{\tt \ \ \ \ pi\ \ \ \ \ \ \ \ \ \ \ \ \ \ \ \ \ \ ::\ a}\\
\mbox{\tt \ \ \ \ exp,\ log,\ sqrt\ \ \ \ \ \ ::\ a\ ->\ a}\\
\mbox{\tt \ \ \ \ (**),\ logBase\ \ \ \ \ \ \ ::\ a\ ->\ a\ ->\ a}\\
\mbox{\tt \ \ \ \ sin,\ cos,\ tan\ \ \ \ \ \ \ ::\ a\ ->\ a}\\
\mbox{\tt \ \ \ \ asin,\ acos,\ atan\ \ \ \ ::\ a\ ->\ a}\\
\mbox{\tt \ \ \ \ sinh,\ cosh,\ tanh\ \ \ \ ::\ a\ ->\ a}\\
\mbox{\tt \ \ \ \ asinh,\ acosh,\ atanh\ ::\ a\ ->\ a}\\
\mbox{\tt \ \ \ \ x\ **\ y\ \ \ \ \ \ \ \ \ \ \ \ \ \ =\ \ exp\ (log\ x\ *\ y)}\\
\mbox{\tt \ \ \ \ logBase\ x\ y\ \ \ \ \ \ \ \ \ =\ \ log\ y\ /\ log\ x}\\
\mbox{\tt \ \ \ \ sqrt\ x\ \ \ \ \ \ \ \ \ \ \ \ \ \ =\ \ x\ **\ 0.5}\\
\mbox{\tt \ \ \ \ tan\ \ x\ \ \ \ \ \ \ \ \ \ \ \ \ \ =\ \ sin\ \ x\ /\ cos\ \ x}\\
\mbox{\tt \ \ \ \ tanh\ x\ \ \ \ \ \ \ \ \ \ \ \ \ \ =\ \ sinh\ x\ /\ cosh\ x}\\
\mbox{\tt }\\[-8pt]
\mbox{\tt class\ \ (Real\ a,\ Fractional\ a)\ =>\ RealFrac\ a\ \ where}\\
\mbox{\tt \ \ \ \ properFraction\ \ \ \ \ \ ::\ (Integral\ b)\ =>\ a\ ->\ (b,a)}\\
\mbox{\tt \ \ \ \ approxRational\ \ \ \ \ \ ::\ a\ ->\ a\ ->\ Rational}
}
\ecaption{Numeric classes and related operations}
\label{basic-numeric-1}
\indextt{Num}\indextt{+}\indextt{-}\indextt{*}
\indextt{negate}\indextt{abs}\indextt{signum}         
\indextt{fromInteger}
\indextt{Real}\indextt{toRational}
\indextt{Integral}\indextt{divRem}\indextt{mod}\indextt{div}\indextt{rem}
\indextt{even}\indextt{odd}
\indextt{Fractional}\indextt{/}\indextt{fromRational}       
\indextt{Floating}\indextt{pi}\indextt{exp}\indextt{log}\indextt{sqrt} 
\indextt{**}\indextt{logBase}
\indextt{sin}\indextt{cos}\indextt{tan}                        
\indextt{asin}\indextt{acos}\indextt{atan}               
\indextt{sinh}\indextt{cosh}\indextt{tanh}               
\indextt{asinh}\indextt{acosh}\indextt{atanh}      
\indextt{RealFrac}\indextt{properFraction}\indextt{approxRational}
\end{figure}

\begin{figure}
\outline{
\mbox{\tt class\ \ (RealFrac\ a,\ Floating\ a)\ =>\ RealFloat\ a\ \ where}\\
\mbox{\tt \ \ \ \ floatRadix\ \ \ \ \ \ \ \ \ \ ::\ a\ ->\ Integer}\\
\mbox{\tt \ \ \ \ floatDigits\ \ \ \ \ \ \ \ \ ::\ a\ ->\ Int}\\
\mbox{\tt \ \ \ \ floatRange\ \ \ \ \ \ \ \ \ \ ::\ a\ ->\ (Int,Int)}\\
\mbox{\tt \ \ \ \ decodeFloat\ \ \ \ \ \ \ \ \ ::\ a\ ->\ (Integer,Int)}\\
\mbox{\tt \ \ \ \ encodeFloat\ \ \ \ \ \ \ \ \ ::\ Integer\ ->\ Int\ ->\ a}\\
\mbox{\tt \ \ \ \ exponent\ \ \ \ \ \ \ \ \ \ \ \ ::\ a\ ->\ Int}\\
\mbox{\tt \ \ \ \ significand\ \ \ \ \ \ \ \ \ ::\ a\ ->\ a}\\
\mbox{\tt \ \ \ \ scaleFloat\ \ \ \ \ \ \ \ \ \ ::\ Int\ ->\ a\ ->\ a}\\
\mbox{\tt }\\[-8pt]
\mbox{\tt \ \ \ \ exponent\ x\ \ \ \ \ \ \ \ \ \ =\ \ if\ m\ ==\ 0\ then\ 0\ else\ n\ +\ floatDigits\ x}\\
\mbox{\tt \ \ \ \ \ \ \ \ \ \ \ \ \ \ \ \ \ \ \ \ \ \ \ \ \ \ \ where\ (m,n)\ =\ decodeFloat\ x}\\
\mbox{\tt \ \ \ \ significand\ x\ \ \ \ \ \ \ =\ \ encodeFloat\ m\ (-\ (floatDigits\ x))}\\
\mbox{\tt \ \ \ \ \ \ \ \ \ \ \ \ \ \ \ \ \ \ \ \ \ \ \ \ \ \ \ where\ (m,{\char'137})\ =\ decodeFloat\ x}\\
\mbox{\tt \ \ \ \ scaleFloat\ k\ x\ \ \ \ \ \ =\ \ encodeFloat\ m\ (n+k)}\\
\mbox{\tt \ \ \ \ \ \ \ \ \ \ \ \ \ \ \ \ \ \ \ \ \ \ \ \ \ \ \ where\ (m,n)\ =\ decodeFloat\ x}\\
\mbox{\tt }\\[-8pt]
\mbox{\tt instance\ \ Integral\ Int}\\
\mbox{\tt instance\ \ Integral\ Integer}\\
\mbox{\tt }\\[-8pt]
\mbox{\tt minInt,\ maxInt\ \ \ \ \ \ \ \ \ \ ::\ \ Int}\\
\mbox{\tt fromIntegral\ \ \ \ \ \ \ \ \ \ \ \ ::\ \ (Integral\ a,\ Num\ b)\ =>\ a\ ->\ b}\\
\mbox{\tt gcd,\ lcm\ \ \ \ \ \ \ \ \ \ \ \ \ \ \ \ ::\ \ (Integral\ a)\ =>\ a\ ->\ a->\ a}\\
\mbox{\tt ({\char'136})\ \ \ \ \ \ \ \ \ \ \ \ \ \ \ \ \ \ \ \ \ ::\ \ (Num\ a,\ Integral\ b)\ =>\ a\ ->\ b\ ->\ a}\\
\mbox{\tt ({\char'136}{\char'136})\ \ \ \ \ \ \ \ \ \ \ \ \ \ \ \ \ \ \ \ ::\ \ (Fractional\ a,\ Integral\ b)\ =>\ a\ ->\ b\ ->\ a}\\
\mbox{\tt }\\[-8pt]
\mbox{\tt data\ \ (Integral\ a)\ \ \ \ \ \ =>\ Ratio\ a}\\
\mbox{\tt type\ \ Rational\ \ \ \ \ \ \ \ \ \ =\ \ Ratio\ Integer}\\
\mbox{\tt instance\ \ (Integral\ a)\ \ =>\ RealFrac\ (Ratio\ a)}\\
\mbox{\tt }\\[-8pt]
\mbox{\tt ({\char'45})\ \ \ \ \ \ \ \ \ \ \ \ \ \ \ \ \ \ \ \ \ ::\ \ (Integral\ a)\ =>\ a\ ->\ a\ ->\ Ratio\ a}\\
\mbox{\tt numerator,\ denominator\ \ ::\ \ (Integral\ a)\ =>\ Ratio\ a\ ->\ a}\\
\mbox{\tt }\\[-8pt]
\mbox{\tt instance\ \ RealFloat\ Float}\\
\mbox{\tt instance\ \ RealFloat\ Double}\\
\mbox{\tt }\\[-8pt]
\mbox{\tt fromRealFrac\ \ \ \ \ \ \ \ \ \ \ \ ::\ (RealFrac\ a,\ Fractional\ b)\ =>\ a\ ->\ b}\\
\mbox{\tt }\\[-8pt]
\mbox{\tt truncate,\ round\ \ \ \ \ \ \ \ \ ::\ \ (RealFrac\ a,\ Integral\ b)\ =>\ a\ ->\ b}\\
\mbox{\tt ceiling,\ floor\ \ \ \ \ \ \ \ \ \ ::\ \ (RealFrac\ a,\ Integral\ b)\ =>\ a\ ->\ b}\\
\mbox{\tt atan2\ \ \ \ \ \ \ \ \ \ \ \ \ \ \ \ \ \ \ ::\ \ (RealFloat\ a)\ =>\ a\ ->\ a\ ->\ a}
}
\ecaption{Numeric classes and related operations (continued)}
\label{basic-numeric-2}
\indextt{RealFloat}
\indextt{floatRadix}\indextt{floatDigits}\indextt{floatRange} 
\indextt{decodeFloat}\indextt{encodeFloat}
\indextt{exponent}\indextt{significand}\indextt{scaleFloat}
\indextt{Int}\indextt{Integer}\indextt{minInt}\indextt{maxInt}                
\indextt{fromIntegral}
\indextt{gcd}\indextt{lcm} 
\index{^@{\ptt {\char'136}}} % this is ^
\index{^^@{\ptt {\char'136}{\char'136}}} % this is ^^
\indextycon{Ratio}\indexsynonym{Rational}           
\indextt{RealFrac}
\index{%@{\ptt {\char'045}}}\indextt{numerator}\indextt{denominator}         
\indextycon{Float}\indextycon{Double}
\indextt{fromRealFrac}
\indextt{truncate}\indextt{round}
\indextt{ceiling}\indextt{floor}    
\indextt{atan2}
\end{figure}
\begin{figure}
\outline{
\mbox{\tt data\ \ (RealFloat\ a)\ \ \ \ =>\ Complex\ a\ =\ a\ :+\ a\ \ deriving\ (Eq,\ Binary,\ Text)}\\
\mbox{\tt instance\ (RealFloat\ a)\ =>\ Floating\ (Complex\ a)}\\
\mbox{\tt }\\[-8pt]
\mbox{\tt realPart,\ imagPart\ \ \ \ \ ::\ \ (RealFloat\ a)\ =>\ Complex\ a\ ->\ a}\\
\mbox{\tt conjugate\ \ \ \ \ \ \ \ \ \ \ \ \ \ ::\ \ (RealFloat\ a)\ =>\ Complex\ a\ ->\ Complex\ a}\\
\mbox{\tt mkPolar\ \ \ \ \ \ \ \ \ \ \ \ \ \ \ \ ::\ \ (RealFloat\ a)\ =>\ a\ ->\ a\ ->\ Complex\ a}\\
\mbox{\tt cis\ \ \ \ \ \ \ \ \ \ \ \ \ \ \ \ \ \ \ \ ::\ \ (RealFloat\ a)\ =>\ a\ ->\ Complex\ a}\\
\mbox{\tt polar\ \ \ \ \ \ \ \ \ \ \ \ \ \ \ \ \ \ ::\ \ (RealFloat\ a)\ =>\ Complex\ a\ ->\ (a,a)}\\
\mbox{\tt magnitude,\ phase\ \ \ \ \ \ \ ::\ \ (RealFloat\ a)\ =>\ Complex\ a\ ->\ a}
}
\ecaption{Numeric classes and related operations (continued)}
\label{basic-numeric-3}
\indextycon{Complex}
\indextt{:+}\indextt{Floating}
\indextt{realPart}\indextt{imagPart}
\indextt{conjugate}\indextt{mkPolar}\indextt{cis}
\indextt{polar}\indextt{magnitude}\indextt{phase}
\end{figure}

\subsubsection{Numeric Literals}
\label{numeric-literals}

The syntax of numeric literals is given in
Section~\ref{lexemes-numeric}.  An integer literal represents the
application
of the function \mbox{\tt fromInteger}\indextt{fromInteger} to the appropriate value of type
\mbox{\tt Integer}.  Similarly, a floating literal stands for an application of
\mbox{\tt fromRational}\indextt{fromRational} to a value of type \mbox{\tt Rational} (that is, 
\mbox{\tt Ratio\ Integer}).  Given the typings:
\bprog
\mbox{\tt fromInteger\ \ ::\ (Num\ a)\ =>\ Integer\ ->\ a}\\
\mbox{\tt fromRational\ ::\ (Fractional\ a)\ =>\ Rational\ ->\ a}
\eprog\indextt{fromInteger}\indextt{fromRational}%
integer and floating literals have the
typings \mbox{\tt (Num\ a)\ =>\ a} and \mbox{\tt (Fractional\ a)\ =>\ a}, respectively.
Numeric literals are defined in this indirect way so that they may be
interpreted as values of any appropriate numeric type.
For example, \mbox{\tt fromInteger} for complex
numbers is defined as follows:
\bprog
\mbox{\tt fromInteger\ n\ =\ fromInteger\ n\ :+\ 0}
\eprog
See Section~\ref{default-decls} for a discussion of overloading ambiguity.

\subsubsection{Constructed Numbers}
\label{constructed-numbers}

There are two kinds of numeric types formed by data constructors:
namely, \mbox{\tt Ratio} and \mbox{\tt Complex}.  For each \mbox{\tt Integral} type $t$, there is
a type \mbox{\tt Ratio}\indextycon{Ratio} $t$ of rational pairs with components of
type $t$. (The type name \mbox{\tt Rational}\indexsynonym{Rational} is a synonym for
\mbox{\tt Ratio\ Integer}.)  Similarly, for each real floating-point type $t$,
\mbox{\tt Complex}\indextycon{Complex} $t$ is a type of complex numbers with real
and imaginary components of type $t$.

The operator \mbox{\tt ({\char'45})}\index{%@{\ptt {\char'045}}} forms the ratio of two
integral numbers.  The functions \mbox{\tt numerator}\indextt{numerator}
and \mbox{\tt denominator}\indextt{denominator} extract the components of a
ratio; these are in reduced form with a positive denominator.

Complex numbers are an algebraic type:
\index{algebraic datatype}
\bprog
\mbox{\tt data\ \ (RealFloat\ a)\ =>\ Floating\ (Complex\ a)\ \ =\ \ a\ :+\ a}
\eprog\indextycon{Floating}%
The constructor \mbox{\tt (:+)}\indextt{:+} forms a complex number from its
real and imaginary rectangular components.  A complex number may also
be formed from polar components of magnitude and phase by the function
\mbox{\tt mkPolar}\indextt{mkPolar}.  The function \mbox{\tt cis}\indextt{polar}
produces a complex number from an angle \mbox{$\it t$}:
\bprog
\mbox{\tt cis\ t\ =\ cos\ t\ :+\ sin\ t}
\eprog
Put another way, \mbox{\tt cis} \mbox{$\it t$} is a complex value with magnitude \mbox{$\it 1$}
and phase \mbox{$\it t$} (modulo \mbox{$\it 2\pi$}).

The function \mbox{\tt polar}\indextt{polar} takes a complex number and
returns a (magnitude, phase) pair in canonical form: The magnitude is
nonnegative, and the phase, in the range $(- \pi , \pi ]$; if the
magnitude is zero, then so is the phase.  Several
component-extraction functions are provided:
\bprog
\mbox{\tt realPart\ (x:+y)\ =\ \ x}\\
\mbox{\tt imagPart\ (x:+y)\ =\ \ y}\\
\mbox{\tt magnitude\ z\ \ \ \ \ =\ \ r\ \ where\ (r,t)\ =\ polar\ z}\\
\mbox{\tt phase\ z\ \ \ \ \ \ \ \ \ =\ \ t\ \ where\ (r,t)\ =\ polar\ z}
\eprog\indextt{realPart}\indextt{imagPart}%
\indextt{magnitude}\indextt{phase}%
Also defined on complex numbers is the conjugate function:
\bprog
\mbox{\tt conjugate\ (x:+y)\ =\ \ x:+(-y)}
\eprogNoSkip\indextt{conjugate}

\subsubsection{Arithmetic and Number-Theoretic Operations}
\label{arithmetic-operators}
\index{arithmetic}

The infix operations \mbox{\tt (+)}\indextt{+}, \mbox{\tt (*)}\indextt{*},
\mbox{\tt (-)}\indextt{-} and the unary function
\mbox{\tt negate}\indextt{negate} (which can also be written as a prefix minus sign; see
section~\ref{operators}) apply to all numbers.  The operations
\mbox{\tt div}\indextt{div}, \mbox{\tt rem}\indextt{rem}, and
\mbox{\tt mod}\indextt{mod} apply only to integral numbers, while the operation
\mbox{\tt (/)}\indextt{/} applies only to fractional ones.  The \mbox{\tt div} and
\mbox{\tt rem} operations satisfy the law:
\[\ba{c}
\mbox{\tt (x\ }\bkqB\mbox{\tt div}\bkqA\mbox{\tt \ y)*y\ +\ (x\ }\bkqB\mbox{\tt rem}\bkqA\mbox{\tt \ y)\ ==\ x}
\ea\]
The result of \mbox{\tt x\ }\bkqB\mbox{\tt div}\bkqA\mbox{\tt \ y} has the same sign as \mbox{\tt x\ *\ y}
and is truncated toward zero.  The modulo function differs from the
remainder function when the signs of the dividend and divisor differ, the
remainder always having the sign of the dividend, and the modulo
having the sign of the divisor.  For example,
\bprog
\mbox{\tt -13\ }\bkqB\mbox{\tt rem}\bkqA\mbox{\tt \ 4\ ==\ -1}\\
\mbox{\tt -13\ }\bkqB\mbox{\tt mod}\bkqA\mbox{\tt \ 4\ ==\ 3}\\
\mbox{\tt }\\[-8pt]
\mbox{\tt 13\ }\bkqB\mbox{\tt rem}\bkqA\mbox{\tt \ -4\ ==\ 1}\\
\mbox{\tt 13\ }\bkqA\mbox{\tt mod}\bkqA\mbox{\tt \ -4\ ==\ -3}
\eprog
The \mbox{\tt divRem} operation takes a dividend and a divisor as arguments
and returns a (quotient, remainder) pair:
\bprog
\mbox{\tt divRem\ x\ y\ \ =\ \ (x\ }\bkqB\mbox{\tt div}\bkqA\mbox{\tt \ y,\ x\ }\bkqB\mbox{\tt rem}\bkqA\mbox{\tt \ y)}
\eprog
Also available on integers are the even and odd predicates:
\bprog
\mbox{\tt even\ x\ \ \ \ \ \ =\ \ x\ }\bkqB\mbox{\tt rem}\bkqA\mbox{\tt \ 2\ ==\ 0}\\
\mbox{\tt odd\ \ \ \ \ \ \ \ \ =\ \ not\ .\ even}
\eprog\indextt{even}\indextt{odd}
Finally, there are the greatest common divisor and least common
multiple functions: \mbox{\tt gcd}\indextt{gcd} $x$ $y$ is the greatest
integer that divides both $x$ and $y$.  \mbox{\tt lcm}\indextt{lcm} $x$ $y$
is the smallest positive integer that both $x$ and $y$ divide.

\subsubsection{Exponentiation and Logarithms}
\index{exponentiation}\index{logarithm}

The one-argument exponential function \mbox{\tt exp}\indextt{exp} and the
logarithm function \mbox{\tt log}\indextt{log} act on floating-point numbers and
use base $e$.  \mbox{\tt logBase}\indextt{logBase} $a$ $x$ returns the
logarithm of $x$ in base $a$.  \mbox{\tt sqrt}\indextt{sqrt} returns the
principal square root of a floating-point number.
There are three two-argument exponentiation operations:
\mbox{\tt ({\char'136})}\index{^@{\ptt {\char'136}}} raises any
number to a nonnegative integer power,
\mbox{\tt ({\char'136}{\char'136})}\index{^^@{\ptt {\char'136}{\char'136}}} raises a
fractional number to any integer power, and \mbox{\tt (**)}\indextt{**}
takes two floating-point arguments.  The value of $x$\mbox{\tt {\char'136}0} or $x$\mbox{\tt {\char'136}{\char'136}0}
is \mbox{\tt 1} for any $x$, including zero; \mbox{\tt 0**}$y$ is undefined.
  
\subsubsection{Magnitude and Sign}
\label{magnitude-sign}
\index{magnitude}\index{sign}

A number has a {\em magnitude}
and a {\em sign}.  The functions \mbox{\tt abs}\indextt{abs} and
\mbox{\tt signum}\indextt{signum} apply to any number and satisfy the law:
\bprog
\mbox{\tt abs\ x\ *\ signum\ x\ ==\ x}
\eprog
For real numbers, these functions are defined by:
\bprog
\mbox{\tt abs\ x\ \ \ \ |\ x\ >=\ 0\ \ =\ x}\\
\mbox{\tt \ \ \ \ \ \ \ \ \ |\ x\ <\ \ 0\ \ =\ -x}\\
\mbox{\tt }\\[-8pt]
\mbox{\tt signum\ x\ |\ x\ >\ \ 0\ \ =\ 1}\\
\mbox{\tt \ \ \ \ \ \ \ \ \ |\ x\ ==\ 0\ \ =\ 0}\\
\mbox{\tt \ \ \ \ \ \ \ \ \ |\ x\ <\ \ 0\ \ =\ -1}
\eprog
For complex numbers, the definitions are different:\nopagebreak
\bprog
\mbox{\tt abs\ z\ \ \ \ \ \ \ \ \ \ \ \ \ =\ \ magnitude\ z\ :+\ 0}\\
\mbox{\tt signum\ 0\ \ \ \ \ \ \ \ \ \ =\ \ 0}\\
\mbox{\tt signum\ z@(x:+y)\ \ \ =\ \ x/r\ :+\ y/r\ \ where\ r\ =\ magnitude\ z}
\eprog
That is, \mbox{\tt abs} $z$ is a number with the magnitude of $z$, but oriented
in the positive real direction, whereas \mbox{\tt signum} $z$ has the phase of
$z$, but unit magnitude.  (\mbox{\tt abs} for a complex number differs from
\mbox{\tt magnitude} only in type.  See Section~\ref{constructed-numbers}.)

\subsubsection{Trigonometric Functions}
\index{trigonometric function}

The circular and hyperbolic sine\index{sine}, cosine\index{cosine},
and tangent\index{tangent} functions and their inverses are provided
for floating-point numbers.  A version of arctangent\index{arctangent}
taking two real floating-point arguments is also provided: For real floating
$x$ and $y$, \mbox{\tt atan2}\indextt{atan2} $y$ $x$ differs from 
\mbox{\tt atan\ (}$y\/$\mbox{\tt /}$x\/$\mbox{\tt )}\indextt{atan} in that its range is 
$( -\pi , \pi ]$ rather than $(- \pi / 2 , \pi / 2 )$ (because the signs
of the arguments provide quadrant information), and that it is defined
when $x$ is zero.

The precise definition of the above functions is as in Common Lisp
\cite{steele:common-lisp}, which in turn follows Penfield's proposal
for APL
\cite{penfield:complex-apl}.
See these references for discussions of branch cuts, discontinuities,
and implementation.

\subsubsection{Coercions and Component Extraction}
\label{coercion}
\index{coercion}

The \mbox{\tt ceiling}\indextt{ceiling}, \mbox{\tt floor}\indextt{floor},
\mbox{\tt truncate}\indextt{truncate}, and \mbox{\tt round}\indextt{round}
functions each take a real fractional argument and return an integral
result.  \mbox{\mbox{\tt ceiling} $x$} returns the least integer not less than $x$, and
\mbox{\mbox{\tt floor} $x$}, the greatest integer not greater than $x$.  \mbox{\mbox{\tt truncate} $x$}
yields the integer nearest $x$ between $0$ and $x$, inclusive.
\mbox{\mbox{\tt round} $x$} returns the nearest integer to $x$, the even integer if
$x$ is equidistant between two integers.

The function \mbox{\tt properFraction}\indextt{properFraction} takes a real
fractional number $x$ and returns a pair comprising $x$ as a
proper fraction: an integral number with the same sign as $x$ and a
fraction with the same type and sign as $x$ and with absolute
value less than 1.  The \mbox{\tt ceiling}, \mbox{\tt floor}, \mbox{\tt truncate}, and \mbox{\tt round}
functions can be defined in terms of this one.

Two functions convert numbers to type \mbox{\tt Rational}:
\mbox{\tt toRational}\indextt{toRational} returns the rational equivalent of
its real argument with full precision;
\mbox{\tt approxRational}\indextt{approxRational} takes two real fractional arguments and returns an
approximation to the first within the tolerance given by the second.
Subject to the tolerance constraint, the result has the smallest
denominator possible.

The operations of class \mbox{\tt RealFloat}\indextt{RealFloat} allow efficient, machine-independent
access to the components of a floating-point number.
The functions \mbox{\tt floatRadix}\indextt{floatRadix},
\mbox{\tt floatDigits}\indextt{floatDigits}, and
\mbox{\tt floatRange}\indextt{floatRange} give the parameters of a
floating-point type:  the radix of the representation, the number of
digits of this radix in the significand, and the lowest and highest
values the exponent may assume, respectively.
The function \mbox{\tt decodeFloat}\indextt{decodeFloat}
applied to a real floating-point number returns the significand
expressed as an \mbox{\tt Integer} and an appropriately scaled exponent (an
\mbox{\tt Int}).  If \mbox{\mbox{\tt decodeFloat\ x}} yields \mbox{\mbox{\tt (}\mbox{$\it m$}\mbox{\tt ,}\mbox{$\it n$}\mbox{\tt )}}, then \mbox{\tt x} is
equal in value to \mbox{$\it mb^n$}, where \mbox{$\it b$} is the floating-point radix, and
furthermore, either \mbox{$\it m$} and \mbox{$\it n$} are both zero or else
\mbox{$\it b^{d-1}\leq m<b^d$}, where \mbox{$\it d$} is the value of \mbox{\mbox{\tt floatDigits\ x}}.
\mbox{\tt encodeFloat}\indextt{encodeFloat} performs the inverse of this
transformation.  The functions \mbox{\tt significand}\indextt{significand}
and \mbox{\tt exponent}\indextt{exponent} together provide the same
information as \mbox{\tt decodeFloat},  but rather than an \mbox{\tt Integer},
\mbox{\mbox{\tt significand\ x}} yields a value of the same type as \mbox{\tt x}, scaled to lie
in the open interval \mbox{$\it (-1,1)$}.  \mbox{\mbox{\tt exponent\ 0}} is zero.  \mbox{\tt scaleFloat}
multiplies a floating-point number by an integer power of the radix.

Also available are the following coercion functions:
\bprog
\mbox{\tt fromIntegral\ ::\ (Integral\ a,\ Num\ b)\ =>\ a\ ->\ b}\\
\mbox{\tt fromRealFrac\ ::\ (RealFrac\ a,\ Fractional\ b)\ =>\ a\ ->\ b}
\eprogNoSkip\indextt{fromIntegral}\indextt{fromRealFrac}

\subsection{Arrays}
\label{arrays}
\index{array}

\Haskell{} provides indexable {\em arrays}, which may be thought of as
functions whose domains are isomorphic to contiguous subsets of the
integers.
Functions restricted in this way can be
implemented efficiently; in particular, a programmer may
reasonably expect rapid access to the components.  To ensure
the possibility of such an implementation, arrays are treated as data, not as
general functions.

Types that are instances of class \mbox{\tt Ix}\indextt{Ix} (see
Section~\ref{instance-decls}) may be indices of arrays; a one-dimensional 
array might have index type \mbox{\tt Int}, a two-dimensional array
\mbox{\tt (Int,Char)} etc.

\subsubsection{Array Construction}

If \mbox{\tt a} is an index type and \mbox{\tt b} is any type, the type of arrays with
indices in \mbox{\tt a} and elements in \mbox{\tt b} is written \mbox{\tt Array\ a\ b}.\indextycon{Array}
An array may be created by the function \mbox{\tt array}:
\bprog
\mbox{\tt array\ ::\ (Ix\ a)\ =>\ (a,a)\ \ ->\ [Assoc\ a\ b]\ ->\ Array\ a\ b}\\
\mbox{\tt data\ \ Assoc\ a\ b\ \ =\ \ a\ :=\ b}
\eprog\indextt{array}\indextycon{Assoc}\indextt{:=}
The first argument of \mbox{\tt array} is a pair of {\em bounds}, each of the
index type of the array.  These bounds are the lowest and
highest indices in the array, in that order.  For example, a
one-origin vector of length \mbox{\tt 10} has bounds \mbox{\tt (1,10)}, and a one-origin \mbox{\tt 10}
by \mbox{\tt 10} matrix has bounds \mbox{\tt ((1,1),(10,10))}.

The second argument of \mbox{\tt array} is a list of {\em associations}
of the form $index$ \mbox{\tt :=} $value$.  Typically, this list will
be expressed as a comprehension.  An association \mbox{\tt i\ :=\ x} defines the
value of the array at index \mbox{\tt i} to be \mbox{\tt x}.  The array is undefined if
any index in the list is out of bounds.  If any two associations in the
list have the same index, the value at that index is undefined.
Because the indices must be checked for these errors, \mbox{\tt array} is
strict in the bounds argument and in the indices of the association list,
but nonstrict in the values.  Thus, recurrences such as the following are
possible:
\bprog
\mbox{\tt a\ =\ array\ (1,100)\ ((1\ :=\ 1)\ :\ [i\ :=\ i\ *\ a!(i-1)\ |\ i\ <-\ [2..100]])}
\eprog
Not every index within the bounds of the array need
appear in the association list, but the values associated with indices
that do not appear will be undefined.
Figure~\ref{array-examples} shows some examples that use the
\mbox{\tt array} constructor.

\begin{figure}
\outline{
\mbox{\tt --\ Scaling\ an\ array\ of\ numbers\ by\ a\ given\ number:}\\
\mbox{\tt scale\ ::\ (Num\ a,\ Ix\ b)\ =>\ a\ ->\ Array\ b\ a\ ->\ Array\ b\ a}\\
\mbox{\tt scale\ x\ a\ =\ array\ b\ [i\ :=\ a!i\ *\ x\ |\ i\ <-\ range\ b]}\\
\mbox{\tt \ \ \ \ \ \ \ \ \ \ \ \ where\ b\ =\ bounds\ a}\\
\mbox{\tt }\\[-8pt]
\mbox{\tt --\ Inverting\ an\ array\ that\ holds\ a\ permutation\ of\ its\ indices}\\
\mbox{\tt invPerm\ ::\ (Ix\ a)\ =>\ Array\ a\ a\ ->\ Array\ a\ a}\\
\mbox{\tt invPerm\ a\ =\ array\ b\ [a!i\ :=\ i\ |\ i\ <-\ range\ b]}\\
\mbox{\tt \ \ \ \ \ \ \ \ \ \ \ \ where\ b\ =\ bounds\ a}\\
\mbox{\tt }\\[-8pt]
\mbox{\tt --\ The\ inner\ product\ of\ two\ vectors}\\
\mbox{\tt inner\ ::\ (Ix\ a,\ Num\ b)\ =>\ Array\ a\ b\ ->\ Array\ a\ b\ ->\ b}\\
\mbox{\tt inner\ v\ w\ =\ if\ b\ ==\ bounds\ w}\\
\mbox{\tt \ \ \ \ \ \ \ \ \ \ \ \ \ \ \ \ then\ sum\ [v!i\ *\ w!i\ |\ i\ <-\ range\ b]}\\
\mbox{\tt \ \ \ \ \ \ \ \ \ \ \ \ \ \ \ \ else\ error\ "inconformable\ arrays\ for\ inner\ product"}\\
\mbox{\tt \ \ \ \ \ \ \ \ \ \ \ \ where\ b\ =\ bounds\ v}
}
\ecaption{Array examples}
\label{array-examples}
\end{figure}

\mbox{\tt (!)}\index{!@{\ptt {\char'041}}} denotes
array subscripting; the \mbox{\tt bounds}\indextt{bounds} function
applied to an array returns its bounds:
\bprog
\mbox{\tt (!)\ \ \ \ ::\ (Ix\ a)\ =>\ Array\ a\ b\ ->\ a\ ->\ b}\\
\mbox{\tt bounds\ ::\ (Ix\ a)\ =>\ Array\ a\ b\ ->\ (a,a)}
\eprog
The functions \mbox{\tt indices}\indextt{indices}, \mbox{\tt elems}\indextt{elems}, and
\mbox{\tt assocs},\indextt{assocs} when applied to an array, return lists of
the indices, elements, or associations, respectively, in index order:
\bprog
\mbox{\tt indices::\ (Ix\ a)\ =>\ Array\ a\ b\ ->\ [a]}\\
\mbox{\tt indices\ =\ range\ .\ bounds}\\
\mbox{\tt }\\[-8pt]
\mbox{\tt elems::\ (Ix\ a)\ =>\ Array\ a\ b\ ->\ [b]}\\
\mbox{\tt elems\ a\ =\ [a!i\ |\ i\ <-\ indices\ a]}\\
\mbox{\tt }\\[-8pt]
\mbox{\tt assocs:\ (Ix\ a)\ =>\ Array\ a\ b\ ->\ [Assoc\ a\ b]}\\
\mbox{\tt assocs\ a\ =\ [\ i\ :=\ a!i\ |\ i\ <-\ indices\ a]}
\eprog
An array may be constructed from a pair of bounds and a list
of values in index order using the function \mbox{\tt listArray}:\nopagebreak
\bprog
\mbox{\tt listArray::\ (Ix\ a)\ =>\ (a,a)\ ->\ [b]\ ->\ Array\ a\ b}\\
\mbox{\tt listArray\ bnds\ xs\ =\ array\ bnds\ (zipWith\ (:=)\ (range\ bnds)\ xs)}
\eprogNoSkip

\subsubsection{Accumulated Arrays}
\index{array!accumulated}

Another array creation function, \mbox{\tt accumArray},\indextt{accumArray}
relaxes the restriction that a given index may appear at most once in
the association list, using an {\em accumulating function} which
combines the values of associations with the same index
\cite{nikhil:id-nouveau,wadler:array-primitive}:
\bprog
\mbox{\tt accumArray::(Ix\ a)\ =>\ (b->c->b)\ ->\ b\ ->\ (a,a)\ ->\ [Assoc\ a\ c]\ ->\ Array\ a\ b}
\eprog\indextt{accumArray}%
The first argument of \mbox{\tt accumArray} is the accumulating function; the
second is an initial value; the remaining two arguments are a bounds
pair and an association list, as for the \mbox{\tt array} function.
For example, given a list of values of some index type, \mbox{\tt hist}
produces a histogram of the number of occurrences of each index within
a specified range:
\bprog
\mbox{\tt hist\ ::\ (Ix\ a,\ Num\ b)\ =>\ (a,a)\ ->\ [a]\ ->\ Array\ a\ b}\\
\mbox{\tt hist\ bnds\ is\ =\ accumArray\ (+)\ 0\ bnds\ [i\ :=\ 1\ |\ i<-is,\ inRange\ bnds\ i]}
\eprog
If the accumulating function is strict, then \mbox{\tt accumArray} is
strict in the values, as well as the indices, in the
association list.  Thus, unlike ordinary arrays,
accumulated arrays should not in general be recursive.

\subsubsection{Incremental Array Updates}

% no indent, please!
\bprogB
\mbox{\tt (//)\ \ ::\ (Ix\ a)\ =>\ Array\ a\ b\ ->\ [Assoc\ a\ b]\ ->\ Array\ a\ b}\\
\mbox{\tt accum\ ::\ (Ix\ a)\ =>\ (b\ ->\ c\ ->\ b)\ ->\ Array\ a\ b\ ->\ [Assoc\ a\ c]\ ->\ Array\ a\ b}
\eprogNoSkip\indextt{//}\indextt{accum}

The operator \mbox{\tt (//)} takes an array and a list of \mbox{\tt Assoc} pairs and returns
an array identical to the left argument except that it has
been updated by the associations in the right argument.  (As with
the \mbox{\tt array} function, the indices in the association list must
be unique for the updated elements to be defined.)  For example,
if \mbox{\tt m} is a 1-origin, \mbox{\tt n} by \mbox{\tt n} matrix, then
\mbox{\tt m//[(i,i)\ :=\ 0\ |\ i\ <-\ [1..n]]} is the same matrix, except with
the diagonal zeroed.

\mbox{\tt accum} \mbox{$\it f$} takes an array
and an association list and accumulates pairs from the list into
the array with the accumulating function \mbox{$\it f$}.  Thus \mbox{\tt accumArray}
can be defined using \mbox{\tt accum}:\nopagebreak[4]
\bprog
\mbox{\tt accumArray\ f\ z\ b\ =\ accum\ f\ (array\ b\ [i\ :=\ z\ |\ i\ <-\ range\ b])}
\eprogNoSkip

\subsubsection{Derived Arrays}
\index{array!derived}

The two functions \mbox{\tt amap}\indextt{amap} and \mbox{\tt ixmap}\indextt{ixmap}
derive new arrays from existing ones; they may be
thought of as providing function composition on the left and right,
respectively, with the mapping that the original array embodies:
\bprog
\mbox{\tt amap\ ::\ (Ix\ a)\ =>\ (b\ ->\ c)\ ->\ Array\ a\ b\ ->\ Array\ a\ c}\\
\mbox{\tt amap\ f\ a\ =\ array\ b\ [i\ :=\ f\ (a!i)\ |\ i\ <-\ range\ b]}\\
\mbox{\tt \ \ \ \ \ \ \ \ \ \ \ where\ b\ =\ bounds\ a}\\
\mbox{\tt }\\[-8pt]
\mbox{\tt ixmap\ ::\ (Ix\ a,\ Ix\ a')\ =>\ (a',a')\ ->\ (a'->a)\ ->\ Array\ a\ b\ ->\ Array\ a'\ b}\\
\mbox{\tt ixmap\ bnds\ f\ a\ =\ array\ bnds\ [i\ :=\ a\ !\ f\ i\ |\ i\ <-\ range\ bnds]}
\eprog\indextt{amap}\indextt{ixmap}%
\mbox{\tt amap} is the array analogue of the \mbox{\tt map} function on lists, while
\mbox{\tt ixmap} allows for transformations on array indices.
Figure~\ref{derived-array-examples} shows some examples.

\begin{figure}
\outline{
\mbox{\tt --\ A\ rectangular\ subarray}\\
\mbox{\tt subArray\ ::\ (Ix\ a)\ =>\ (a,a)\ ->\ Array\ a\ b\ ->\ Array\ a\ b}\\
\mbox{\tt subArray\ bnds\ =\ ixmap\ bnds\ ({\char'134}i->i)}\\
\mbox{\tt }\\[-8pt]
\mbox{\tt --\ A\ row\ of\ a\ matrix}\\
\mbox{\tt row\ ::\ (Ix\ a,\ Ix\ b)\ =>\ a\ ->\ Array\ (a,b)\ c\ ->\ Array\ b\ c}\\
\mbox{\tt row\ i\ x\ =\ ixmap\ (l',u')\ ({\char'134}j->(i,j))\ x\ where\ ((l,l'),(u,u'))\ =\ bounds\ x}\\
\mbox{\tt }\\[-8pt]
\mbox{\tt --\ Diagonal\ of\ a\ square\ matrix}\\
\mbox{\tt diag\ ::\ (Ix\ a)\ =>\ Array\ (a,a)\ b\ ->\ Array\ a\ b}\\
\mbox{\tt diag\ x\ =\ ixmap\ (l,u)\ ({\char'134}i->(i,i))\ x}\\
\mbox{\tt \ \ \ \ \ \ \ \ \ where\ ((l,l'),(u,u'))\ |\ l\ ==\ l'\ {\char'46}{\char'46}\ u\ ==\ u'\ \ =\ bounds\ x}\\
\mbox{\tt }\\[-8pt]
\mbox{\tt --\ Projection\ of\ first\ components\ of\ an\ array\ of\ pairs}\\
\mbox{\tt firstArray\ ::\ (Ix\ a)\ =>\ Array\ a\ (b,c)\ ->\ Array\ a\ b}\\
\mbox{\tt firstArray\ =\ amap\ ({\char'134}(x,y)->x)}
}
\ecaption{Derived array examples}
\label{derived-array-examples}
\end{figure}

\subsection{Errors}
\label{basic-errors}\index{error}

All errors in \Haskell{} are semantically equivalent to $\bot$.
\mbox{\tt error::\ String\ ->\ a}\indextt{error} takes a string
argument and returns $\bot$.  An application of
\mbox{\tt error} terminates evaluation of the
program and displays the string as appropriate.

% Local Variables: 
% mode: latex
% End:

\startnewsection
%
% $Header$
%
\section{Input/Output}
\label{io}
\index{input/output}

\Haskell{}'s I/O system is based on the view that a program
communicates to the outside world via {\em streams of messages}:
a program issues a stream of {\em requests} to the
operating system and in return receives a stream of {\em responses}.
Since a stream in \Haskell{} is only a lazy list,
a \Haskell{} program has the type:
\bprog
\mbox{\tt type\ \ Dialogue\ =\ [Response]\ ->\ [Request]}
\eprog\indexsynonym{Dialogue}%
The datatypes \mbox{\tt Response}\indextycon{Response} and
\mbox{\tt Request}\indextycon{Request} are defined below.  Intuitively,
\mbox{\tt [Response]} is an ordered list of {\em responses} and \mbox{\tt [Request]} is
an ordered list of {\em requests}; the $n\/$th response is the
operating system's reply to the $n\/$th request.

With this view of I/O, there is no need for any special-purpose
syntax or constructs for I/O; the I/O system is defined
entirely in terms of how the operating system responds to a program
with the above type---i.e.~what response it issues for each request.
An abstract specification of this behaviour is defined by
giving a definition of the operating system as a function that takes
as input an initial state and a collection of \Haskell{} programs,
each with the above type.  This specification appears in
Appendix~\ref{io-semantics}, using standard \Haskell{} syntax
augmented with a single non-deterministic merge operator.

% Due to the equivalence of stream-based and continuation-based I/O (see
% the technical report by Hudak and Sundaresh \cite{hudak:io}),
One can
define a continuation-based
version of I/O in terms of a stream-based version.  Such a definition
is provided in Section~\ref{continuation-io}.
The specific I/O requests available in each style are identical;
what differs is the way they are expressed.  This means that programs
in either style may be combined with a well-defined semantics.  In
both cases arbitrary I/O requests within conventional operating
systems may be induced while retaining referential transparency within
a \Haskell{} program.

The required requests for a valid implementation are:
\bprog
\mbox{\tt data\ \ Request\ =\ }\\
\mbox{\tt \ \ \ \ \ --\ file\ system\ requests:}\\
\mbox{\tt \ \ \ \ \ \ \ \ \ \ \ \ \ \ \ ReadFile\ \ \ \ \ \ String\ \ \ \ \ \ \ \ \ }\\
\mbox{\tt \ \ \ \ \ \ \ \ \ \ \ \ \ |\ WriteFile\ \ \ \ \ String\ String}\\
\mbox{\tt \ \ \ \ \ \ \ \ \ \ \ \ \ |\ AppendFile\ \ \ \ String\ String}\\
\mbox{\tt \ \ \ \ \ \ \ \ \ \ \ \ \ |\ ReadBinFile\ \ \ String\ }\\
\mbox{\tt \ \ \ \ \ \ \ \ \ \ \ \ \ |\ WriteBinFile\ \ String\ Bin}\\
\mbox{\tt \ \ \ \ \ \ \ \ \ \ \ \ \ |\ AppendBinFile\ String\ Bin}\\
\mbox{\tt \ \ \ \ \ \ \ \ \ \ \ \ \ |\ DeleteFile\ \ \ \ String}\\
\mbox{\tt \ \ \ \ \ \ \ \ \ \ \ \ \ |\ StatusFile\ \ \ \ String}\\
\mbox{\tt \ \ \ \ \ --\ channel\ system\ requests:}\\
\mbox{\tt \ \ \ \ \ \ \ \ \ \ \ \ \ |\ ReadChan\ \ \ \ \ \ String\ }\\
\mbox{\tt \ \ \ \ \ \ \ \ \ \ \ \ \ |\ AppendChan\ \ \ \ String\ String}\\
\mbox{\tt \ \ \ \ \ \ \ \ \ \ \ \ \ |\ ReadBinChan\ \ \ String\ }\\
\mbox{\tt \ \ \ \ \ \ \ \ \ \ \ \ \ |\ AppendBinChan\ String\ Bin}\\
\mbox{\tt \ \ \ \ \ \ \ \ \ \ \ \ \ |\ StatusChan\ \ \ \ String}
\eprogNoSkip
% utter page-breaking hack
\bprog
\mbox{\tt \ \ \ \ \ --\ environment\ requests:}\\
\mbox{\tt \ \ \ \ \ \ \ \ \ \ \ \ \ |\ Echo\ \ \ \ \ \ \ \ \ \ Bool}\\
\mbox{\tt \ \ \ \ \ \ \ \ \ \ \ \ \ |\ GetArgs}\\
\mbox{\tt \ \ \ \ \ \ \ \ \ \ \ \ \ |\ GetEnv\ \ \ \ \ \ \ \ String}\\
\mbox{\tt \ \ \ \ \ \ \ \ \ \ \ \ \ |\ SetEnv\ \ \ \ \ \ \ \ String\ String}\\
\mbox{\tt }\\[-8pt]
\mbox{\tt stdin\ \ \ \ \ \ \ =\ "stdin"}\\
\mbox{\tt stdout\ \ \ \ \ \ =\ "stdout"}\\
\mbox{\tt stderr\ \ \ \ \ \ =\ "stderr"}\\
\mbox{\tt stdecho\ \ \ \ \ =\ "stdecho"}
\eprog%
\indextycon{Request}\indextt{ReadFile}\indextt{WriteFile}%
\indextt{AppendFile}\indextt{ReadBinFile}\indextt{WriteBinFile}%
\indextt{AppendBinFile}\indextt{DeleteFile}\indextt{StatusFile}%
\indextt{ReadChan}\indextt{AppendChan}%
\indextt{ReadBinChan}\indextt{AppendBinChan}%
\indextt{StatusChan}\indextt{GetArgs}%
\indextt{GetEnv}\indextt{SetEnv}%
\indextt{stdin}\indextt{stdout}\indextt{stderr}\indextt{stdecho}%
Conceptually the above requests can be organised into three groups:
those relating to the {\em file system}\index{file} component of the
operating system (the first eight), those relating to the {\em channel
system}\index{channel} (the next five), and those relating to the
{\em environment} (the last four).  

The file system is fairly conventional: a mapping of file names to
contents.  The channel system consists of a collection of {\em
channels}, examples of which include standard input (\mbox{\tt stdin}),
standard output (\mbox{\tt stdout}), standard error (\mbox{\tt stderr}), and standard
echo (\mbox{\tt stdecho}) channels.  A channel is a one-way communication
medium---it either consumes values from the program (via \mbox{\tt AppendChan}
or \mbox{\tt AppendBinChan}) or produces values for the program (by responding
to \mbox{\tt ReadChan} or \mbox{\tt ReadBinChan}).  Channels communicate to and from
{\em agents}\index{agent} (a concept made more precise in
Appendix~\ref{io-semantics}).  Examples of agents include line printers, disk
controllers, networks, and human beings.  As an example of the latter,
the {\em user} is normally the consumer of standard output and the
producer of standard input.  Channels cannot be deleted, nor is there
a notion of creating a channel.

Apart from these required requests, several optional requests are
described in Appendix~\ref{io-options}.  Although not required for a
valid \Haskell{} implementation, they may be useful in
particular implementations.

Requests to the file system are in general order-dependent; if $i>j$
then the response to the $i$th request may depend on the $j$th
request.  In the case of the channel system the nature of the
dependencies is dictated by the agents.  In all cases external
effects may also be felt ``between'' internal effects.
%All of this is formalised in Appendix~\ref{io-semantics}.

Responses are defined by:
\bprog
\mbox{\tt data\ \ Response\ =\ Success}\\
\mbox{\tt \ \ \ \ \ \ \ \ \ \ \ \ \ \ \ |\ Str\ String}\\
\mbox{\tt \ \ \ \ \ \ \ \ \ \ \ \ \ \ \ |\ StrList\ [String]}\\
\mbox{\tt \ \ \ \ \ \ \ \ \ \ \ \ \ \ \ |\ Bn\ \ Bin}\\
\mbox{\tt \ \ \ \ \ \ \ \ \ \ \ \ \ \ \ |\ Failure\ IOError}\\
\mbox{\tt }\\[-8pt]
\mbox{\tt data\ \ IOError\ \ =\ WriteError\ \ \ String}\\
\mbox{\tt \ \ \ \ \ \ \ \ \ \ \ \ \ \ \ |\ ReadError\ \ \ \ String}\\
\mbox{\tt \ \ \ \ \ \ \ \ \ \ \ \ \ \ \ |\ SearchError\ \ String}\\
\mbox{\tt \ \ \ \ \ \ \ \ \ \ \ \ \ \ \ |\ FormatError\ \ String}\\
\mbox{\tt \ \ \ \ \ \ \ \ \ \ \ \ \ \ \ |\ OtherError\ \ \ String}
\eprog%
\indextycon{Response}%
\indextt{Success}\indextt{Str}%
\indextt{Failure}\indextt{Bn}%
\indextt{IOError}%
\indextt{WriteError}\indextt{ReadError}\indextt{SearchError}%
\indextt{FormatError}\indextt{OtherError}%
The response to a request is either \mbox{\tt Success}, when no value is
returned; \mbox{$\it \makebox{\tt Str\ }s$} [\mbox{$\it \makebox{\tt Bn\ }b$}], when a string [binary] value \mbox{$\it s$} [\mbox{$\it b$}]
is returned; or \mbox{$\it \makebox{\tt Failure\ }e$},
indicating failure with I/O error $e$.

The nature of a failure is defined by the \mbox{\tt IOError} datatype, which
captures the most common kinds of errors.  The \mbox{\tt String} components of
these errors are implementation dependent, and may be used to refine
the description of the error (for example, for \mbox{\tt ReadError}, the
string might be \mbox{\tt "file\ locked"}, \mbox{\tt "access\ rights\ violation"}, etc.).
An implementation is free to extend \mbox{\tt IOError} as required.

\subsection{I/O Modes}
\label{io-modes}
\index{input/output!mode}

The I/O requests \mbox{\tt ReadFile}, \mbox{\tt WriteFile}, \mbox{\tt AppendFile}, \mbox{\tt ReadChan},
and \mbox{\tt AppendChan} all work with {\em text} values---i.e.~strings.  Any
value whose type is an instance of the class \mbox{\tt Text} may be written to
a file (or communicated on a channel) by using the appropriate output
request if it is first converted to a string, using \mbox{\tt shows} (see
Section~\ref{derived-decls}).  Similarly, \mbox{\tt reads} can be used with the
appropriate input request to read such a value from a file (or a
channel).  This is text mode I/O.

For both efficiency and transparency,
\Haskell{} also supports a corresponding set of {\em binary} I/O
requests---\mbox{\tt ReadBinFile}, \mbox{\tt WriteBinFile}, \mbox{\tt AppendBinFile},
\mbox{\tt ReadBinChan}, and \mbox{\tt AppendBinChan}.  \mbox{\tt showBin} and \mbox{\tt readBin} are using
analogously to \mbox{\tt shows} and \mbox{\tt reads} (see Section~\ref{derived-decls})
for values whose types are instances of the class \mbox{\tt Binary}
(see Section~\ref{bin-type}).

Binary mode I/O ensures transparency {\em within} an
implementation---i.e.~``what is read is what was written.''
Implementations on conventional machines
will probably be able to realise binary mode more efficiently than text
mode.  On the other hand, the \mbox{\tt Bin} datatype itself is implementation
dependent, and thus binary mode {\em should not} be used as a method
to ensure transparency {\em between} implementations.

In the remainder of this section, various aspects of text mode will be
discussed, including the behaviour of standard channels such as \mbox{\tt stdin}
and \mbox{\tt stdout}.

\subsubsection{Transparent Character Set}
\index{input/output!transparency}

The {\em transparent character set}\index{transparent character set}
is defined by:

\begin{tabular}{l}
the 52 uppercase and lowercase alphabetic characters \\
the 10 decimal digits \\
the 32 graphic characters: \\
\ \ \ \ \mbox{\tt !} \mbox{\tt "} \mbox{\tt {\char'43}} \mbox{\tt {\char'44}} \mbox{\tt {\char'45}} \mbox{\tt {\char'46}} \fwq\ \mbox{\tt (} \mbox{\tt )} \mbox{\tt *} \mbox{\tt +} \mbox{\tt ,} \mbox{\tt -} \mbox{\tt .} \mbox{\tt /} \mbox{\tt :} \mbox{\tt ;} \mbox{\tt <} \mbox{\tt =} \mbox{\tt >} \mbox{\tt ?} @ \mbox{\tt [} \mbox{\tt {\char'134}} \mbox{\tt ]} \mbox{\tt {\char'136}} \mbox{\tt {\char'137}} \bkq\ \mbox{\tt {\char'173}} \mbox{\tt |} \mbox{\tt {\char'175}} \mbox{\tt {\char'176}}\\ 
the space character\\
\end{tabular}

\noindent (This is identical to the \mbox{$\it any$} syntactic category defined
in Section~\ref{whitespace}, with \mbox{$\it tab$} excluded.)

A {\em transparent line}\index{transparent line} is a list of no more
than 254 transparent characters followed by a \mbox{\tt {\char'134}n} character (i.e.~no
more than 255
characters in total).  A {\em transparent string}\index{transparent
string} is the finite concatenation of zero or more transparent lines.

\Haskell{}'s {\em text mode for files is transparent whenever the string
being used is transparent}.  An implementation must
ensure that a transparent string written to a file in text mode is
identical to the string read back from the same file in text mode
(assuming there were no intervening external effects).

The transparent character set is restricted because of
the inconsistent treatment of text files by operating systems.  For
example, some systems translate the newline character \mbox{\tt {\char'134}n} into
\mbox{\tt CR/LF}, and others into just \mbox{\tt CR} or just \mbox{\tt LF}---so none of these
characters can be in the transparent character set.  Similarly, some
systems truncate lines exceeding a certain length, others do not.
\Haskell{}'s transparent string is intended to provide a useful degree
of portability of text file manipulating programs.  Of course, an
implementation is free to guarantee a higher degree of transparency
than that defined here (such as longer lines or more character types).

Besides this definition of text mode transparency, the standard
input and output channels carry with them notions of standard {\em
presentation} and {\em acceptance}, as defined below.

\subsubsection{Presentation}
\label{presentation}
\index{input/output!presentation}

{\em Standard text mode presentation}
guarantees a minimum kind of presentable output on standard output
devices; thus it is only defined for \mbox{\tt AppendChan} using the channels
\mbox{\tt stdout}, \mbox{\tt stderr}, and \mbox{\tt stdecho}.  Abstractly, these channels are
assumed to be attached to a sequence of rectangular grids of
characters called {\em pages}; each page consists of a number of lines
and columns, with the first line presented at the ``top'' and the
first column presented to the ``left.''  The width of a column is
assumed to be constant.  (On a paper printing device, we expect an
abstract page to correspond to a physical page; on a terminal
display, it will correspond to whatever abstraction is presented by the
terminal, but at a minimum the terminal should support display of at
least one full page.)

Characters obtained from \mbox{\tt AppendChan} requests are written
sequentially into these pages starting at the top left hand corner of
the first page.  The characters are written in order horizontally
across the page until a newline character (\mbox{\tt {\char'134}n}) is processed, at
which point the subsequent characters are written starting in column
one of line two, and so on.  If a form feed character (\mbox{\tt {\char'134}f}) is
processed, writing starts at the top left hand corner of the second
page, and so on.

Maximum line length and page length for the output channels \mbox{\tt stdout},
\mbox{\tt stdecho}, and \mbox{\tt stderr} may be obtained via the \mbox{\tt StatusChan} request
as described in Section~\ref{channel-system-requests}.  These are
implementation-dependent constants, but must be at least 40 characters
and 20 lines, respectively.  \mbox{\tt AppendChan} may induce a
\mbox{\tt FormatError} if either of these limits is exceeded.

Presentation of the transparent character set may be in any readable
font.  Presentation of \mbox{\tt {\char'134}n} and \mbox{\tt {\char'134}f} is as defined above.
Presentation of any other character is not defined---presentation of
such a character may invalidate standard presentation
of all subsequent characters.  An implementation, of course, may
guarantee other forms of useful presentation beyond what is
specified here.

To facilitate processing of text to and from standard input/output
channels, the auxiliary functions shown in Figure~\ref{auxiliary-io}
are provided in the standard prelude.
\begin{figure}
\outline{
\mbox{\tt span,\ break\ \ \ \ \ \ \ \ \ \ \ \ \ ::\ (a\ ->\ Bool)\ ->\ [a]\ ->\ ([a],[a])}\\
\mbox{\tt span\ p\ xs\ \ \ \ \ \ \ \ \ \ \ \ \ \ \ =\ \ (takeWhile\ p\ xs,\ dropWhile\ p\ xs)}\\
\mbox{\tt break\ p\ \ \ \ \ \ \ \ \ \ \ \ \ \ \ \ \ =\ \ span\ (not\ .\ p)}\\
\mbox{\tt }\\[-8pt]
\mbox{\tt lines\ \ \ \ \ \ \ \ \ \ \ \ \ \ \ \ \ \ \ ::\ String\ ->\ [String]}\\
\mbox{\tt lines\ ""\ \ \ \ \ \ \ \ \ \ \ \ \ \ \ \ =\ \ []}\\
\mbox{\tt lines\ s\ \ \ \ \ \ \ \ \ \ \ \ \ \ \ \ \ =\ \ l\ :\ (if\ null\ s'\ then\ []\ else\ lines\ (tail\ s'))}\\
\mbox{\tt \ \ \ \ \ \ \ \ \ \ \ \ \ \ \ \ \ \ \ \ \ \ \ \ \ \ \ where\ (l,\ s')\ =\ break\ ((==)\ '{\char'134}n')\ s}\\
\mbox{\tt }\\[-8pt]
\mbox{\tt words\ \ \ \ \ \ \ \ \ \ \ \ \ \ \ \ \ \ \ ::\ String\ ->\ [String]}\\
\mbox{\tt words\ s\ \ \ \ \ \ \ \ \ \ \ \ \ \ \ \ \ =\ \ case\ dropWhile\ isSpace\ s\ of}\\
\mbox{\tt \ \ \ \ \ \ \ \ \ \ \ \ \ \ \ \ \ \ \ \ \ \ \ \ \ \ \ \ \ \ \ \ ""\ ->\ []}\\
\mbox{\tt \ \ \ \ \ \ \ \ \ \ \ \ \ \ \ \ \ \ \ \ \ \ \ \ \ \ \ \ \ \ \ \ s'\ ->\ w\ :\ words\ s''}\\
\mbox{\tt \ \ \ \ \ \ \ \ \ \ \ \ \ \ \ \ \ \ \ \ \ \ \ \ \ \ \ \ \ \ \ \ \ \ \ \ \ \ where\ (w,\ s'')\ =\ break\ isSpace\ s'}\\
\mbox{\tt }\\[-8pt]
\mbox{\tt unlines\ \ \ \ \ \ \ \ \ \ \ \ \ \ \ \ \ ::\ [String]\ ->\ String}\\
\mbox{\tt unlines\ ls\ \ \ \ \ \ \ \ \ \ \ \ \ \ =\ concat\ (map\ ({\char'134}l\ ->\ l\ ++\ "{\char'134}n")\ ls)}\\
\mbox{\tt }\\[-8pt]
\mbox{\tt unwords\ \ \ \ \ \ \ \ \ \ \ \ \ \ \ \ \ ::\ [String]\ ->\ String}\\
\mbox{\tt unwords\ []\ \ \ \ \ \ \ \ \ \ \ \ \ \ =\ ""}\\
\mbox{\tt unwords\ [w]\ \ \ \ \ \ \ \ \ \ \ \ \ =\ w}\\
\mbox{\tt unwords\ (w:ws)\ \ \ \ \ \ \ \ \ \ =\ w\ ++\ concat\ (map\ ('\ '\ :)\ ws)}
}
\caption{Auxiliary Functions for Text Processing of Standard Output}
\label{auxiliary-io} 
\indextt{span}\indextt{break}\indextt{lines}\indextt{words}
\indextt{unlines}\indextt{unwords}
\end{figure}

\subsubsection{Acceptance}
\index{input/output!acceptance}

{\em Standard text mode acceptance}
guarantees a minimum kind of character input from standard input
devices; thus it is only defined for \mbox{\tt ReadChan} using the channel
\mbox{\tt stdin}.  Abstractly, \mbox{\tt stdin} is assumed to be attached to a {\em
keyboard}.  The only requirement of the keyboard is that it have keys
to support the transparent character set plus the newline (\mbox{\tt {\char'134}n})
character.

\subsubsection{Echoing}
\label{echoing}
\index{input/output!echoing}

The channel \mbox{\tt stdecho} is assumed connected to the display associated
with the device to which \mbox{\tt stdin} is connected.  It may be possible for
\mbox{\tt stdout} and \mbox{\tt stdecho} to be connected to the same device, but this is
not required.  It may be possible in some operating systems to
redirect \mbox{\tt stdout} to a file while still displaying information to
the user on \mbox{\tt stdecho}.

The \mbox{\tt Echo} request (described in Section~\ref{environment-requests})
controls echoing of \mbox{\tt stdin} on \mbox{\tt stdecho}.  When echoing
is enabled, characters typed at the terminal connected to \mbox{\tt stdin} are
echoed onto \mbox{\tt stdecho}, with
optional implementation-specific line-editing functions available.
% each \mbox{\tt DEL} character treated as a
% request to ``delete previous character'' (up to the previous newline).
The list of characters returned by a read request to
\mbox{\tt stdin} should be the result of this processing.  As an
entire line may be erased by the user, a program will not see
any of the line until a \mbox{\tt {\char'134}n} character is typed.

A display may receive data from four
different sources: echoing from \mbox{\tt stdin}, and explicit output to
\mbox{\tt stdecho}, \mbox{\tt stdout}, and \mbox{\tt stderr}.  The result is an interleaving of
these character streams, but it is not an arbitrary one, because of
two constraints: (1)~{\em explicit} output (via \mbox{\tt AppendChan}) must
appear as the concatenation of the individual streams; i.e.~they
cannot be interleaved (this is consistent with the hyperstrict nature
of \mbox{\tt AppendChan}), and (2)~if echoing is on, characters from \mbox{\tt stdin}
that a program depends on for some I/O request must appear on the
display before that I/O occurs.  These constraints permit a
user to type ahead, but prevent a system from printing a reply
before echoing the user's request.

\subsection{File System Requests}
\label{file-system-requests}
\index{file system request}

In this section, each request is described using the
stream model---the corresponding behaviour using the
continuation model should be obvious.  Optional requests, not required
of a valid \Haskell{} implementation, are described in
Appendix~\ref{io-options}.

\begin{itemize}
\item
\mbox{\tt ReadFile\ \ \ \ name}\\
\mbox{\tt ReadBinFile\ name}
\indextt{ReadFile}\indextt{ReadBinFile}

Returns the contents of file \mbox{\tt name} treated as a text
[binary] file.  If successful, the response will be of the form
\mbox{\tt Str\ s} [\mbox{\tt Bn\ b}], where \mbox{\tt s} [\mbox{\tt b}] is a string [binary] value.  If the
file is not found, the response \mbox{\tt Failure\ (SearchError\ string)} is
induced; if it is unreadable for some other reason, the 
\mbox{\tt Failure\ (ReadError\ string)} error is induced.

%For example, to sum together all of the elements of a file named
%\mbox{\tt "grades"} that was written with a list of integers (i.e.~written in
%text mode with \mbox{\tt show\ gradeList}), one would first issue
%the request \mbox{\tt ReadFile\ "grades"}.  If the response is of the form
%\mbox{\tt Str\ filedGradeList}, then the grades may be summed with:
%\bprog
%@
%foldl (+) 0 (read filedGradeList)
%@
%\eprog

\item
\mbox{\tt WriteFile\ \ \ \ name\ string}\\
\mbox{\tt WriteBinFile\ name\ bin}
\indextt{WriteFile}\indextt{WriteBinFile}

Writes \mbox{\tt string} [\mbox{\tt bin}] to file \mbox{\tt name}.  If
the file does not exist, it is created.  If it already exists, it is
overwritten.  A successful response has form \mbox{\tt Success}; the only
failure possible has the form \mbox{\tt Failure\ (WriteError\ string)}.

Both of these requests are ``hyperstrict'' in their second argument:
no response is returned until the entire list of values is
completely evaluated.

\item
\mbox{\tt AppendFile\ \ \ \ name\ string}\\
\mbox{\tt AppendBinFile\ name\ bin}
\indextt{AppendFile}\indextt{AppendBinFile}

Identical to \mbox{\tt WriteFile} [\mbox{\tt WriteBinFile}], except that (1)~the \mbox{\tt string} [\mbox{\tt bin}] argument
is appended to the current contents of the file named \mbox{\tt name}; (2)~if
the I/O mode does not match the previous mode with which \mbox{\tt name} was
written, the behaviour is not specified; and (3)~if the file does not
exist, the response \mbox{\tt Failure\ (SearchError\ string)} is induced.
All other errors have form
\mbox{\tt Failure\ (WriteError\ string)}, and both
requests are hyperstrict in their second argument.

\item
\mbox{\tt DeleteFile\ name}
\indextt{DeleteFile}

Deletes file \mbox{\tt name}, with successful response \mbox{\tt Success}.  If
the file does not exist, the response \mbox{\tt Failure\ (SearchError\ string)}
is induced.  If it cannot be deleted for some other reason, a response
of the form \mbox{\tt Failure\ (WriteError\ string)} is induced.

\item
\mbox{\tt StatusFile\ name}
\indextt{StatusFile}

Induces \mbox{\tt Failure\ (SearchError\ string)} if an object \mbox{\tt name}
does not exist, otherwise induces \mbox{\tt Str\ status} where \mbox{\tt status} is a
string containing, in this order: (1)~either \fwq\mbox{\tt t}\fwq,\
\fwq\mbox{\tt b}\fwq,\ \fwq\mbox{\tt d}\fwq, or \fwq\mbox{\tt u}\fwq\ depending on whether the
object is a text file, binary file, directory, or something else,
respectively (if text and binary files cannot be distinguished,
\fwq\mbox{\tt f}\fwq\ indicates either text or binary file);
(2)~\fwq\mbox{\tt r}\fwq\ if the object is readable by this program,
\fwq\mbox{\tt -}\fwq\ if not; and (3)~\fwq\mbox{\tt w}\fwq\ if the object is writable by this
program, \fwq\mbox{\tt -}\fwq\ if not.  For example \mbox{\tt "dr-"} denotes a directory
that can be read but not written.  An implementation is
free to append more status information to this string.
% for example, if \mbox{\tt name} is a directory, an implementation may
% include a directory listing in an implementation-dependent format.
\end{itemize}

{\em Note 1.} A proper implementation of \mbox{\tt ReadFile} or \mbox{\tt ReadBinFile} may
have to make copies of files in order to preserve referential
transparency---a successful read of a file returns a {\em lazy list}
whose contents should be preserved, despite future writes to or
deletions of that file, even if the lazy list has not yet been completely
evaluated.

{\em Note 2.} Given the two juxtaposed requests:
\bprog
\mbox{\tt [\ ...,\ WriteFile\ name\ contents1,\ ReadFile\ name,\ ...\ ]}
\eprog
with the corresponding responses:
\bprog
\mbox{\tt [\ ...,\ Success,\ Str\ contents2,\ ...\ ]}
\eprog
then \mbox{\tt contents1\ ==\ contents2} if \mbox{\tt contents1} is a transparent string,
assuming that there were no external effects.  A
similar result would hold if the binary versions were used.


\subsection{Channel System Requests}
\label{channel-system-requests}
\index{channel system request}

Channels are inherently different from files---they contain
ephemeral streams of data as opposed to persistent stationary
values.  The most common channels are standard input (\mbox{\tt stdin}),
standard output (\mbox{\tt stdout}), standard error (\mbox{\tt stderr}), and standard
echo (\mbox{\tt stdecho}); these four are the only required channels
in a valid implementation.

\begin{itemize}
\item
\mbox{\tt ReadChan\ \ \ \ name}\\
\mbox{\tt ReadBinChan\ name}
\indextt{ReadChan}\indextt{ReadBinChan}

Opens channel \mbox{\tt name} for input.  A successful response
returns the contents of the channel as a lazy stream of characters [a binary
value].  If the channel does not exist the response 
\mbox{\tt Failure\ (SearchError\ string)} is induced; all other errors have form
\mbox{\tt Failure\ (ReadError\ string)}.

Unlike files, once a \mbox{\tt ReadChan} or \mbox{\tt ReadBinChan} request has been
issued for a particular channel, it cannot be issued again for the
same channel in that program.  This reflects the ephemeral nature of
its contents and prevents a serious space leak.

\item
\mbox{\tt AppendChan\ \ \ \ name\ string}\\
\mbox{\tt AppendBinChan\ name\ bin}
\indextt{AppendChan}\indextt{AppendBinChan}

Writes \mbox{\tt string} [\mbox{\tt bin}] to channel
\mbox{\tt name}.  The semantics is as for \mbox{\tt AppendFile}, except:
(1)~the second argument is appended to whatever was
previously written (if anything); (2)~if \mbox{\tt AppendChan} and
\mbox{\tt AppendBinChan} are both issued to the same channel, the resulting
behaviour is not specified; (3)~if the channel does not exist, the
% hacked format because of impossible linebreaking
response \mbox{\tt Failure} \mbox{\tt (SearchError} \mbox{\tt string)} is induced; and (4)~if the
maximum line or page length of \mbox{\tt stdout}, \mbox{\tt stderr}, or \mbox{\tt stdecho} is
exceeded, the response \mbox{\tt Failure} \mbox{\tt (FormatError} \mbox{\tt string)} is induced (see
Section~\ref{presentation}).  All other errors have form 
\mbox{\tt Failure} \mbox{\tt (WriteError} \mbox{\tt string)}.
Both requests are hyperstrict in their second argument.

\item
\mbox{\tt StatusChan\ name}
\indextt{StatusChan}

Induces \mbox{\tt Failure\ (SearchError\ string)} if channel \mbox{\tt name}
does not exist, otherwise induces \mbox{\tt Str\ status} where \mbox{\tt status} is a
string containing implementation-dependent information about the named
channel.  The only information required of a valid implementation is
that for the output channels \mbox{\tt stdout}, \mbox{\tt stdecho}, and \mbox{\tt stderr}: the
beginning of the status string must contain two integers separated by
a space, the first integer indicating the maximum line length (in
characters) allowed on the channel, the second indicating the maximum
page length (in lines) allowed (see Section~\ref{presentation}).
A zero length implies that there is no bound.

\end{itemize}

\subsection{Environment Requests}
\label{environment-requests}
\index{environment request}

\begin{itemize}
\item
\mbox{\tt Echo\ bool}
\indextt{Echo}

\nopagebreak[4]
\mbox{\tt Echo\ True} enables echoing of \mbox{\tt stdin} on \mbox{\tt stdecho}; \mbox{\tt Echo\ False}
disables it (see Section~\ref{echoing}).  Either \mbox{\tt Success} or 
\mbox{\tt Failure\ (OtherError\ string)} is induced.

The echo mode can only be set once by a particular program, and it
must be done before any I/O operation involving \mbox{\tt stdin}.  If no \mbox{\tt Echo}
request is made, a valid implementation is expected to use the echoing
mode of the OS at the time the program is run.

\item
\mbox{\tt GetArgs}
\indextt{GetArgs}

\nopagebreak[4]
Induces the response \mbox{\tt StrList\ str{\char'137}list},
where \mbox{\tt str{\char'137}list} is a list of the program's command line arguments.

\item
\mbox{\tt GetEnv\ name}
\indextt{GetEnv}

\nopagebreak[4]
Returns the value of environment variable \mbox{\tt name}.  If successful,
the response will be of the form \mbox{\tt Str\ s}, where \mbox{\tt s} is a string.  If
the environment variable does not exist, a \mbox{\tt SearchError} is induced.

\item
\mbox{\tt SetEnv\ name\ string}
\indextt{SetEnv}

\nopagebreak[4]
Sets environment variable \mbox{\tt name} to value \mbox{\tt string}, with response \mbox{\tt Success}.
If the environment variable does not exist, it is created.
\end{itemize}

\subsection{Continuation-based I/O}
\label{continuation-io}

\Haskell{} supports an alternative
style of I/O called {\em continuation-based I/O}.  Under this model, a
\Haskell{} program still has type
\mbox{\tt [Response]->[Request]}, but instead of the user manipulating the
requests and responses directly, a collection of {\em 
transactions}\index{transaction} defined in a continuation style,
captures the effect of each request/response pair.

Transactions are functions.  For each request \mbox{\tt Req} there
corresponds a transaction \mbox{\tt req}, as shown in
Figure~\ref{continuation-fig}.  For example, \mbox{\tt ReadFile}
induces
either a failure response \mbox{\tt Failure\ msg} or success response 
\mbox{\tt Str\ contents}.  In contrast the transaction \mbox{\tt readFile} would be used
in continuation-based I/O, as for example,
\bprog
\mbox{\tt readFile\ name\ ({\char'134}\ msg\ ->\ errorTransaction)\ }\\
\mbox{\tt \ \ \ \ \ \ \ \ \ \ \ \ \ \ ({\char'134}\ contents\ ->\ successTransaction)}
\eprog
where the second and third arguments are the {\em failure
continuation} and {\em success continuation}, respectively.  If the
transaction fails then the error continuation is applied to the error
message; if it succeeds then the success continuation is applied to the
contents of the file.
The following type synonyms and auxiliary functions are defined for
continuation-based I/O:
\bprog
\mbox{\tt type\ \ Dialogue\ \ \ \ \ =\ \ [Response]\ ->\ [Request]}\\
\mbox{\tt type\ \ SuccCont\ \ \ \ \ =\ \ \ \ \ \ \ \ \ \ \ \ \ \ \ \ Dialogue}\\
\mbox{\tt type\ \ StrCont\ \ \ \ \ \ =\ \ String\ \ \ \ \ ->\ Dialogue}\\
\mbox{\tt type\ \ StrListCont\ \ =\ \ [String]\ \ \ ->\ Dialogue}\\
\mbox{\tt type\ \ BinCont\ \ \ \ \ \ =\ \ Bin\ \ \ \ \ \ \ \ ->\ Dialogue}\\
\mbox{\tt type\ \ FailCont\ \ \ \ \ =\ \ IOError\ \ \ \ ->\ Dialogue}
\eprogNoSkip%
\indexsynonym{Dialogue}\indexsynonym{SuccCont}%
\indexsynonym{StrCont}\indexsynonym{BinCont}\indexsynonym{FailCont}%
\indexsynonym{StrListCont}%
\bprog
\mbox{\tt strDispatch\ fail\ succ\ (resp:resps)\ =\ }\\
\mbox{\tt \ \ \ \ \ \ \ \ \ \ \ \ case\ resp\ of\ Str\ val\ \ \ \ \ ->\ succ\ val\ resps}\\
\mbox{\tt \ \ \ \ \ \ \ \ \ \ \ \ \ \ \ \ \ \ \ \ \ \ \ \ \ Failure\ msg\ ->\ fail\ msg\ resps}\\
\mbox{\tt strListDispatch\ fail\ succ\ (resp:resps)\ =\ }\\
\mbox{\tt \ \ \ \ \ \ \ \ \ \ \ \ case\ resp\ of\ StrList\ val\ ->\ succ\ val\ resps}\\
\mbox{\tt \ \ \ \ \ \ \ \ \ \ \ \ \ \ \ \ \ \ \ \ \ \ \ \ \ Failure\ msg\ ->\ fail\ msg\ resps}\\
\mbox{\tt binDispatch\ fail\ succ\ (resp:resps)\ =\ }\\
\mbox{\tt \ \ \ \ \ \ \ \ \ \ \ \ case\ resp\ of\ Bn\ val\ \ \ \ \ \ ->\ succ\ val\ resps}\\
\mbox{\tt \ \ \ \ \ \ \ \ \ \ \ \ \ \ \ \ \ \ \ \ \ \ \ \ \ Failure\ msg\ ->\ fail\ msg\ resps}\\
\mbox{\tt succDispatch\ fail\ succ\ (resp:resps)\ =\ }\\
\mbox{\tt \ \ \ \ \ \ \ \ \ \ \ \ case\ resp\ of\ Success\ \ \ \ \ ->\ succ\ resps}\\
\mbox{\tt \ \ \ \ \ \ \ \ \ \ \ \ \ \ \ \ \ \ \ \ \ \ \ \ \ Failure\ msg\ ->\ fail\ msg\ resps}
\eprogNoSkip
\indextt{strDispatch}\indextt{binDispatch}\indextt{succDispatch}%
\indextt{strListDispatch}
\begin{figure}
\outline{
\mbox{\tt done\ \ \ \ \ \ \ \ \ \ ::\ \ \ \ \ \ \ \ \ \ \ \ \ \ \ \ \ \ \ \ \ \ \ \ \ \ \ \ \ \ \ \ \ \ \ \ \ \ \ \ \ \ \ \ \ \ \ \ Dialogue}\\
\mbox{\tt readFile\ \ \ \ \ \ ::\ String\ ->\ \ \ \ \ \ \ \ \ \ \ FailCont\ ->\ StrCont\ \ \ \ \ ->\ Dialogue}\\
\mbox{\tt writeFile\ \ \ \ \ ::\ String\ ->\ String\ ->\ FailCont\ ->\ SuccCont\ \ \ \ ->\ Dialogue}\\
\mbox{\tt appendFile\ \ \ \ ::\ String\ ->\ String\ ->\ FailCont\ ->\ SuccCont\ \ \ \ ->\ Dialogue}\\
\mbox{\tt readBinFile\ \ \ ::\ String\ ->\ \ \ \ \ \ \ \ \ \ \ FailCont\ ->\ BinCont\ \ \ \ \ ->\ Dialogue}\\
\mbox{\tt writeBinFile\ \ ::\ String\ ->\ Bin\ \ \ \ ->\ FailCont\ ->\ SuccCont\ \ \ \ ->\ Dialogue}\\
\mbox{\tt appendBinFile\ ::\ String\ ->\ Bin\ \ \ \ ->\ FailCont\ ->\ SuccCont\ \ \ \ ->\ Dialogue}\\
\mbox{\tt deleteFile\ \ \ \ ::\ String\ ->\ \ \ \ \ \ \ \ \ \ \ FailCont\ ->\ SuccCont\ \ \ \ ->\ Dialogue}\\
\mbox{\tt statusFile\ \ \ \ ::\ String\ ->\ \ \ \ \ \ \ \ \ \ \ FailCont\ ->\ StrCont\ \ \ \ \ ->\ Dialogue}\\
\mbox{\tt readChan\ \ \ \ \ \ ::\ String\ ->\ \ \ \ \ \ \ \ \ \ \ FailCont\ ->\ StrCont\ \ \ \ \ ->\ Dialogue}\\
\mbox{\tt appendChan\ \ \ \ ::\ String\ ->\ String\ ->\ FailCont\ ->\ SuccCont\ \ \ \ ->\ Dialogue}\\
\mbox{\tt readBinChan\ \ \ ::\ String\ ->\ \ \ \ \ \ \ \ \ \ \ FailCont\ ->\ BinCont\ \ \ \ \ ->\ Dialogue}\\
\mbox{\tt appendBinChan\ ::\ String\ ->\ Bin\ \ \ \ ->\ FailCont\ ->\ SuccCont\ \ \ \ ->\ Dialogue}\\
\mbox{\tt statusChan\ \ \ \ ::\ String\ ->\ \ \ \ \ \ \ \ \ \ \ FailCont\ ->\ StrCont\ \ \ \ \ ->\ Dialogue}\\
\mbox{\tt echo\ \ \ \ \ \ \ \ \ \ ::\ Bool\ \ \ ->\ \ \ \ \ \ \ \ \ \ \ FailCont\ ->\ SuccCont\ \ \ \ ->\ Dialogue}\\
\mbox{\tt getArgs\ \ \ \ \ \ \ ::\ \ \ \ \ \ \ \ \ \ \ \ \ \ \ \ \ \ \ \ \ FailCont\ ->\ StrListCont\ ->\ Dialogue}\\
\mbox{\tt getEnv\ \ \ \ \ \ \ \ ::\ String\ ->\ \ \ \ \ \ \ \ \ \ \ FailCont\ ->\ StrCont\ \ \ \ \ ->\ Dialogue}\\
\mbox{\tt setEnv\ \ \ \ \ \ \ \ ::\ String\ ->\ String\ ->\ FailCont\ ->\ SuccCont\ \ \ \ ->\ Dialogue}\\
\mbox{\tt }\\[-8pt]
\mbox{\tt done\ resps\ =\ []}\\
\mbox{\tt readFile\ name\ fail\ succ\ resps\ =\ \ \ \ \ \ \ \ \ \ \ \ --similarly\ for\ readBinFile}\\
\mbox{\tt \ \ \ (ReadFile\ name)\ :\ strDispatch\ fail\ succ\ resps}\\
\mbox{\tt writeFile\ name\ contents\ fail\ succ\ resps\ =\ \ --similarly\ for\ writeBinFile}\\
\mbox{\tt \ \ \ (WriteFile\ name\ contents)\ :\ succDispatch\ fail\ succ\ resps}\\
\mbox{\tt appendFile\ name\ contents\ fail\ succ\ resps\ =\ --similarly\ for\ appendBinFile}\\
\mbox{\tt \ \ \ (AppendFile\ name\ contents)\ :\ succDispatch\ fail\ succ\ resps}\\
\mbox{\tt deleteFile\ name\ fail\ succ\ resps\ =}\\
\mbox{\tt \ \ \ (DeleteFile\ name)\ :\ succDispatch\ fail\ succ\ resps}\\
\mbox{\tt statusFile\ name\ fail\ succ\ resps\ =\ \ \ \ \ \ \ \ \ \ --similarly\ for\ statusChan}\\
\mbox{\tt \ \ \ (StatusFile\ name)\ :\ strDispatch\ fail\ succ\ resps}\\
\mbox{\tt readChan\ name\ fail\ succ\ resps\ =\ \ \ \ \ \ \ \ \ \ \ \ --similarly\ for\ readBinChan}\\
\mbox{\tt \ \ \ (ReadChan\ name)\ :\ strDispatch\ fail\ succ\ resps}\\
\mbox{\tt appendChan\ name\ contents\ fail\ succ\ resps\ =\ --similarly\ for\ appendBinChan}\\
\mbox{\tt \ \ \ (AppendChan\ name\ contents)\ :\ succDispatch\ fail\ succ\ resps}\\
\mbox{\tt echo\ bool\ fail\ succ\ resps\ =}\\
\mbox{\tt \ \ \ (Echo\ bool)\ :\ succDispatch\ fail\ succ\ resps}\\
\mbox{\tt getArgs\ fail\ succ\ resps\ =}\\
\mbox{\tt \ \ \ GetArgs\ :\ strListDispatch\ fail\ succ\ resps}\\
\mbox{\tt getEnv\ name\ fail\ succ\ resps\ =}\\
\mbox{\tt \ \ \ (GetEnv\ name)\ :\ strDispatch\ fail\ succ\ resps}\\
\mbox{\tt setEnv\ name\ contents\ fail\ succ\ resps\ =}\\
\mbox{\tt \ \ \ (SetEnv\ name\ contents)\ :\ succDispatch\ fail\ succ\ resps}
}
\caption{Transactions of continuation-based I/O.}
\label{continuation-fig}
\indextt{done}\indextt{readFile}\indextt{writeFile}
\indextt{appendFile}\indextt{readBinFile}\indextt{writeBinFile}
\indextt{appendBinFile}\indextt{deleteFile}\indextt{statusFile}
\indextt{readChan}\indextt{appendChan}
\indextt{readBinChan}\indextt{appendBinChan}\indextt{statusChan}
\indextt{echo}\indextt{getArgs}\indextt{getEnv}\indextt{setEnv}
\end{figure}
\bprog
\mbox{\tt abort\ \ \ \ \ ::\ \ FailCont}\\
\mbox{\tt abort\ err\ \ =\ \ done}\\
\mbox{\tt }\\[-8pt]
\mbox{\tt exit\ \ \ \ \ ::\ \ FailCont}\\
\mbox{\tt exit\ err\ \ =\ \ appendChan\ stdout\ msg\ abort\ done}\\
\mbox{\tt \ \ \ \ \ \ \ \ \ \ \ \ \ where\ msg\ =\ case\ err\ of\ ReadError\ \ \ s\ ->\ s}\\
\mbox{\tt \ \ \ \ \ \ \ \ \ \ \ \ \ \ \ \ \ \ \ \ \ \ \ \ \ \ \ \ \ \ \ \ \ \ \ \ \ WriteError\ \ s\ ->\ s}\\
\mbox{\tt \ \ \ \ \ \ \ \ \ \ \ \ \ \ \ \ \ \ \ \ \ \ \ \ \ \ \ \ \ \ \ \ \ \ \ \ \ SearchError\ s\ ->\ s}\\
\mbox{\tt \ \ \ \ \ \ \ \ \ \ \ \ \ \ \ \ \ \ \ \ \ \ \ \ \ \ \ \ \ \ \ \ \ \ \ \ \ FormatError\ s\ ->\ s}\\
\mbox{\tt \ \ \ \ \ \ \ \ \ \ \ \ \ \ \ \ \ \ \ \ \ \ \ \ \ \ \ \ \ \ \ \ \ \ \ \ \ OtherError\ \ s\ ->\ s}\\
\mbox{\tt }\\[-8pt]
\mbox{\tt print\ \ \ \ \ \ \ \ \ \ \ ::\ (Text\ a)\ =>\ a\ ->\ Dialogue}\\
\mbox{\tt print\ x\ \ \ \ \ \ \ \ \ =\ \ appendChan\ stdout\ (show\ x)\ abort\ done}\\
\mbox{\tt prints\ \ \ \ \ \ \ \ \ \ ::\ (Text\ a)\ =>\ a\ ->\ String\ ->\ Dialogue}\\
\mbox{\tt prints\ x\ s\ \ \ \ \ \ =\ \ appendChan\ stdout\ (shows\ x\ s)\ abort\ done}\\
\mbox{\tt }\\[-8pt]
\mbox{\tt interact\ \ ::\ \ (String\ ->\ String)\ ->\ Dialogue}\\
\mbox{\tt interact\ f\ =\ readChan\ stdin\ abort}\\
\mbox{\tt \ \ \ \ \ \ \ \ \ \ \ \ \ \ \ \ \ \ \ \ \ \ ({\char'134}x\ ->\ appendChan\ stdout\ (f\ x)\ abort\ done)}
\eprogNoSkip
\indextt{abort}\indextt{exit}%
\indextt{print}\indextt{prints}\indextt{interact}

\subsection{A Small Example}

Both of the following programs prompt the user for the name of a
file, and then look up and display the contents of the file on
standard-output.  The filename as typed by the user is also echoed.
The first program uses the stream-based style (note the irrefutable
patterns):\nopagebreak[4]
\bprog
\mbox{\tt main\ {\char'176}(Success\ :\ {\char'176}((Str\ userInput)\ :\ {\char'176}(Success\ :\ {\char'176}(r4\ :\ {\char'137}))))\ =}\\
\mbox{\tt \ \ [\ AppendChan\ stdout\ "please\ type\ a\ filename{\char'134}n",}\\
\mbox{\tt \ \ \ \ ReadChan\ stdin,}\\
\mbox{\tt \ \ \ \ AppendChan\ stdout\ name,}\\
\mbox{\tt \ \ \ \ ReadFile\ name,}\\
\mbox{\tt \ \ \ \ AppendChan\ stdout\ (case\ r4\ of\ Str\ contents\ \ \ \ ->\ contents}\\
\mbox{\tt \ \ \ \ \ \ \ \ \ \ \ \ \ \ \ \ \ \ \ \ \ \ \ \ \ \ \ \ \ \ \ \ \ \ Failure\ ioerror\ ->\ "can't\ open\ file")}\\
\mbox{\tt \ \ ]\ where\ (name\ :\ {\char'137})\ =\ lines\ userInput}
\eprog
The second program uses the continuation-based style:\nopagebreak[4]
\bprog
\mbox{\tt main\ =\ appendChan\ stdout\ "please\ type\ a\ filename{\char'134}n"\ abort\ (}\\
\mbox{\tt \ \ \ \ \ \ \ readChan\ stdin\ abort\ ({\char'134}\ userInput\ ->}\\
\mbox{\tt \ \ \ \ \ \ \ let\ (name\ :\ {\char'137})\ =\ lines\ userInput\ in}\\
\mbox{\tt \ \ \ \ \ \ \ appendChan\ stdout\ name\ abort\ (}\\
\mbox{\tt \ \ \ \ \ \ \ readFile\ name\ ({\char'134}\ ioerror\ ->\ appendChan\ stdout\ }\\
\mbox{\tt \ \ \ \ \ \ \ \ \ \ \ \ \ \ \ \ \ \ \ \ \ \ \ \ \ \ \ \ \ \ \ \ \ \ \ "can't\ open\ file"\ abort\ done)}\\
\mbox{\tt \ \ \ \ \ \ \ \ \ \ \ \ \ \ \ \ \ \ \ \ \ ({\char'134}\ contents\ ->}\\
\mbox{\tt \ \ \ \ \ \ \ appendChan\ stdout\ contents\ abort\ done)))))}
\eprog

More examples and a general discussion of both forms of I/O may
be found in a report by Hudak and Sundaresh \cite{hudak:io}.
\pagebreak[4]

\subsection{An Example Involving Synchronisation}

The following program reads two numbers and prints their sum.  After
the initial \mbox{\tt readChan} request, the value of the input stream must be
passed in and out of the functions which actually obtain the user
input.  The programmer must control the synchronisation between the
\mbox{\tt appendChan} requests and when the program stops to read input.  The
\mbox{\tt readChan} request does not actually cause the program to stop and
wait for the user to enter the entire input stream; only at demands
for actual input characters will execution pause for input.  This
program assures that this demand is properly synchronised with the
\mbox{\tt appendChan} requests by verifying input values in the \mbox{\tt readInt}
function.
\bprog
\mbox{\tt main\ ::\ Dialogue}\\
\mbox{\tt }\\[-8pt]
\mbox{\tt main\ =\ readChan\ stdin\ abort\ ({\char'134}\ userInput\ ->\ readNums\ (lines\ userInput))}\\
\mbox{\tt }\\[-8pt]
\mbox{\tt readNums\ ::\ [String]\ ->\ Dialogue}\\
\mbox{\tt }\\[-8pt]
\mbox{\tt readNums\ inputLines\ =}\\
\mbox{\tt \ \ \ readInt\ "Enter\ first\ number:\ "\ inputLines}\\
\mbox{\tt \ \ \ \ \ ({\char'134}\ num1\ inputLines1\ ->}\\
\mbox{\tt \ \ \ \ \ \ \ \ \ \ readInt\ "Enter\ second\ number:\ "\ inputLines1}\\
\mbox{\tt \ \ \ \ \ \ \ \ \ \ \ \ \ ({\char'134}\ num2\ {\char'137}\ ->\ reportResult\ num1\ num2))}\\
\mbox{\tt reportResult\ ::\ Int\ ->\ Int\ ->\ Dialogue}\\
\mbox{\tt }\\[-8pt]
\mbox{\tt reportResult\ num1\ num2\ =}\\
\mbox{\tt \ \ appendChan\ stdout\ ("Their\ sum\ is:\ "\ ++\ show\ (num1\ +\ num2))\ abort\ done}
\eprog
\bprog
\mbox{\tt --\ readInt\ prints\ a\ prompt\ and\ then\ reads\ a\ line\ of\ input.\ \ If\ the}\\
\mbox{\tt --\ line\ contains\ an\ integer,\ the\ value\ of\ the\ integer\ is\ passed\ to\ the}\\
\mbox{\tt --\ success\ continuation.\ \ If\ a\ line\ cannot\ be\ parsed\ as\ an\ integer,}\\
\mbox{\tt --\ an\ error\ message\ is\ printed\ and\ the\ user\ is\ asked\ to\ try\ again.}\\
\mbox{\tt --\ If\ EOF\ is\ detected,\ the\ program\ is\ aborted.}\\
\mbox{\tt }\\[-8pt]
\mbox{\tt readInt\ ::\ String\ ->\ [String]\ ->\ (Int\ ->\ [String]\ ->\ Dialogue)\ ->\ Dialogue}\\
\mbox{\tt }\\[-8pt]
\mbox{\tt readInt\ prompt\ inputLines\ succ\ =}\\
\mbox{\tt \ \ appendChan\ stdout\ prompt\ abort}\\
\mbox{\tt \ \ \ \ (case\ inputLines\ of}\\
\mbox{\tt \ \ \ \ \ \ \ (l1\ :\ rest)\ ->\ case\ (reads\ l1)\ of}\\
\mbox{\tt \ \ \ \ \ \ \ \ \ \ \ \ \ \ \ \ \ \ \ \ \ \ \ \ [(x,"")]\ ->\ succ\ x\ rest}\\
\mbox{\tt \ \ \ \ \ \ \ \ \ \ \ \ \ \ \ \ \ \ \ \ \ \ \ \ {\char'137}\ \ \ \ \ \ \ \ ->\ appendChan\ stdout}\\
\mbox{\tt \ \ \ \ \ \ \ \ \ \ \ \ \ \ \ \ \ \ \ \ \ \ \ \ \ \ \ \ \ \ \ \ \ \ \ \ \ \ "Error\ -\ retype\ the\ number{\char'134}n"\ abort}\\
\mbox{\tt \ \ \ \ \ \ \ \ \ \ \ \ \ \ \ \ \ \ \ \ \ \ \ \ \ \ \ \ \ \ \ \ \ \ \ \ \ \ \ (readInt\ prompt\ rest\ succ)}\\
\mbox{\tt \ \ \ \ \ \ \ {\char'137}\ \ \ \ \ \ \ \ \ \ \ ->\ appendChan\ stdout\ "Early\ EOF"\ abort\ done)}
\eprog

% Local Variables: 
% mode: latex
% End:

%
%
%\startnewstuff
\startnewsection
\appendix
%
% $Header$
%
\section{Standard Prelude}
\label{stdprelude}

In this appendix the entire \Haskell{} prelude is given.  It is
organised into a root module and eight sub-modules.

\medskip
\noindent\bprogB
\mbox{\tt --\ Standard\ value\ bindings}\\
\mbox{\tt }\\
\mbox{\tt module\ Prelude\ (}\\
\mbox{\tt \ \ \ \ PreludeCore..,\ PreludeRatio..,\ PreludeList..,\ PreludeArray..,\ }\\
\mbox{\tt \ \ \ \ PreludeText..,\ PreludeIO..,\ }\\
\mbox{\tt \ \ \ \ nullBin,\ isNullBin,}\\
\mbox{\tt \ \ \ \ ({\char'46}{\char'46}),\ (||),\ not,}\\
\mbox{\tt \ \ \ \ ord,\ chr,\ }\\
\mbox{\tt \ \ \ \ isAscii,\ isControl,\ isPrint,\ isSpace,\ }\\
\mbox{\tt \ \ \ \ isUpper,\ isLower,\ isAlpha,\ isDigit,\ isAlphanum,}\\
\mbox{\tt \ \ \ \ toUpper,\ toLower,}\\
\mbox{\tt \ \ \ \ gcd,\ lcm,\ ({\char'136}),\ ({\char'136}{\char'136}),\ }\\
\mbox{\tt \ \ \ \ truncate,\ round,\ ceiling,\ floor,\ fromIntegral,\ fromRealFrac,\ atan2,}\\
\mbox{\tt \ \ \ \ realPart,\ imagPart,\ conjugate,\ mkPolar,\ cis,\ polar,\ magnitude,\ phase,}\\
\mbox{\tt \ \ \ \ fst,\ snd,\ (.),\ until,\ error,\ asTypeOf,\ otherwise\ )\ where}
\eprogB\noindent\bprogB
\mbox{\tt import\ PreludeBuiltin}\\
\mbox{\tt import\ PreludeCore}\\
\mbox{\tt import\ PreludeList}\\
\mbox{\tt import\ PreludeArray}\\
\mbox{\tt import\ PreludeRatio}\\
\mbox{\tt import\ PreludeText}\\
\mbox{\tt import\ PreludeIO}
\eprogB\noindent\bprogB
\mbox{\tt infixr\ 9\ \ .}\\
\mbox{\tt infixr\ 8\ \ {\char'136},\ {\char'136}{\char'136}}\\
\mbox{\tt infixr\ 1\ \ {\char'46}{\char'46}}\\
\mbox{\tt infixr\ 0\ \ ||}
\eprogB\noindent\bprogB
\mbox{\tt --\ Binary\ functions}\\
\mbox{\tt }\\
\mbox{\tt nullBin\ \ \ \ \ \ \ \ \ \ \ \ \ \ \ \ \ ::\ Bin}\\
\mbox{\tt nullBin\ \ \ \ \ \ \ \ \ \ \ \ \ \ \ \ \ =\ \ primNullBin}
\indextt{nullBin}%
\eprogB\noindent\bprogB
\mbox{\tt isNullBin\ \ \ \ \ \ \ \ \ \ \ \ \ \ \ ::\ Bin\ ->\ Bool}\\
\mbox{\tt isNullBin\ \ \ \ \ \ \ \ \ \ \ \ \ \ \ =\ \ primIsNullBin}
\indextt{isNullBin}%
\eprogB\noindent\bprogB
\mbox{\tt --\ Boolean\ functions}\\
\mbox{\tt }\\
\mbox{\tt ({\char'46}{\char'46}),\ (||)\ \ \ \ \ \ \ \ \ \ \ \ \ \ ::\ Bool\ ->\ Bool\ ->\ Bool}\\
\mbox{\tt True\ \ {\char'46}{\char'46}\ x\ \ \ \ \ \ \ \ \ \ \ \ \ \ =\ \ x}\\
\mbox{\tt False\ {\char'46}{\char'46}\ x\ \ \ \ \ \ \ \ \ \ \ \ \ \ =\ \ False}\\
\mbox{\tt True\ \ ||\ x\ \ \ \ \ \ \ \ \ \ \ \ \ \ =\ \ True}\\
\mbox{\tt False\ ||\ x\ \ \ \ \ \ \ \ \ \ \ \ \ \ =\ \ x}
\index{&&@{\tt \&\&}}%
\index{||@{\tt {\char'174}{\char'174}}}%
\eprogB\noindent\bprogB
\mbox{\tt not\ \ \ \ \ \ \ \ \ \ \ \ \ \ \ \ \ \ \ \ \ ::\ Bool\ ->\ Bool}\\
\mbox{\tt not\ True\ \ \ \ \ \ \ \ \ \ \ \ \ \ \ \ =\ \ False}\\
\mbox{\tt not\ False\ \ \ \ \ \ \ \ \ \ \ \ \ \ \ =\ \ True}
\indextt{not}%
\eprogB\noindent\bprogB
\mbox{\tt otherwise\ \ \ \ \ \ \ \ \ \ \ \ \ \ \ ::\ Bool}\\
\mbox{\tt otherwise\ \ \ \ \ \ \ \ \ \ \ \ \ \ \ =\ True}
\indextt{otherwise}%
\eprogB\noindent\bprogB
\mbox{\tt --\ Character\ functions}\\
\mbox{\tt }\\
\mbox{\tt ord\ \ \ \ \ \ \ \ \ \ \ \ \ \ \ \ \ \ \ \ \ ::\ Char\ ->\ Int}\\
\mbox{\tt ord\ \ \ \ \ \ \ \ \ \ \ \ \ \ \ \ \ \ \ \ \ =\ primCharToInt}
\indextt{ord}%
\eprogB\noindent\bprogB
\mbox{\tt chr\ \ \ \ \ \ \ \ \ \ \ \ \ \ \ \ \ \ \ \ \ ::\ Int\ ->\ Char}\\
\mbox{\tt chr\ \ \ \ \ \ \ \ \ \ \ \ \ \ \ \ \ \ \ \ \ =\ primIntToChar}
\indextt{chr}%
\eprogB\noindent\bprogB
\mbox{\tt isAscii,\ isControl,\ isPrint,\ isSpace\ \ \ \ \ \ \ \ \ \ \ \ ::\ Char\ ->\ Bool}\\
\mbox{\tt isUpper,\ isLower,\ isAlpha,\ isDigit,\ isAlphanum\ \ ::\ Char\ ->\ Bool}
\indextt{isAscii}%
\indextt{isControl}%
\indextt{isPrint}%
\indextt{isSpace}%
\indextt{isUpper}%
\indextt{isLower}%
\indextt{isAlpha}%
\indextt{isDigit}%
\indextt{isAlphanum}%
\eprogB\noindent\bprogB
\mbox{\tt isAscii\ c\ \ \ \ \ \ \ \ \ \ \ \ \ \ \ =\ \ ord\ c\ <\ 128}\\
\mbox{\tt isControl\ c\ \ \ \ \ \ \ \ \ \ \ \ \ =\ \ c\ <\ '\ '\ ||\ c\ ==\ '{\char'134}DEL'}\\
\mbox{\tt isPrint\ c\ \ \ \ \ \ \ \ \ \ \ \ \ \ \ =\ \ c\ >=\ '\ '\ {\char'46}{\char'46}\ c\ <=\ '{\char'176}'}\\
\mbox{\tt isSpace\ c\ \ \ \ \ \ \ \ \ \ \ \ \ \ \ =\ \ c\ ==\ '\ '\ ||\ c\ ==\ '{\char'134}t'\ ||\ c\ ==\ '{\char'134}n'\ ||\ }\\
\mbox{\tt \ \ \ \ \ \ \ \ \ \ \ \ \ \ \ \ \ \ \ \ \ \ \ \ \ \ \ c\ ==\ '{\char'134}r'\ ||\ c\ ==\ '{\char'134}f'\ ||\ c\ ==\ '{\char'134}v'}\\
\mbox{\tt isUpper\ c\ \ \ \ \ \ \ \ \ \ \ \ \ \ \ =\ \ c\ >=\ 'A'\ {\char'46}{\char'46}\ c\ <=\ 'Z'}\\
\mbox{\tt isLower\ c\ \ \ \ \ \ \ \ \ \ \ \ \ \ \ =\ \ c\ >=\ 'a'\ {\char'46}{\char'46}\ c\ <=\ 'z'}\\
\mbox{\tt isAlpha\ c\ \ \ \ \ \ \ \ \ \ \ \ \ \ \ =\ \ isUpper\ c\ ||\ isLower\ c}\\
\mbox{\tt isDigit\ c\ \ \ \ \ \ \ \ \ \ \ \ \ \ \ =\ \ c\ >=\ '0'\ {\char'46}{\char'46}\ c\ <=\ '9'}\\
\mbox{\tt isAlphanum\ c\ \ \ \ \ \ \ \ \ \ \ \ =\ \ isAlpha\ c\ ||\ isDigit\ c}
\eprogB\noindent\bprogB
\mbox{\tt toUpper,\ toLower\ \ \ \ \ \ \ \ ::\ Char\ ->\ Char}\\
\mbox{\tt toUpper\ c\ |\ isLower\ c\ \ \ =\ \ chr\ (ord\ c\ -\ ord\ 'a'\ +\ ord\ 'A')}\\
\mbox{\tt \ \ \ \ \ \ \ \ \ \ |\ otherwise\ \ \ =\ \ c}
\indextt{toUpper}%
\indextt{toLower}%
\eprogB\noindent\bprogB
\mbox{\tt toLower\ c\ |\ isUpper\ c\ \ \ =\ \ chr\ (ord\ c\ -\ ord\ 'A'\ +\ ord\ 'a')}\\
\mbox{\tt \ \ \ \ \ \ \ \ \ \ |\ otherwise\ \ \ =\ \ c}
\eprogB\noindent\bprogB
\mbox{\tt --\ Numeric\ functions}\\
\mbox{\tt }\\
\mbox{\tt minInt,\ maxInt\ \ ::\ Int}\\
\mbox{\tt minInt\ \ \ \ \ \ \ \ \ \ =\ \ primMinInt}\\
\mbox{\tt maxInt\ \ \ \ \ \ \ \ \ \ =\ \ primMaxInt}
\indextt{minInt}%
\indextt{maxInt}%
\eprogB\noindent\bprogB
\mbox{\tt gcd\ \ \ \ \ \ \ \ \ \ \ \ \ ::\ (Integral\ a)\ =>\ a\ ->\ a->\ a}\\
\mbox{\tt gcd\ x\ 0\ \ \ \ \ \ \ \ \ =\ \ abs\ x}\\
\mbox{\tt gcd\ 0\ y\ \ \ \ \ \ \ \ \ =\ \ abs\ y}\\
\mbox{\tt gcd\ x\ y\ \ \ \ \ \ \ \ \ =\ \ gcd'\ (abs\ x)\ (abs\ y)}\\
\mbox{\tt \ \ \ \ \ \ \ \ \ \ \ \ \ \ \ \ \ \ \ where\ gcd'\ x\ y\ \ =\ \ if\ x\ ==\ y\ then\ x}\\
\mbox{\tt \ \ \ \ \ \ \ \ \ \ \ \ \ \ \ \ \ \ \ \ \ \ \ \ \ \ \ \ \ \ \ \ \ \ \ \ \ \ else\ gcd'\ y\ (x\ `rem`\ y)}
\indextt{gcd}%
\eprogB\noindent\bprogB
\mbox{\tt lcm\ \ \ \ \ \ \ \ \ \ \ \ \ ::\ (Integral\ a)\ =>\ a\ ->\ a->\ a}\\
\mbox{\tt lcm\ {\char'137}\ 0\ \ \ \ \ \ \ \ \ =\ \ 0}\\
\mbox{\tt lcm\ 0\ {\char'137}\ \ \ \ \ \ \ \ \ =\ \ 0}\\
\mbox{\tt lcm\ x\ y\ \ \ \ \ \ \ \ \ =\ \ abs\ ((x\ `div`\ (gcd\ x\ y))\ *\ y)}
\indextt{lcm}%
\eprogB\noindent\bprogB
\mbox{\tt ({\char'136})\ \ \ \ \ \ \ \ \ \ \ \ \ ::\ (Num\ a,\ Integral\ b)\ =>\ a\ ->\ b\ ->\ a}\\
\mbox{\tt x\ {\char'136}\ 0\ \ \ \ \ \ \ \ \ \ \ =\ \ 1}\\
\mbox{\tt x\ {\char'136}\ (n+1)\ \ \ \ \ \ \ =\ \ f\ x\ n\ x}\\
\mbox{\tt \ \ \ \ \ \ \ \ \ \ \ \ \ \ \ \ \ \ \ where\ f\ {\char'137}\ 0\ y\ =\ y}\\
\mbox{\tt \ \ \ \ \ \ \ \ \ \ \ \ \ \ \ \ \ \ \ \ \ \ \ \ \ f\ x\ n\ y\ =\ g\ x\ n\ \ where}\\
\mbox{\tt \ \ \ \ \ \ \ \ \ \ \ \ \ \ \ \ \ \ \ \ \ \ \ \ \ \ \ \ \ \ \ \ \ \ \ g\ x\ n\ |\ even\ n\ \ =\ g\ (x*x)\ (n`div`2)}\\
\mbox{\tt \ \ \ \ \ \ \ \ \ \ \ \ \ \ \ \ \ \ \ \ \ \ \ \ \ \ \ \ \ \ \ \ \ \ \ \ \ \ \ \ \ |\ otherwise\ =\ f\ x\ (n-1)\ (x*y)}
\index{^@{\tt {\char'136}}}%
\eprogB\noindent\bprogB
\mbox{\tt ({\char'136}{\char'136})\ \ \ \ \ \ \ \ \ \ \ \ ::\ (Fractional\ a,\ Integral\ b)\ =>\ a\ ->\ b\ ->\ a}\\
\mbox{\tt x\ {\char'136}{\char'136}\ n\ \ \ \ \ \ \ \ \ \ =\ \ if\ n\ >=\ 0\ then\ x{\char'136}n\ else\ 1/x{\char'136}(-n)}
\index{^^@{\tt {\char'136}{\char'136}}}%
\eprogB\noindent\bprogB
\mbox{\tt truncate\ \ \ \ \ \ \ \ ::\ (RealFrac\ a,\ Integral\ b)\ =>\ a\ ->\ b}\\
\mbox{\tt truncate\ x\ \ \ \ \ \ =\ \ fromInteger\ m\ \ where\ (m,r)\ =\ properFraction\ x}
\indextt{truncate}%
\eprogB\noindent\bprogB
\mbox{\tt round\ \ \ \ \ \ \ \ \ \ \ ::\ (RealFrac\ a,\ Integral\ b)\ =>\ a\ ->\ b}\\
\mbox{\tt round\ x\ \ \ \ \ \ \ \ \ =\ \ fromInteger\ y}\\
\mbox{\tt \ \ \ \ \ \ \ \ \ \ \ \ \ \ \ \ \ \ \ where\ y\ =\ case\ signum\ (abs\ r\ -\ 0.5)\ of}\\
\mbox{\tt \ \ \ \ \ \ \ \ \ \ \ \ \ \ \ \ \ \ \ \ \ \ \ \ \ \ \ \ \ \ \ \ -1\ ->\ n}\\
\mbox{\tt \ \ \ \ \ \ \ \ \ \ \ \ \ \ \ \ \ \ \ \ \ \ \ \ \ \ \ \ \ \ \ \ 0\ \ ->\ if\ even\ n\ then\ n\ else\ m}\\
\mbox{\tt \ \ \ \ \ \ \ \ \ \ \ \ \ \ \ \ \ \ \ \ \ \ \ \ \ \ \ \ \ \ \ \ 1\ \ ->\ m}\\
\mbox{\tt \ \ \ \ \ \ \ \ \ \ \ \ \ \ \ \ \ \ \ \ \ \ \ \ \ (n,r)\ =\ properFraction\ x}\\
\mbox{\tt \ \ \ \ \ \ \ \ \ \ \ \ \ \ \ \ \ \ \ \ \ \ \ \ \ m\ \ \ \ \ =\ if\ r\ <\ 0\ then\ n\ -\ 1\ else\ n\ +\ 1}
\indextt{round}%
\eprogB\noindent\bprogB
\mbox{\tt ceiling\ \ \ \ \ \ \ \ \ ::\ (RealFrac\ a,\ Integral\ b)\ =>\ a\ ->\ b}\\
\mbox{\tt ceiling\ x\ \ \ \ \ \ \ =\ \ fromInteger\ (if\ r\ >\ 0\ then\ n\ +\ 1\ else\ n)}\\
\mbox{\tt \ \ \ \ \ \ \ \ \ \ \ \ \ \ \ \ \ \ \ where\ (n,r)\ =\ properFraction\ x}
\indextt{ceiling}%
\eprogB\noindent\bprogB
\mbox{\tt floor\ \ \ \ \ \ \ \ \ \ \ ::\ (RealFrac\ a,\ Integral\ b)\ =>\ a\ ->\ b}\\
\mbox{\tt floor\ x\ \ \ \ \ \ \ \ \ =\ \ fromInteger\ (if\ r\ <\ 0\ then\ n\ -\ 1\ else\ n)}\\
\mbox{\tt \ \ \ \ \ \ \ \ \ \ \ \ \ \ \ \ \ \ \ where\ (n,r)\ =\ properFraction\ x}
\indextt{floor}%
\eprogB\noindent\bprogB
\mbox{\tt fromIntegral\ \ \ \ ::\ (Integral\ a,\ Num\ b)\ =>\ a\ ->\ b}\\
\mbox{\tt fromIntegral\ \ \ \ =\ \ fromInteger\ .\ toInteger}
\indextt{fromIntegral}%
\eprogB\noindent\bprogB
\mbox{\tt fromRealFrac\ \ \ \ ::\ (RealFrac\ a,\ Fractional\ b)\ =>\ a\ ->\ b}\\
\mbox{\tt fromRealFrac\ \ \ \ =\ \ fromRational\ .\ toRational}
\indextt{fromRealFrac}%
\eprogB\noindent\bprogB
\mbox{\tt atan2\ \ \ \ \ \ \ \ \ \ \ ::\ (RealFloat\ a)\ =>\ a\ ->\ a\ ->\ a}\\
\mbox{\tt atan2\ y\ x\ \ \ \ \ \ \ =\ \ case\ (signum\ y,\ signum\ x)\ of}\\
\mbox{\tt \ \ \ \ \ \ \ \ \ \ \ \ \ \ \ \ \ \ \ \ \ \ \ \ (\ 0,\ 1)\ ->\ \ 0}\\
\mbox{\tt \ \ \ \ \ \ \ \ \ \ \ \ \ \ \ \ \ \ \ \ \ \ \ \ (\ 1,\ 0)\ ->\ \ pi/2}\\
\mbox{\tt \ \ \ \ \ \ \ \ \ \ \ \ \ \ \ \ \ \ \ \ \ \ \ \ (\ 0,-1)\ ->\ \ pi}\\
\mbox{\tt \ \ \ \ \ \ \ \ \ \ \ \ \ \ \ \ \ \ \ \ \ \ \ \ (-1,\ 0)\ ->\ -pi/2}\\
\mbox{\tt \ \ \ \ \ \ \ \ \ \ \ \ \ \ \ \ \ \ \ \ \ \ \ \ (\ {\char'137},\ 1)\ ->\ \ atan\ (y/x)}\\
\mbox{\tt \ \ \ \ \ \ \ \ \ \ \ \ \ \ \ \ \ \ \ \ \ \ \ \ (\ {\char'137},-1)\ ->\ \ atan\ (y/x)\ +\ pi}\\
\mbox{\tt \ \ \ \ \ \ \ \ \ \ \ \ \ \ \ \ \ \ \ \ \ \ \ \ --\ (0,0)\ is\ an\ error}
\indextt{atan2}%
\eprogB\noindent\bprogB
\mbox{\tt realPart,\ imagPart\ ::\ (RealFloat\ a)\ =>\ Complex\ a\ ->\ a}\\
\mbox{\tt realPart\ (x:+y)\ \ =\ \ x}\\
\mbox{\tt imagPart\ (x:+y)\ \ =\ \ y}
\indextt{realPart}%
\indextt{imagPart}%
\eprogB\noindent\bprogB
\mbox{\tt conjugate\ \ \ \ \ \ \ \ ::\ (RealFloat\ a)\ =>\ Complex\ a\ ->\ Complex\ a}\\
\mbox{\tt conjugate\ (x:+y)\ =\ \ x\ :+\ (-y)}
\indextt{conjugate}%
\eprogB\noindent\bprogB
\mbox{\tt mkPolar\ \ \ \ \ \ \ \ \ \ ::\ (RealFloat\ a)\ =>\ a\ ->\ a\ ->\ Complex\ a}\\
\mbox{\tt mkPolar\ r\ theta\ \ =\ \ r\ *\ sin\ theta\ :+\ r\ *\ cos\ theta}
\indextt{mkPolar}%
\eprogB\noindent\bprogB
\mbox{\tt cis\ \ \ \ \ \ \ \ \ \ \ \ \ \ ::\ (RealFloat\ a)\ =>\ a\ ->\ Complex\ a}\\
\mbox{\tt cis\ theta\ \ \ \ \ \ \ \ =\ \ sin\ theta\ :+\ cos\ theta}
\indextt{cis}%
\eprogB\noindent\bprogB
\mbox{\tt polar\ \ \ \ \ \ \ \ \ \ \ \ ::\ (RealFloat\ a)\ =>\ Complex\ a\ ->\ (a,a)}\\
\mbox{\tt polar\ z\ \ \ \ \ \ \ \ \ \ =\ \ (magnitude\ z,\ phase\ z)}
\indextt{polar}%
\eprogB\noindent\bprogB
\mbox{\tt magnitude,\ phase\ ::\ (RealFloat\ a)\ =>\ Complex\ a\ ->\ a}\\
\mbox{\tt magnitude\ (x:+y)\ =\ \ scaleFloat\ k}\\
\mbox{\tt \ \ \ \ \ \ \ \ \ \ \ \ \ \ \ \ \ \ \ \ \ (sqrt\ ((scaleFloat\ mk\ x){\char'136}2\ +\ (scaleFloat\ mk\ y){\char'136}2))}\\
\mbox{\tt \ \ \ \ \ \ \ \ \ \ \ \ \ \ \ \ \ \ \ \ where\ k\ \ =\ max\ (exponent\ x)\ (exponent\ y)}\\
\mbox{\tt \ \ \ \ \ \ \ \ \ \ \ \ \ \ \ \ \ \ \ \ \ \ \ \ \ \ mk\ =\ -\ k}
\indextt{magnitude}%
\indextt{phase}%
\eprogB\noindent\bprogB
\mbox{\tt phase\ (x:+y)\ \ \ \ \ =\ \ atan2\ y\ x}
\eprogB\noindent\bprogB
\mbox{\tt --\ Some\ standard\ functions}\\
\mbox{\tt }\\
\mbox{\tt fst\ \ \ \ \ \ \ \ \ \ \ \ \ \ \ \ \ \ \ \ \ ::\ (a,b)\ ->\ a}\\
\mbox{\tt fst\ (x,y)\ \ \ \ \ \ \ \ \ \ \ \ \ \ \ =\ \ x}
\indextt{fst}%
\eprogB\noindent\bprogB
\mbox{\tt snd\ \ \ \ \ \ \ \ \ \ \ \ \ \ \ \ \ \ \ \ \ ::\ (a,b)\ ->\ b}\\
\mbox{\tt snd\ (x,y)\ \ \ \ \ \ \ \ \ \ \ \ \ \ \ =\ \ y}
\indextt{snd}%
\eprogB\noindent\bprogB
\mbox{\tt (.)\ \ \ \ \ \ \ \ \ \ \ \ \ \ \ \ \ \ \ \ \ ::\ (b\ ->\ c)\ ->\ (a\ ->\ b)\ ->\ a\ ->\ c}\\
\mbox{\tt (f\ .\ g)\ x\ \ \ \ \ \ \ \ \ \ \ \ \ \ \ =\ \ f\ (g\ x)}
\index{.@{\ptt .}}%
\eprogB\noindent\bprogB
\mbox{\tt until\ \ \ \ \ \ \ \ \ \ \ \ \ \ \ \ \ \ \ ::\ (a\ ->\ Bool)\ ->\ (a\ ->\ a)\ ->\ a\ ->\ a}\\
\mbox{\tt until\ p\ f\ x\ |\ p\ x\ \ \ \ \ \ \ =\ \ x}\\
\mbox{\tt \ \ \ \ \ \ \ \ \ \ \ \ |\ otherwise\ =\ \ until\ p\ f\ (f\ x)}
\indextt{until}%
\eprogB\noindent\bprogB
\mbox{\tt error\ \ \ \ \ \ \ \ \ \ \ \ \ \ \ \ \ \ \ ::\ String\ ->\ a}\\
\mbox{\tt error\ msg\ |\ False\ \ \ \ \ \ \ =\ \ error\ msg}
\indextt{error}%
\eprogB\noindent\bprogB
\mbox{\tt asTypeOf\ \ \ \ \ \ \ \ \ \ \ \ \ \ \ \ ::\ a\ ->\ a\ ->\ a}\\
\mbox{\tt x\ `asTypeOf`\ {\char'137}\ \ \ \ \ \ \ \ \ \ =\ x}
\indextt{asTypeOf}%
\eprogB
\clearpage

\subsection{Prelude {\tt PreludeBuiltin}}
\label{preludebuiltin}
\noindent\bprogB
\mbox{\tt interface\ PreludeBuiltin\ \ where}
\index{PreludeBuiltin@{\ptt PreludeBuiltin} (module)}%
\eprogB\noindent\bprogB
\mbox{\tt infixr\ 5\ :}
\index{:@{\ptt :}}%
\eprogB\noindent\bprogB
\mbox{\tt --\ The\ following\ are\ algebraic\ types\ with\ special\ syntax.\ \ All\ of\ their}\\
\mbox{\tt --\ standard\ instances\ are\ derived\ here,\ except\ for\ class\ Text,\ for}\\
\mbox{\tt --\ which\ the\ special\ syntax\ must\ be\ taken\ into\ account.\ \ See\ PreludeText}\\
\mbox{\tt --\ for\ the\ Text\ instances\ of\ lists\ and\ the\ trivial\ type\ and\ a\ scheme}\\
\mbox{\tt --\ for\ Tuple\ Text\ instances.}\\
\mbox{\tt --}\\
\mbox{\tt --\ data\ [a]\ =\ []\ |\ a\ :\ [a]\ \ deriving\ (Eq,\ Ord,\ Binary)\ \ \ \ \ Lists}\\
\mbox{\tt --\ data\ ()\ =\ ()\ \ deriving\ (Eq,\ Ord,\ Ix,\ Enum,\ Binary)\ \ \ \ \ \ Trivial\ Type}\\
\mbox{\tt --\ data\ (a,b)\ =\ (a,b)\ \ deriving\ (Eq,\ Ord,\ Ix,\ Binary)\ \ \ \ \ \ Pairs}\\
\mbox{\tt --\ data\ (a,b,c)\ =\ (a,b,c)\ \ deriving\ (Eq,\ Ord,\ Ix,\ Binary)\ \ Triples}\\
\mbox{\tt --\ et\ cetera\ \ \ \ \ \ \ \ \ \ \ \ \ \ \ \ \ \ \ \ \ \ \ \ \ \ \ \ \ \ \ \ \ \ \ \ \ \ \ \ \ \ \ \ \ \ \ Other\ Tuples}\\
\mbox{\tt }\\[-8pt]
\mbox{\tt }\\[-8pt]
\mbox{\tt --\ The\ primitive\ types:}\\
\mbox{\tt }\\[-8pt]
\mbox{\tt data\ Char}\\
\mbox{\tt data\ Int}\\
\mbox{\tt data\ Integer}\\
\mbox{\tt data\ Float}\\
\mbox{\tt data\ Double}\\
\mbox{\tt data\ Bin}
\index{Char@{\ptt Char} (datatype)}%
\index{Int@{\ptt Int} (datatype)}%
\index{Integer@{\ptt Integer} (datatype)}%
\index{Float@{\ptt Float} (datatype)}%
\index{Double@{\ptt Double} (datatype)}%
\index{Bin@{\ptt Bin} (datatype)}%
\eprogB\noindent\bprogB
\mbox{\tt instance\ Binary\ Char}\\
\mbox{\tt instance\ Binary\ Int}\\
\mbox{\tt instance\ Binary\ Integer}\\
\mbox{\tt instance\ Binary\ Float}\\
\mbox{\tt instance\ Binary\ Double}
\eprogB\noindent\bprogB
\mbox{\tt primMinInt,\ primMaxInt\ \ \ \ \ \ \ \ \ \ ::\ Int}\\
\mbox{\tt primCharToInt\ \ \ \ \ \ \ \ \ \ \ \ \ \ \ \ \ \ \ ::\ Char\ ->\ Int}\\
\mbox{\tt primIntToChar\ \ \ \ \ \ \ \ \ \ \ \ \ \ \ \ \ \ \ ::\ Int\ ->\ Char}\\
\mbox{\tt primIntToInteger\ \ \ \ \ \ \ \ \ \ \ \ \ \ \ \ ::\ Int\ ->\ Integer}\\
\mbox{\tt primIntegerToInt\ \ \ \ \ \ \ \ \ \ \ \ \ \ \ \ ::\ Integer\ ->\ Int}
\eprogB\noindent\bprogB
\mbox{\tt primEqInt,\ primLeInt\ \ \ \ \ \ \ \ \ \ \ \ ::\ Int\ ->\ Int\ ->\ Bool}\\
\mbox{\tt primPlusInt,\ primMulInt\ \ \ \ \ \ \ \ \ ::\ Int\ ->\ Int\ ->\ Int}\\
\mbox{\tt primNegInt\ \ \ \ \ \ \ \ \ \ \ \ \ \ \ \ \ \ \ \ \ \ ::\ Int\ ->\ Int}\\
\mbox{\tt primDivRemInt\ \ \ \ \ \ \ \ \ \ \ \ \ \ \ \ \ \ \ ::\ Int\ ->\ Int\ ->\ (Int,Int)}
\eprogB\noindent\bprogB
\mbox{\tt primEqInteger,\ primLeInteger\ \ \ \ ::\ Integer\ ->\ Integer\ ->\ Bool}\\
\mbox{\tt primPlusInteger,\ primMulInteger\ ::\ Integer\ ->\ Integer\ ->\ Integer}\\
\mbox{\tt primNegInteger\ \ \ \ \ \ \ \ \ \ \ \ \ \ \ \ \ \ ::\ Integer\ ->\ Integer}\\
\mbox{\tt primDivRemInteger\ \ \ \ \ \ \ \ \ \ \ \ \ \ \ ::\ Integer\ ->\ Integer\ ->\ (Integer,Integer)}
\eprogB\noindent\bprogB
\mbox{\tt primFloatRadix\ \ \ \ \ \ \ \ \ \ \ \ \ \ \ \ \ \ \ \ \ \ \ \ \ \ ::\ Integer}\\
\mbox{\tt primFloatDigits,\ primFloatMinExp,}\\
\mbox{\tt \ \ \ \ primFloatMaxExp\ \ \ \ \ \ \ \ \ \ \ \ \ \ \ \ \ \ \ \ \ ::\ Int}\\
\mbox{\tt primDecodeFloat\ \ \ \ \ \ \ \ \ \ \ \ \ \ \ \ \ \ \ \ \ \ \ \ \ ::\ Float\ ->\ (Integer,Int)}\\
\mbox{\tt primEncodeFloat\ \ \ \ \ \ \ \ \ \ \ \ \ \ \ \ \ \ \ \ \ \ \ \ \ ::\ Integer\ ->\ Int\ ->\ Float}\\
\mbox{\tt primEqFloat,\ primLeFloat\ \ \ \ \ \ \ \ \ \ \ \ \ \ \ \ ::\ Float\ ->\ Float\ ->\ Bool}\\
\mbox{\tt primPlusFloat,\ primMulFloat,}\\
\mbox{\tt \ \ \ \ primDivFloat\ \ \ \ \ \ \ \ \ \ \ \ \ \ \ \ \ \ \ \ \ \ \ \ ::\ Float\ ->\ Float\ ->\ Float}\\
\mbox{\tt primNegFloat\ \ \ \ \ \ \ \ \ \ \ \ \ \ \ \ \ \ \ \ \ \ \ \ \ \ \ \ ::\ Float\ ->\ Float}
\eprogB\noindent\bprogB
\mbox{\tt primPiFloat\ \ \ \ \ \ \ \ \ \ \ \ \ \ \ \ \ \ \ \ \ \ \ \ \ \ \ \ \ ::\ Float}\\
\mbox{\tt primExpFloat,\ primLogFloat,}\\
\mbox{\tt \ \ \ \ primSqrtFloat,\ primSinFloat,}\\
\mbox{\tt \ \ \ \ primCosFloat,\ primTanFloat,}\\
\mbox{\tt \ \ \ \ primAsinFloat,\ primAcosFloat,}\\
\mbox{\tt \ \ \ \ primAtanFloat,\ primSinhFloat,}\\
\mbox{\tt \ \ \ \ primCoshFloat,\ primTanhFloat,}\\
\mbox{\tt \ \ \ \ primAsinhFloat,\ primAcoshFloat,}\\
\mbox{\tt \ \ \ \ primAtanhFloat\ \ \ \ \ \ \ \ \ \ \ \ \ \ \ \ \ \ \ \ \ \ ::\ Float\ ->\ Float}
\eprogB\noindent\bprogB
\mbox{\tt primDoubleRadix\ \ \ \ \ \ \ \ \ \ \ \ \ \ \ \ \ \ \ \ \ \ \ \ \ ::\ Integer}\\
\mbox{\tt primDoubleDigits,\ primDoubleMinExp,}\\
\mbox{\tt \ \ \ \ primDoubleMaxExp\ \ \ \ \ \ \ \ \ \ \ \ \ \ \ \ \ \ \ \ ::\ Int}\\
\mbox{\tt primDecodeDouble\ \ \ \ \ \ \ \ \ \ \ \ \ \ \ \ \ \ \ \ \ \ \ \ ::\ Double\ ->\ (Integer,Int)}\\
\mbox{\tt primEncodeDouble\ \ \ \ \ \ \ \ \ \ \ \ \ \ \ \ \ \ \ \ \ \ \ \ ::\ Integer\ ->\ Int\ ->\ Double}\\
\mbox{\tt primEqDouble,\ primLeDouble\ \ \ \ \ \ \ \ \ \ \ \ \ \ ::\ Double\ ->\ Double\ ->\ Bool}\\
\mbox{\tt primPlusDouble,\ primMulDouble,}\\
\mbox{\tt \ \ \ \ primDivDouble\ \ \ \ \ \ \ \ \ \ \ \ \ \ \ \ \ \ \ \ \ \ \ ::\ Double\ ->\ Double\ ->\ Double}\\
\mbox{\tt primNegDouble\ \ \ \ \ \ \ \ \ \ \ \ \ \ \ \ \ \ \ \ \ \ \ \ \ \ \ ::\ Double\ ->\ Double}\\
\mbox{\tt primPiDouble\ \ \ \ \ \ \ \ \ \ \ \ \ \ \ \ \ \ \ \ \ \ \ \ \ \ \ \ ::\ Double}\\
\mbox{\tt primExpDouble,\ primLogDouble,}\\
\mbox{\tt \ \ \ \ primSqrtDouble,\ primSinDouble,}\\
\mbox{\tt \ \ \ \ primCosDouble,\ primTanDouble,}\\
\mbox{\tt \ \ \ \ primAsinDouble,\ primAcosDouble,}\\
\mbox{\tt \ \ \ \ primAtanDouble,\ primSinhDouble,}\\
\mbox{\tt \ \ \ \ primCoshDouble,\ primTanhDouble,}\\
\mbox{\tt \ \ \ \ primAsinhDouble,\ primAcoshDouble,}\\
\mbox{\tt \ \ \ \ primAtanhDouble\ \ \ \ \ \ \ \ \ \ \ \ \ \ \ \ \ \ \ \ \ ::\ Double\ ->\ Double}
\eprogB\noindent\bprogB
\mbox{\tt primNullBin\ \ \ \ \ \ \ \ \ \ \ \ \ \ \ \ \ \ \ \ \ \ \ \ \ \ \ \ \ ::\ Bin}\\
\mbox{\tt primIsNullBin\ \ \ \ \ \ \ \ \ \ \ \ \ \ \ \ \ \ \ \ \ \ \ \ \ \ \ ::\ Bin\ ->\ Bool}\\
\mbox{\tt primAppendBin\ \ \ \ \ \ \ \ \ \ \ \ \ \ \ \ \ \ \ \ \ \ \ \ \ \ \ ::\ Bin\ ->\ Bin\ ->\ Bin}
\eprogB
\clearpage

\subsection{Prelude {\tt PreludeCore}}
\label{preludecore}
% The index entries for :, [], (), and tuples are here
% it just so HAPPENS that they'll end up referring to the right page
% HHAACCKK!!
\index{[t]@\mbox{$\it \makebox{\tt [}t\makebox{\tt ]}$} (list type)}%
\index{[]@{\ptt []} (nil)}%
\index{(t1,...,tn)@\mbox{$\it \makebox{\tt (}t_1,\ldots ,t_n\makebox{\tt )}$} (tuple type)}
\noindent\bprogB
\mbox{\tt --\ Standard\ types,\ classes,\ and\ instances}\\
\mbox{\tt }\\
\mbox{\tt module\ PreludeCore\ (}\\
\mbox{\tt \ \ \ \ Eq((=),\ (/=)),}\\
\mbox{\tt \ \ \ \ Ord((<),\ (<=),\ (>=),\ (>),\ max,\ min),}\\
\mbox{\tt \ \ \ \ Num((+),\ (-),\ (*),\ negate,\ abs,\ signum,\ fromInteger),}\\
\mbox{\tt \ \ \ \ Integral(divRem,\ div,\ rem,\ mod,\ even,\ odd,\ toInteger),}\\
\mbox{\tt \ \ \ \ Fractional((/),\ fromRational),}\\
\mbox{\tt \ \ \ \ Floating(pi,\ exp,\ log,\ sqrt,\ (**),\ logBase,}\\
\mbox{\tt \ \ \ \ \ \ \ \ \ \ \ \ \ sin,\ cos,\ tan,\ asin,\ acos,\ atan,}\\
\mbox{\tt \ \ \ \ \ \ \ \ \ \ \ \ \ sinh,\ cosh,\ tanh,\ asinh,\ acosh,\ atanh),}\\
\mbox{\tt \ \ \ \ Real(toRational),}\\
\mbox{\tt \ \ \ \ RealFrac(properFraction,\ approxRational),}\\
\mbox{\tt \ \ \ \ RealFloat(floatRadix,\ floatDigits,\ floatRange,}\\
\mbox{\tt \ \ \ \ \ \ \ \ \ \ \ \ \ \ encodeFloat,\ decodeFloat,\ exponent,\ significand,\ scaleFloat),}\\
\mbox{\tt \ \ \ \ Ix(range,\ index,\ inRange),}\\
\mbox{\tt \ \ \ \ Enum(enumFrom,\ enumFromThen,\ enumFromTo,\ enumFromThenTo),}\\
\mbox{\tt \ \ \ \ Text(readsPrec,\ showsPrec,\ readList,\ showList),}\\
\mbox{\tt \ \ \ \ Binary(readBin,\ showBin),}\\
\mbox{\tt --\ \ List\ type:\ [{\char'137}]((:),\ [])}\\
\mbox{\tt --\ \ Tuple\ types:\ ({\char'137},{\char'137}),\ ({\char'137},{\char'137},{\char'137}),\ etc.}\\
\mbox{\tt --\ \ Trivial\ type:\ ()\ }\\
\mbox{\tt \ \ \ \ Bool(True,\ False),}\\
\mbox{\tt \ \ \ \ Char,\ Int,\ Integer,\ Float,\ Double,\ Bin,}\\
\mbox{\tt \ \ \ \ Ratio,\ Complex((:+)),\ Assoc((:=)),\ Array,}\\
\mbox{\tt \ \ \ \ String,\ Rational\ )\ \ where}
\eprogB\noindent\bprogB
\mbox{\tt import\ PreludeBuiltin}\\
\mbox{\tt import\ PreludeText(Text(readsPrec,\ showsPrec,\ readList,\ showList))}\\
\mbox{\tt import\ PreludeRatio(Ratio,\ Rational)}\\
\mbox{\tt import\ PreludeComplex}\\
\mbox{\tt import\ PreludeArray(Assoc(:=),\ Array)}\\
\mbox{\tt import\ PreludeIO(Name,\ Request,\ Response,\ IOError,}\\
\mbox{\tt \ \ \ \ \ \ \ \ \ \ \ \ \ \ \ \ \ Dialogue,\ SuccCont,\ StrCont,\ BinCont,\ FailCont)}
\eprogB\noindent\bprogB
\mbox{\tt infixr\ 8\ \ **}\\
\mbox{\tt infixl\ 7\ \ *}\\
\mbox{\tt infix\ \ 7\ \ /,\ `div`,\ `rem`,\ `mod`}\\
\mbox{\tt infixl\ 6\ \ +,\ -}\\
\mbox{\tt infixr\ 3\ \ :}\\
\mbox{\tt infix\ \ 2\ \ ==,\ /=,\ <,\ <=,\ >=,\ >}
\eprogB\noindent\bprogB
\mbox{\tt --\ Equality\ and\ Ordered\ classes}\\
\mbox{\tt }\\
\mbox{\tt class\ \ Eq\ a\ \ where}\\
\mbox{\tt \ \ \ \ (==),\ (/=)\ \ \ \ \ \ \ \ \ \ ::\ a\ ->\ a\ ->\ Bool}\\
\mbox{\tt }\\
\mbox{\tt \ \ \ \ x\ /=\ y\ \ \ \ \ \ \ \ \ \ \ \ \ \ =\ \ not\ (x\ ==\ y)}
\eprogB\noindent\bprogB
\mbox{\tt class\ \ (Eq\ a)\ =>\ Ord\ a\ \ where}\\
\mbox{\tt \ \ \ \ (<),\ (<=),\ (>=),\ (>)::\ a\ ->\ a\ ->\ Bool}\\
\mbox{\tt \ \ \ \ max,\ min\ \ \ \ \ \ \ \ \ \ \ \ ::\ a\ ->\ a\ ->\ Bool}\\
\mbox{\tt }\\
\mbox{\tt \ \ \ \ x\ <\ \ y\ \ \ \ \ \ \ \ \ \ \ \ \ \ =\ \ x\ <=\ y\ {\char'46}{\char'46}\ x\ /=\ y}\\
\mbox{\tt \ \ \ \ x\ >=\ y\ \ \ \ \ \ \ \ \ \ \ \ \ \ =\ \ y\ <=\ x}\\
\mbox{\tt \ \ \ \ x\ >\ \ y\ \ \ \ \ \ \ \ \ \ \ \ \ \ =\ \ y\ <\ \ x}\\
\mbox{\tt \ \ \ \ max\ x\ y\ |\ x\ >=\ y\ \ \ \ =\ \ x}\\
\mbox{\tt \ \ \ \ \ \ \ \ \ \ \ \ |\ y\ >=\ x\ \ \ \ =\ \ y}\\
\mbox{\tt \ \ \ \ min\ x\ y\ |\ x\ <=\ y\ \ \ \ =\ \ x}\\
\mbox{\tt \ \ \ \ \ \ \ \ \ \ \ \ |\ y\ <=\ x\ \ \ \ =\ \ y}
\eprogB\noindent\bprogB
\mbox{\tt --\ Numeric\ classes}\\
\mbox{\tt }\\
\mbox{\tt class\ \ (Eq\ a)\ =>\ Num\ a\ \ where}\\
\mbox{\tt \ \ \ \ (+),\ (-),\ (*)\ \ \ \ \ \ \ ::\ a\ ->\ a\ ->\ a}\\
\mbox{\tt \ \ \ \ negate\ \ \ \ \ \ \ \ \ \ \ \ \ \ ::\ a\ ->\ a}\\
\mbox{\tt \ \ \ \ abs,\ signum\ \ \ \ \ \ \ \ \ ::\ a\ ->\ a}\\
\mbox{\tt \ \ \ \ fromInteger\ \ \ \ \ \ \ \ \ ::\ Integer\ ->\ a}\\
\mbox{\tt }\\
\mbox{\tt \ \ \ \ x\ -\ y\ \ \ \ \ \ \ \ \ \ \ \ \ \ \ =\ \ x\ +\ negate\ y}
\eprogB\noindent\bprogB
\mbox{\tt class\ \ (Num\ a,\ Ord\ a)\ =>\ Real\ a\ where}\\
\mbox{\tt \ \ \ \ toRational\ \ \ \ \ \ \ \ \ \ ::\ \ a\ ->\ Rational}
\eprogB\noindent\bprogB
\mbox{\tt class\ \ (Real\ a)\ =>\ Integral\ a\ \ where}\\
\mbox{\tt \ \ \ \ div,\ rem,\ mod\ \ \ \ \ \ \ ::\ a\ ->\ a\ ->\ a}\\
\mbox{\tt \ \ \ \ divRem\ \ \ \ \ \ \ \ \ \ \ \ \ \ ::\ a\ ->\ a\ ->\ (a,a)}\\
\mbox{\tt \ \ \ \ even,\ odd\ \ \ \ \ \ \ \ \ \ \ ::\ a\ ->\ Bool}\\
\mbox{\tt \ \ \ \ toInteger\ \ \ \ \ \ \ \ \ \ \ ::\ a\ ->\ Integer}\\
\mbox{\tt }\\
\mbox{\tt \ \ \ \ x\ `div`\ y\ \ \ \ \ \ \ \ \ \ \ =\ \ q\ \ where\ (q,r)\ =\ divRem\ x\ y}\\
\mbox{\tt \ \ \ \ x\ `rem`\ y\ \ \ \ \ \ \ \ \ \ \ =\ \ r\ \ where\ (q,r)\ =\ divRem\ x\ y}\\
\mbox{\tt \ \ \ \ x\ `mod`\ y\ \ \ \ \ \ \ \ \ \ \ =\ \ if\ signum\ x\ ==\ -\ (signum\ y)\ then\ r\ +\ y\ else\ r}\\
\mbox{\tt \ \ \ \ \ \ \ \ \ \ \ \ \ \ \ \ \ \ \ \ \ \ \ \ \ \ \ where\ r\ =\ x\ `rem`\ y}\\
\mbox{\tt \ \ \ \ even\ x\ \ \ \ \ \ \ \ \ \ \ \ \ \ =\ \ x\ `rem`\ 2\ ==\ 0}\\
\mbox{\tt \ \ \ \ odd\ \ \ \ \ \ \ \ \ \ \ \ \ \ \ \ \ =\ \ not\ .\ even}
\eprogB\noindent\bprogB
\mbox{\tt class\ \ (Num\ a)\ =>\ Fractional\ a\ \ where}\\
\mbox{\tt \ \ \ \ (/)\ \ \ \ \ \ \ \ \ \ \ \ \ \ \ \ \ ::\ a\ ->\ a\ ->\ a}\\
\mbox{\tt \ \ \ \ fromRational\ \ \ \ \ \ \ \ ::\ Rational\ ->\ a}
\eprogB\noindent\bprogB
\mbox{\tt class\ \ (Fractional\ a)\ =>\ Floating\ a\ \ where}\\
\mbox{\tt \ \ \ \ pi\ \ \ \ \ \ \ \ \ \ \ \ \ \ \ \ \ \ ::\ a}\\
\mbox{\tt \ \ \ \ exp,\ log,\ sqrt\ \ \ \ \ \ ::\ a\ ->\ a}\\
\mbox{\tt \ \ \ \ (**),\ logBase\ \ \ \ \ \ \ ::\ a\ ->\ a\ ->\ a}\\
\mbox{\tt \ \ \ \ sin,\ cos,\ tan\ \ \ \ \ \ \ ::\ a\ ->\ a}\\
\mbox{\tt \ \ \ \ asin,\ acos,\ atan\ \ \ \ ::\ a\ ->\ a}\\
\mbox{\tt \ \ \ \ sinh,\ cosh,\ tanh\ \ \ \ ::\ a\ ->\ a}\\
\mbox{\tt \ \ \ \ asinh,\ acosh,\ atanh\ ::\ a\ ->\ a}\\
\mbox{\tt }\\
\mbox{\tt \ \ \ \ x\ **\ y\ \ \ \ \ \ \ \ \ \ \ \ \ \ =\ \ exp\ (log\ x\ *\ y)}\\
\mbox{\tt \ \ \ \ logBase\ x\ y\ \ \ \ \ \ \ \ \ =\ \ log\ y\ /\ log\ x}\\
\mbox{\tt \ \ \ \ sqrt\ x\ \ \ \ \ \ \ \ \ \ \ \ \ \ =\ \ x\ **\ 0.5}\\
\mbox{\tt \ \ \ \ tan\ \ x\ \ \ \ \ \ \ \ \ \ \ \ \ \ =\ \ sin\ \ x\ /\ cos\ \ x}\\
\mbox{\tt \ \ \ \ tanh\ x\ \ \ \ \ \ \ \ \ \ \ \ \ \ =\ \ sinh\ x\ /\ cosh\ x}
\eprogB\noindent\bprogB
\mbox{\tt class\ \ (Real\ a,\ Fractional\ a)\ =>\ RealFrac\ a\ \ where}\\
\mbox{\tt \ \ \ \ properFraction\ \ \ \ \ \ ::\ a\ ->\ (Integer,a)}\\
\mbox{\tt \ \ \ \ approxRational\ \ \ \ \ \ ::\ a\ ->\ a\ ->\ Rational}
\eprogB\noindent\bprogB
\mbox{\tt class\ \ (RealFrac\ a,\ Floating\ a)\ =>\ RealFloat\ a\ \ where}\\
\mbox{\tt \ \ \ \ floatRadix\ \ \ \ \ \ \ \ \ \ ::\ a\ ->\ Integer}\\
\mbox{\tt \ \ \ \ floatDigits\ \ \ \ \ \ \ \ \ ::\ a\ ->\ Int}\\
\mbox{\tt \ \ \ \ floatRange\ \ \ \ \ \ \ \ \ \ ::\ a\ ->\ (Int,Int)}\\
\mbox{\tt \ \ \ \ decodeFloat\ \ \ \ \ \ \ \ \ ::\ a\ ->\ (Integer,Int)}\\
\mbox{\tt \ \ \ \ encodeFloat\ \ \ \ \ \ \ \ \ ::\ Integer\ ->\ Int\ ->\ a}\\
\mbox{\tt \ \ \ \ exponent\ \ \ \ \ \ \ \ \ \ \ \ ::\ a\ ->\ Int}\\
\mbox{\tt \ \ \ \ significand\ \ \ \ \ \ \ \ \ ::\ a\ ->\ a}\\
\mbox{\tt \ \ \ \ scaleFloat\ \ \ \ \ \ \ \ \ \ ::\ Int\ ->\ a\ ->\ a}\\
\mbox{\tt }\\
\mbox{\tt \ \ \ \ exponent\ x\ \ \ \ \ \ \ \ \ \ =\ \ if\ m\ ==\ 0\ then\ 0\ else\ n\ +\ floatDigits\ x}\\
\mbox{\tt \ \ \ \ \ \ \ \ \ \ \ \ \ \ \ \ \ \ \ \ \ \ \ \ \ \ \ where\ (m,n)\ =\ decodeFloat\ x}\\
\mbox{\tt }\\
\mbox{\tt \ \ \ \ significand\ x\ \ \ \ \ \ \ =\ \ encodeFloat\ m\ (-\ (floatDigits\ x))}\\
\mbox{\tt \ \ \ \ \ \ \ \ \ \ \ \ \ \ \ \ \ \ \ \ \ \ \ \ \ \ \ where\ (m,{\char'137})\ =\ decodeFloat\ x}\\
\mbox{\tt }\\
\mbox{\tt \ \ \ \ scaleFloat\ k\ x\ \ \ \ \ \ =\ \ encodeFloat\ m\ (n+k)}\\
\mbox{\tt \ \ \ \ \ \ \ \ \ \ \ \ \ \ \ \ \ \ \ \ \ \ \ \ \ \ \ where\ (m,n)\ =\ decodeFloat\ x}
\eprogB\noindent\bprogB
\mbox{\tt --\ Index\ and\ Enumeration\ classes}\\
\mbox{\tt }\\
\mbox{\tt class\ \ (Ord\ a)\ =>\ Ix\ a\ \ where}\\
\mbox{\tt \ \ \ \ range\ \ \ \ \ \ \ \ \ \ \ \ \ \ \ ::\ (a,a)\ ->\ [a]}\\
\mbox{\tt \ \ \ \ index\ \ \ \ \ \ \ \ \ \ \ \ \ \ \ ::\ (a,a)\ ->\ a\ ->\ Int}\\
\mbox{\tt \ \ \ \ inRange\ \ \ \ \ \ \ \ \ \ \ \ \ ::\ (a,a)\ ->\ a\ ->\ Bool}
\eprogB\noindent\bprogB
\mbox{\tt class\ \ (Ix\ a)\ =>\ Enum\ a\ where}\\
\mbox{\tt \ \ \ \ enumFrom\ \ \ \ \ \ \ \ \ \ \ \ ::\ a\ ->\ [a]\ \ \ \ \ \ \ \ \ \ \ \ \ --\ [n..]}\\
\mbox{\tt \ \ \ \ enumFromThen\ \ \ \ \ \ \ \ ::\ a\ ->\ a\ ->\ [a]\ \ \ \ \ \ \ \ --\ [n,n'..]}\\
\mbox{\tt \ \ \ \ enumFromTo\ \ \ \ \ \ \ \ \ \ ::\ a\ ->\ a\ ->\ [a]\ \ \ \ \ \ \ \ --\ [n..m]}\\
\mbox{\tt \ \ \ \ enumFromThenTo\ \ \ \ \ \ ::\ a\ ->\ a\ ->\ a\ ->\ [a]\ \ \ --\ [n,n'..m]}\\
\mbox{\tt }\\
\mbox{\tt \ \ \ \ enumFromTo\ n\ m\ \ \ \ \ \ =\ \ takeWhile\ ((>=)\ m)\ (enumFrom\ n)}\\
\mbox{\tt \ \ \ \ enumFromThenTo\ n\ n'\ m}\\
\mbox{\tt \ \ \ \ \ \ \ \ \ \ \ \ \ \ \ \ \ \ \ \ \ \ \ \ =\ \ takeWhile\ ((if\ n'\ >=\ n\ then\ (>=)\ else\ (<=))\ m)}\\
\mbox{\tt \ \ \ \ \ \ \ \ \ \ \ \ \ \ \ \ \ \ \ \ \ \ \ \ \ \ \ \ \ \ \ \ \ \ \ \ \ (enumFromThen\ n\ n')}
\eprogB\noindent\bprogB
\mbox{\tt --\ Binary\ class}\\
\mbox{\tt }\\
\mbox{\tt class\ \ Binary\ a\ \ where}\\
\mbox{\tt \ \ \ \ readBin\ \ \ \ \ \ \ \ \ \ \ \ \ ::\ Bin\ ->\ (a,Bin)}\\
\mbox{\tt \ \ \ \ showBin\ \ \ \ \ \ \ \ \ \ \ \ \ ::\ a\ ->\ Bin\ ->\ Bin}
\eprogB\noindent\bprogB
\mbox{\tt --\ Boolean\ type}\\
\mbox{\tt }\\
\mbox{\tt data\ \ Bool\ \ =\ \ False\ |\ True}
\eprogB\noindent\bprogB
\mbox{\tt --\ Character\ type}\\
\mbox{\tt }\\
\mbox{\tt instance\ \ Eq\ Char\ \ where}\\
\mbox{\tt \ \ \ \ c\ ==\ c'\ \ \ \ \ \ \ \ \ \ \ \ \ =\ \ ord\ c\ ==\ \ ord\ c'}
\eprogB\noindent\bprogB
\mbox{\tt instance\ \ Ord\ Char\ \ where}\\
\mbox{\tt \ \ \ \ c\ <=\ c'\ \ \ \ \ \ \ \ \ \ \ \ \ =\ \ ord\ c\ <=\ ord\ c'}
\eprogB\noindent\bprogB
\mbox{\tt instance\ \ Ix\ Char\ \ where}\\
\mbox{\tt \ \ \ \ range\ (c,c')\ \ \ \ \ \ \ \ =\ \ [c..c']}\\
\mbox{\tt \ \ \ \ index\ (c,c')\ ci\ \ \ \ \ =\ \ ord\ ci\ -\ ord\ c}\\
\mbox{\tt \ \ \ \ inRange\ (c,c')\ ci\ \ \ =\ \ ord\ c\ <=\ i\ {\char'46}{\char'46}\ i\ <=\ ord\ c'}\\
\mbox{\tt \ \ \ \ \ \ \ \ \ \ \ \ \ \ \ \ \ \ \ \ \ \ \ \ \ \ \ where\ i\ =\ ord\ ci}
\eprogB\noindent\bprogB
\mbox{\tt instance\ \ Enum\ Char\ \ where}\\
\mbox{\tt \ \ \ \ enumFrom\ c\ \ \ \ \ \ \ \ \ \ =\ \ map\ chr\ [ord\ c\ ..]}\\
\mbox{\tt \ \ \ \ enumFromThen\ c\ c'\ \ \ =\ \ map\ chr\ [ord\ c,\ ord\ c'\ ..]}
\eprogB\noindent\bprogB
\mbox{\tt type\ \ String\ =\ [Char]}
\eprogB\noindent\bprogB
\mbox{\tt --\ Standard\ Integral\ types}\\
\mbox{\tt }\\
\mbox{\tt instance\ \ Eq\ Int\ \ where}\\
\mbox{\tt \ \ \ \ (==)\ \ \ \ \ \ \ \ \ \ \ \ \ \ \ \ =\ \ primEqInt}
\eprogB\noindent\bprogB
\mbox{\tt instance\ \ Eq\ Integer\ \ where}\\
\mbox{\tt \ \ \ \ (==)\ \ \ \ \ \ \ \ \ \ \ \ \ \ \ \ =\ \ primEqInteger}
\eprogB\noindent\bprogB
\mbox{\tt instance\ \ Ord\ Int\ \ where}\\
\mbox{\tt \ \ \ \ (<=)\ \ \ \ \ \ \ \ \ \ \ \ \ \ \ \ =\ \ primLeInt}
\eprogB\noindent\bprogB
\mbox{\tt instance\ \ Ord\ Integer\ \ where}\\
\mbox{\tt \ \ \ \ (<=)\ \ \ \ \ \ \ \ \ \ \ \ \ \ \ \ =\ \ primLeInteger}
\eprogB\noindent\bprogB
\mbox{\tt instance\ \ Num\ Int\ \ where}\\
\mbox{\tt \ \ \ \ (+)\ \ \ \ \ \ \ \ \ \ \ \ \ \ \ \ \ =\ \ primPlusInt}\\
\mbox{\tt \ \ \ \ negate\ \ \ \ \ \ \ \ \ \ \ \ \ \ =\ \ primNegInt}\\
\mbox{\tt \ \ \ \ (*)\ \ \ \ \ \ \ \ \ \ \ \ \ \ \ \ \ =\ \ primMulInt}\\
\mbox{\tt \ \ \ \ abs\ \ \ \ \ \ \ \ \ \ \ \ \ \ \ \ \ =\ \ absReal}\\
\mbox{\tt \ \ \ \ signum\ \ \ \ \ \ \ \ \ \ \ \ \ \ =\ \ signumReal}\\
\mbox{\tt \ \ \ \ fromInteger\ \ \ \ \ \ \ \ \ =\ \ primIntegerToInt}
\eprogB\noindent\bprogB
\mbox{\tt instance\ \ Num\ Integer\ \ where}\\
\mbox{\tt \ \ \ \ (+)\ \ \ \ \ \ \ \ \ \ \ \ \ \ \ \ \ =\ \ primPlusInteger}\\
\mbox{\tt \ \ \ \ negate\ \ \ \ \ \ \ \ \ \ \ \ \ \ =\ \ primNegInteger}\\
\mbox{\tt \ \ \ \ (*)\ \ \ \ \ \ \ \ \ \ \ \ \ \ \ \ \ =\ \ primMulInteger}\\
\mbox{\tt \ \ \ \ abs\ \ \ \ \ \ \ \ \ \ \ \ \ \ \ \ \ =\ \ absReal}\\
\mbox{\tt \ \ \ \ signum\ \ \ \ \ \ \ \ \ \ \ \ \ \ =\ \ signumReal}\\
\mbox{\tt \ \ \ \ fromInteger\ x\ \ \ \ \ \ \ =\ \ x}
\eprogB\noindent\bprogB
\mbox{\tt absReal\ x\ \ \ \ |\ x\ >=\ 0\ \ \ \ =\ \ x}\\
\mbox{\tt \ \ \ \ \ \ \ \ \ \ \ \ \ |\ otherwise\ =\ \ -\ x}
\eprogB\noindent\bprogB
\mbox{\tt signumReal\ x\ |\ x\ ==\ 0\ \ \ \ =\ \ 0}\\
\mbox{\tt \ \ \ \ \ \ \ \ \ \ \ \ \ |\ x\ >\ 0\ \ \ \ \ =\ \ 1}\\
\mbox{\tt \ \ \ \ \ \ \ \ \ \ \ \ \ |\ otherwise\ =\ -1}
\eprogB\noindent\bprogB
\mbox{\tt instance\ \ Real\ Int\ \ where}\\
\mbox{\tt \ \ \ \ toRational\ x\ \ \ \ \ \ \ \ =\ \ toInteger\ x\ {\char'45}\ 1}
\eprogB\noindent\bprogB
\mbox{\tt instance\ \ Real\ Integer\ \ where}\\
\mbox{\tt \ \ \ \ toRational\ x\ \ \ \ \ \ \ \ =\ \ x\ {\char'45}\ 1}
\eprogB\noindent\bprogB
\mbox{\tt instance\ \ Integral\ Int\ \ where}\\
\mbox{\tt \ \ \ \ divRem\ \ \ \ \ \ \ \ \ \ \ \ \ \ =\ \ primDivRemInt}\\
\mbox{\tt \ \ \ \ toInteger\ \ \ \ \ \ \ \ \ \ \ =\ \ primIntToInteger}
\eprogB\noindent\bprogB
\mbox{\tt instance\ \ Integral\ Integer\ \ where}\\
\mbox{\tt \ \ \ \ divRem\ \ \ \ \ \ \ \ \ \ \ \ \ \ =\ \ primDivRemInteger}\\
\mbox{\tt \ \ \ \ toInteger\ x\ \ \ \ \ \ \ \ \ =\ \ x}
\eprogB\noindent\bprogB
\mbox{\tt instance\ \ Ix\ Int\ \ where}\\
\mbox{\tt \ \ \ \ range\ (m,n)\ \ \ \ \ \ \ \ \ =\ \ [m..n]}\\
\mbox{\tt \ \ \ \ index\ (m,n)\ i\ \ \ \ \ \ \ =\ \ i\ -\ m}\\
\mbox{\tt \ \ \ \ inRange\ (m,n)\ i\ \ \ \ \ =\ \ m\ <=\ i\ {\char'46}{\char'46}\ i\ <=\ n}
\eprogB\noindent\bprogB
\mbox{\tt instance\ \ Ix\ Integer\ \ where}\\
\mbox{\tt \ \ \ \ range\ (m,n)\ \ \ \ \ \ \ \ \ =\ \ [m..n]}\\
\mbox{\tt \ \ \ \ index\ (m,n)\ i\ \ \ \ \ \ \ =\ \ fromInteger\ (i\ -\ m)}\\
\mbox{\tt \ \ \ \ inRange\ (m,n)\ i\ \ \ \ \ =\ \ m\ <=\ i\ {\char'46}{\char'46}\ i\ <=\ n}
\eprogB\noindent\bprogB
\mbox{\tt instance\ \ Enum\ Int\ \ where}\\
\mbox{\tt \ \ \ \ enumFrom\ n\ \ \ \ \ \ \ \ \ \ =\ \ enumFromBy\ n\ 1}\\
\mbox{\tt \ \ \ \ enumFromThen\ n\ m\ \ \ \ =\ \ enumFromBy\ n\ (m\ -\ n)}
\eprogB\noindent\bprogB
\mbox{\tt instance\ \ Enum\ Integer\ \ where}\\
\mbox{\tt \ \ \ \ enumFrom\ n\ \ \ \ \ \ \ \ \ \ =\ \ enumFromBy\ n\ 1}\\
\mbox{\tt \ \ \ \ enumFromThen\ n\ m\ \ \ \ =\ \ enumFromBy\ n\ (m\ -\ n)}
\eprogB\noindent\bprogB
\mbox{\tt enumFromBy\ n\ k\ \ \ \ \ \ \ \ \ \ =\ \ n\ :\ enumFromBy\ (n+k)\ k}
\eprogB\noindent\bprogB
\mbox{\tt --\ Standard\ Floating\ types}\\
\mbox{\tt }\\
\mbox{\tt instance\ \ Eq\ Float\ \ where}\\
\mbox{\tt \ \ \ \ (==)\ \ \ \ \ \ \ \ \ \ \ \ \ \ \ \ =\ \ primEqFloat}
\eprogB\noindent\bprogB
\mbox{\tt instance\ \ Eq\ Double\ \ where}\\
\mbox{\tt \ \ \ \ (==)\ \ \ \ \ \ \ \ \ \ \ \ \ \ \ \ =\ \ primEqDouble}
\eprogB\noindent\bprogB
\mbox{\tt instance\ \ Ord\ Float\ \ where}\\
\mbox{\tt \ \ \ \ (<=)\ \ \ \ \ \ \ \ \ \ \ \ \ \ \ \ =\ \ primLeFloat}
\eprogB\noindent\bprogB
\mbox{\tt instance\ \ Ord\ Double\ \ where}\\
\mbox{\tt \ \ \ \ (<=)\ \ \ \ \ \ \ \ \ \ \ \ \ \ \ \ =\ \ primLeDouble}
\eprogB\noindent\bprogB
\mbox{\tt instance\ \ Num\ Float\ \ where}\\
\mbox{\tt \ \ \ \ (+)\ \ \ \ \ \ \ \ \ \ \ \ \ \ \ \ \ =\ \ primPlusFloat}\\
\mbox{\tt \ \ \ \ negate\ \ \ \ \ \ \ \ \ \ \ \ \ \ =\ \ primNegFloat}\\
\mbox{\tt \ \ \ \ (*)\ \ \ \ \ \ \ \ \ \ \ \ \ \ \ \ \ =\ \ primMulFloat}\\
\mbox{\tt \ \ \ \ abs\ \ \ \ \ \ \ \ \ \ \ \ \ \ \ \ \ =\ \ absReal}\\
\mbox{\tt \ \ \ \ signum\ \ \ \ \ \ \ \ \ \ \ \ \ \ =\ \ signumReal}\\
\mbox{\tt \ \ \ \ fromInteger\ n\ \ \ \ \ \ \ =\ \ encodeFloat\ n\ 0}
\eprogB\noindent\bprogB
\mbox{\tt instance\ \ Num\ Double\ \ where}\\
\mbox{\tt \ \ \ \ (+)\ \ \ \ \ \ \ \ \ \ \ \ \ \ \ \ \ =\ \ primPlusDouble}\\
\mbox{\tt \ \ \ \ negate\ \ \ \ \ \ \ \ \ \ \ \ \ \ =\ \ primNegDouble}\\
\mbox{\tt \ \ \ \ (*)\ \ \ \ \ \ \ \ \ \ \ \ \ \ \ \ \ =\ \ primMulDouble}\\
\mbox{\tt \ \ \ \ abs\ \ \ \ \ \ \ \ \ \ \ \ \ \ \ \ \ =\ \ absReal}\\
\mbox{\tt \ \ \ \ signum\ \ \ \ \ \ \ \ \ \ \ \ \ \ =\ \ signumReal}\\
\mbox{\tt \ \ \ \ fromInteger\ n\ \ \ \ \ \ \ =\ \ encodeFloat\ n\ 0}
\eprogB\noindent\bprogB
\mbox{\tt instance\ \ Real\ Float\ \ where}\\
\mbox{\tt \ \ \ \ toRational\ \ \ \ \ \ \ \ \ \ =\ \ floatingToRational}
\eprogB\noindent\bprogB
\mbox{\tt instance\ \ Real\ Double\ \ where}\\
\mbox{\tt \ \ \ \ toRational\ \ \ \ \ \ \ \ \ \ =\ \ floatingToRational}
\eprogB\noindent\bprogB
\mbox{\tt floatingToRational\ x\ \ \ \ =\ \ (m{\char'45}1)*(b{\char'45}1){\char'136}{\char'136}n}\\
\mbox{\tt \ \ \ \ \ \ \ \ \ \ \ \ \ \ \ \ \ \ \ \ \ \ \ \ \ \ \ where\ (m,n)\ =\ decodeFloat\ x}\\
\mbox{\tt \ \ \ \ \ \ \ \ \ \ \ \ \ \ \ \ \ \ \ \ \ \ \ \ \ \ \ \ \ \ \ \ \ b\ \ \ \ \ =\ floatRadix\ \ x}
\eprogB\noindent\bprogB
\mbox{\tt instance\ \ Fractional\ Float\ \ where}\\
\mbox{\tt \ \ \ \ (/)\ \ \ \ \ \ \ \ \ \ \ \ \ \ \ \ \ =\ \ primDivFloat}\\
\mbox{\tt \ \ \ \ fromRational\ \ \ \ \ \ \ \ =\ \ rationalToFloating}
\eprogB\noindent\bprogB
\mbox{\tt instance\ \ Fractional\ Double\ \ where}\\
\mbox{\tt \ \ \ \ (/)\ \ \ \ \ \ \ \ \ \ \ \ \ \ \ \ \ =\ \ primDivDouble}\\
\mbox{\tt \ \ \ \ fromRational\ \ \ \ \ \ \ \ =\ \ rationalToFloating}
\eprogB\noindent\bprogB
\mbox{\tt rationalToFloating\ x\ \ \ \ =\ \ fromInteger\ (numerator\ x)}\\
\mbox{\tt \ \ \ \ \ \ \ \ \ \ \ \ \ \ \ \ \ \ \ \ \ \ \ \ \ \ \ \ \ \ \ \ /\ fromInteger\ (denominator\ x)}
\eprogB\noindent\bprogB
\mbox{\tt instance\ \ Floating\ Float\ \ where}\\
\mbox{\tt \ \ \ \ pi\ \ \ \ \ \ \ \ \ \ \ \ \ \ \ \ \ \ =\ \ primPiFloat}\\
\mbox{\tt \ \ \ \ exp\ \ \ \ \ \ \ \ \ \ \ \ \ \ \ \ \ =\ \ primExpFloat}\\
\mbox{\tt \ \ \ \ log\ \ \ \ \ \ \ \ \ \ \ \ \ \ \ \ \ =\ \ primLogFloat}\\
\mbox{\tt \ \ \ \ sqrt\ \ \ \ \ \ \ \ \ \ \ \ \ \ \ \ =\ \ primSqrtFloat}\\
\mbox{\tt \ \ \ \ sin\ \ \ \ \ \ \ \ \ \ \ \ \ \ \ \ \ =\ \ primSinFloat}\\
\mbox{\tt \ \ \ \ cos\ \ \ \ \ \ \ \ \ \ \ \ \ \ \ \ \ =\ \ primCosFloat}\\
\mbox{\tt \ \ \ \ tan\ \ \ \ \ \ \ \ \ \ \ \ \ \ \ \ \ =\ \ primTanFloat}\\
\mbox{\tt \ \ \ \ asin\ \ \ \ \ \ \ \ \ \ \ \ \ \ \ \ =\ \ primAsinFloat}\\
\mbox{\tt \ \ \ \ acos\ \ \ \ \ \ \ \ \ \ \ \ \ \ \ \ =\ \ primAcosFloat}\\
\mbox{\tt \ \ \ \ atan\ \ \ \ \ \ \ \ \ \ \ \ \ \ \ \ =\ \ primAtanFloat}\\
\mbox{\tt \ \ \ \ sinh\ \ \ \ \ \ \ \ \ \ \ \ \ \ \ \ =\ \ primSinhFloat}\\
\mbox{\tt \ \ \ \ cosh\ \ \ \ \ \ \ \ \ \ \ \ \ \ \ \ =\ \ primCoshFloat}\\
\mbox{\tt \ \ \ \ tanh\ \ \ \ \ \ \ \ \ \ \ \ \ \ \ \ =\ \ primTanhFloat}\\
\mbox{\tt \ \ \ \ asinh\ \ \ \ \ \ \ \ \ \ \ \ \ \ \ =\ \ primAsinhFloat}\\
\mbox{\tt \ \ \ \ acosh\ \ \ \ \ \ \ \ \ \ \ \ \ \ \ =\ \ primAcoshFloat}\\
\mbox{\tt \ \ \ \ atanh\ \ \ \ \ \ \ \ \ \ \ \ \ \ \ =\ \ primAtanhFloat}
\eprogB\noindent\bprogB
\mbox{\tt instance\ \ Floating\ Double\ \ where}\\
\mbox{\tt \ \ \ \ pi\ \ \ \ \ \ \ \ \ \ \ \ \ \ \ \ \ \ =\ \ primPiDouble}\\
\mbox{\tt \ \ \ \ exp\ \ \ \ \ \ \ \ \ \ \ \ \ \ \ \ \ =\ \ primExpDouble}\\
\mbox{\tt \ \ \ \ log\ \ \ \ \ \ \ \ \ \ \ \ \ \ \ \ \ =\ \ primLogDouble}\\
\mbox{\tt \ \ \ \ sqrt\ \ \ \ \ \ \ \ \ \ \ \ \ \ \ \ =\ \ primSqrtDouble}\\
\mbox{\tt \ \ \ \ sin\ \ \ \ \ \ \ \ \ \ \ \ \ \ \ \ \ =\ \ primSinDouble}\\
\mbox{\tt \ \ \ \ cos\ \ \ \ \ \ \ \ \ \ \ \ \ \ \ \ \ =\ \ primCosDouble}\\
\mbox{\tt \ \ \ \ tan\ \ \ \ \ \ \ \ \ \ \ \ \ \ \ \ \ =\ \ primTanDouble}\\
\mbox{\tt \ \ \ \ asin\ \ \ \ \ \ \ \ \ \ \ \ \ \ \ \ =\ \ primAsinDouble}\\
\mbox{\tt \ \ \ \ acos\ \ \ \ \ \ \ \ \ \ \ \ \ \ \ \ =\ \ primAcosDouble}\\
\mbox{\tt \ \ \ \ atan\ \ \ \ \ \ \ \ \ \ \ \ \ \ \ \ =\ \ primAtanDouble}\\
\mbox{\tt \ \ \ \ sinh\ \ \ \ \ \ \ \ \ \ \ \ \ \ \ \ =\ \ primSinhDouble}\\
\mbox{\tt \ \ \ \ cosh\ \ \ \ \ \ \ \ \ \ \ \ \ \ \ \ =\ \ primCoshDouble}\\
\mbox{\tt \ \ \ \ tanh\ \ \ \ \ \ \ \ \ \ \ \ \ \ \ \ =\ \ primTanhDouble}\\
\mbox{\tt \ \ \ \ asinh\ \ \ \ \ \ \ \ \ \ \ \ \ \ \ =\ \ primAsinhDouble}\\
\mbox{\tt \ \ \ \ acosh\ \ \ \ \ \ \ \ \ \ \ \ \ \ \ =\ \ primAcoshDouble}\\
\mbox{\tt \ \ \ \ atanh\ \ \ \ \ \ \ \ \ \ \ \ \ \ \ =\ \ primAtanhDouble}
\eprogB\noindent\bprogB
\mbox{\tt instance\ \ RealFrac\ Float\ \ where}\\
\mbox{\tt \ \ \ \ properFraction\ \ \ \ \ \ =\ \ floatProperFraction}\\
\mbox{\tt \ \ \ \ approxRational\ \ \ \ \ \ =\ \ floatApproxRational}
\eprogB\noindent\bprogB
\mbox{\tt instance\ \ RealFrac\ Double\ \ where}\\
\mbox{\tt \ \ \ \ properFraction\ \ \ \ \ \ =\ \ floatProperFraction}\\
\mbox{\tt \ \ \ \ approxRational\ \ \ \ \ \ =\ \ floatApproxRational}
\eprogB\noindent\bprogB
\mbox{\tt floatProperFraction\ x\ \ \ =\ \ if\ n\ >=\ 0}\\
\mbox{\tt \ \ \ \ \ \ \ \ \ \ \ \ \ \ \ \ \ \ \ \ \ \ \ \ \ \ \ \ \ \ then\ (m\ *\ b{\char'136}n,\ 0)}\\
\mbox{\tt \ \ \ \ \ \ \ \ \ \ \ \ \ \ \ \ \ \ \ \ \ \ \ \ \ \ \ \ \ \ else\ (m',\ fromInteger\ k\ /\ fromInteger\ d)}\\
\mbox{\tt \ \ \ \ \ \ \ \ \ \ \ \ \ \ \ \ \ \ \ \ \ \ \ \ \ \ \ where\ (m,n)\ \ =\ decodeFloat\ x}\\
\mbox{\tt \ \ \ \ \ \ \ \ \ \ \ \ \ \ \ \ \ \ \ \ \ \ \ \ \ \ \ \ \ \ \ \ \ b\ \ \ \ \ \ =\ floatRadix\ x}\\
\mbox{\tt \ \ \ \ \ \ \ \ \ \ \ \ \ \ \ \ \ \ \ \ \ \ \ \ \ \ \ \ \ \ \ \ \ (m',k)\ =\ divRem\ m\ d}\\
\mbox{\tt \ \ \ \ \ \ \ \ \ \ \ \ \ \ \ \ \ \ \ \ \ \ \ \ \ \ \ \ \ \ \ \ \ d\ \ \ \ \ \ =\ b{\char'136}(-n)}\\
\mbox{\tt floatApproxRational\ x\ eps\ =}\\
\mbox{\tt \ \ \ \ case\ withinEps\ of}\\
\mbox{\tt \ \ \ \ \ \ \ \ r:r':{\char'137}\ |\ denominator\ r\ ==\ denominator\ r'\ ->\ r'}\\
\mbox{\tt \ \ \ \ \ \ \ \ r:{\char'137}\ \ \ \ \ \ \ \ \ \ \ \ \ \ \ \ \ \ \ \ \ \ \ \ \ \ \ \ \ \ \ \ \ \ \ \ \ \ ->\ r}\\
\mbox{\tt \ \ \ \ where\ withinEps\ =\ dropWhile\ ({\char'134}r\ ->\ abs\ (fromRational\ r\ -\ x)\ >\ eps)}\\
\mbox{\tt \ \ \ \ \ \ \ \ \ \ \ \ \ \ \ \ \ \ \ \ \ \ \ \ \ \ \ \ \ \ \ \ (approximants\ p\ q)}\\
\mbox{\tt \ \ \ \ \ \ \ \ \ \ (p,q)\ \ \ \ \ =\ if\ n\ <\ 0\ then\ (m,\ b{\char'136}(-n))\ else\ (m*b{\char'136}n,\ 1)}\\
\mbox{\tt \ \ \ \ \ \ \ \ \ \ (m,n)\ \ \ \ \ =\ decodeFloat\ x}\\
\mbox{\tt \ \ \ \ \ \ \ \ \ \ b\ \ \ \ \ \ \ \ \ =\ toInteger\ (floatRadix\ x)}
\eprogB\noindent\bprogB
\mbox{\tt instance\ \ RealFloat\ Float\ \ where}\\
\mbox{\tt \ \ \ \ floatRadix\ {\char'137}\ \ \ \ \ \ \ \ =\ \ primFloatRadix}\\
\mbox{\tt \ \ \ \ floatDigits\ {\char'137}\ \ \ \ \ \ \ =\ \ primFloatDigits}\\
\mbox{\tt \ \ \ \ floatRange\ {\char'137}\ \ \ \ \ \ \ \ =\ \ (primFloatMinExp,primFloatMaxExp)}\\
\mbox{\tt \ \ \ \ decodeFloat\ \ \ \ \ \ \ \ \ =\ \ primDecodeFloat}\\
\mbox{\tt \ \ \ \ encodeFloat\ \ \ \ \ \ \ \ \ =\ \ primEncodeFloat}
\eprogB\noindent\bprogB
\mbox{\tt instance\ \ RealFloat\ Double\ \ where}\\
\mbox{\tt \ \ \ \ floatRadix\ {\char'137}\ \ \ \ \ \ \ \ =\ \ primDoubleRadix}\\
\mbox{\tt \ \ \ \ floatDigits\ {\char'137}\ \ \ \ \ \ \ =\ \ primDoubleDigits}\\
\mbox{\tt \ \ \ \ floatRange\ {\char'137}\ \ \ \ \ \ \ \ =\ \ (primDoubleMinExp,primDoubleMaxExp)}\\
\mbox{\tt \ \ \ \ decodeFloat\ \ \ \ \ \ \ \ \ =\ \ primDecodeDouble}\\
\mbox{\tt \ \ \ \ encodeFloat\ \ \ \ \ \ \ \ \ =\ \ primEncodeDouble}
\eprogB\noindent\bprogB
\mbox{\tt instance\ \ Ix\ Float\ \ where}\\
\mbox{\tt \ \ \ \ range\ (x,y)\ \ \ \ \ \ \ \ \ =\ \ [x..y]}\\
\mbox{\tt \ \ \ \ index\ (x,y)\ i\ \ \ \ \ \ \ =\ \ floor\ (i\ -\ x)}\\
\mbox{\tt \ \ \ \ inRange\ (x,y)\ i\ \ \ \ \ =\ \ x\ <=\ i\ {\char'46}{\char'46}\ i\ <=\ y}
\eprogB\noindent\bprogB
\mbox{\tt instance\ \ Ix\ Double\ \ where}\\
\mbox{\tt \ \ \ \ range\ (x,y)\ \ \ \ \ \ \ \ \ =\ \ [x..y]}\\
\mbox{\tt \ \ \ \ index\ (x,y)\ i\ \ \ \ \ \ \ =\ \ floor\ (i\ -\ x)}\\
\mbox{\tt \ \ \ \ inRange\ (x,y)\ i\ \ \ \ \ =\ \ x\ <=\ i\ {\char'46}{\char'46}\ i\ <=\ y}
\eprogB\noindent\bprogB
\mbox{\tt instance\ \ Enum\ Float\ \ where}\\
\mbox{\tt \ \ \ \ enumFrom\ x\ \ \ \ \ \ \ \ \ \ =\ \ enumFromBy\ x\ 1}\\
\mbox{\tt \ \ \ \ enumFromThen\ x\ y\ \ \ \ =\ \ enumFromBy\ x\ (y\ -\ x)}
\eprogB\noindent\bprogB
\mbox{\tt instance\ \ Enum\ Double\ \ where}\\
\mbox{\tt \ \ \ \ enumFrom\ x\ \ \ \ \ \ \ \ \ \ =\ \ enumFromBy\ x\ 1}\\
\mbox{\tt \ \ \ \ enumFromThen\ x\ y\ \ \ \ =\ \ enumFromBy\ x\ (y\ -\ x)}
\eprogB
\clearpage

\subsection{Prelude {\tt PreludeRatio}}
\label{preluderatio}
\noindent\bprogB
\mbox{\tt --\ Standard\ functions\ on\ rational\ numbers}\\
\mbox{\tt }\\
\mbox{\tt module\ \ PreludeRatio\ (}\\
\mbox{\tt \ \ \ \ Ratio,\ Rational,\ ({\char'45}),\ numerator,\ denominator,}\\
\mbox{\tt \ \ \ \ approximants,\ partialQuotients\ )\ where}
\eprogB\noindent\bprogB
\mbox{\tt infix\ \ 7\ \ {\char'45},\ :{\char'45}}
\eprogB\noindent\bprogB
\mbox{\tt prec\ =\ 7}
\eprogB\noindent\bprogB
\mbox{\tt data\ \ (Integral\ a)\ \ \ \ \ \ =>\ Ratio\ a\ =\ a\ :{\char'45}\ a\ \ deriving\ (Eq,\ Binary)}\\
\mbox{\tt type\ \ Rational\ \ \ \ \ \ \ \ \ \ =\ \ Ratio\ Integer}
\eprogB\noindent\bprogB
\mbox{\tt ({\char'45})\ \ \ \ \ \ \ \ \ \ \ \ \ \ \ \ \ \ \ \ \ ::\ (Integral\ a)\ =>\ a\ ->\ a\ ->\ Ratio\ a}\\
\mbox{\tt numerator,\ denominator\ \ ::\ (Integral\ a)\ =>\ Ratio\ a\ ->\ a}\\
\mbox{\tt approximants\ \ \ \ \ \ \ \ \ \ \ \ ::\ (Integral\ a)\ =>\ a\ ->\ a\ ->\ [Ratio\ a]}\\
\mbox{\tt partialQuotients\ \ \ \ \ \ \ \ ::\ (Integral\ a)\ =>\ a\ ->\ a\ ->\ [a]}
\index{%@{\tt {\char'045}}}%
\indextt{numerator}%
\indextt{denominator}%
\indextt{approximants}%
\indextt{partialQuotients}%
\eprogB\noindent\bprogB
\mbox{\tt reduce\ x\ y\ \ \ \ \ \ \ \ \ \ \ \ \ \ =\ \ (x\ `div`\ d)\ :{\char'45}\ (y\ `div`\ d)}\\
\mbox{\tt \ \ \ \ \ \ \ \ \ \ \ \ \ \ \ \ \ \ \ \ \ \ \ \ \ \ \ where\ d\ =\ gcd\ x\ y}
\eprogB\noindent\bprogB
\mbox{\tt x\ {\char'45}\ y\ \ \ \ \ \ \ \ \ \ \ \ \ \ \ \ \ \ \ =\ \ reduce\ (x\ *\ signum\ y)\ (abs\ y)}
\eprogB\noindent\bprogB
\mbox{\tt numerator\ (x:{\char'45}y)\ \ \ \ \ \ \ \ =\ \ x}
\eprogB\noindent\bprogB
\mbox{\tt denominator\ (x:{\char'45}y)\ \ \ \ \ \ =\ \ y}
\eprogB\noindent\bprogB
\mbox{\tt approximants\ p\ q\ \ \ \ \ \ \ \ =\ \ zipWith\ (:{\char'45})\ ps\ qs}\\
\mbox{\tt \ \ \ \ \ \ \ \ \ \ \ \ \ \ \ \ \ \ \ \ \ \ \ \ \ \ \ where}\\
\mbox{\tt \ \ \ \ \ \ \ \ \ \ \ \ \ \ \ \ \ \ \ \ \ \ \ \ \ \ \ ps\ =\ gen\ unit\ (unit*a)}\\
\mbox{\tt \ \ \ \ \ \ \ \ \ \ \ \ \ \ \ \ \ \ \ \ \ \ \ \ \ \ \ qs\ =\ gen\ 0\ 1}\\
\mbox{\tt \ \ \ \ \ \ \ \ \ \ \ \ \ \ \ \ \ \ \ \ \ \ \ \ \ \ \ unit\ =\ signum\ p\ *\ signum\ q}\\
\mbox{\tt \ \ \ \ \ \ \ \ \ \ \ \ \ \ \ \ \ \ \ \ \ \ \ \ \ \ \ a:as\ =\ partialQuotients\ (abs\ p)\ (abs\ q)}\\
\mbox{\tt \ \ \ \ \ \ \ \ \ \ \ \ \ \ \ \ \ \ \ \ \ \ \ \ \ \ \ gen\ x\ x'\ =\ xs}\\
\mbox{\tt \ \ \ \ \ \ \ \ \ \ \ \ \ \ \ \ \ \ \ \ \ \ \ \ \ \ \ \ \ \ \ \ \ \ \ \ \ \ where}\\
\mbox{\tt \ \ \ \ \ \ \ \ \ \ \ \ \ \ \ \ \ \ \ \ \ \ \ \ \ \ \ \ \ \ \ \ \ \ \ \ \ \ xs\ =\ x'\ :\ zipWith3\ next\ as\ (x:xs)\ xs}\\
\mbox{\tt \ \ \ \ \ \ \ \ \ \ \ \ \ \ \ \ \ \ \ \ \ \ \ \ \ \ \ \ \ \ \ \ \ \ \ \ \ \ next\ a\ x\ x'\ =\ x'*a\ +\ x}
\eprogB\noindent\bprogB
\mbox{\tt partialQuotients\ p\ q\ \ \ \ =\ \ a\ :\ (if\ r==0\ then\ []\ else\ partialQuotients\ q\ r)}\\
\mbox{\tt \ \ \ \ \ \ \ \ \ \ \ \ \ \ \ \ \ \ \ \ \ \ \ \ \ \ \ where\ (a,r)\ =\ divRem\ p\ q}
\eprogB\noindent\bprogB
\mbox{\tt instance\ \ (Integral\ a)\ \ =>\ Ord\ (Ratio\ a)\ \ where}\\
\mbox{\tt \ \ \ \ (x:{\char'45}y)\ <=\ (x':{\char'45}y')\ \ =\ \ x\ *\ y'\ <=\ x'\ *\ y}
\eprogB\noindent\bprogB
\mbox{\tt instance\ \ (Integral\ a)\ \ =>\ Num\ (Ratio\ a)\ \ where}\\
\mbox{\tt \ \ \ \ (x:{\char'45}y)\ +\ (x':{\char'45}y')\ \ \ =\ \ ((x*m)\ `div`\ y\ +\ (x'*m)\ `div`\ y')\ :{\char'45}\ m}\\
\mbox{\tt \ \ \ \ \ \ \ \ \ \ \ \ \ \ \ \ \ \ \ \ \ \ \ \ \ \ \ where\ m\ =\ lcm\ y\ y'}\\
\mbox{\tt \ \ \ \ (x:{\char'45}y)\ *\ (x':{\char'45}y')\ \ \ =\ \ reduce\ (x\ *\ x')\ (y\ *\ y')}\\
\mbox{\tt \ \ \ \ negate\ (x:{\char'45}y)\ \ \ \ \ \ \ =\ \ (-x)\ :{\char'45}\ y}\\
\mbox{\tt \ \ \ \ abs\ (x:{\char'45}y)\ \ \ \ \ \ \ \ \ \ =\ \ abs\ x\ :{\char'45}\ y}\\
\mbox{\tt \ \ \ \ signum\ (x:{\char'45}y)\ \ \ \ \ \ \ =\ \ signum\ x\ :{\char'45}\ 1}\\
\mbox{\tt \ \ \ \ fromInteger\ x\ \ \ \ \ \ \ =\ \ fromInteger\ x\ :{\char'45}\ 1}
\eprogB\noindent\bprogB
\mbox{\tt instance\ \ (Integral\ a)\ \ =>\ Real\ (Ratio\ a)\ \ where}\\
\mbox{\tt \ \ \ \ toRational\ (x:{\char'45}y)\ \ \ =\ \ toInteger\ x\ :{\char'45}\ toInteger\ y}
\eprogB\noindent\bprogB
\mbox{\tt instance\ \ (Integral\ a)\ \ =>\ Fractional\ (Ratio\ a)\ \ where}\\
\mbox{\tt \ \ \ \ (x:{\char'45}y)\ /\ (x':{\char'45}y')\ \ \ =\ \ (x*y')\ {\char'45}\ (y*x')}\\
\mbox{\tt \ \ \ \ fromRational\ (x:{\char'45}y)\ =\ \ fromInteger\ x\ :{\char'45}\ fromInteger\ y}
\eprogB\noindent\bprogB
\mbox{\tt instance\ \ (Integral\ a)\ \ =>\ RealFrac\ (Ratio\ a)\ \ where}\\
\mbox{\tt \ \ \ \ properFraction\ (x:{\char'45}y)\ =\ (toInteger\ q,\ r:{\char'45}y)}\\
\mbox{\tt \ \ \ \ \ \ \ \ \ \ \ \ \ \ \ \ \ \ \ \ \ \ \ \ \ \ \ \ where\ (q,r)\ =\ divRem\ x\ y}\\
\mbox{\tt }\\
\mbox{\tt \ \ \ \ approxRational\ x@(p:{\char'45}q)\ eps\ =}\\
\mbox{\tt \ \ \ \ \ \ \ \ case\ withinEps\ of}\\
\mbox{\tt \ \ \ \ \ \ \ \ \ \ \ \ r:r':{\char'137}\ |\ denominator\ r\ ==\ denominator\ r'\ ->\ r'}\\
\mbox{\tt \ \ \ \ \ \ \ \ \ \ \ \ r:{\char'137}\ \ \ \ \ \ \ \ \ \ \ \ \ \ \ \ \ \ \ \ \ \ \ \ \ \ \ \ \ \ \ \ \ \ \ \ \ \ ->\ r}\\
\mbox{\tt \ \ \ \ \ \ \ \ where\ withinEps\ =\ dropWhile\ ({\char'134}r\ ->\ abs\ (r-x)\ >\ eps)}\\
\mbox{\tt \ \ \ \ \ \ \ \ \ \ \ \ \ \ \ \ \ \ \ \ \ \ \ \ \ \ \ \ \ \ \ \ \ \ \ \ (approximants\ p\ q)}
\eprogB\noindent\bprogB
\mbox{\tt instance\ \ (Integral\ a,\ Text\ a)\ =>\ Text\ (Ratio\ a)\ \ where}\\
\mbox{\tt \ \ \ \ readsPrec\ p\ \ =\ \ readParen\ (p\ >\ prec)}\\
\mbox{\tt \ \ \ \ \ \ \ \ \ \ \ \ \ \ \ \ \ \ \ \ \ \ \ \ \ \ \ \ \ \ ({\char'134}r\ ->\ [(x{\char'45}y,u)\ |\ (x,s)\ \ \ <-\ reads\ r,}\\
\mbox{\tt \ \ \ \ \ \ \ \ \ \ \ \ \ \ \ \ \ \ \ \ \ \ \ \ \ \ \ \ \ \ \ \ \ \ \ \ \ \ \ \ \ \ \ \ \ \ \ \ ("{\char'45}",t)\ <-\ [lex\ s],}\\
\mbox{\tt \ \ \ \ \ \ \ \ \ \ \ \ \ \ \ \ \ \ \ \ \ \ \ \ \ \ \ \ \ \ \ \ \ \ \ \ \ \ \ \ \ \ \ \ \ \ \ \ (y,u)\ \ \ <-\ reads\ t\ ])}\\
\mbox{\tt }\\
\mbox{\tt \ \ \ \ showsPrec\ p\ (x:{\char'45}y)\ \ =\ \ showParen\ (p\ >\ prec)}\\
\mbox{\tt \ \ \ \ \ \ \ \ \ \ \ \ \ \ \ \ \ \ \ \ \ \ \ \ \ \ \ \ \ \ \ (shows\ x\ .\ showString\ "\ {\char'45}\ "\ .\ shows\ y)}
\eprogB
\clearpage

\subsection{Prelude {\tt PreludeComplex}}
\label{preludecomplex}
\noindent\bprogB
\mbox{\tt --\ Complex\ Numbers}\\
\mbox{\tt }\\[-8pt]
\mbox{\tt module\ PreludeComplex\ (\ Complex((:+))\ )\ \ where}
\index{PreludeComplex@{\ptt PreludeComplex} (module)}%
\eprogB\noindent\bprogB
\mbox{\tt infix\ \ 6\ \ :+}
\index{:+@{\ptt :+}}%
\eprogB\noindent\bprogB
\mbox{\tt data\ \ (RealFloat\ a)\ \ \ \ \ =>\ Complex\ a\ =\ a\ :+\ a\ \ deriving\ (Eq,Binary,Text)}
\eprogB\noindent\bprogB
\mbox{\tt instance\ \ (RealFloat\ a)\ =>\ Num\ (Complex\ a)\ \ where}\\
\mbox{\tt \ \ \ \ (x:+y)\ +\ (x':+y')\ \ \ =\ \ (x+x')\ :+\ (y+y')}\\
\mbox{\tt \ \ \ \ (x:+y)\ -\ (x':+y')\ \ \ =\ \ (x-x')\ :+\ (y-y')}\\
\mbox{\tt \ \ \ \ (x:+y)\ *\ (x':+y')\ \ \ =\ \ (x*x'-y*y')\ :+\ (x*y'+y*x')}\\
\mbox{\tt \ \ \ \ negate\ (x:+y)\ \ \ \ \ \ \ =\ \ negate\ x\ :+\ negate\ y}\\
\mbox{\tt \ \ \ \ abs\ z\ \ \ \ \ \ \ \ \ \ \ \ \ \ \ =\ \ magnitude\ z\ :+\ 0}\\
\mbox{\tt \ \ \ \ signum\ 0\ \ \ \ \ \ \ \ \ \ \ \ =\ \ 0}\\
\mbox{\tt \ \ \ \ signum\ z@(x:+y)\ \ \ \ \ =\ \ x/r\ :+\ y/r\ \ where\ r\ =\ magnitude\ z}\\
\mbox{\tt \ \ \ \ fromInteger\ n\ \ \ \ \ \ \ =\ \ fromInteger\ n\ :+\ 0}
\index{Num@{\ptt Num}!instance for {\ptt Complex}}%
\eprogB\noindent\bprogB
\mbox{\tt instance\ \ (RealFloat\ a)\ =>\ Fractional\ (Complex\ a)\ \ where}\\
\mbox{\tt \ \ \ \ (x:+y)\ /\ (x':+y')\ \ \ =\ \ (x*x''+y*y'')\ /\ d\ :+\ (y*x''-x*y'')\ /\ d}\\
\mbox{\tt \ \ \ \ \ \ \ \ \ \ \ \ \ \ \ \ \ \ \ \ \ \ \ \ \ \ \ where\ x''\ =\ scaleFloat\ k\ x'}\\
\mbox{\tt \ \ \ \ \ \ \ \ \ \ \ \ \ \ \ \ \ \ \ \ \ \ \ \ \ \ \ \ \ \ \ \ \ y''\ =\ scaleFloat\ k\ y'}\\
\mbox{\tt \ \ \ \ \ \ \ \ \ \ \ \ \ \ \ \ \ \ \ \ \ \ \ \ \ \ \ \ \ \ \ \ \ k\ \ \ =\ -\ max\ (exponent\ x')\ (exponent\ y')}\\
\mbox{\tt \ \ \ \ \ \ \ \ \ \ \ \ \ \ \ \ \ \ \ \ \ \ \ \ \ \ \ \ \ \ \ \ \ d\ \ \ =\ x'*x''\ +\ y'*y''}\\
\mbox{\tt }\\[-8pt]
\mbox{\tt \ \ \ \ fromRational\ a\ \ \ \ \ \ =\ \ fromRational\ a\ :+\ 0}
\index{Fractional@{\ptt Fractional}!instance for {\ptt Complex}}%
\eprogB\noindent\bprogB
\mbox{\tt instance\ \ (RealFloat\ a)\ =>\ Floating\ (Complex\ a)\ where}\\
\mbox{\tt \ \ \ \ pi\ \ \ \ \ \ \ \ \ \ \ \ \ =\ \ pi\ :+\ 0}\\
\mbox{\tt \ \ \ \ exp\ (x:+y)\ \ \ \ \ =\ \ expx\ *\ cos\ y\ :+\ expx\ *\ sin\ y}\\
\mbox{\tt \ \ \ \ \ \ \ \ \ \ \ \ \ \ \ \ \ \ \ \ \ \ where\ expx\ =\ exp\ x}\\
\mbox{\tt \ \ \ \ log\ z\ \ \ \ \ \ \ \ \ \ =\ \ log\ (magnitude\ z)\ :+\ phase\ z}\\
\mbox{\tt }\\[-8pt]
\mbox{\tt \ \ \ \ sqrt\ 0\ \ \ \ \ \ \ \ \ =\ \ 0}\\
\mbox{\tt \ \ \ \ sqrt\ z@(x:+y)\ \ =\ \ u\ :+\ (if\ y\ <\ 0\ then\ -v\ else\ v)}\\
\mbox{\tt \ \ \ \ \ \ \ \ \ \ \ \ \ \ \ \ \ \ \ \ \ \ where\ (u,v)\ =\ if\ x\ <\ 0\ then\ (v',u')\ else\ (u',v')}\\
\mbox{\tt \ \ \ \ \ \ \ \ \ \ \ \ \ \ \ \ \ \ \ \ \ \ \ \ \ \ \ \ v'\ \ \ \ =\ abs\ y\ /\ (u'*2)}\\
\mbox{\tt \ \ \ \ \ \ \ \ \ \ \ \ \ \ \ \ \ \ \ \ \ \ \ \ \ \ \ \ u'\ \ \ \ =\ sqrt\ ((magnitude\ z\ +\ abs\ x)\ /\ 2)}\\
\mbox{\tt }\\[-8pt]
\mbox{\tt \ \ \ \ sin\ (x:+y)\ \ \ \ \ =\ \ sin\ x\ *\ cosh\ y\ :+\ cos\ x\ *\ sinh\ y}\\
\mbox{\tt \ \ \ \ cos\ (x:+y)\ \ \ \ \ =\ \ cos\ x\ *\ cosh\ y\ :+\ sin\ x\ *\ sinh\ y}\\
\mbox{\tt \ \ \ \ tan\ (x:+y)\ \ \ \ \ =\ \ (sinx*coshy:+cosx*sinhy)/(cosx*coshy:+sinx*sinhy)}\\
\mbox{\tt \ \ \ \ \ \ \ \ \ \ \ \ \ \ \ \ \ \ \ \ \ \ where\ sinx\ \ =\ sin\ x}\\
\mbox{\tt \ \ \ \ \ \ \ \ \ \ \ \ \ \ \ \ \ \ \ \ \ \ \ \ \ \ \ \ cosx\ \ =\ cos\ x}\\
\mbox{\tt \ \ \ \ \ \ \ \ \ \ \ \ \ \ \ \ \ \ \ \ \ \ \ \ \ \ \ \ sinhy\ =\ sinh\ y}\\
\mbox{\tt \ \ \ \ \ \ \ \ \ \ \ \ \ \ \ \ \ \ \ \ \ \ \ \ \ \ \ \ coshy\ =\ cosh\ y}\\
\mbox{\tt }\\[-8pt]
\mbox{\tt \ \ \ \ sinh\ (x:+y)\ \ \ \ =\ \ cos\ y\ *\ sinh\ x\ :+\ sin\ \ y\ *\ cosh\ x}\\
\mbox{\tt \ \ \ \ cosh\ (x:+y)\ \ \ \ =\ \ cos\ y\ *\ cosh\ x\ :+\ (-\ sin\ y\ *\ sinh\ x)}\\
\mbox{\tt \ \ \ \ tanh\ (x:+y)\ \ \ \ =\ \ (cosy*sinhx:+siny*coshx)/(cosy*coshx:+(-siny*sinhx))}\\
\mbox{\tt \ \ \ \ \ \ \ \ \ \ \ \ \ \ \ \ \ \ \ \ \ \ where\ siny\ \ =\ sin\ y}\\
\mbox{\tt \ \ \ \ \ \ \ \ \ \ \ \ \ \ \ \ \ \ \ \ \ \ \ \ \ \ \ \ cosy\ \ =\ cos\ y}\\
\mbox{\tt \ \ \ \ \ \ \ \ \ \ \ \ \ \ \ \ \ \ \ \ \ \ \ \ \ \ \ \ sinhx\ =\ sinh\ x}\\
\mbox{\tt \ \ \ \ \ \ \ \ \ \ \ \ \ \ \ \ \ \ \ \ \ \ \ \ \ \ \ \ coshx\ =\ cosh\ x}\\
\mbox{\tt }\\[-8pt]
\mbox{\tt \ \ \ \ asin\ z@(x:+y)\ \ =\ \ y':+(-x')}\\
\mbox{\tt \ \ \ \ \ \ \ \ \ \ \ \ \ \ \ \ \ \ \ \ \ \ where\ \ (x':+y')\ =\ log\ ((-y:+x)\ +\ sqrt\ (1\ -\ z*z))}\\
\mbox{\tt \ \ \ \ acos\ z@(x:+y)\ \ =\ \ y'':+(-x'')}\\
\mbox{\tt \ \ \ \ \ \ \ \ \ \ \ \ \ \ \ \ \ \ \ \ \ \ where\ (x'':+y'')\ =\ log\ (z\ +\ ((-y'):+x'))}\\
\mbox{\tt \ \ \ \ \ \ \ \ \ \ \ \ \ \ \ \ \ \ \ \ \ \ \ \ \ \ \ \ (x':+y')\ \ \ =\ sqrt\ (1\ -\ z*z)}\\
\mbox{\tt \ \ \ \ atan\ z@(x:+y)\ \ =\ \ y':+(-x')}\\
\mbox{\tt \ \ \ \ \ \ \ \ \ \ \ \ \ \ \ \ \ \ \ \ \ \ where}\\
\mbox{\tt \ \ \ \ \ \ \ \ \ \ \ \ \ \ \ \ \ \ \ \ \ \ (x':+y')\ =\ log\ (((-y+1):+x)\ *\ sqrt\ (1/(1+z*z)))}\\
\mbox{\tt }\\[-8pt]
\mbox{\tt \ \ \ \ asinh\ z\ \ \ \ \ \ \ \ =\ \ log\ (z\ +\ sqrt\ (1+z*z))}\\
\mbox{\tt \ \ \ \ acosh\ z\ \ \ \ \ \ \ \ =\ \ log\ (z\ +\ (z+1)\ *\ sqrt\ ((z-1)/(z+1)))}\\
\mbox{\tt \ \ \ \ atanh\ z\ \ \ \ \ \ \ \ =\ \ log\ ((z+1)\ *\ sqrt\ (1\ -\ 1/(z*z)))}
\index{Floating@{\ptt Floating}!instance for {\ptt Complex}}%
\eprogB\noindent\bprogB
\mbox{\tt realPart,\ imagPart\ ::\ (RealFloat\ a)\ =>\ Complex\ a\ ->\ a}\\
\mbox{\tt realPart\ (x:+y)\ \ =\ \ x}\\
\mbox{\tt imagPart\ (x:+y)\ \ =\ \ y}
\indextt{realPart}%
\indextt{imagPart}%
\eprogB\noindent\bprogB
\mbox{\tt conjugate\ \ \ \ \ \ \ \ ::\ (RealFloat\ a)\ =>\ Complex\ a\ ->\ Complex\ a}\\
\mbox{\tt conjugate\ (x:+y)\ =\ \ x\ :+\ (-y)}
\indextt{conjugate}%
\eprogB\noindent\bprogB
\mbox{\tt mkPolar\ \ \ \ \ \ \ \ \ \ ::\ (RealFloat\ a)\ =>\ a\ ->\ a\ ->\ Complex\ a}\\
\mbox{\tt mkPolar\ r\ theta\ \ =\ \ r\ *\ cos\ theta\ :+\ r\ *\ sin\ theta}
\indextt{mkPolar}%
\eprogB\noindent\bprogB
\mbox{\tt cis\ \ \ \ \ \ \ \ \ \ \ \ \ \ ::\ (RealFloat\ a)\ =>\ a\ ->\ Complex\ a}\\
\mbox{\tt cis\ theta\ \ \ \ \ \ \ \ =\ \ cos\ theta\ :+\ sin\ theta}
\indextt{cis}%
\eprogB\noindent\bprogB
\mbox{\tt polar\ \ \ \ \ \ \ \ \ \ \ \ ::\ (RealFloat\ a)\ =>\ Complex\ a\ ->\ (a,a)}\\
\mbox{\tt polar\ z\ \ \ \ \ \ \ \ \ \ =\ \ (magnitude\ z,\ phase\ z)}
\indextt{polar}%
\eprogB\noindent\bprogB
\mbox{\tt magnitude,\ phase\ ::\ (RealFloat\ a)\ =>\ Complex\ a\ ->\ a}\\
\mbox{\tt magnitude\ (x:+y)\ =\ \ scaleFloat\ k}\\
\mbox{\tt \ \ \ \ \ \ \ \ \ \ \ \ \ \ \ \ \ \ \ \ \ (sqrt\ ((scaleFloat\ mk\ x){\char'136}2\ +\ (scaleFloat\ mk\ y){\char'136}2))}\\
\mbox{\tt \ \ \ \ \ \ \ \ \ \ \ \ \ \ \ \ \ \ \ \ where\ k\ \ =\ max\ (exponent\ x)\ (exponent\ y)}\\
\mbox{\tt \ \ \ \ \ \ \ \ \ \ \ \ \ \ \ \ \ \ \ \ \ \ \ \ \ \ mk\ =\ -\ k}
\indextt{magnitude}%
\indextt{phase}%
\eprogB\noindent\bprogB
\mbox{\tt phase\ (x:+y)\ \ \ \ \ =\ \ atan2\ y\ x}
\eprogB
\clearpage

\subsection{Prelude {\tt PreludeList}}
\label{preludelist}
\noindent\bprogB
\mbox{\tt --\ Standard\ list\ functions}\\
\mbox{\tt }\\
\mbox{\tt module\ PreludeList\ \ where}
\eprogB\noindent\bprogB
\mbox{\tt infixl\ 9\ \ !!}\\
\mbox{\tt infixl\ 3\ \ {\char'134}{\char'134}}\\
\mbox{\tt infixr\ 3\ \ ++}\\
\mbox{\tt infix\ 2\ \ `in`}
\eprogB\noindent\bprogB
\mbox{\tt head\ \ \ \ \ \ \ \ \ \ \ \ \ \ \ \ \ \ \ \ ::\ [a]\ ->\ a}\\
\mbox{\tt head\ (x:{\char'137})\ \ \ \ \ \ \ \ \ \ \ \ \ \ =\ \ x}
\indextt{head}%
\eprogB\noindent\bprogB
\mbox{\tt last\ \ \ \ \ \ \ \ \ \ \ \ \ \ \ \ \ \ \ \ ::\ [a]\ ->\ a}\\
\mbox{\tt last\ [x]\ \ \ \ \ \ \ \ \ \ \ \ \ \ \ \ =\ \ x}\\
\mbox{\tt last\ ({\char'137}:xs)\ \ \ \ \ \ \ \ \ \ \ \ \ =\ \ last\ xs}
\indextt{last}%
\eprogB\noindent\bprogB
\mbox{\tt tail\ \ \ \ \ \ \ \ \ \ \ \ \ \ \ \ \ \ \ \ ::\ [a]\ ->\ [a]}\\
\mbox{\tt tail\ ({\char'137}:xs)\ \ \ \ \ \ \ \ \ \ \ \ \ =\ \ xs}
\indextt{tail}%
\eprogB\noindent\bprogB
\mbox{\tt init\ \ \ \ \ \ \ \ \ \ \ \ \ \ \ \ \ \ \ \ ::\ [a]\ ->\ [a]}\\
\mbox{\tt init\ [x]\ \ \ \ \ \ \ \ \ \ \ \ \ \ \ \ =\ \ []}\\
\mbox{\tt init\ (x:xs)\ \ \ \ \ \ \ \ \ \ \ \ \ =\ \ x\ :\ init\ xs}
\indextt{init}%
\eprogB\noindent\bprogB
\mbox{\tt null\ \ \ \ \ \ \ \ \ \ \ \ \ \ \ \ \ \ \ \ ::\ [a]\ ->\ Bool}\\
\mbox{\tt null\ []\ \ \ \ \ \ \ \ \ \ \ \ \ \ \ \ \ =\ \ True}\\
\mbox{\tt null\ ({\char'137}:{\char'137})\ \ \ \ \ \ \ \ \ \ \ \ \ \ =\ \ False}
\indextt{null}%
\eprogB\noindent\bprogB
\mbox{\tt (++)\ \ \ \ \ \ \ \ \ \ \ \ \ \ \ \ \ \ \ \ ::\ [a]\ ->\ [a]\ ->\ [a]}\\
\mbox{\tt []\ ++\ ys\ \ \ \ \ \ \ \ \ \ \ \ \ \ \ \ =\ \ ys}\\
\mbox{\tt (x:xs)\ ++\ ys\ \ \ \ \ \ \ \ \ \ \ \ =\ \ x\ :\ (xs\ ++\ ys)}
\index{++@{\tt ++}}%
\eprogB\noindent\bprogB
\mbox{\tt length\ \ \ \ \ \ \ \ \ \ \ \ \ \ \ \ \ \ ::\ (Integral\ a)\ =>\ [b]\ ->\ a}\\
\mbox{\tt length\ \ \ \ \ \ \ \ \ \ \ \ \ \ \ \ \ \ =\ \ foldl\ ({\char'134}n\ {\char'137}\ ->\ n+1)\ 0}
\indextt{length}%
\eprogB\noindent\bprogB
\mbox{\tt (!!)\ \ \ \ \ \ \ \ \ \ \ \ \ \ \ \ \ \ \ \ ::\ (Integral\ a)\ =>\ [b]\ ->\ a\ ->\ b}\\
\mbox{\tt (x:{\char'137})\ \ !!\ 0\ \ \ \ \ \ \ \ \ \ \ \ \ =\ \ x}\\
\mbox{\tt ({\char'137}:xs)\ !!\ (n+1)\ \ \ \ \ \ \ \ \ =\ \ xs\ !!\ n}
\index{!!@{\tt {\char'041}{\char'041}}}%
\eprogB\noindent\bprogB
\mbox{\tt map\ \ \ \ \ \ \ \ \ \ \ \ \ \ \ \ \ \ \ \ \ ::\ (a\ ->\ b)\ ->\ [a]\ ->\ [b]}\\
\mbox{\tt map\ f\ []\ \ \ \ \ \ \ \ \ \ \ \ \ \ \ \ =\ \ []}\\
\mbox{\tt map\ f\ (x:xs)\ \ \ \ \ \ \ \ \ \ \ \ =\ \ f\ x\ :\ map\ f\ xs}
\indextt{map}%
\eprogB\noindent\bprogB
\mbox{\tt filter\ \ \ \ \ \ \ \ \ \ \ \ \ \ \ \ \ \ ::\ (a\ ->\ Bool)\ ->\ [a]\ ->\ [a]}\\
\mbox{\tt filter\ p\ xs\ \ \ \ \ \ \ \ \ \ \ \ \ =\ \ [x\ |\ x\ <-\ xs,\ p\ x]}
\indextt{filter}%
\eprogB\noindent\bprogB
\mbox{\tt foldr\ \ \ \ \ \ \ \ \ \ \ \ \ \ \ \ \ \ \ ::\ (a\ ->\ b\ ->\ b)\ ->\ b\ ->\ [a]\ ->\ b}\\
\mbox{\tt foldr\ f\ z\ []\ \ \ \ \ \ \ \ \ \ \ \ =\ \ z}\\
\mbox{\tt foldr\ f\ z\ (x:xs)\ \ \ \ \ \ \ \ =\ \ f\ x\ (foldr\ f\ z\ xs)}
\indextt{foldr}%
\eprogB\noindent\bprogB
\mbox{\tt foldl\ \ \ \ \ \ \ \ \ \ \ \ \ \ \ \ \ \ \ ::\ (a\ ->\ b\ ->\ a)\ ->\ a\ ->\ [b]\ ->\ a}\\
\mbox{\tt foldl\ f\ z\ []\ \ \ \ \ \ \ \ \ \ \ \ =\ \ z}\\
\mbox{\tt foldl\ f\ z\ (x:xs)\ \ \ \ \ \ \ \ =\ \ foldl\ f\ (f\ z\ x)\ xs}
\indextt{foldl}%
\eprogB\noindent\bprogB
\mbox{\tt foldr1\ \ \ \ \ \ \ \ \ \ \ \ \ \ \ \ \ \ ::\ (a\ ->\ a\ ->\ a)\ ->\ [a]\ ->\ a}\\
\mbox{\tt foldr1\ f\ [x]\ \ \ \ \ \ \ \ \ \ \ \ =\ \ x}\\
\mbox{\tt foldr1\ f\ (x:xs)\ \ \ \ \ \ \ \ \ =\ \ f\ x\ (foldr1\ f\ xs)}
\indextt{foldr1}%
\eprogB\noindent\bprogB
\mbox{\tt foldl1\ \ \ \ \ \ \ \ \ \ \ \ \ \ \ \ \ \ ::\ (a\ ->\ a\ ->\ a)\ ->\ [a]\ ->\ a}\\
\mbox{\tt foldl1\ f\ (x:xs)\ \ \ \ \ \ \ \ \ =\ \ foldl\ f\ x\ xs}
\indextt{foldl1}%
\eprogB\noindent\bprogB
\mbox{\tt scan\ \ \ \ \ \ \ \ \ \ \ \ \ \ \ \ \ \ \ \ ::\ (a\ ->\ b\ ->\ a)\ ->\ a\ ->\ [b]\ ->\ [a]}\\
\mbox{\tt scan\ f\ q\ xs\ \ \ \ \ \ \ \ \ \ \ \ \ =\ \ q\ :\ case\ xs\ of}\\
\mbox{\tt \ \ \ \ \ \ \ \ \ \ \ \ \ \ \ \ \ \ \ \ \ \ \ \ \ \ \ \ \ \ \ \ []\ \ \ ->\ []}\\
\mbox{\tt \ \ \ \ \ \ \ \ \ \ \ \ \ \ \ \ \ \ \ \ \ \ \ \ \ \ \ \ \ \ \ \ x:xs\ ->\ scan\ f\ (f\ q\ x)\ xs}
\indextt{scan}%
\eprogB\noindent\bprogB
\mbox{\tt iterate\ \ \ \ \ \ \ \ \ \ \ \ \ \ \ \ \ ::\ (a\ ->\ a)\ ->\ a\ ->\ [a]}\\
\mbox{\tt iterate\ f\ x\ \ \ \ \ \ \ \ \ \ \ \ \ =\ \ x\ :\ iterate\ f\ (f\ x)}
\indextt{iterate}%
\eprogB\noindent\bprogB
\mbox{\tt repeat\ \ \ \ \ \ \ \ \ \ \ \ \ \ \ \ \ \ ::\ a\ ->\ [a]}\\
\mbox{\tt repeat\ x\ \ \ \ \ \ \ \ \ \ \ \ \ \ \ \ =\ \ xs\ where\ xs\ =\ x:xs}
\indextt{repeat}%
\eprogB\noindent\bprogB
\mbox{\tt cycle\ \ \ \ \ \ \ \ \ \ \ \ \ \ \ \ \ \ \ ::\ [a]\ ->\ [a]}\\
\mbox{\tt cycle\ xs\ \ \ \ \ \ \ \ \ \ \ \ \ \ \ \ =\ \ xs'\ where\ xs'\ =\ xs\ ++\ xs'}
\indextt{cycle}%
\eprogB\noindent\bprogB
\mbox{\tt take\ \ \ \ \ \ \ \ \ \ \ \ \ \ \ \ \ \ \ \ ::\ (Integral\ a)\ =>\ a\ ->\ [b]\ ->\ [b]}\\
\mbox{\tt take\ \ {\char'137}\ \ \ \ \ []\ \ \ \ \ \ \ \ \ \ =\ \ []}\\
\mbox{\tt take\ \ 0\ \ \ \ \ {\char'137}\ \ \ \ \ \ \ \ \ \ \ =\ \ []}\\
\mbox{\tt take\ (n+1)\ (x:xs)\ \ \ \ \ \ \ =\ \ x\ :\ take\ n\ xs}
\indextt{take}%
\eprogB\noindent\bprogB
\mbox{\tt drop\ \ \ \ \ \ \ \ \ \ \ \ \ \ \ \ \ \ \ \ ::\ (Integral\ a)\ =>\ a\ \ ->\ [b]\ ->\ [b]}\\
\mbox{\tt drop\ \ {\char'137}\ \ \ \ \ []\ \ \ \ \ \ \ \ \ \ =\ \ []}\\
\mbox{\tt drop\ \ 0\ \ \ \ \ xs\ \ \ \ \ \ \ \ \ \ =\ \ xs}\\
\mbox{\tt drop\ (n+1)\ ({\char'137}:xs)\ \ \ \ \ \ \ =\ \ drop\ n\ xs}
\indextt{drop}%
\eprogB\noindent\bprogB
\mbox{\tt takeWhile\ \ \ \ \ \ \ \ \ \ \ \ \ \ \ ::\ (a\ ->\ Bool)\ ->\ [a]\ ->\ [a]}\\
\mbox{\tt takeWhile\ p\ []\ \ \ \ \ \ \ \ \ \ =\ \ []}\\
\mbox{\tt takeWhile\ p\ (x:xs)\ }\\
\mbox{\tt \ \ \ \ \ \ \ \ \ \ \ \ |\ p\ x\ \ \ \ \ \ \ =\ \ x\ :\ takeWhile\ p\ xs}\\
\mbox{\tt \ \ \ \ \ \ \ \ \ \ \ \ |\ otherwise\ =\ \ []}
\indextt{takeWhile}%
\eprogB\noindent\bprogB
\mbox{\tt dropWhile\ \ \ \ \ \ \ \ \ \ \ \ \ \ \ ::\ (a\ ->\ Bool)\ ->\ [a]\ ->\ [a]}\\
\mbox{\tt dropWhile\ p\ []\ \ \ \ \ \ \ \ \ \ =\ \ []}\\
\mbox{\tt dropWhile\ p\ xs@(x:xs')}\\
\mbox{\tt \ \ \ \ \ \ \ \ \ \ \ \ |\ p\ x\ \ \ \ \ \ \ =\ \ dropWhile\ p\ xs'}\\
\mbox{\tt \ \ \ \ \ \ \ \ \ \ \ \ |\ otherwise\ =\ \ xs}
\indextt{dropWhile}%
\eprogB\noindent\bprogB
\mbox{\tt span,\ break\ \ \ \ \ \ \ \ \ \ \ \ \ ::\ (a\ ->\ Bool)\ ->\ [a]\ ->\ ([a],[a])}\\
\mbox{\tt span\ p\ xs\ \ \ \ \ \ \ \ \ \ \ \ \ \ \ =\ \ (takeWhile\ p\ xs,\ dropWhile\ p\ xs)}\\
\mbox{\tt break\ p\ \ \ \ \ \ \ \ \ \ \ \ \ \ \ \ \ =\ \ span\ (not\ .\ p)}
\indextt{span}%
\indextt{break}%
\eprogB\noindent\bprogB
\mbox{\tt lines\ \ \ \ \ \ \ \ \ \ \ \ \ \ \ \ \ \ \ ::\ String\ ->\ [String]}\\
\mbox{\tt lines\ ""\ \ \ \ \ \ \ \ \ \ \ \ \ \ \ \ =\ \ []}\\
\mbox{\tt lines\ s\ \ \ \ \ \ \ \ \ \ \ \ \ \ \ \ \ =\ \ l\ :\ (if\ null\ s'\ then\ []\ else\ lines\ (tail\ s'))}\\
\mbox{\tt \ \ \ \ \ \ \ \ \ \ \ \ \ \ \ \ \ \ \ \ \ \ \ \ \ \ \ where\ (l,\ s')\ =\ break\ ((==)\ '{\char'134}n')\ s}
\indextt{lines}%
\eprogB\noindent\bprogB
\mbox{\tt words\ \ \ \ \ \ \ \ \ \ \ \ \ \ \ \ \ \ \ ::\ String\ ->\ [String]}\\
\mbox{\tt words\ s\ \ \ \ \ \ \ \ \ \ \ \ \ \ \ \ \ =\ \ case\ dropWhile\ isSpace\ s\ of}\\
\mbox{\tt \ \ \ \ \ \ \ \ \ \ \ \ \ \ \ \ \ \ \ \ \ \ \ \ \ \ \ \ \ \ \ \ ""\ ->\ []}\\
\mbox{\tt \ \ \ \ \ \ \ \ \ \ \ \ \ \ \ \ \ \ \ \ \ \ \ \ \ \ \ \ \ \ \ \ s'\ ->\ w\ :\ words\ s''}\\
\mbox{\tt \ \ \ \ \ \ \ \ \ \ \ \ \ \ \ \ \ \ \ \ \ \ \ \ \ \ \ \ \ \ \ \ \ \ \ \ \ \ where\ (w,\ s'')\ =\ break\ isSpace\ s'}
\indextt{words}%
\eprogB\noindent\bprogB
\mbox{\tt unlines\ \ \ \ \ \ \ \ \ \ \ \ \ \ \ \ \ ::\ [String]\ ->\ String}\\
\mbox{\tt unlines\ ls\ \ \ \ \ \ \ \ \ \ \ \ \ \ =\ concat\ (map\ ({\char'134}l\ ->\ l\ ++\ "{\char'134}n")\ ls)}
\indextt{unlines}%
\eprogB\noindent\bprogB
\mbox{\tt unwords\ \ \ \ \ \ \ \ \ \ \ \ \ \ \ \ \ ::\ [String]\ ->\ String}\\
\mbox{\tt unwords\ []\ \ \ \ \ \ \ \ \ \ \ \ \ \ =\ ""}\\
\mbox{\tt unwords\ [w]\ \ \ \ \ \ \ \ \ \ \ \ \ =\ w}\\
\mbox{\tt unwords\ (w:ws)\ \ \ \ \ \ \ \ \ \ =\ w\ ++\ concat\ (map\ ((:)\ '\ ')\ ws)}
\indextt{unwords}%
\eprogB\noindent\bprogB
\mbox{\tt in\ \ \ \ \ \ \ \ \ \ \ \ \ \ \ \ \ \ \ \ \ \ ::\ (Eq\ a)\ =>\ a\ ->\ [a]\ ->\ Bool}\\
\mbox{\tt x\ `in`\ []\ \ \ \ \ \ \ \ \ \ \ \ \ \ \ =\ \ False}\\
\mbox{\tt x\ `in`\ (y:ys)\ \ \ \ \ \ \ \ \ \ \ =\ \ x\ ==\ y\ ||\ x\ `in`\ ys}
\indextt{in}%
\eprogB\noindent\bprogB
\mbox{\tt ({\char'134}{\char'134})\ \ \ \ \ \ \ \ \ \ \ \ \ \ \ \ \ \ \ \ ::\ (Eq\ a)\ =>\ [a]\ ->\ [a]\ ->\ [a]}\\
\mbox{\tt xs\ {\char'134}{\char'134}\ ys\ \ \ \ \ \ \ \ \ \ \ \ \ \ \ \ =\ \ foldr\ remove\ ys\ xs}
\index{\\@{\tt {\char'134}{\char'134}}}%
\eprogB\noindent\bprogB
\mbox{\tt remove\ \ \ \ \ \ \ \ \ \ \ \ \ \ \ \ \ \ ::\ (Eq\ a)\ =>\ a\ ->\ [a]\ ->\ [a]}\\
\mbox{\tt remove\ x\ \ \ \ \ \ \ \ \ \ \ \ \ \ \ \ =\ \ filter\ ((/=)\ x)}
\indextt{remove}%
\eprogB\noindent\bprogB
\mbox{\tt nub\ \ \ \ \ \ \ \ \ \ \ \ \ \ \ \ \ \ \ \ \ ::\ (Eq\ a)\ =>\ [a]\ ->\ [a]}\\
\mbox{\tt nub\ []\ \ \ \ \ \ \ \ \ \ \ \ \ \ \ \ \ \ =\ \ []}\\
\mbox{\tt nub\ (x:xs)\ \ \ \ \ \ \ \ \ \ \ \ \ \ =\ \ x\ :\ nub\ (remove\ x\ xs)}
\indextt{nub}%
\eprogB\noindent\bprogB
\mbox{\tt partition\ \ \ \ \ \ \ \ \ \ \ \ \ \ \ ::\ (a\ ->\ Bool)\ ->\ [a]\ ->\ ([a],[a])}\\
\mbox{\tt partition\ p\ xs\ \ \ \ \ \ \ \ \ \ =\ \ (filter\ p\ xs,\ filter\ (not\ .\ p)\ xs)}
\indextt{partition}%
\eprogB\noindent\bprogB
\mbox{\tt reverse\ \ \ \ \ \ \ \ \ \ \ \ \ \ \ \ \ ::\ [a]\ ->\ [a]}\\
\mbox{\tt reverse\ \ \ \ \ \ \ \ \ \ \ \ \ \ \ \ \ =\ \ foldl\ ({\char'134}xs\ x\ ->\ x:xs)\ []}
\indextt{reverse}%
\eprogB\noindent\bprogB
\mbox{\tt and,\ or\ \ \ \ \ \ \ \ \ \ \ \ \ \ \ \ \ ::\ [Bool]\ ->\ Bool}\\
\mbox{\tt and\ \ \ \ \ \ \ \ \ \ \ \ \ \ \ \ \ \ \ \ \ =\ \ foldr\ ({\char'46}{\char'46})\ True}\\
\mbox{\tt or\ \ \ \ \ \ \ \ \ \ \ \ \ \ \ \ \ \ \ \ \ \ =\ \ foldr\ (||)\ False}
\indextt{and}%
\indextt{or}%
\eprogB\noindent\bprogB
\mbox{\tt sum,\ product\ \ \ \ \ \ \ \ \ \ \ \ ::\ (Num\ a)\ =>\ [a]\ ->\ a}\\
\mbox{\tt sum\ \ \ \ \ \ \ \ \ \ \ \ \ \ \ \ \ \ \ \ \ =\ \ foldl\ (+)\ 0}\\
\mbox{\tt product\ \ \ \ \ \ \ \ \ \ \ \ \ \ \ \ \ =\ \ foldl\ (*)\ 1}
\indextt{sum}%
\indextt{product}%
\eprogB\noindent\bprogB
\mbox{\tt maximum,\ minimum\ \ \ \ \ \ \ \ ::\ (Ord\ a)\ =>\ [a]\ ->\ a}\\
\mbox{\tt maximum\ \ \ \ \ \ \ \ \ \ \ \ \ \ \ \ \ =\ \ foldl1\ max}\\
\mbox{\tt minimum\ \ \ \ \ \ \ \ \ \ \ \ \ \ \ \ \ =\ \ foldl1\ min}
\indextt{maximum}%
\indextt{minimum}%
\eprogB\noindent\bprogB
\mbox{\tt concat\ \ \ \ \ \ \ \ \ \ \ \ \ \ \ \ \ \ ::\ [[a]]\ ->\ [a]}\\
\mbox{\tt concat\ \ \ \ \ \ \ \ \ \ \ \ \ \ \ \ \ \ =\ \ foldr\ (++)\ []}
\indextt{concat}%
\eprogB\noindent\bprogB
\mbox{\tt zip\ \ \ \ \ \ \ \ \ \ \ \ \ \ \ \ \ \ \ \ \ ::\ [a]\ ->\ [b]\ ->\ [(a,b)]}\\
\mbox{\tt zip\ \ \ \ \ \ \ \ \ \ \ \ \ \ \ \ \ \ \ \ \ =\ \ zipWith\ ({\char'134}a\ b\ ->\ (a,b))}
\indextt{zip}%
\eprogB\noindent\bprogB
\mbox{\tt zip3\ \ \ \ \ \ \ \ \ \ \ \ \ \ \ \ \ \ \ \ ::\ [a]\ ->\ [b]\ ->\ [c]\ ->\ [(a,b,c)]}\\
\mbox{\tt zip3\ \ \ \ \ \ \ \ \ \ \ \ \ \ \ \ \ \ \ \ =\ \ zipWith3\ ({\char'134}a\ b\ c\ ->\ (a,b,c))}
\indextt{zip3}%
\eprogB\noindent\bprogB
\mbox{\tt zip4\ \ \ \ \ \ \ \ \ \ \ \ \ \ \ \ \ \ \ \ ::\ [a]\ ->\ [b]\ ->\ [c]\ ->\ [d]\ ->\ [(a,b,c,d)]}\\
\mbox{\tt zip4\ \ \ \ \ \ \ \ \ \ \ \ \ \ \ \ \ \ \ \ =\ \ zipWith4\ ({\char'134}a\ b\ c\ d\ ->\ (a,b,c,d))}
\indextt{zip4}%
\eprogB\noindent\bprogB
\mbox{\tt zip5\ \ \ \ \ \ \ \ \ \ \ \ \ \ \ \ \ \ \ \ ::\ [a]\ ->\ [b]\ ->\ [c]\ ->\ [d]\ ->\ [e]\ ->\ [(a,b,c,d,e)]}\\
\mbox{\tt zip5\ \ \ \ \ \ \ \ \ \ \ \ \ \ \ \ \ \ \ \ =\ \ zipWith5\ ({\char'134}a\ b\ c\ d\ e\ ->\ (a,b,c,d,e))}
\indextt{zip5}%
\eprogB\noindent\bprogB
\mbox{\tt zip6\ \ \ \ \ \ \ \ \ \ \ \ \ \ \ \ \ \ \ \ ::\ [a]\ ->\ [b]\ ->\ [c]\ ->\ [d]\ ->\ [e]\ ->\ [f]}\\
\mbox{\tt \ \ \ \ \ \ \ \ \ \ \ \ \ \ \ \ \ \ \ \ \ \ \ \ \ \ \ ->\ [(a,b,c,d,e,f)]}\\
\mbox{\tt zip6\ \ \ \ \ \ \ \ \ \ \ \ \ \ \ \ \ \ \ \ =\ \ zipWith6\ ({\char'134}a\ b\ c\ d\ e\ f\ ->\ (a,b,c,d,e,f))}
\indextt{zip6}%
\eprogB\noindent\bprogB
\mbox{\tt zip7\ \ \ \ \ \ \ \ \ \ \ \ \ \ \ \ \ \ \ \ ::\ [a]\ ->\ [b]\ ->\ [c]\ ->\ [d]\ ->\ [e]\ ->\ [f]\ ->\ [g]}\\
\mbox{\tt \ \ \ \ \ \ \ \ \ \ \ \ \ \ \ \ \ \ \ \ \ \ \ \ \ \ \ ->\ [(a,b,c,d,e,f,g)]}\\
\mbox{\tt zip7\ \ \ \ \ \ \ \ \ \ \ \ \ \ \ \ \ \ \ \ =\ \ zipWith7\ ({\char'134}a\ b\ c\ d\ e\ f\ g\ ->\ (a,b,c,d,e,f,g))}
\indextt{zip7}%
\eprogB\noindent\bprogB
\mbox{\tt zipWith\ \ \ \ \ \ \ \ \ \ \ \ \ \ \ \ \ ::\ (a->b->c)\ ->\ [a]->[b]->[c]}\\
\mbox{\tt zipWith\ z\ (a:as)\ (b:bs)\ =\ \ z\ a\ b\ :\ zipWith\ z\ as\ bs}\\
\mbox{\tt zipWith\ {\char'137}\ {\char'137}\ {\char'137}\ \ \ \ \ \ \ \ \ \ \ =\ \ []}
\indextt{zipWith}%
\eprogB\noindent\bprogB
\mbox{\tt zipWith3\ \ \ \ \ \ \ \ \ \ \ \ \ \ \ \ ::\ (a->b->c->d)\ ->\ [a]->[b]->[c]->[d]}\\
\mbox{\tt zipWith3\ z\ (a:as)\ (b:bs)\ (c:cs)}\\
\mbox{\tt \ \ \ \ \ \ \ \ \ \ \ \ \ \ \ \ \ \ \ \ \ \ \ \ =\ \ z\ a\ b\ c\ :\ zipWith3\ z\ as\ bs\ cs}\\
\mbox{\tt zipWith3\ {\char'137}\ {\char'137}\ {\char'137}\ {\char'137}\ \ \ \ \ \ \ \ =\ \ []}
\indextt{zipWith3}%
\eprogB\noindent\bprogB
\mbox{\tt zipWith4\ \ \ \ \ \ \ \ \ \ \ \ \ \ \ \ ::\ (a->b->c->d->e)\ ->\ [a]->[b]->[c]->[d]->[e]}\\
\mbox{\tt zipWith4\ z\ (a:as)\ (b:bs)\ (c:cs)\ (d:ds)}\\
\mbox{\tt \ \ \ \ \ \ \ \ \ \ \ \ \ \ \ \ \ \ \ \ \ \ \ \ =\ \ z\ a\ b\ c\ d\ :\ zipWith4\ z\ as\ bs\ cs\ ds}\\
\mbox{\tt zipWith4\ {\char'137}\ {\char'137}\ {\char'137}\ {\char'137}\ {\char'137}\ \ \ \ \ \ =\ \ []}
\indextt{zipWith4}%
\eprogB\noindent\bprogB
\mbox{\tt zipWith5\ \ \ \ \ \ \ \ \ \ \ \ \ \ \ \ ::\ (a->b->c->d->e->f)}\\
\mbox{\tt \ \ \ \ \ \ \ \ \ \ \ \ \ \ \ \ \ \ \ \ \ \ \ \ \ \ \ ->\ [a]->[b]->[c]->[d]->[e]->[f]}\\
\mbox{\tt zipWith5\ z\ (a:as)\ (b:bs)\ (c:cs)\ (d:ds)\ (e:es)}\\
\mbox{\tt \ \ \ \ \ \ \ \ \ \ \ \ \ \ \ \ \ \ \ \ \ \ \ \ =\ \ z\ a\ b\ c\ d\ e\ :\ zipWith5\ z\ as\ bs\ cs\ ds\ es}\\
\mbox{\tt zipWith5\ {\char'137}\ {\char'137}\ {\char'137}\ {\char'137}\ {\char'137}\ {\char'137}\ \ \ \ =\ \ []}
\indextt{zipWith5}%
\eprogB\noindent\bprogB
\mbox{\tt zipWith6\ \ \ \ \ \ \ \ \ \ \ \ \ \ \ \ ::\ (a->b->c->d->e->f->g)}\\
\mbox{\tt \ \ \ \ \ \ \ \ \ \ \ \ \ \ \ \ \ \ \ \ \ \ \ \ \ \ \ ->\ [a]->[b]->[c]->[d]->[e]->[f]->[g]}\\
\mbox{\tt zipWith6\ z\ (a:as)\ (b:bs)\ (c:cs)\ (d:ds)\ (e:es)\ (f:fs)}\\
\mbox{\tt \ \ \ \ \ \ \ \ \ \ \ \ \ \ \ \ \ \ \ \ \ \ \ \ =\ \ z\ a\ b\ c\ d\ e\ f\ :\ zipWith6\ z\ as\ bs\ cs\ ds\ es\ fs}\\
\mbox{\tt zipWith6\ {\char'137}\ {\char'137}\ {\char'137}\ {\char'137}\ {\char'137}\ {\char'137}\ {\char'137}\ \ =\ \ []}
\indextt{zipWith6}%
\eprogB\noindent\bprogB
\mbox{\tt zipWith7\ \ \ \ \ \ \ \ \ \ \ \ \ \ \ \ ::\ (a->b->c->d->e->f->g->h)}\\
\mbox{\tt \ \ \ \ \ \ \ \ \ \ \ \ \ \ \ \ \ \ \ \ \ \ \ \ \ \ \ ->\ [a]->[b]->[c]->[d]->[e]->[f]->[g]->[h]}\\
\mbox{\tt zipWith7\ z\ (a:as)\ (b:bs)\ (c:cs)\ (d:ds)\ (e:es)\ (f:fs)\ (g:gs)}\\
\mbox{\tt \ \ \ \ \ \ \ \ \ \ \ \ \ \ \ \ \ \ \ =\ \ z\ a\ b\ c\ d\ e\ f\ g\ :\ zipWith7\ z\ as\ bs\ cs\ ds\ es\ fs\ gs}\\
\mbox{\tt zipWith7\ {\char'137}\ {\char'137}\ {\char'137}\ {\char'137}\ {\char'137}\ {\char'137}\ {\char'137}\ {\char'137}\ =\ \ []}
\indextt{zipWith7}%
\eprogB
\clearpage

\subsection{Prelude {\tt PreludeArray}}
\label {preludearray}
\noindent\bprogB
\mbox{\tt module\ \ PreludeArray\ (\ Array,\ Assoc((:=)),\ array,\ listArray,\ (!),\ bounds,}\\
\mbox{\tt \ \ \ \ \ \ \ \ \ \ \ \ \ \ \ \ \ \ \ \ \ indices,\ elems,\ assocs,\ accumArray,\ (//),\ accum,\ amap,}\\
\mbox{\tt \ \ \ \ \ \ \ \ \ \ \ \ \ \ \ \ \ \ \ \ \ ixmap}\\
\mbox{\tt \ \ \ \ \ \ \ \ \ \ \ \ \ \ \ \ \ \ \ )\ where}
\index{PreludeArray@{\ptt PreludeArray} (module)}%
\eprogB\noindent\bprogB
\mbox{\tt --\ This\ module\ specifies\ the\ semantics\ of\ arrays\ only:\ it\ is\ not}\\
\mbox{\tt --\ intended\ as\ an\ efficient\ implementation.}\\
\mbox{\tt }\\[-8pt]
\mbox{\tt infixl\ 9\ \ !}\\
\mbox{\tt infixl\ 9\ \ //}\\
\mbox{\tt infix\ \ 1\ \ :=}
\index{!@{\ptt {\char'041}}}%
\index{//@{\ptt //}}%
\index{:=@{\ptt :=}}%
\eprogB\noindent\bprogB
\mbox{\tt data\ \ Assoc\ a\ b\ =\ \ a\ :=\ b\ \ deriving\ (Eq,\ Ord,\ Ix,\ Text,\ Binary)}\\
\mbox{\tt data\ \ (Ix\ a)\ \ \ \ =>\ Array\ a\ b\ =\ MkArray\ (a,a)\ (a\ ->\ b)\ deriving\ ()}
\index{Assoc@{\ptt Assoc} (datatype)}%
\eprogB\noindent\bprogB
\mbox{\tt array\ \ \ \ \ \ \ \ \ \ \ ::\ (Ix\ a)\ =>\ (a,a)\ ->\ [Assoc\ a\ b]\ ->\ Array\ a\ b}\\
\mbox{\tt listArray\ \ \ \ \ \ \ ::\ (Ix\ a)\ =>\ (a,a)\ ->\ [b]\ ->\ Array\ a\ b}\\
\mbox{\tt (!)\ \ \ \ \ \ \ \ \ \ \ \ \ ::\ (Ix\ a)\ =>\ Array\ a\ b\ ->\ a\ ->\ b}\\
\mbox{\tt bounds\ \ \ \ \ \ \ \ \ \ ::\ (Ix\ a)\ =>\ Array\ a\ b\ ->\ (a,a)}\\
\mbox{\tt indices\ \ \ \ \ \ \ \ \ ::\ (Ix\ a)\ =>\ Array\ a\ b\ ->\ [a]}\\
\mbox{\tt elems\ \ \ \ \ \ \ \ \ \ \ ::\ (Ix\ a)\ =>\ Array\ a\ b\ ->\ [b]}\\
\mbox{\tt assocs\ \ \ \ \ \ \ \ \ \ ::\ (Ix\ a)\ =>\ Array\ a\ b\ ->\ [Assoc\ a\ b]}\\
\mbox{\tt accumArray\ \ \ \ \ \ ::\ (Ix\ a)\ =>\ (b\ ->\ c\ ->\ b)\ ->\ b\ ->\ (a,a)\ ->\ [Assoc\ a\ c]}\\
\mbox{\tt \ \ \ \ \ \ \ \ \ \ \ \ \ \ \ \ \ \ \ \ \ \ \ \ \ \ \ \ \ ->\ Array\ a\ b}\\
\mbox{\tt (//)\ \ \ \ \ \ \ \ \ \ \ \ ::\ (Ix\ a)\ =>\ Array\ a\ b\ ->\ [Assoc\ a\ b]\ ->\ Array\ a\ b}\\
\mbox{\tt accum\ \ \ \ \ \ \ \ \ \ \ ::\ (Ix\ a)\ =>\ (b\ ->\ c\ ->\ b)\ ->\ Array\ a\ b\ ->\ [Assoc\ a\ c]}\\
\mbox{\tt \ \ \ \ \ \ \ \ \ \ \ \ \ \ \ \ \ \ \ \ \ \ \ \ \ \ \ \ \ ->\ Array\ a\ b}\\
\mbox{\tt amap\ \ \ \ \ \ \ \ \ \ \ \ ::\ (Ix\ a)\ =>\ (b\ ->\ c)\ ->\ Array\ a\ b\ ->\ Array\ a\ c}\\
\mbox{\tt ixmap\ \ \ \ \ \ \ \ \ \ \ ::\ (Ix\ a,\ Ix\ b)\ =>\ (a,a)\ ->\ (a\ ->\ b)\ ->\ Array\ b\ c}\\
\mbox{\tt \ \ \ \ \ \ \ \ \ \ \ \ \ \ \ \ \ \ \ \ \ \ \ \ \ \ \ \ \ ->\ Array\ a\ c}
\indextt{array}%
\indextt{listArray}%
\index{!@{\ptt {\char'041}}}%
\indextt{bounds}%
\indextt{indices}%
\indextt{elems}%
\indextt{assocs}%
\indextt{accumArray}%
\index{//@{\ptt //}}%
\indextt{accum}%
\indextt{amap}%
\indextt{ixmap}%
\eprogB\noindent\bprogB
\mbox{\tt array\ b\ ivs\ =}\\
\mbox{\tt \ \ \ \ if\ and\ [inRange\ b\ i\ |\ i:={\char'137}\ <-\ ivs]}\\
\mbox{\tt \ \ \ \ \ \ \ \ then\ MkArray\ b}\\
\mbox{\tt \ \ \ \ \ \ \ \ \ \ \ \ \ \ \ \ \ \ \ \ \ ({\char'134}j\ ->\ case\ [v\ |\ (i\ :=\ v)\ <-\ ivs,\ i\ ==\ j]\ of}\\
\mbox{\tt \ \ \ \ \ \ \ \ \ \ \ \ \ \ \ \ \ \ \ \ \ \ \ \ \ \ \ \ [v]\ \ \ ->\ v}\\
\mbox{\tt \ \ \ \ \ \ \ \ \ \ \ \ \ \ \ \ \ \ \ \ \ \ \ \ \ \ \ \ []\ \ \ \ ->\ error\ "(!){\char'173}PreludeArray{\char'175}:\ {\char'134}}\\
\mbox{\tt \ \ \ \ \ \ \ \ \ \ \ \ \ \ \ \ \ \ \ \ \ \ \ \ \ \ \ \ \ \ \ \ \ \ \ \ \ \ \ \ \ \ \ {\char'134}undefined\ array\ element"}\\
\mbox{\tt \ \ \ \ \ \ \ \ \ \ \ \ \ \ \ \ \ \ \ \ \ \ \ \ \ \ \ \ {\char'137}\ \ \ \ \ ->\ error\ "(!){\char'173}PreludeArray{\char'175}:\ {\char'134}}\\
\mbox{\tt \ \ \ \ \ \ \ \ \ \ \ \ \ \ \ \ \ \ \ \ \ \ \ \ \ \ \ \ \ \ \ \ \ \ \ \ \ \ \ \ \ \ \ {\char'134}multiply\ defined\ array\ element")}\\
\mbox{\tt \ \ \ \ \ \ \ \ else\ error\ "array{\char'173}PreludeArray{\char'175}:\ out-of-range\ array\ association"}
\eprogB\noindent\bprogB
\mbox{\tt listArray\ b\ vs\ \ \ \ \ \ \ \ =\ array\ b\ (zipWith\ (:=)\ (range\ b)\ vs)}
\eprogB\noindent\bprogB
\mbox{\tt (!)\ (MkArray\ {\char'137}\ f)\ \ \ \ \ =\ f}
\eprogB\noindent\bprogB
\mbox{\tt bounds\ (MkArray\ b\ {\char'137})\ \ =\ b}
\eprogB\noindent\bprogB
\mbox{\tt indices\ \ \ \ \ \ \ \ \ \ \ \ \ \ \ =\ range\ .\ bounds}
\eprogB\noindent\bprogB
\mbox{\tt elems\ a\ \ \ \ \ \ \ \ \ \ \ \ \ \ \ =\ [a!i\ |\ i\ <-\ indices\ a]}
\eprogB\noindent\bprogB
\mbox{\tt assocs\ a\ \ \ \ \ \ \ \ \ \ \ \ \ \ =\ [i\ :=\ a!i\ |\ i\ <-\ indices\ a]}
\eprogB\noindent\bprogB
\mbox{\tt a\ //\ us\ \ \ \ \ \ \ \ \ \ \ \ \ \ \ =\ array\ (bounds\ a)}\\
\mbox{\tt \ \ \ \ \ \ \ \ \ \ \ \ \ \ \ \ \ \ \ \ \ \ \ \ \ \ \ \ ([i\ :=\ a!i\ |\ i\ <-\ indices\ a\ {\char'134}{\char'134}\ [i\ |\ i:={\char'137}\ <-\ us]]}\\
\mbox{\tt \ \ \ \ \ \ \ \ \ \ \ \ \ \ \ \ \ \ \ \ \ \ \ \ \ \ \ \ \ ++\ us)}
\eprogB\noindent\bprogB
\mbox{\tt accum\ f\ \ \ \ \ \ \ \ \ \ \ \ \ \ \ =\ foldl\ ({\char'134}a\ (i\ :=\ v)\ ->\ a\ //\ [i\ :=\ f\ (a!i)\ v])}
\eprogB\noindent\bprogB
\mbox{\tt accumArray\ f\ z\ b\ \ \ \ \ \ =\ accum\ f\ (array\ b\ [i\ :=\ z\ |\ i\ <-\ range\ b])}
\eprogB\noindent\bprogB
\mbox{\tt amap\ f\ a\ \ \ \ \ \ \ \ \ \ \ \ \ \ =\ array\ b\ [i\ :=\ f\ (a!i)\ |\ i\ <-\ range\ b]}\\
\mbox{\tt \ \ \ \ \ \ \ \ \ \ \ \ \ \ \ \ \ \ \ \ \ \ \ \ where\ b\ =\ bounds\ a}
\eprogB\noindent\bprogB
\mbox{\tt ixmap\ b\ f\ a\ \ \ \ \ \ \ \ \ \ \ =\ array\ b\ [i\ :=\ a\ !\ f\ i\ |\ i\ <-\ range\ b]}
\eprogB\noindent\bprogB
\mbox{\tt instance\ \ (Ix\ a,\ Eq\ b)\ \ =>\ Eq\ (Array\ a\ b)\ \ where}\\
\mbox{\tt \ \ \ \ a\ ==\ a'\ \ \ \ \ \ \ \ \ \ \ \ \ =\ \ assocs\ a\ ==\ assocs\ a'}
\index{Eq@{\ptt Eq}!instance for {\ptt Array}}%
\eprogB\noindent\bprogB
\mbox{\tt instance\ \ (Ix\ a,\ Ord\ b)\ =>\ Ord\ (Array\ a\ b)\ \ where}\\
\mbox{\tt \ \ \ \ a\ <=\ \ a'\ \ \ \ \ \ \ \ \ \ \ \ =\ \ assocs\ a\ <=\ \ assocs\ a'}
\index{Ord@{\ptt Ord}!instance for {\ptt Array}}%
\eprogB\noindent\bprogB
\mbox{\tt instance\ \ (Ix\ a,\ Text\ a,\ Text\ b)\ =>\ Text\ (Array\ a\ b)\ \ where}\\
\mbox{\tt \ \ \ \ showsPrec\ p\ a\ =\ showParen\ (p\ >\ 9)\ (}\\
\mbox{\tt \ \ \ \ \ \ \ \ \ \ \ \ \ \ \ \ \ \ \ \ showString\ "array\ "\ .}\\
\mbox{\tt \ \ \ \ \ \ \ \ \ \ \ \ \ \ \ \ \ \ \ \ shows\ (bounds\ a)\ .\ showChar\ '\ '\ .}\\
\mbox{\tt \ \ \ \ \ \ \ \ \ \ \ \ \ \ \ \ \ \ \ \ shows\ (assocs\ a)\ \ \ \ \ \ \ \ \ \ \ \ \ \ \ \ \ \ )}\\
\mbox{\tt }\\[-8pt]
\mbox{\tt \ \ \ \ readsPrec\ p\ =\ readParen\ (p\ >\ 9)}\\
\mbox{\tt \ \ \ \ \ \ \ \ \ \ \ ({\char'134}r\ ->\ [(array\ b\ as,\ u)\ |\ ("array",s)\ <-\ [lex\ r],}\\
\mbox{\tt \ \ \ \ \ \ \ \ \ \ \ \ \ \ \ \ \ \ \ \ \ \ \ \ \ \ \ \ \ \ \ \ \ \ \ \ \ (b,t)\ \ \ \ \ \ \ <-\ reads\ s,}\\
\mbox{\tt \ \ \ \ \ \ \ \ \ \ \ \ \ \ \ \ \ \ \ \ \ \ \ \ \ \ \ \ \ \ \ \ \ \ \ \ \ (as,u)\ \ \ \ \ \ <-\ reads\ t\ \ \ ]}\\
\mbox{\tt \ \ \ \ \ \ \ \ \ \ \ \ \ \ \ \ \ \ ++}\\
\mbox{\tt \ \ \ \ \ \ \ \ \ \ \ \ \ \ \ \ \ \ [(listArray\ b\ xs,\ u)\ |\ ("listArray",s)\ <-\ [lex\ r],}\\
\mbox{\tt \ \ \ \ \ \ \ \ \ \ \ \ \ \ \ \ \ \ \ \ \ \ \ \ \ \ \ \ \ \ \ \ \ \ \ \ \ \ \ \ \ (b,t)\ \ \ \ \ \ \ \ \ \ \ <-\ reads\ s,}\\
\mbox{\tt \ \ \ \ \ \ \ \ \ \ \ \ \ \ \ \ \ \ \ \ \ \ \ \ \ \ \ \ \ \ \ \ \ \ \ \ \ \ \ \ \ (xs,u)\ \ \ \ \ \ \ \ \ \ <-\ reads\ t\ ])}
\index{Text@{\ptt Text}!instance for {\ptt Array}}%
\eprogB\noindent\bprogB
\mbox{\tt instance\ \ (Ix\ a,\ Binary\ a,\ Binary\ b)\ =>\ Binary\ (Array\ a\ b)\ \ where}\\
\mbox{\tt \ \ \ \ showBin\ a\ =\ showBin\ (bounds\ a)\ .\ showBin\ (elems\ a)}\\
\mbox{\tt }\\[-8pt]
\mbox{\tt \ \ \ \ readBin\ bin\ =\ (listArray\ b\ vs,\ bin'')}\\
\mbox{\tt \ \ \ \ \ \ \ \ \ \ \ \ \ \ \ \ \ where\ (b,bin')\ \ \ =\ readBin\ bin}\\
\mbox{\tt \ \ \ \ \ \ \ \ \ \ \ \ \ \ \ \ \ \ \ \ \ \ \ (vs,bin'')\ =\ readBin\ bin'}
\index{Binary@{\ptt Binary}!instance for {\ptt Array}}%
\eprogB
\clearpage

\subsection{Prelude {\tt PreludeText}}
\label{preludetext}
\noindent\bprogB
\mbox{\tt module\ \ PreludeText\ (}\\
\mbox{\tt \ \ \ \ \ \ \ \ Text(readsPrec,showsPrec,readList,showList),}\\
\mbox{\tt \ \ \ \ \ \ \ \ ReadS,\ ShowS,\ reads,\ shows,\ show,\ read,\ lex,}\\
\mbox{\tt \ \ \ \ \ \ \ \ showChar,\ showString,\ readParen,\ showParen\ )\ where}
\index{PreludeText@{\ptt PreludeText} (module)}%
\eprogB\noindent\bprogB
\mbox{\tt type\ \ ReadS\ a\ =\ String\ ->\ [(a,String)]}\\
\mbox{\tt type\ \ ShowS\ \ \ =\ String\ ->\ String}
\index{ReadS@{\ptt ReadS} (type synonym)}%
\index{ShowS@{\ptt ShowS} (type synonym)}%
\eprogB\noindent\bprogB
\mbox{\tt class\ \ Text\ a\ \ where}\\
\mbox{\tt \ \ \ \ readsPrec\ ::\ Int\ ->\ ReadS\ a}\\
\mbox{\tt \ \ \ \ showsPrec\ ::\ Int\ ->\ a\ ->\ ShowS}\\
\mbox{\tt \ \ \ \ readList\ \ ::\ ReadS\ [a]}\\
\mbox{\tt \ \ \ \ showList\ \ ::\ [a]\ ->\ ShowS}\\
\mbox{\tt }\\[-8pt]
\mbox{\tt \ \ \ \ readList\ \ \ \ =\ readParen\ False\ ({\char'134}r\ ->\ [pr\ |\ ("[",s)\ \ <-\ lex\ r,}\\
\mbox{\tt \ \ \ \ \ \ \ \ \ \ \ \ \ \ \ \ \ \ \ \ \ \ \ \ \ \ \ \ \ \ \ \ \ \ \ \ \ \ \ \ \ \ \ \ \ \ \ pr\ \ \ \ \ \ \ <-\ readl\ s])}\\
\mbox{\tt \ \ \ \ \ \ \ \ \ \ \ \ \ \ \ \ \ \ where\ readl\ \ s\ =\ [([],t)\ \ \ |\ ("]",t)\ \ <-\ lex\ s]\ ++}\\
\mbox{\tt \ \ \ \ \ \ \ \ \ \ \ \ \ \ \ \ \ \ \ \ \ \ \ \ \ \ \ \ \ \ \ \ \ \ \ [(x:xs,u)\ |\ (x,t)\ \ \ \ <-\ reads\ s,}\\
\mbox{\tt \ \ \ \ \ \ \ \ \ \ \ \ \ \ \ \ \ \ \ \ \ \ \ \ \ \ \ \ \ \ \ \ \ \ \ \ \ \ \ \ \ \ \ \ \ \ \ (xs,u)\ \ \ <-\ readl'\ t]}\\
\mbox{\tt \ \ \ \ \ \ \ \ \ \ \ \ \ \ \ \ \ \ \ \ \ \ \ \ readl'\ s\ =\ [([],t)\ \ \ |\ ("]",t)\ \ <-\ lex\ s]\ ++}\\
\mbox{\tt \ \ \ \ \ \ \ \ \ \ \ \ \ \ \ \ \ \ \ \ \ \ \ \ \ \ \ \ \ \ \ \ \ \ \ [(x:xs,v)\ |\ (",",t)\ \ <-\ lex\ s,}\\
\mbox{\tt \ \ \ \ \ \ \ \ \ \ \ \ \ \ \ \ \ \ \ \ \ \ \ \ \ \ \ \ \ \ \ \ \ \ \ \ \ \ \ \ \ \ \ \ \ \ \ (x,u)\ \ \ \ <-\ read\ t,}\\
\mbox{\tt \ \ \ \ \ \ \ \ \ \ \ \ \ \ \ \ \ \ \ \ \ \ \ \ \ \ \ \ \ \ \ \ \ \ \ \ \ \ \ \ \ \ \ \ \ \ \ (xs,v)\ \ \ <-\ readl'\ u]}\\
\mbox{\tt \ \ \ \ showList\ []\ =\ showString\ "[]"}\\
\mbox{\tt \ \ \ \ showList\ (x:xs)}\\
\mbox{\tt \ \ \ \ \ \ \ \ \ \ \ \ \ \ \ \ =\ showChar\ '['\ .\ shows\ x\ .\ showl\ xs}\\
\mbox{\tt \ \ \ \ \ \ \ \ \ \ \ \ \ \ \ \ \ \ where\ showl\ []\ \ \ \ \ =\ showChar\ ']'}\\
\mbox{\tt \ \ \ \ \ \ \ \ \ \ \ \ \ \ \ \ \ \ \ \ \ \ \ \ showl\ (x:xs)\ =\ showChar\ ','\ .\ shows\ x\ .\ showl\ xs}
\indextt{readsPrec}%
\indextt{showsPrec}%
\indextt{readList}%
\indextt{showList}%
\indextt{Text}%
\eprogB\noindent\bprogB
\mbox{\tt reads\ \ \ \ \ \ \ \ \ \ \ ::\ (Text\ a)\ =>\ ReadS\ a}\\
\mbox{\tt reads\ \ \ \ \ \ \ \ \ \ \ =\ \ readsPrec\ 0}
\indextt{reads}%
\eprogB\noindent\bprogB
\mbox{\tt shows\ \ \ \ \ \ \ \ \ \ \ ::\ (Text\ a)\ =>\ a\ ->\ ShowS}\\
\mbox{\tt shows\ \ \ \ \ \ \ \ \ \ \ =\ \ showsPrec\ 0}
\indextt{shows}%
\eprogB\noindent\bprogB
\mbox{\tt read\ \ \ \ \ \ \ \ \ \ \ \ ::\ (Text\ a)\ =>\ String\ ->\ a}\\
\mbox{\tt read\ s\ \ \ \ \ \ \ \ \ \ =\ \ case\ [x\ |\ (x,t)\ <-\ reads\ s,\ ("","")\ <-\ lex\ t]\ of}\\
\mbox{\tt \ \ \ \ \ \ \ \ \ \ \ \ \ \ \ \ \ \ \ \ \ \ \ \ [x]\ ->\ x}\\
\mbox{\tt \ \ \ \ \ \ \ \ \ \ \ \ \ \ \ \ \ \ \ \ \ \ \ \ []\ \ ->\ error\ "read{\char'173}PreludeText{\char'175}:\ no\ parse"}\\
\mbox{\tt \ \ \ \ \ \ \ \ \ \ \ \ \ \ \ \ \ \ \ \ \ \ \ \ {\char'137}\ \ \ ->\ error\ "read{\char'173}PreludeText{\char'175}:\ ambiguous\ parse"}
\indextt{read}%
\eprogB\noindent\bprogB
\mbox{\tt show\ \ \ \ \ \ \ \ \ \ \ \ ::\ (Text\ a)\ =>\ a\ ->\ String}\\
\mbox{\tt show\ x\ \ \ \ \ \ \ \ \ \ =\ \ shows\ x\ ""}
\indextt{show}%
\eprogB\noindent\bprogB
\mbox{\tt showChar\ \ \ \ \ \ \ \ ::\ Char\ ->\ ShowS}\\
\mbox{\tt showChar\ \ \ \ \ \ \ \ =\ \ (:)}
\indextt{showChar}%
\eprogB\noindent\bprogB
\mbox{\tt showString\ \ \ \ \ \ ::\ String\ ->\ ShowS}\\
\mbox{\tt showString\ \ \ \ \ \ =\ \ (++)}
\indextt{showString}%
\eprogB\noindent\bprogB
\mbox{\tt showParen\ \ \ \ \ \ \ ::\ Bool\ ->\ ShowS\ ->\ ShowS}\\
\mbox{\tt showParen\ b\ p\ \ \ =\ \ if\ b\ then\ showChar\ '('\ .\ p\ .\ showChar\ ')'\ else\ p}
\indextt{showParen}%
\eprogB\noindent\bprogB
\mbox{\tt readParen\ \ \ \ \ \ \ ::\ Bool\ ->\ ReadS\ a\ ->\ ReadS\ a}\\
\mbox{\tt readParen\ b\ g\ \ \ =\ \ if\ b\ then\ mandatory\ else\ optional}\\
\mbox{\tt \ \ \ \ \ \ \ \ \ \ \ \ \ \ \ \ \ \ \ where\ optional\ r\ \ =\ g\ r\ ++\ mandatory\ r}\\
\mbox{\tt \ \ \ \ \ \ \ \ \ \ \ \ \ \ \ \ \ \ \ \ \ \ \ \ \ mandatory\ r\ =\ [(x,u)\ |\ ("(",s)\ <-\ lex\ r,}\\
\mbox{\tt \ \ \ \ \ \ \ \ \ \ \ \ \ \ \ \ \ \ \ \ \ \ \ \ \ \ \ \ \ \ \ \ \ \ \ \ \ \ \ \ \ \ \ \ \ \ \ \ (x,t)\ \ \ <-\ optional\ s,}\\
\mbox{\tt \ \ \ \ \ \ \ \ \ \ \ \ \ \ \ \ \ \ \ \ \ \ \ \ \ \ \ \ \ \ \ \ \ \ \ \ \ \ \ \ \ \ \ \ \ \ \ \ (")",u)\ <-\ lex\ t\ \ \ \ ]}
\indextt{readParen}%
\eprogB\noindent\bprogB
\mbox{\tt lex\ \ \ \ \ \ \ \ \ \ \ \ \ \ \ \ \ \ \ \ \ ::\ ReadS\ String}\\
\mbox{\tt lex\ ""\ \ \ \ \ \ \ \ \ \ \ \ \ \ \ \ \ \ =\ [("","")]}\\
\mbox{\tt lex\ (c:s)\ |\ isSpace\ c\ \ \ =\ lex\ (dropWhile\ isSpace\ s)}\\
\mbox{\tt lex\ ('-':'-':s)\ \ \ \ \ \ \ \ \ =\ case\ dropWhile\ (/=\ '{\char'134}n')\ s\ of}\\
\mbox{\tt \ \ \ \ \ \ \ \ \ \ \ \ \ \ \ \ \ \ \ \ \ \ \ \ \ \ \ \ \ \ \ \ \ '{\char'134}n':t\ ->\ lex\ t}\\
\mbox{\tt \ \ \ \ \ \ \ \ \ \ \ \ \ \ \ \ \ \ \ \ \ \ \ \ \ \ \ \ \ \ \ \ \ {\char'137}\ \ \ \ \ \ ->\ []\ --\ unterminated\ end-of-line}\\
\mbox{\tt \ \ \ \ \ \ \ \ \ \ \ \ \ \ \ \ \ \ \ \ \ \ \ \ \ \ \ \ \ \ \ \ \ \ \ \ \ \ \ \ \ \ \ \ \ \ --\ comment}
\indextt{lex}%
\eprogB\noindent\bprogB
\mbox{\tt lex\ ('{\char'173}':'-':s)\ \ \ \ \ \ \ \ \ =\ lexNest\ lex\ s}\\
\mbox{\tt \ \ \ \ \ \ \ \ \ \ \ \ \ \ \ \ \ \ \ \ \ \ \ \ \ \ where}\\
\mbox{\tt \ \ \ \ \ \ \ \ \ \ \ \ \ \ \ \ \ \ \ \ \ \ \ \ \ \ lexNest\ f\ ('-':'{\char'175}':s)\ =\ f\ s}\\
\mbox{\tt \ \ \ \ \ \ \ \ \ \ \ \ \ \ \ \ \ \ \ \ \ \ \ \ \ \ lexNest\ f\ ('{\char'173}':'-':s)\ =\ lexNest\ (lexNest\ f)\ s}\\
\mbox{\tt \ \ \ \ \ \ \ \ \ \ \ \ \ \ \ \ \ \ \ \ \ \ \ \ \ \ lexNest\ f\ (c:s)\ \ \ \ \ \ \ \ \ \ \ \ \ =\ lexNest\ f\ s}\\
\mbox{\tt \ \ \ \ \ \ \ \ \ \ \ \ \ \ \ \ \ \ \ \ \ \ \ \ \ \ lexNest\ {\char'137}\ ""\ \ \ \ \ \ \ \ \ \ =\ []\ --\ unterminated}\\
\mbox{\tt \ \ \ \ \ \ \ \ \ \ \ \ \ \ \ \ \ \ \ \ \ \ \ \ \ \ \ \ \ \ \ \ \ \ \ \ \ \ \ \ \ \ \ \ \ \ \ \ \ \ \ \ \ --\ nested\ comment}
\eprogB\noindent\bprogB
\mbox{\tt lex\ ('-':'>':s)\ \ \ \ \ \ \ \ \ =\ [("->",s)]}\\
\mbox{\tt lex\ ('<':'-':s)\ \ \ \ \ \ \ \ \ =\ [("<-",s)]}\\
\mbox{\tt lex\ ('{\char'134}'':s)\ \ \ \ \ \ \ \ \ \ \ \ =\ [('{\char'134}'':ch++"'",\ t)\ |\ (ch,t)\ \ <-\ lexLitChar\ s,}\\
\mbox{\tt \ \ \ \ \ \ \ \ \ \ \ \ \ \ \ \ \ \ \ \ \ \ \ \ \ \ \ \ \ \ \ \ \ \ \ \ \ \ \ \ \ \ \ \ \ \ \ ch\ /=\ "'"\ \ \ \ \ \ \ \ \ \ \ \ \ \ \ \ ]}\\
\mbox{\tt lex\ ('"':s)\ \ \ \ \ \ \ \ \ \ \ \ \ =\ [('"':str,\ t)\ \ \ \ \ \ |\ (str,t)\ <-\ lexString\ s]}\\
\mbox{\tt \ \ \ \ \ \ \ \ \ \ \ \ \ \ \ \ \ \ \ \ \ \ \ \ \ \ where}\\
\mbox{\tt \ \ \ \ \ \ \ \ \ \ \ \ \ \ \ \ \ \ \ \ \ \ \ \ \ \ lexString\ ('"':s)\ =\ [("{\char'134}"",s)]}\\
\mbox{\tt \ \ \ \ \ \ \ \ \ \ \ \ \ \ \ \ \ \ \ \ \ \ \ \ \ \ lexString\ s\ =\ [(ch++str,\ u)}\\
\mbox{\tt \ \ \ \ \ \ \ \ \ \ \ \ \ \ \ \ \ \ \ \ \ \ \ \ \ \ \ \ \ \ \ \ \ \ \ \ \ \ \ \ \ \ \ \ \ \ \ \ |\ (ch,t)\ \ <-\ lexStrItem\ s,}\\
\mbox{\tt \ \ \ \ \ \ \ \ \ \ \ \ \ \ \ \ \ \ \ \ \ \ \ \ \ \ \ \ \ \ \ \ \ \ \ \ \ \ \ \ \ \ \ \ \ \ \ \ \ \ (str,u)\ <-\ lexString\ t\ \ ]}\\
\mbox{\tt }\\[-8pt]
\mbox{\tt \ \ \ \ \ \ \ \ \ \ \ \ \ \ \ \ \ \ \ \ \ \ \ \ \ \ lexStrItem\ ('{\char'134}{\char'134}':'{\char'46}':s)\ =\ [("{\char'134}{\char'134}{\char'46}",s)]}\\
\mbox{\tt \ \ \ \ \ \ \ \ \ \ \ \ \ \ \ \ \ \ \ \ \ \ \ \ \ \ lexStrItem\ ('{\char'134}{\char'134}':c:s)\ |\ isSpace\ c}\\
\mbox{\tt \ \ \ \ \ \ \ \ \ \ \ \ \ \ \ \ \ \ \ \ \ \ \ \ \ \ \ \ \ \ =\ [("{\char'134}{\char'134}{\char'46}",t)\ |\ '{\char'134}{\char'134}':t\ <-\ [dropWhile\ isSpace\ s]]}\\
\mbox{\tt \ \ \ \ \ \ \ \ \ \ \ \ \ \ \ \ \ \ \ \ \ \ \ \ \ \ lexStrItem\ s\ \ \ \ \ \ \ \ \ \ \ \ =\ lexLitChar\ s}
\eprogB\noindent\bprogB
\mbox{\tt lex\ (c:s)\ |\ isSingle\ c\ \ =\ [([c],s)]}\\
\mbox{\tt \ \ \ \ \ \ \ \ \ \ |\ isSym1\ c\ \ \ \ =\ [(c:sym,t)\ \ \ \ \ \ \ \ \ |\ (sym,t)\ <-\ [span\ isSym\ s]]}\\
\mbox{\tt \ \ \ \ \ \ \ \ \ \ |\ isAlpha\ c\ \ \ =\ [(c:nam,t)\ \ \ \ \ \ \ \ \ |\ (nam,t)\ <-\ [span\ isIdChar\ s]]}\\
\mbox{\tt \ \ \ \ \ \ \ \ \ \ |\ isDigit\ c\ \ \ =\ [(c:ds++fe,t)\ \ \ \ \ \ |\ (ds,s)\ \ <-\ [span\ isDigit\ s],}\\
\mbox{\tt \ \ \ \ \ \ \ \ \ \ \ \ \ \ \ \ \ \ \ \ \ \ \ \ \ \ \ \ \ \ \ \ \ \ \ \ \ \ \ \ \ \ \ \ \ \ \ (fe,t)\ \ <-\ lexFracExp\ s\ \ \ \ \ ]}\\
\mbox{\tt \ \ \ \ \ \ \ \ \ \ |\ otherwise\ \ \ =\ []\ \ \ \ --\ bad\ character}\\
\mbox{\tt \ \ \ \ \ \ \ \ \ \ \ \ \ \ \ \ where}\\
\mbox{\tt \ \ \ \ \ \ \ \ \ \ \ \ \ \ \ \ isSingle\ c\ \ =\ \ c\ `elem`\ ",;()[]{\char'173}{\char'175}{\char'137}"}\\
\mbox{\tt \ \ \ \ \ \ \ \ \ \ \ \ \ \ \ \ isSym1\ c\ \ \ \ =\ \ c\ `elem`\ "-{\char'176}"\ ||\ isSym\ c}\\
\mbox{\tt \ \ \ \ \ \ \ \ \ \ \ \ \ \ \ \ isSym\ c\ \ \ \ \ =\ \ c\ `elem`\ "!@{\char'43}{\char'44}{\char'45}{\char'46}*+./<=>?{\char'134}{\char'134}{\char'136}|:"}\\
\mbox{\tt \ \ \ \ \ \ \ \ \ \ \ \ \ \ \ \ isIdChar\ c\ \ =\ \ isAlphanum\ c\ ||\ c\ `elem`\ "{\char'137}'"}\\
\mbox{\tt }\\[-8pt]
\mbox{\tt \ \ \ \ \ \ \ \ \ \ \ \ \ \ \ \ lexFracExp\ ('.':s)\ =\ [('.':ds++e,u)\ |\ (ds,t)\ <-\ lexDigits\ s,}\\
\mbox{\tt \ \ \ \ \ \ \ \ \ \ \ \ \ \ \ \ \ \ \ \ \ \ \ \ \ \ \ \ \ \ \ \ \ \ \ \ \ \ \ \ \ \ \ \ \ \ \ \ \ \ \ \ \ \ (e,u)\ \ <-\ lexExp\ t\ \ \ \ ]}\\
\mbox{\tt \ \ \ \ \ \ \ \ \ \ \ \ \ \ \ \ lexFracExp\ s\ \ \ \ \ \ \ =\ [("",s)]}\\
\mbox{\tt }\\[-8pt]
\mbox{\tt \ \ \ \ \ \ \ \ \ \ \ \ \ \ \ \ lexExp\ (e:s)\ |\ e\ `elem`\ "eE"}\\
\mbox{\tt \ \ \ \ \ \ \ \ \ \ \ \ \ \ \ \ \ \ \ \ \ \ \ \ \ =\ [(e:c:ds,u)\ |\ (c:t)\ \ <-\ [s],\ c\ `elem`\ "+-",}\\
\mbox{\tt \ \ \ \ \ \ \ \ \ \ \ \ \ \ \ \ \ \ \ \ \ \ \ \ \ \ \ \ \ \ \ \ \ \ \ \ \ \ \ \ \ \ \ \ \ \ \ \ \ \ \ (ds,u)\ <-\ lexDigits\ t]\ ++}\\
\mbox{\tt \ \ \ \ \ \ \ \ \ \ \ \ \ \ \ \ \ \ \ \ \ \ \ \ \ \ \ [(e:ds,t)\ \ \ |\ (ds,t)\ <-\ lexDigits\ s]}\\
\mbox{\tt \ \ \ \ \ \ \ \ \ \ \ \ \ \ \ \ lexExp\ s\ =\ [("",s)]}
\eprogB\noindent\bprogB
\mbox{\tt lexDigits\ \ \ \ \ \ \ \ \ \ \ \ \ \ \ ::\ ReadS\ String\ }\\
\mbox{\tt lexDigits\ \ \ \ \ \ \ \ \ \ \ \ \ \ \ =\ \ nonnull\ isDigit}
\indextt{lexDigits}%
\eprogB\noindent\bprogB
\mbox{\tt nonnull\ \ \ \ \ \ \ \ \ \ \ \ \ \ \ \ \ ::\ (char\ ->\ Bool)\ ->\ ReadS\ String}\\
\mbox{\tt nonnull\ p\ s\ \ \ \ \ \ \ \ \ \ \ \ \ =\ \ [(cs,t)\ |\ (cs@({\char'137}:{\char'137}),t)\ <-\ [span\ p\ s]]}
\indextt{nonnull}%
\eprogB\noindent\bprogB
\mbox{\tt lexLitChar\ \ \ \ \ \ \ \ \ \ \ \ \ \ ::\ ReadS\ String}\\
\mbox{\tt lexLitChar\ ('{\char'134}{\char'134}':s)\ \ \ \ \ =\ \ [('{\char'134}{\char'134}':esc,\ t)\ |\ (esc,t)\ <-\ lexEsc\ s]}\\
\mbox{\tt \ \ \ \ \ \ \ \ where}\\
\mbox{\tt \ \ \ \ \ \ \ \ lexEsc\ (c:s)\ \ \ \ \ |\ c\ `elem`\ "abfnrtv{\char'134}{\char'134}{\char'134}"'"\ =\ [([c],s)]}\\
\mbox{\tt \ \ \ \ \ \ \ \ lexEsc\ ('{\char'136}':c:s)\ |\ c\ >=\ '@'\ {\char'46}{\char'46}\ c\ <=\ '{\char'137}'\ \ =\ [(['{\char'136}',c],s)]}\\
\mbox{\tt \ \ \ \ \ \ \ \ lexEsc\ s@(d:{\char'137})\ \ \ |\ isDigit\ d\ \ \ \ \ \ \ \ \ \ \ \ \ =\ lexDigits\ s}\\
\mbox{\tt \ \ \ \ \ \ \ \ lexEsc\ ('o':s)\ \ =\ \ [('o':os,\ t)\ |\ (os,t)\ <-\ nonnull\ isOctDigit\ s]}\\
\mbox{\tt \ \ \ \ \ \ \ \ lexEsc\ ('x':s)\ \ =\ \ [('x':xs,\ t)\ |\ (xs,t)\ <-\ nonnull\ isHexDigit\ s]}\\
\mbox{\tt \ \ \ \ \ \ \ \ lexEsc\ s@(c:{\char'137})\ \ \ |\ isUpper\ c}\\
\mbox{\tt \ \ \ \ \ \ \ \ \ \ \ \ \ \ \ \ \ \ \ \ \ \ \ \ =\ \ case\ [(mne,s')\ |\ mne\ <-\ "DEL"\ :\ elems\ asciiTab,}\\
\mbox{\tt \ \ \ \ \ \ \ \ \ \ \ \ \ \ \ \ \ \ \ \ \ \ \ \ \ \ \ \ \ \ \ \ \ \ \ \ \ \ \ \ \ \ \ \ ([],s')\ <-\ [match\ mne\ s]\ \ \ \ \ \ ]}\\
\mbox{\tt \ \ \ \ \ \ \ \ \ \ \ \ \ \ \ \ \ \ \ \ \ \ \ \ \ \ \ of\ (pr:{\char'137})\ ->\ [pr]}\\
\mbox{\tt \ \ \ \ \ \ \ \ \ \ \ \ \ \ \ \ \ \ \ \ \ \ \ \ \ \ \ \ \ \ []\ \ \ \ \ ->\ []}\\
\mbox{\tt \ \ \ \ \ \ \ \ lexEsc\ {\char'137}\ \ \ \ \ \ \ \ =\ \ []}\\
\mbox{\tt lexLitChar\ (c:s)\ \ \ \ \ \ \ \ =\ \ [([c],s)]}
\indextt{lexLitChar}%
\eprogB\noindent\bprogB
\mbox{\tt isOctDigit\ c\ \ =\ \ c\ >=\ '0'\ {\char'46}{\char'46}\ c\ <=\ '7'}\\
\mbox{\tt isHexDigit\ c\ \ =\ \ isDigit\ c\ ||\ c\ >=\ 'A'\ {\char'46}{\char'46}\ c\ <=\ 'F'}\\
\mbox{\tt \ \ \ \ \ \ \ \ \ \ \ \ \ \ \ \ \ \ \ \ \ \ \ \ \ \ \ ||\ c\ >=\ 'a'\ {\char'46}{\char'46}\ c\ <=\ 'f'}
\eprogB\noindent\bprogB
\mbox{\tt match\ \ \ \ \ \ \ \ \ \ \ \ \ \ \ \ \ \ \ ::\ (Eq\ a)\ =>\ [a]\ ->\ [a]\ ->\ ([a],[a])}\\
\mbox{\tt match\ (x:xs)\ (y:ys)\ |\ x\ ==\ y\ \ =\ \ match\ xs\ ys}\\
\mbox{\tt match\ xs\ \ \ \ \ ys\ \ \ \ \ \ \ \ \ \ \ \ \ \ \ =\ \ (xs,ys)}
\indextt{match}%
\eprogB\noindent\bprogB
\mbox{\tt asciiTab\ =\ listArray\ ('{\char'134}NUL',\ '\ ')}\\
\mbox{\tt \ \ \ \ \ \ \ \ \ \ \ ["NUL",\ "SOH",\ "STX",\ "ETX",\ "EOT",\ "ENQ",\ "ACK",\ "BEL",}\\
\mbox{\tt \ \ \ \ \ \ \ \ \ \ \ \ "BS",\ \ "HT",\ \ "LF",\ \ "VT",\ \ "FF",\ \ "CR",\ \ "SO",\ \ "SI",\ }\\
\mbox{\tt \ \ \ \ \ \ \ \ \ \ \ \ "DLE",\ "DC1",\ "DC2",\ "DC3",\ "DC4",\ "NAK",\ "SYN",\ "ETB",}\\
\mbox{\tt \ \ \ \ \ \ \ \ \ \ \ \ "CAN",\ "EM",\ \ "SUB",\ "ESC",\ "FS",\ \ "GS",\ \ "RS",\ \ "US",\ }\\
\mbox{\tt \ \ \ \ \ \ \ \ \ \ \ \ "SP"]\ }
\eprogB\noindent\bprogB
\mbox{\tt --\ Trivial\ type}\\
\mbox{\tt }\\[-8pt]
\mbox{\tt instance\ \ Text\ ()\ \ where}\\
\mbox{\tt \ \ \ \ readsPrec\ p\ \ \ \ =\ readParen\ False}\\
\mbox{\tt \ \ \ \ \ \ \ \ \ \ \ \ \ \ \ \ \ \ \ \ \ \ \ \ \ \ \ \ ({\char'134}r\ ->\ [((),t)\ |\ ("(",s)\ <-\ lex\ r,}\\
\mbox{\tt \ \ \ \ \ \ \ \ \ \ \ \ \ \ \ \ \ \ \ \ \ \ \ \ \ \ \ \ \ \ \ \ \ \ \ \ \ \ \ \ \ \ \ \ \ (")",t)\ <-\ lex\ s\ ]\ )}\\
\mbox{\tt \ \ \ \ showsPrec\ p\ ()\ =\ showString\ "()"}
\eprogB\noindent\bprogB
\mbox{\tt --\ Binary\ type}\\
\mbox{\tt }\\[-8pt]
\mbox{\tt instance\ \ Text\ Bin\ \ where}\\
\mbox{\tt \ \ \ \ readsPrec\ p\ s\ \ =\ \ error\ "readsPrec{\char'173}PreludeText{\char'175}:\ Cannot\ read\ Bin."}\\
\mbox{\tt \ \ \ \ showsPrec\ p\ b\ \ =\ \ showString\ "<<Bin>>"}
\index{Text@{\ptt Text}!instance for {\ptt Bin}}%
\eprogB\noindent\bprogB
\mbox{\tt --\ Character\ type}\\
\mbox{\tt }\\[-8pt]
\mbox{\tt instance\ \ Text\ Char\ \ where}\\
\mbox{\tt \ \ \ \ readsPrec\ p\ \ \ \ \ \ =\ readParen\ False}\\
\mbox{\tt \ \ \ \ \ \ \ \ \ \ \ \ \ \ \ \ \ \ \ \ \ \ \ \ \ \ \ \ ({\char'134}r\ ->\ [(c,t)\ |\ ('{\char'134}'':s,t)<-\ lex\ r,}\\
\mbox{\tt \ \ \ \ \ \ \ \ \ \ \ \ \ \ \ \ \ \ \ \ \ \ \ \ \ \ \ \ \ \ \ \ \ \ \ \ \ \ \ \ \ \ \ \ (c,{\char'137})\ \ \ \ \ <-\ readLitChar\ s])}\\
\mbox{\tt }\\[-8pt]
\mbox{\tt \ \ \ \ showsPrec\ p\ '{\char'134}''\ =\ showString\ "'{\char'134}{\char'134}''"}\\
\mbox{\tt \ \ \ \ showsPrec\ p\ c\ \ \ \ =\ showChar\ '{\char'134}''\ .\ showLitChar\ c\ .\ showChar\ '{\char'134}''}\\
\mbox{\tt }\\[-8pt]
\mbox{\tt \ \ \ \ readList\ =\ readParen\ False\ ({\char'134}r\ ->\ [pr\ |\ ('"':s,\ t)\ <-\ lex\ r,}\\
\mbox{\tt \ \ \ \ \ \ \ \ \ \ \ \ \ \ \ \ \ \ \ \ \ \ \ \ \ \ \ \ \ \ \ \ \ \ \ \ \ \ \ \ \ \ \ \ pr\ <-\ readl\ s\ \ \ \ \ \ \ ])}\\
\mbox{\tt \ \ \ \ \ \ \ \ \ \ \ \ \ \ \ where\ readl\ ('"':s)\ \ \ \ \ \ =\ [("",s)]}\\
\mbox{\tt \ \ \ \ \ \ \ \ \ \ \ \ \ \ \ \ \ \ \ \ \ readl\ ('{\char'134}{\char'134}':'{\char'46}':s)\ =\ readl\ s}\\
\mbox{\tt \ \ \ \ \ \ \ \ \ \ \ \ \ \ \ \ \ \ \ \ \ readl\ s\ \ \ \ \ \ \ \ \ \ \ \ =\ [(c:cs,u)\ |\ (c\ ,t)\ <-\ readLitChar\ s,}\\
\mbox{\tt \ \ \ \ \ \ \ \ \ \ \ \ \ \ \ \ \ \ \ \ \ \ \ \ \ \ \ \ \ \ \ \ \ \ \ \ \ \ \ \ \ \ \ \ \ \ \ \ \ \ \ \ \ \ (cs,u)\ <-\ readl\ t\ \ \ \ \ \ \ ]}\\
\mbox{\tt }\\[-8pt]
\mbox{\tt \ \ \ \ showList\ cs\ =\ showChar\ '"'\ .\ showl\ cs}\\
\mbox{\tt \ \ \ \ \ \ \ \ \ \ \ \ \ \ \ \ \ where\ showl\ ""\ \ \ \ \ \ \ =\ showChar\ '"'}\\
\mbox{\tt \ \ \ \ \ \ \ \ \ \ \ \ \ \ \ \ \ \ \ \ \ \ \ showl\ ('"':cs)\ =\ showString\ "{\char'134}{\char'134}{\char'134}""\ .\ showl\ cs}\\
\mbox{\tt \ \ \ \ \ \ \ \ \ \ \ \ \ \ \ \ \ \ \ \ \ \ \ showl\ (c:cs)\ \ \ =\ showLitChar\ c\ .\ showl\ cs}
\index{Text@{\ptt Text}!instance for {\ptt Char}}%
\eprogB\noindent\bprogB
\mbox{\tt readLitChar\ \ \ \ \ \ \ \ \ \ \ \ \ ::\ ReadS\ Char}\\
\mbox{\tt readLitChar\ ('{\char'134}{\char'134}':s)\ \ \ \ =\ \ readEsc\ s}\\
\mbox{\tt \ \ \ \ \ \ \ \ where}\\
\mbox{\tt \ \ \ \ \ \ \ \ readEsc\ ('a':s)\ \ =\ [('{\char'134}a',s)]}\\
\mbox{\tt \ \ \ \ \ \ \ \ readEsc\ ('b':s)\ \ =\ [('{\char'134}b',s)]}\\
\mbox{\tt \ \ \ \ \ \ \ \ readEsc\ ('f':s)\ \ =\ [('{\char'134}f',s)]}\\
\mbox{\tt \ \ \ \ \ \ \ \ readEsc\ ('n':s)\ \ =\ [('{\char'134}n',s)]}\\
\mbox{\tt \ \ \ \ \ \ \ \ readEsc\ ('r':s)\ \ =\ [('{\char'134}r',s)]}\\
\mbox{\tt \ \ \ \ \ \ \ \ readEsc\ ('t':s)\ \ =\ [('{\char'134}t',s)]}\\
\mbox{\tt \ \ \ \ \ \ \ \ readEsc\ ('v':s)\ \ =\ [('{\char'134}v',s)]}\\
\mbox{\tt \ \ \ \ \ \ \ \ readEsc\ ('{\char'134}{\char'134}':s)\ =\ [('{\char'134}{\char'134}',s)]}\\
\mbox{\tt \ \ \ \ \ \ \ \ readEsc\ ('"':s)\ \ =\ [('"',s)]}\\
\mbox{\tt \ \ \ \ \ \ \ \ readEsc\ ('{\char'134}'':s)\ =\ [('{\char'134}'',s)]}\\
\mbox{\tt \ \ \ \ \ \ \ \ readEsc\ ('{\char'136}':c:s)\ |\ c\ >=\ '@'\ {\char'46}{\char'46}\ c\ <=\ '{\char'137}'}\\
\mbox{\tt \ \ \ \ \ \ \ \ \ \ \ \ \ \ \ \ \ \ \ \ \ \ \ \ \ =\ [(chr\ (ord\ c\ -\ ord\ '@'),\ s)]}\\
\mbox{\tt \ \ \ \ \ \ \ \ readEsc\ s@(d:{\char'137})\ |\ isDigit\ d}\\
\mbox{\tt \ \ \ \ \ \ \ \ \ \ \ \ \ \ \ \ \ \ \ \ \ \ \ \ \ =\ [(chr\ n,\ t)\ |\ (n,t)\ <-\ readDec\ s]}\\
\mbox{\tt \ \ \ \ \ \ \ \ readEsc\ ('o':s)\ \ =\ [(chr\ n,\ t)\ |\ (n,t)\ <-\ readOct\ s]}\\
\mbox{\tt \ \ \ \ \ \ \ \ readEsc\ ('x':s)\ \ =\ [(chr\ n,\ t)\ |\ (n,t)\ <-\ readHex\ s]}\\
\mbox{\tt \ \ \ \ \ \ \ \ readEsc\ s@(c:{\char'137})\ |\ isUpper\ c}\\
\mbox{\tt \ \ \ \ \ \ \ \ \ \ \ \ \ \ \ \ \ \ \ \ \ \ \ \ \ =\ let\ table\ =\ ('{\char'134}DEL'\ :=\ "DEL")\ :\ assocs\ asciiTab}\\
\mbox{\tt \ \ \ \ \ \ \ \ \ \ \ \ \ \ \ \ \ \ \ \ \ \ \ \ \ \ \ in\ case\ [(c,s')\ |\ (c\ :=\ mne)\ <-\ table,}\\
\mbox{\tt \ \ \ \ \ \ \ \ \ \ \ \ \ \ \ \ \ \ \ \ \ \ \ \ \ \ \ \ \ \ \ \ \ \ \ \ \ \ \ \ \ \ \ \ \ ([],s')\ <-\ [match\ mne\ s]]}\\
\mbox{\tt \ \ \ \ \ \ \ \ \ \ \ \ \ \ \ \ \ \ \ \ \ \ \ \ \ \ \ \ \ \ of\ (pr:{\char'137})\ ->\ [pr]}\\
\mbox{\tt \ \ \ \ \ \ \ \ \ \ \ \ \ \ \ \ \ \ \ \ \ \ \ \ \ \ \ \ \ \ \ \ \ []\ \ \ \ \ ->\ []}\\
\mbox{\tt \ \ \ \ \ \ \ \ readEsc\ {\char'137}\ \ \ \ \ \ \ \ =\ []}\\
\mbox{\tt readLitChar\ (c:s)\ \ \ \ \ \ \ =\ \ [(c,s)]}
\indextt{readLitChar}%
\eprogB\noindent\bprogB
\mbox{\tt showLitChar\ \ \ \ \ \ \ \ \ \ \ \ \ \ \ \ ::\ Char\ ->\ ShowS}\\
\mbox{\tt showLitChar\ c\ |\ c\ >\ '{\char'134}DEL'\ =\ \ protectEsc\ isDigit\ (showInt\ (ord\ c))}\\
\mbox{\tt showLitChar\ '{\char'134}DEL'\ \ \ \ \ \ \ \ \ =\ \ showString\ "{\char'134}{\char'134}DEL"}\\
\mbox{\tt showLitChar\ '{\char'134}{\char'134}'\ \ \ \ \ \ \ \ \ \ \ =\ \ showString\ "{\char'134}{\char'134}{\char'134}{\char'134}"}\\
\mbox{\tt showLitChar\ c\ |\ c\ >=\ '\ '\ \ \ =\ \ showChar\ c}\\
\mbox{\tt showLitChar\ '{\char'134}a'\ \ \ \ \ \ \ \ \ \ \ =\ \ showString\ "{\char'134}{\char'134}a"}\\
\mbox{\tt showLitChar\ '{\char'134}b'\ \ \ \ \ \ \ \ \ \ \ =\ \ showString\ "{\char'134}{\char'134}b"}\\
\mbox{\tt showLitChar\ '{\char'134}f'\ \ \ \ \ \ \ \ \ \ \ =\ \ showString\ "{\char'134}{\char'134}f"}\\
\mbox{\tt showLitChar\ '{\char'134}n'\ \ \ \ \ \ \ \ \ \ \ =\ \ showString\ "{\char'134}{\char'134}n"}\\
\mbox{\tt showLitChar\ '{\char'134}r'\ \ \ \ \ \ \ \ \ \ \ =\ \ showString\ "{\char'134}{\char'134}r"}\\
\mbox{\tt showLitChar\ '{\char'134}t'\ \ \ \ \ \ \ \ \ \ \ =\ \ showString\ "{\char'134}{\char'134}t"}\\
\mbox{\tt showLitChar\ '{\char'134}v'\ \ \ \ \ \ \ \ \ \ \ =\ \ showString\ "{\char'134}{\char'134}v"}\\
\mbox{\tt showLitChar\ '{\char'134}SO'\ \ \ \ \ \ \ \ \ \ =\ \ protectEsc\ (==\ 'H')\ (showString\ "{\char'134}{\char'134}SO")}\\
\mbox{\tt showLitChar\ c\ \ \ \ \ \ \ \ \ \ \ \ \ \ =\ \ showString\ ('{\char'134}{\char'134}'\ :\ asciiTab!c)}
\indextt{showLitChar}%
\eprogB\noindent\bprogB
\mbox{\tt protectEsc\ p\ f\ \ \ \ \ \ \ \ \ \ \ \ \ =\ f\ .\ cont}\\
\mbox{\tt \ \ \ \ \ \ \ \ \ \ \ \ \ \ \ \ \ \ \ \ \ \ \ \ \ \ \ \ \ where\ cont\ s@(c:{\char'137})\ |\ p\ c\ =\ "{\char'134}{\char'134}{\char'46}"\ ++\ s}\\
\mbox{\tt \ \ \ \ \ \ \ \ \ \ \ \ \ \ \ \ \ \ \ \ \ \ \ \ \ \ \ \ \ \ \ \ \ \ \ cont\ s\ \ \ \ \ \ \ \ \ \ \ \ \ =\ s}
\eprogB\noindent\bprogB
\mbox{\tt readDec,\ readOct,\ readHex\ ::\ (Integral\ a)\ =>\ ReadS\ a}\\
\mbox{\tt readDec\ =\ readInt\ 10\ isDigit\ ({\char'134}d\ ->\ ord\ d\ -\ ord\ '0')}\\
\mbox{\tt readOct\ =\ readInt\ \ 8\ isOctDigit\ ({\char'134}d\ ->\ ord\ d\ -\ ord\ '0')}\\
\mbox{\tt readHex\ =\ readInt\ 16\ isHexDigit\ hex}\\
\mbox{\tt \ \ \ \ \ \ \ \ \ \ \ \ where\ hex\ d\ =\ ord\ d\ -\ (if\ isDigit\ d\ then\ ord\ '0'}\\
\mbox{\tt \ \ \ \ \ \ \ \ \ \ \ \ \ \ \ \ \ \ \ \ \ \ \ \ \ \ \ \ \ \ \ \ \ \ \ else\ ord\ (if\ isUpper\ d\ then\ 'A'\ else\ 'a')}\\
\mbox{\tt \ \ \ \ \ \ \ \ \ \ \ \ \ \ \ \ \ \ \ \ \ \ \ \ \ \ \ \ \ \ \ \ \ \ \ \ \ \ \ \ -\ 10)}
\indextt{readDec}%
\indextt{readOct}%
\indextt{readHex}%
\eprogB\noindent\bprogB
\mbox{\tt readInt\ ::\ (Integral\ a)\ =>\ a\ ->\ (Char\ ->\ Bool)\ ->\ (Char\ ->\ Int)\ ->\ ReadS\ a}\\
\mbox{\tt readInt\ radix\ isDig\ digToInt\ s\ =}\\
\mbox{\tt \ \ \ \ [(foldl1\ ({\char'134}n\ d\ ->\ n\ *\ radix\ +\ d)\ (map\ (fromIntegral\ .\ digToInt)\ ds),\ r)}\\
\mbox{\tt \ \ \ \ \ \ \ \ |\ (ds,r)\ <-\ nonnull\ isDig\ s\ ]}
\indextt{readInt}%
\eprogB\noindent\bprogB
\mbox{\tt showInt\ ::\ (Integral\ a)\ =>\ a\ ->\ ShowS}\\
\mbox{\tt showInt\ n\ =\ if\ n\ <\ 0\ then\ showChar\ '-'\ .\ showInt'\ (-n)\ else\ showInt'\ n}\\
\mbox{\tt \ \ \ \ \ \ \ \ \ \ \ \ where\ showInt'\ n\ r\ =\ let\ (n',d)\ =\ divRem\ n\ 10}\\
\mbox{\tt \ \ \ \ \ \ \ \ \ \ \ \ \ \ \ \ \ \ \ \ \ \ \ \ \ \ \ \ \ \ \ \ \ \ \ \ \ r'\ =\ chr\ (ord\ '0'\ +\ fromIntegral\ d)\ :\ r}\\
\mbox{\tt \ \ \ \ \ \ \ \ \ \ \ \ \ \ \ \ \ \ \ \ \ \ \ \ \ \ \ \ \ \ \ \ \ in\ if\ n'\ ==\ 0\ then\ r'\ else\ showInt\ n'\ r'}
\indextt{showInt}%
\eprogB\noindent\bprogB
\mbox{\tt --\ Standard\ integral\ types}\\
\mbox{\tt }\\[-8pt]
\mbox{\tt instance\ \ Text\ Int\ \ where}\\
\mbox{\tt \ \ \ \ readsPrec\ p\ =\ readSigned\ readDec}\\
\mbox{\tt \ \ \ \ showsPrec\ \ \ =\ showSigned\ showInt}
\index{Text@{\ptt Text}!instance for {\ptt Int}}%
\eprogB\noindent\bprogB
\mbox{\tt instance\ \ Text\ Integer\ \ where}\\
\mbox{\tt \ \ \ \ readsPrec\ p\ =\ readSigned\ readDec}\\
\mbox{\tt \ \ \ \ showsPrec\ \ \ =\ showSigned\ showInt}
\index{Text@{\ptt Text}!instance for {\ptt Integer}}%
\eprogB\noindent\bprogB
\mbox{\tt readSigned::\ (Real\ a)\ =>\ ReadS\ a\ ->\ ReadS\ a}\\
\mbox{\tt readSigned\ readPos\ =\ readParen\ False\ read'}\\
\mbox{\tt \ \ \ \ \ \ \ \ \ \ \ \ \ \ \ \ \ \ \ \ \ where\ read'\ r\ \ =\ read''\ r\ ++}\\
\mbox{\tt \ \ \ \ \ \ \ \ \ \ \ \ \ \ \ \ \ \ \ \ \ \ \ \ \ \ \ \ \ \ \ \ \ \ \ \ \ \ [(-x,t)\ |\ ("-",s)\ <-\ lex\ r,}\\
\mbox{\tt \ \ \ \ \ \ \ \ \ \ \ \ \ \ \ \ \ \ \ \ \ \ \ \ \ \ \ \ \ \ \ \ \ \ \ \ \ \ \ \ \ \ \ \ \ \ \ \ (x,t)\ \ \ <-\ read''\ s]}\\
\mbox{\tt \ \ \ \ \ \ \ \ \ \ \ \ \ \ \ \ \ \ \ \ \ \ \ \ \ \ \ read''\ r\ =\ [(n,s)\ \ |\ (str,s)\ <-\ lex\ r,}\\
\mbox{\tt \ \ \ \ \ \ \ \ \ \ \ \ \ \ \ \ \ \ \ \ \ \ \ \ \ \ \ \ \ \ \ \ \ \ \ \ \ \ \ \ \ \ \ \ \ \ \ \ (n,"")\ \ <-\ readPos\ str]}
\indextt{readSigned}%
\eprogB\noindent\bprogB
\mbox{\tt showSigned::\ (Real\ a)\ =>\ (a\ ->\ ShowS)\ ->\ Int\ ->\ a\ ->\ ShowS}\\
\mbox{\tt showSigned\ showPos\ p\ x\ =\ showParen\ (x\ <\ 0\ {\char'46}{\char'46}\ p\ >\ 6)\ (showPos\ x)}
\indextt{showSigned}%
\eprogB\noindent\bprogB
\mbox{\tt --\ Standard\ real\ floating-point\ types}\\
\mbox{\tt }\\[-8pt]
\mbox{\tt instance\ \ Text\ Float\ \ where}\\
\mbox{\tt \ \ \ \ readsPrec\ p\ =\ readSigned\ readFloat}\\
\mbox{\tt \ \ \ \ showsPrec\ \ \ =\ showSigned\ showFloat}
\index{Text@{\ptt Text}!instance for {\ptt Float}}%
\eprogB\noindent\bprogB
\mbox{\tt instance\ \ Text\ Double\ \ where}\\
\mbox{\tt \ \ \ \ readsPrec\ p\ =\ readSigned\ readFloat}\\
\mbox{\tt \ \ \ \ showsPrec\ \ \ =\ showSigned\ showFloat}
\index{Text@{\ptt Text}!instance for {\ptt Double}}%
\eprogB\noindent\bprogB
\mbox{\tt --\ The\ functions\ readFloat\ and\ showFloat\ below\ use\ rational\ arithmetic}\\
\mbox{\tt --\ to\ insure\ correct\ conversion\ between\ the\ floating-point\ radix\ and}\\
\mbox{\tt --\ decimal.\ \ It\ is\ often\ possible\ to\ use\ a\ higher-precision\ floating-}\\
\mbox{\tt --\ point\ type\ to\ obtain\ the\ same\ results.}\\
\mbox{\tt }\\[-8pt]
\mbox{\tt readFloat\ r\ =\ [(fromRational\ ((n{\char'45}1)*10{\char'136}{\char'136}(k-d)),\ t)\ |\ (n,d,s)\ <-\ readFix\ r,}\\
\mbox{\tt \ \ \ \ \ \ \ \ \ \ \ \ \ \ \ \ \ \ \ \ \ \ \ \ \ \ \ \ \ \ \ \ \ \ \ \ \ \ \ \ \ \ \ \ \ \ \ \ \ \ \ \ \ (k,t)\ \ \ <-\ readExp\ s]}\\
\mbox{\tt \ \ \ \ \ \ \ \ \ \ \ \ \ \ where\ readFix\ r\ =\ [(read\ (ds++ds'),\ length\ ds',\ t)}\\
\mbox{\tt \ \ \ \ \ \ \ \ \ \ \ \ \ \ \ \ \ \ \ \ \ \ \ \ \ \ \ \ \ \ \ \ \ \ \ \ \ \ \ \ |\ (ds,'.':s)\ <-\ lexDigits\ r,}\\
\mbox{\tt \ \ \ \ \ \ \ \ \ \ \ \ \ \ \ \ \ \ \ \ \ \ \ \ \ \ \ \ \ \ \ \ \ \ \ \ \ \ \ \ \ \ (ds',t)\ \ \ \ <-\ lexDigits\ s\ ]}\\
\mbox{\tt }\\[-8pt]
\mbox{\tt \ \ \ \ \ \ \ \ \ \ \ \ \ \ \ \ \ \ \ \ readExp\ (e:s)\ |\ e\ `elem`\ "eE"\ =\ readExp'\ s}\\
\mbox{\tt \ \ \ \ \ \ \ \ \ \ \ \ \ \ \ \ \ \ \ \ readExp\ s\ \ \ \ \ \ \ \ \ \ \ \ \ \ \ \ \ \ \ \ \ =\ [(0,s)]}\\
\mbox{\tt }\\[-8pt]
\mbox{\tt \ \ \ \ \ \ \ \ \ \ \ \ \ \ \ \ \ \ \ \ readExp'\ ('-':s)\ =\ [(-k,t)\ |\ (k,t)\ <-\ readDec\ s]}\\
\mbox{\tt \ \ \ \ \ \ \ \ \ \ \ \ \ \ \ \ \ \ \ \ readExp'\ ('+':s)\ =\ readDec\ s}\\
\mbox{\tt \ \ \ \ \ \ \ \ \ \ \ \ \ \ \ \ \ \ \ \ readExp'\ s\ \ \ \ \ \ \ =\ readDec\ s}
\eprogB\noindent\bprogB
\mbox{\tt --\ The\ number\ of\ decimal\ digits\ m\ below\ is\ chosen\ to\ guarantee\ }\\
\mbox{\tt --\ read(show\ x)\ =\ x.\ \ See}\\
\mbox{\tt --\ \ \ \ \ \ Matula,\ D.\ W.\ \ A\ formalization\ of\ floating-point\ numeric\ base}\\
\mbox{\tt --\ \ \ \ \ \ conversion.\ \ IEEE\ Transactions\ on\ Computers\ C-19,\ 8\ (1970\ August),}\\
\mbox{\tt --\ \ \ \ \ \ 681-692.}\\
\mbox{\tt }\\[-8pt]
\mbox{\tt showFloat\ x\ =}\\
\mbox{\tt \ \ \ \ if\ x\ ==\ 0\ then\ showString\ ("0."\ ++\ take\ (m-1)\ (repeat\ '0'))}\\
\mbox{\tt \ \ \ \ \ \ \ \ \ \ \ \ \ \ else\ if\ e\ >=\ m-1\ ||\ e\ <\ 0\ then\ showSci\ else\ showFix}\\
\mbox{\tt \ \ \ \ where}\\
\mbox{\tt \ \ \ \ showFix\ \ \ \ \ =\ showString\ whole\ .\ showChar\ '.'\ .\ showString\ frac}\\
\mbox{\tt \ \ \ \ \ \ \ \ \ \ \ \ \ \ \ \ \ \ where\ (whole,frac)\ =\ splitAt\ (e+1)\ (show\ sig)}\\
\mbox{\tt \ \ \ \ showSci\ \ \ \ \ =\ showChar\ d\ .\ showChar\ '.'\ .\ showString\ frac}\\
\mbox{\tt \ \ \ \ \ \ \ \ \ \ \ \ \ \ \ \ \ \ \ \ \ \ .\ showChar\ 'e'\ .\ showInt\ e}\\
\mbox{\tt \ \ \ \ \ \ \ \ \ \ \ \ \ \ \ \ \ \ where\ (d:frac)\ =\ show\ sig}\\
\mbox{\tt \ \ \ \ (m,\ sig,\ e)\ =\ if\ b\ ==\ 10\ then\ (w,\ \ \ s,\ \ \ n+w-1)}\\
\mbox{\tt \ \ \ \ \ \ \ \ \ \ \ \ \ \ \ \ \ \ \ \ \ \ \ \ \ \ \ \ \ else\ (m',\ sig',\ e'\ \ \ )}\\
\mbox{\tt \ \ \ \ m'\ \ \ \ \ \ \ \ \ \ =\ ceiling\ ((fromInt\ w\ *\ log\ (fromInteger\ b))/log\ 10)\ +\ 1}\\
\mbox{\tt \ \ \ \ (sig',\ e')\ \ =\ if\ \ \ \ \ \ sig1\ >=\ 10{\char'136}m'\ \ \ \ \ then\ (round\ (t/10),\ e1+1)}\\
\mbox{\tt \ \ \ \ \ \ \ \ \ \ \ \ \ \ \ \ \ \ else\ if\ sig1\ <\ \ 10{\char'136}(m'-1)\ then\ (round\ (t*10),\ e1-1)}\\
\mbox{\tt \ \ \ \ \ \ \ \ \ \ \ \ \ \ \ \ \ \ \ \ \ \ \ \ \ \ \ \ \ \ \ \ \ \ \ \ \ \ \ \ \ \ \ \ else\ (sig1,\ \ \ \ \ \ \ \ \ e1\ \ )}\\
\mbox{\tt \ \ \ \ sig1\ \ \ \ \ \ \ \ =\ round\ t}\\
\mbox{\tt \ \ \ \ t\ \ \ \ \ \ \ \ \ \ \ =\ s{\char'45}1\ *\ (b{\char'45}1){\char'136}{\char'136}n\ *\ 10{\char'136}{\char'136}(m'-e1-1)}\\
\mbox{\tt \ \ \ \ e1\ \ \ \ \ \ \ \ \ \ =\ floor\ (logBase\ 10\ x)}\\
\mbox{\tt \ \ \ \ (s,\ n)\ \ \ \ \ \ =\ decodeFloat\ x}\\
\mbox{\tt \ \ \ \ b\ \ \ \ \ \ \ \ \ \ \ =\ floatRadix\ x}\\
\mbox{\tt \ \ \ \ w\ \ \ \ \ \ \ \ \ \ \ =\ floatDigits\ x}
\eprogB\noindent\bprogB
\mbox{\tt --\ Lists}\\
\mbox{\tt }\\[-8pt]
\mbox{\tt instance\ \ (Text\ a)\ =>\ Text\ [a]\ \ where}\\
\mbox{\tt \ \ \ \ readsPrec\ p\ =\ readList}\\
\mbox{\tt \ \ \ \ showsPrec\ p\ =\ showList}
\eprogB\noindent\bprogB
\mbox{\tt --\ Tuples}\\
\mbox{\tt }\\[-8pt]
\mbox{\tt instance\ \ (Text\ a,\ Text\ b)\ =>\ Text\ (a,b)\ \ where}\\
\mbox{\tt \ \ \ \ readsPrec\ p\ =\ readParen\ False}\\
\mbox{\tt \ \ \ \ \ \ \ \ \ \ \ \ \ \ \ \ \ \ \ \ \ \ \ \ \ \ \ \ ({\char'134}r\ ->\ [((x,y),\ w)\ |\ ("(",s)\ <-\ lex\ r,}\\
\mbox{\tt \ \ \ \ \ \ \ \ \ \ \ \ \ \ \ \ \ \ \ \ \ \ \ \ \ \ \ \ \ \ \ \ \ \ \ \ \ \ \ \ \ \ \ \ \ \ \ \ \ (x,t)\ \ \ <-\ reads\ s,}\\
\mbox{\tt \ \ \ \ \ \ \ \ \ \ \ \ \ \ \ \ \ \ \ \ \ \ \ \ \ \ \ \ \ \ \ \ \ \ \ \ \ \ \ \ \ \ \ \ \ \ \ \ \ (",",u)\ <-\ lex\ t,}\\
\mbox{\tt \ \ \ \ \ \ \ \ \ \ \ \ \ \ \ \ \ \ \ \ \ \ \ \ \ \ \ \ \ \ \ \ \ \ \ \ \ \ \ \ \ \ \ \ \ \ \ \ \ (y,v)\ \ \ <-\ reads\ u,}\\
\mbox{\tt \ \ \ \ \ \ \ \ \ \ \ \ \ \ \ \ \ \ \ \ \ \ \ \ \ \ \ \ \ \ \ \ \ \ \ \ \ \ \ \ \ \ \ \ \ \ \ \ \ (")",w)\ <-\ lex\ v\ ]\ )}\\
\mbox{\tt }\\[-8pt]
\mbox{\tt \ \ \ \ showsPrec\ p\ (x,y)\ =\ showChar\ '('\ .\ shows\ x\ .\ showChar\ ','\ .}\\
\mbox{\tt \ \ \ \ \ \ \ \ \ \ \ \ \ \ \ \ \ \ \ \ \ \ \ \ \ \ \ \ \ \ \ \ \ \ \ \ \ \ \ shows\ y\ .\ showChar\ ')'}\\
\mbox{\tt --\ et\ cetera}\\
\mbox{\tt }\\[-8pt]
\mbox{\tt }\\[-8pt]
\mbox{\tt --\ Functions}\\
\mbox{\tt }\\[-8pt]
\mbox{\tt instance\ \ Text\ (a\ ->\ b)\ \ where}\\
\mbox{\tt \ \ \ \ readsPrec\ p\ s\ \ =\ \ error\ "readsPrec{\char'173}PreludeText{\char'175}:\ Cannot\ read\ functions."}\\
\mbox{\tt \ \ \ \ showsPrec\ p\ f\ \ =\ \ showString\ "<<function>>"}
\eprogB
\clearpage

\subsection{Prelude {\tt PreludeIO}}
\label{preludeio}
\noindent\bprogB
\mbox{\tt --\ I/O\ functions\ and\ definitions}\\
\mbox{\tt }\\
\mbox{\tt module\ PreludeIO\ \ where}
\eprogB\noindent\bprogB
\mbox{\tt --\ File\ and\ channel\ names:}\\
\mbox{\tt }\\
\mbox{\tt type\ \ Name\ \ =\ String}
\eprogB\noindent\bprogB
\mbox{\tt stdin\ \ \ \ \ \ \ =\ \ "stdin"}\\
\mbox{\tt stdout\ \ \ \ \ \ =\ \ "stdout"}\\
\mbox{\tt stderr\ \ \ \ \ \ =\ \ "stderr"}\\
\mbox{\tt stdecho\ \ \ \ \ =\ \ "stdecho"}
\eprogB\noindent\bprogB
\mbox{\tt --\ Requests\ and\ responses:}\\
\mbox{\tt }\\
\mbox{\tt data\ Request\ =\ \ --\ file\ system\ requests:}\\
\mbox{\tt \ \ \ \ \ \ \ \ \ \ \ \ \ \ \ \ \ \ \ \ \ \ \ \ \ \ ReadFile\ \ \ \ \ \ Name\ \ \ \ \ \ \ \ \ }\\
\mbox{\tt \ \ \ \ \ \ \ \ \ \ \ \ \ \ \ \ \ \ \ \ \ \ \ \ |\ WriteFile\ \ \ \ \ Name\ String}\\
\mbox{\tt \ \ \ \ \ \ \ \ \ \ \ \ \ \ \ \ \ \ \ \ \ \ \ \ |\ AppendFile\ \ \ \ Name\ String}\\
\mbox{\tt \ \ \ \ \ \ \ \ \ \ \ \ \ \ \ \ \ \ \ \ \ \ \ \ |\ ReadBinFile\ \ \ Name\ }\\
\mbox{\tt \ \ \ \ \ \ \ \ \ \ \ \ \ \ \ \ \ \ \ \ \ \ \ \ |\ WriteBinFile\ \ Name\ Bin}\\
\mbox{\tt \ \ \ \ \ \ \ \ \ \ \ \ \ \ \ \ \ \ \ \ \ \ \ \ |\ AppendBinFile\ Name\ Bin}\\
\mbox{\tt \ \ \ \ \ \ \ \ \ \ \ \ \ \ \ \ \ \ \ \ \ \ \ \ |\ DeleteFile\ \ \ \ Name}\\
\mbox{\tt \ \ \ \ \ \ \ \ \ \ \ \ \ \ \ \ \ \ \ \ \ \ \ \ |\ StatusFile\ \ \ \ Name}\\
\mbox{\tt \ \ \ \ \ \ \ \ \ \ \ \ \ \ \ \ --\ channel\ system\ requests:}\\
\mbox{\tt \ \ \ \ \ \ \ \ \ \ \ \ \ \ \ \ \ \ \ \ \ \ \ \ |\ ReadChan\ \ \ \ \ \ \ \ \ \ \ Name\ }\\
\mbox{\tt \ \ \ \ \ \ \ \ \ \ \ \ \ \ \ \ \ \ \ \ \ \ \ \ |\ AppendChan\ \ \ \ Name\ String}\\
\mbox{\tt \ \ \ \ \ \ \ \ \ \ \ \ \ \ \ \ \ \ \ \ \ \ \ \ |\ ReadBinChan\ \ \ Name\ }\\
\mbox{\tt \ \ \ \ \ \ \ \ \ \ \ \ \ \ \ \ \ \ \ \ \ \ \ \ |\ AppendBinChan\ Name\ Bin}\\
\mbox{\tt \ \ \ \ \ \ \ \ \ \ \ \ \ \ \ \ \ \ \ \ \ \ \ \ |\ StatusChan\ \ \ \ Name}\\
\mbox{\tt \ \ \ \ \ \ \ \ \ \ \ \ \ \ \ \ --\ environment\ requests:}\\
\mbox{\tt \ \ \ \ \ \ \ \ \ \ \ \ \ \ \ \ \ \ \ \ \ \ \ \ |\ Echo\ \ \ \ \ \ \ \ \ \ Bool}\\
\mbox{\tt \ \ \ \ \ \ \ \ \ \ \ \ \ \ \ \ \ \ \ \ \ \ \ \ |\ GetArgs}\\
\mbox{\tt \ \ \ \ \ \ \ \ \ \ \ \ \ \ \ \ \ \ \ \ \ \ \ \ |\ GetEnv\ \ \ \ \ \ \ \ Name}\\
\mbox{\tt \ \ \ \ \ \ \ \ \ \ \ \ \ \ \ \ \ \ \ \ \ \ \ \ |\ SetEnv\ \ \ \ \ \ \ \ Name\ String}
\eprogB\noindent\bprogB
\mbox{\tt data\ Response\ =\ \ \ \ \ \ \ \ \ \ \ Success}\\
\mbox{\tt \ \ \ \ \ \ \ \ \ \ \ \ \ \ \ \ \ \ \ \ \ \ \ \ |\ Str\ String\ }\\
\mbox{\tt \ \ \ \ \ \ \ \ \ \ \ \ \ \ \ \ \ \ \ \ \ \ \ \ |\ Bn\ \ Bin}\\
\mbox{\tt \ \ \ \ \ \ \ \ \ \ \ \ \ \ \ \ \ \ \ \ \ \ \ \ |\ Failure\ IOError}
\eprogB\noindent\bprogB
\mbox{\tt data\ IOError\ =\ \ \ \ \ \ \ \ \ \ \ \ WriteError\ \ \ String}\\
\mbox{\tt \ \ \ \ \ \ \ \ \ \ \ \ \ \ \ \ \ \ \ \ \ \ \ \ |\ ReadError\ \ \ \ String}\\
\mbox{\tt \ \ \ \ \ \ \ \ \ \ \ \ \ \ \ \ \ \ \ \ \ \ \ \ |\ SearchError\ \ String}\\
\mbox{\tt \ \ \ \ \ \ \ \ \ \ \ \ \ \ \ \ \ \ \ \ \ \ \ \ |\ FormatError\ \ String}\\
\mbox{\tt \ \ \ \ \ \ \ \ \ \ \ \ \ \ \ \ \ \ \ \ \ \ \ \ |\ OtherError\ \ \ String}
\eprogB\noindent\bprogB
\mbox{\tt --\ Continuation-based\ I/O:}\\
\mbox{\tt }\\
\mbox{\tt type\ Dialogue\ \ \ \ =\ \ [Response]\ ->\ [Request]}\\
\mbox{\tt type\ SuccCont\ \ \ \ =\ \ \ \ \ \ \ \ \ \ \ \ \ \ \ \ Dialogue}\\
\mbox{\tt type\ StrCont\ \ \ \ \ =\ \ String\ \ \ \ \ ->\ Dialogue}\\
\mbox{\tt type\ BinCont\ \ \ \ \ =\ \ Bin\ \ \ \ \ \ \ \ ->\ Dialogue}\\
\mbox{\tt type\ FailCont\ \ \ \ =\ \ IOError\ \ \ \ ->\ Dialogue}
\eprogB\noindent\bprogB
\mbox{\tt done\ \ \ \ \ \ \ \ \ \ ::\ \ \ \ \ \ \ \ \ \ \ \ \ \ \ \ \ \ \ \ \ \ \ \ \ \ \ \ \ \ \ \ \ \ \ \ \ \ \ \ \ \ \ Dialogue}\\
\mbox{\tt readFile\ \ \ \ \ \ ::\ Name\ ->\ \ \ \ \ \ \ \ \ \ \ FailCont\ ->\ StrCont\ \ ->\ Dialogue}\\
\mbox{\tt writeFile\ \ \ \ \ ::\ Name\ ->\ String\ ->\ FailCont\ ->\ SuccCont\ ->\ Dialogue}\\
\mbox{\tt appendFile\ \ \ \ ::\ Name\ ->\ String\ ->\ FailCont\ ->\ SuccCont\ ->\ Dialogue}\\
\mbox{\tt readBinFile\ \ \ ::\ Name\ ->\ \ \ \ \ \ \ \ \ \ \ FailCont\ ->\ BinCont\ \ ->\ Dialogue}\\
\mbox{\tt writeBinFile\ \ ::\ Name\ ->\ Bin\ \ \ \ ->\ FailCont\ ->\ SuccCont\ ->\ Dialogue}\\
\mbox{\tt appendBinFile\ ::\ Name\ ->\ Bin\ \ \ \ ->\ FailCont\ ->\ SuccCont\ ->\ Dialogue}\\
\mbox{\tt deleteFile\ \ \ \ ::\ Name\ ->\ \ \ \ \ \ \ \ \ \ \ FailCont\ ->\ SuccCont\ ->\ Dialogue}\\
\mbox{\tt statusFile\ \ \ \ ::\ Name\ ->\ \ \ \ \ \ \ \ \ \ \ FailCont\ ->\ StrCont\ \ ->\ Dialogue}\\
\mbox{\tt readChan\ \ \ \ \ \ ::\ Name\ ->\ \ \ \ \ \ \ \ \ \ \ FailCont\ ->\ StrCont\ \ ->\ Dialogue}\\
\mbox{\tt appendChan\ \ \ \ ::\ Name\ ->\ String\ ->\ FailCont\ ->\ SuccCont\ ->\ Dialogue}\\
\mbox{\tt readBinChan\ \ \ ::\ Name\ ->\ \ \ \ \ \ \ \ \ \ \ FailCont\ ->\ BinCont\ \ ->\ Dialogue}\\
\mbox{\tt appendBinChan\ ::\ Name\ ->\ Bin\ \ \ \ ->\ FailCont\ ->\ SuccCont\ ->\ Dialogue}\\
\mbox{\tt echo\ \ \ \ \ \ \ \ \ \ ::\ Bool\ ->\ \ \ \ \ \ \ \ \ \ \ FailCont\ ->\ SuccCont\ ->\ Dialogue}\\
\mbox{\tt getArgs\ \ \ \ \ \ \ ::\ \ \ \ \ \ \ \ \ \ \ \ \ \ \ \ \ \ \ FailCont\ ->\ StrCont\ \ ->\ Dialogue}\\
\mbox{\tt getEnv\ \ \ \ \ \ \ \ ::\ Name\ ->\ \ \ \ \ \ \ \ \ \ \ FailCont\ ->\ StrCont\ \ ->\ Dialogue}\\
\mbox{\tt setEnv\ \ \ \ \ \ \ \ ::\ Name\ ->\ String\ ->\ FailCont\ ->\ SuccCont\ ->\ Dialogue}
\indextt{done}%
\indextt{readFile}%
\indextt{writeFile}%
\indextt{appendFile}%
\indextt{readBinFile}%
\indextt{writeBinFile}%
\indextt{appendBinFile}%
\indextt{deleteFile}%
\indextt{statusFile}%
\indextt{readChan}%
\indextt{appendChan}%
\indextt{readBinChan}%
\indextt{appendBinChan}%
\indextt{echo}%
\indextt{getArgs}%
\indextt{getEnv}%
\indextt{setEnv}%
\eprogB\noindent\bprogB
\mbox{\tt done\ resps\ \ \ \ =\ \ []}
\eprogB\noindent\bprogB
\mbox{\tt readFile\ name\ fail\ succ\ resps\ =}\\
\mbox{\tt \ \ \ \ \ (ReadFile\ name)\ :\ strDispatch\ fail\ succ\ resps}
\eprogB\noindent\bprogB
\mbox{\tt writeFile\ name\ contents\ fail\ succ\ resps\ =}\\
\mbox{\tt \ \ \ \ (WriteFile\ name\ contents)\ :\ succDispatch\ fail\ succ\ resps}
\eprogB\noindent\bprogB
\mbox{\tt appendFile\ name\ contents\ fail\ succ\ resps\ =}\\
\mbox{\tt \ \ \ \ (AppendFile\ name\ contents)\ :\ succDispatch\ fail\ succ\ resps}
\eprogB\noindent\bprogB
\mbox{\tt readBinFile\ name\ fail\ succ\ resps\ =}\\
\mbox{\tt \ \ \ \ (ReadBinFile\ name)\ :\ binDispatch\ fail\ succ\ resps}
\eprogB\noindent\bprogB
\mbox{\tt writeBinFile\ name\ contents\ fail\ succ\ resps\ =}\\
\mbox{\tt \ \ \ \ (WriteBinFile\ name\ contents)\ :\ succDispatch\ fail\ succ\ resps}
\eprogB\noindent\bprogB
\mbox{\tt appendBinFile\ name\ contents\ fail\ succ\ resps\ =}\\
\mbox{\tt \ \ \ \ (AppendBinFile\ name\ contents)\ :\ succDispatch\ fail\ succ\ resps}
\eprogB\noindent\bprogB
\mbox{\tt deleteFile\ name\ fail\ succ\ resps\ =}\\
\mbox{\tt \ \ \ \ (DeleteFile\ name)\ :\ succDispatch\ fail\ succ\ resps}
\eprogB\noindent\bprogB
\mbox{\tt statusFile\ name\ fail\ succ\ resps\ =}\\
\mbox{\tt \ \ \ \ (StatusFile\ name)\ :\ strDispatch\ fail\ succ\ resps}
\eprogB\noindent\bprogB
\mbox{\tt readChan\ name\ fail\ succ\ resps\ =}\\
\mbox{\tt \ \ \ \ (ReadChan\ name)\ :\ strDispatch\ fail\ succ\ resps}
\eprogB\noindent\bprogB
\mbox{\tt appendChan\ name\ contents\ fail\ succ\ resps\ =}\\
\mbox{\tt \ \ \ \ (AppendChan\ name\ contents)\ :\ succDispatch\ fail\ succ\ resps}
\eprogB\noindent\bprogB
\mbox{\tt readBinChan\ name\ fail\ succ\ resps\ =}\\
\mbox{\tt \ \ \ \ (ReadBinChan\ name)\ :\ binDispatch\ fail\ succ\ resps}
\eprogB\noindent\bprogB
\mbox{\tt appendBinChan\ name\ contents\ fail\ succ\ resps\ =}\\
\mbox{\tt \ \ \ \ (AppendBinChan\ name\ contents)\ :\ succDispatch\ fail\ succ\ resps}
\eprogB\noindent\bprogB
\mbox{\tt echo\ bool\ fail\ succ\ resps\ =}\\
\mbox{\tt \ \ \ \ (Echo\ bool)\ :\ succDispatch\ fail\ succ\ resps}
\eprogB\noindent\bprogB
\mbox{\tt getArgs\ fail\ succ\ resps\ =}\\
\mbox{\tt \ \ \ \ \ \ \ \ GetArgs\ :\ strDispatch\ fail\ succ\ resps}
\eprogB\noindent\bprogB
\mbox{\tt getEnv\ name\ fail\ succ\ resps\ =}\\
\mbox{\tt \ \ \ \ \ \ \ \ (GetEnv\ name)\ :\ strDispatch\ fail\ succ\ resps}
\eprogB\noindent\bprogB
\mbox{\tt setEnv\ name\ val\ fail\ succ\ resps\ =}\\
\mbox{\tt \ \ \ \ \ \ \ \ (SetEnv\ name\ val)\ :\ succDispatch\ fail\ succ\ resps}
\eprogB\noindent\bprogB
\mbox{\tt strDispatch\ \ fail\ succ\ (resp:resps)\ =\ case\ resp\ of\ }\\
\mbox{\tt \ \ \ \ \ \ \ \ \ \ \ \ \ \ \ \ \ \ \ \ \ \ \ \ \ \ \ \ \ \ \ \ \ \ \ \ \ \ \ \ Str\ val\ \ \ \ \ \ ->\ succ\ val\ resps}\\
\mbox{\tt \ \ \ \ \ \ \ \ \ \ \ \ \ \ \ \ \ \ \ \ \ \ \ \ \ \ \ \ \ \ \ \ \ \ \ \ \ \ \ \ Failure\ msg\ \ ->\ fail\ msg\ resps}
\eprogB\noindent\bprogB
\mbox{\tt binDispatch\ \ fail\ succ\ (resp:resps)\ =\ case\ resp\ of\ }\\
\mbox{\tt \ \ \ \ \ \ \ \ \ \ \ \ \ \ \ \ \ \ \ \ \ \ \ \ \ \ \ \ \ \ \ \ \ \ \ \ \ \ \ \ Bn\ val\ \ \ \ \ \ \ ->\ succ\ val\ resps}\\
\mbox{\tt \ \ \ \ \ \ \ \ \ \ \ \ \ \ \ \ \ \ \ \ \ \ \ \ \ \ \ \ \ \ \ \ \ \ \ \ \ \ \ \ Failure\ msg\ \ ->\ fail\ msg\ resps}
\eprogB\noindent\bprogB
\mbox{\tt succDispatch\ fail\ succ\ (resp:resps)\ =\ case\ resp\ of}\\
\mbox{\tt \ \ \ \ \ \ \ \ \ \ \ \ \ \ \ \ \ \ \ \ \ \ \ \ \ \ \ \ \ \ \ \ \ \ \ \ \ \ \ \ Success\ \ \ \ \ ->\ succ\ resps}\\
\mbox{\tt \ \ \ \ \ \ \ \ \ \ \ \ \ \ \ \ \ \ \ \ \ \ \ \ \ \ \ \ \ \ \ \ \ \ \ \ \ \ \ \ Failure\ msg\ ->\ fail\ msg\ resps}
\eprogB\noindent\bprogB
\mbox{\tt abort\ \ \ \ \ \ \ \ \ \ \ ::\ FailCont}\\
\mbox{\tt abort\ msg\ \ \ \ \ \ \ =\ \ done}
\indextt{abort}%
\eprogB\noindent\bprogB
\mbox{\tt exit\ \ \ \ \ \ \ \ \ \ \ \ ::\ FailCont}\\
\mbox{\tt exit\ err\ \ \ \ \ \ \ \ =\ appendChan\ stdout\ msg\ abort\ done}\\
\mbox{\tt \ \ \ \ \ \ \ \ \ \ \ \ \ \ \ \ \ \ where\ msg\ =\ case\ err\ of\ ReadError\ s\ \ \ ->\ s}\\
\mbox{\tt \ \ \ \ \ \ \ \ \ \ \ \ \ \ \ \ \ \ \ \ \ \ \ \ \ \ \ \ \ \ \ \ \ \ \ \ \ \ \ \ \ \ WriteError\ s\ \ ->\ s}\\
\mbox{\tt \ \ \ \ \ \ \ \ \ \ \ \ \ \ \ \ \ \ \ \ \ \ \ \ \ \ \ \ \ \ \ \ \ \ \ \ \ \ \ \ \ \ SearchError\ s\ ->\ s}\\
\mbox{\tt \ \ \ \ \ \ \ \ \ \ \ \ \ \ \ \ \ \ \ \ \ \ \ \ \ \ \ \ \ \ \ \ \ \ \ \ \ \ \ \ \ \ FormatError\ s\ ->\ s}\\
\mbox{\tt \ \ \ \ \ \ \ \ \ \ \ \ \ \ \ \ \ \ \ \ \ \ \ \ \ \ \ \ \ \ \ \ \ \ \ \ \ \ \ \ \ \ OtherError\ s\ \ ->\ s}
\indextt{exit}%
\eprogB\noindent\bprogB
\mbox{\tt let\ \ \ \ \ \ \ \ \ \ \ \ \ ::\ \ a\ ->\ (a\ ->\ b)\ ->\ b}\\
\mbox{\tt let\ x\ k\ \ \ \ \ \ \ \ \ =\ \ \ k\ x}
\indextt{let}%
\eprogB\noindent\bprogB
\mbox{\tt print\ \ \ \ \ \ \ \ \ \ \ ::\ (Text\ a)\ =>\ a\ ->\ Dialogue}\\
\mbox{\tt print\ x\ \ \ \ \ \ \ \ \ =\ \ appendChan\ stdout\ (show\ x)\ abort\ done}\\
\mbox{\tt prints\ \ \ \ \ \ \ \ \ \ ::\ (Text\ a)\ =>\ a\ ->\ String\ ->\ Dialogue}\\
\mbox{\tt prints\ x\ s\ \ \ \ \ \ =\ \ appendChan\ stdout\ (shows\ x\ s)\ abort\ done}
\indextt{print}%
\indextt{prints}%
\eprogB\noindent\bprogB
\mbox{\tt interact\ \ \ \ \ \ \ \ ::\ (String\ ->\ String)\ ->\ Dialogue}\\
\mbox{\tt interact\ f\ \ \ \ \ \ =\ \ readChan\ stdin\ abort}\\
\mbox{\tt \ \ \ \ \ \ \ \ \ \ \ \ \ \ \ \ \ \ \ \ \ \ \ \ \ \ \ \ ({\char'134}x\ ->\ appendChan\ stdout\ (f\ x)\ abort\ done)}
\indextt{interact}%
\eprogB


% Local Variables: 
% mode: latex
% End:
\startnewsection
%
% $Header$
%
\section{Syntax}
\label{syntax}
\index{syntax}

\subsection{Notational Conventions}

These notational conventions are used for presenting syntax:

\[\ba{cl}
\mbox{$\it [pattern]$}             & \tr{optional} \\
\mbox{$\it \{pattern\}$}           & \tr{zero or more repetitions} \\
\mbox{$\it (pattern)$}             & \tr{grouping} \\
\mbox{$\it pat_1\ |\ pat_2$}         & \tr{choice} \\
\mbox{$\it pat_{\{pat'\}}$}        & \tr{difference---elements generated by \mbox{$\it pat$}} \\
                        & \tr{except those generated by \mbox{$\it pat'$}} \\
\mbox{$\it \makebox{\tt fibonacci}$}           & \tr{terminal syntax in typewriter font}
\ea\]

BNF-like syntax is used throughout, with productions having form:
\begin{flushleft}\it\begin{tabbing}
\hspace{0.5in}\=\hspace{3.0in}\=\kill
$\it nonterm$\>\makebox[3.5em]{$\rightarrow$}$\it alt_1\ |\ alt_2\ |\ \ldots \ |\ alt_n$
\end{tabbing}\end{flushleft}

There are some families of nonterminals indexed by
precedence levels (written as a superscript).  Similarly, the
lexeme classes \mbox{$\it op$}, \mbox{$\it varop$}, and \mbox{$\it conop$} have a double index:  a letter \mbox{$\it l$},
\mbox{$\it r$}, or \mbox{$\it n$} for left-, right- or nonassociativity and a precedence
level.  So, for example
\begin{flushleft}\it\begin{tabbing}
\hspace{0.5in}\=\hspace{3.0in}\=\kill
$\it exp^i$\>\makebox[3.5em]{$\rightarrow$}$\it exp^{i+1}\ [op^{({\rm\ n},i)}\ exp^{i+1}]\ (0\leq i\leq 9)$
\end{tabbing}\end{flushleft}
actually stands for 10 productions where \mbox{$\it op$} is non-associative.

% here we input a list of the main changes in version 1.1
\subsection{Minor Syntax Changes in Version 1.1}
\label{syntax-changes}

This section is a list of the non-trivial changes to the \Haskell{}
syntax between versions~1.0 and 1.1 of this report, {\em excluding} those
mentioned in the 1.1~preface (page~\ref{preface-changes-11}).  Other
clarifications and corrections are reflected in the full syntax in the
following sections.

\begin{enumerate}
\item
Empty declarations and declaration lists ending with \mbox{\tt ;} have been
added, to aid automatic program generation.

\item
Guards have been eliminated from lambda expressions. 

\item
List comprehensions must have at least one qualifier. 

\item
\mbox{\tt Case} expressions may have more than one guard per clause.

\item
Instance declarations can only have \mbox{$\it valdefs$}
in their body; in particular, they cannot have type signatures in
their body.
\end{enumerate}

% Local Variables: 
% mode: latex
% End:


\subsection{Lexical Syntax}

\begin{flushleft}\it\begin{tabbing}
\hspace{0.5in}\=\hspace{3.0in}\=\kill
$\it program$\>\makebox[3.5em]{$\rightarrow$}$\it \{\ lexeme\ |\ whitespace\ \}$\\ 
$\it lexeme$\>\makebox[3.5em]{$\rightarrow$}$\it varid\ |\ conid\ |\ varop\ |\ conop\ |\ literal\ |\ special\ |\ reservedop\ |\ reservedid$\\ 
$\it literal$\>\makebox[3.5em]{$\rightarrow$}$\it integer\ |\ float\ |\ char\ |\ string$\\ 
$\it special$\>\makebox[3.5em]{$\rightarrow$}$\it \makebox{\tt (}\ |\ \makebox{\tt )}\ |\ \makebox{\tt ,}\ |\ \makebox{\tt ;}\ |\ \makebox{\tt [}\ |\ \makebox{\tt ]}\ |\ \makebox{\tt {\char'137}}\ |\ \makebox{\tt {\char'173}}\ |\ \makebox{\tt {\char'175}}$\\ 
$\it $\\ 
$\it whitespace$\>\makebox[3.5em]{$\rightarrow$}$\it whitestuff\ \{whitestuff\}$\\ 
$\it whitestuff$\>\makebox[3.5em]{$\rightarrow$}$\it whitechar\ |\ comment\ |\ ncomment$\\ 
$\it whitechar$\>\makebox[3.5em]{$\rightarrow$}$\it newline\ |\ space\ |\ tab\ |\ vertab\ |\ formfeed$\\ 
$\it newline$\>\makebox[3.5em]{$\rightarrow$}$\it \tr{a\ newline\ (system\ dependent)}$\\ 
$\it space$\>\makebox[3.5em]{$\rightarrow$}$\it \tr{a\ space}$\\ 
$\it tab$\>\makebox[3.5em]{$\rightarrow$}$\it \tr{a\ horizontal\ tab}$\\ 
$\it vertab$\>\makebox[3.5em]{$\rightarrow$}$\it \tr{a\ vertical\ tab}$\\ 
$\it formfeed$\>\makebox[3.5em]{$\rightarrow$}$\it \tr{a\ form\ feed}$\\ 
$\it comment$\>\makebox[3.5em]{$\rightarrow$}$\it \makebox{\tt --}\ \{any\}\ newline$\\ 
$\it ncomment$\>\makebox[3.5em]{$\rightarrow$}$\it \makebox{\tt {\char'173}-}\ ANYseq\ \{ncomment\ ANYseq\}\ \makebox{\tt -{\char'175}}$\\ 
$\it ANYseq$\>\makebox[3.5em]{$\rightarrow$}$\it \{ANY\}_{\{ANY\}\ (\ \makebox{\tt {\char'173}-}\ |\ \makebox{\tt -{\char'175}}\ )\ \{ANY\}}$\\ 
$\it ANY$\>\makebox[3.5em]{$\rightarrow$}$\it any\ |\ newline\ |\ vertab\ |\ formfeed$\\ 
$\it any$\>\makebox[3.5em]{$\rightarrow$}$\it graphic\ |\ space\ |\ tab$\\ 
$\it graphic$\>\makebox[3.5em]{$\rightarrow$}$\it large\ |\ small\ |\ digit$\\ 
$\it $\>\makebox[3.5em]{$|$}$\it \makebox{\tt !}\ |\ \makebox{\tt "}\ |\ \makebox{\tt {\char'43}}\ |\ \makebox{\tt {\char'44}}\ |\ \makebox{\tt {\char'45}}\ |\ \makebox{\tt {\char'46}}\ |\ \fwq\ |\ \makebox{\tt (}\ |\ \makebox{\tt )}\ |\ \makebox{\tt *}\ |\ \makebox{\tt +}$\\ 
$\it $\>\makebox[3.5em]{$|$}$\it \makebox{\tt ,}\ |\ \makebox{\tt -}\ |\ \makebox{\tt .}\ |\ \makebox{\tt /}\ |\ \makebox{\tt :}\ |\ \makebox{\tt ;}\ |\ \makebox{\tt <}\ |\ \makebox{\tt =}\ |\ \makebox{\tt >}\ |\ \makebox{\tt ?}\ |\ @$\\ 
$\it $\>\makebox[3.5em]{$|$}$\it \makebox{\tt [}\ |\ \makebox{\tt {\char'134}}\ |\ \makebox{\tt ]}\ |\ \makebox{\tt {\char'136}}\ |\ \makebox{\tt {\char'137}}\ |\ \bkq\ |\ \makebox{\tt {\char'173}}\ |\ \makebox{\tt |}\ |\ \makebox{\tt {\char'175}}\ |\ \makebox{\tt {\char'176}}$\\ 
$\it $\\ 
$\it small$\>\makebox[3.5em]{$\rightarrow$}$\it \makebox{\tt a}\ |\ \makebox{\tt b}\ |\ \ldots \ |\ \makebox{\tt z}$\\ 
$\it large$\>\makebox[3.5em]{$\rightarrow$}$\it \makebox{\tt A}\ |\ \makebox{\tt B}\ |\ \ldots \ |\ \makebox{\tt Z}$\\ 
$\it digit$\>\makebox[3.5em]{$\rightarrow$}$\it \makebox{\tt 0}\ |\ \makebox{\tt 1}\ |\ \ldots \ |\ \makebox{\tt 9}$
\end{tabbing}\end{flushleft}
\indexsyn{program}%
\indexsyn{lexeme}%
\indexsyn{literal}%
\indexsyn{special}%
\indexsyn{whitespace}%
\indexsyn{whitestuff}%
\indexsyn{whitechar}%
\indexsyn{newline}%
\indexsyn{space}%
\indexsyn{tab}%
\indexsyn{vertab}%
\indexsyn{formfeed}%
\indexsyn{comment}%
\indexsyn{ncomment}%
\indexsyn{ANYseq}%
\indexsyn{ANY}%
\indexsyn{any}%
\indexsyn{graphic}%
\indexsyn{small}%
\indexsyn{large}%
\indexsyn{digit}%

\begin{flushleft}\it\begin{tabbing}
\hspace{0.5in}\=\hspace{3.0in}\=\kill
$\it avarid$\>\makebox[3.5em]{$\rightarrow$}$\it (small\ \{small\ |\ large\ |\ digit\ |\ \fwq\ |\ \makebox{\tt {\char'137}}\})_{\{reservedid\}}$\\ 
$\it varid$\>\makebox[3.5em]{$\rightarrow$}$\it avarid\ |\ \makebox{\tt (}avarop\makebox{\tt )}$\\ 
$\it aconid$\>\makebox[3.5em]{$\rightarrow$}$\it large\ \{small\ |\ large\ |\ digit\ |\ \fwq\ |\ \makebox{\tt {\char'137}}\}$\\ 
$\it conid$\>\makebox[3.5em]{$\rightarrow$}$\it aconid\ |\ \makebox{\tt (}aconop\makebox{\tt )}$\\ 
$\it reservedid$\>\makebox[3.5em]{$\rightarrow$}$\it \makebox{\tt case}\ |\ \makebox{\tt class}\ |\ \makebox{\tt data}\ |\ \makebox{\tt default}\ |\ \makebox{\tt deriving}\ |\ \makebox{\tt else}\ |\ \makebox{\tt hiding}$\\ 
$\it $\>\makebox[3.5em]{$|$}$\it \makebox{\tt if}\ |\ \makebox{\tt import}\ |\ \makebox{\tt in}\ |\ \makebox{\tt infix}\ |\ \makebox{\tt infixl}\ |\ \makebox{\tt infixr}\ |\ \makebox{\tt instance}\ |\ \makebox{\tt interface}$\\ 
$\it $\>\makebox[3.5em]{$|$}$\it \makebox{\tt let}\ |\ \makebox{\tt module}\ |\ \makebox{\tt of}\ |\ \makebox{\tt renaming}\ |\ \makebox{\tt then}\ |\ \makebox{\tt to}\ |\ \makebox{\tt type}\ |\ \makebox{\tt where}$
\end{tabbing}\end{flushleft}
\indexsyn{avarid}%
\indexsyn{varid}%
\indexsyn{aconid}%
\indexsyn{conid}%
\indexsyn{reservedid}%

\begin{flushleft}\it\begin{tabbing}
\hspace{0.5in}\=\hspace{3.0in}\=\kill
$\it avarop$\>\makebox[3.5em]{$\rightarrow$}$\it (\ (\ symbol\ |\ presymbol\ )\ \{symbol\ |\ \makebox{\tt :}\}\ )_{\{reservedop\}}$\\ 
$\it varop$\>\makebox[3.5em]{$\rightarrow$}$\it avarop\ |\ \bkqB{avarid}\bkqA$\\ 
$\it aconop$\>\makebox[3.5em]{$\rightarrow$}$\it (\makebox{\tt :}\ \{symbol\ |\ \makebox{\tt :}\})_{\{reservedop\}}$\\ 
$\it conop$\>\makebox[3.5em]{$\rightarrow$}$\it aconop\ |\ \bkqB{aconid}\bkqA$\\ 
$\it presymbol$\>\makebox[3.5em]{$\rightarrow$}$\it \makebox{\tt -}\ |\ \makebox{\tt {\char'176}}$\\ 
$\it symbol$\>\makebox[3.5em]{$\rightarrow$}$\it \makebox{\tt !}\ |\ \makebox{\tt {\char'43}}\ |\ \makebox{\tt {\char'44}}\ |\ \makebox{\tt {\char'45}}\ |\ \makebox{\tt {\char'46}}\ |\ \makebox{\tt *}\ |\ \makebox{\tt +}\ |\ \makebox{\tt .}\ |\ \makebox{\tt /}\ |\ \makebox{\tt <}\ |\ \makebox{\tt =}\ |\ \makebox{\tt >}\ |\ \makebox{\tt ?}\ |\ @\ |\ \makebox{\tt {\char'134}}\ |\ \makebox{\tt {\char'136}}\ |\ \makebox{\tt |}$\\ 
$\it reservedop$\>\makebox[3.5em]{$\rightarrow$}$\it \makebox{\tt ..}\ |\ \makebox{\tt ::}\ |\ \makebox{\tt =>}\ |\ \makebox{\tt =}\ |\ @\ |\ \makebox{\tt {\char'134}}\ |\ \makebox{\tt |}\ |\ \makebox{\tt {\char'176}}\ |\ \makebox{\tt <-}\ |\ \makebox{\tt ->}$
\end{tabbing}\end{flushleft}
\indexsyn{avarop}%
\indexsyn{varop}%
\indexsyn{aconop}%
\indexsyn{conop}%
\indexsyn{presymbol}%
\indexsyn{symbol}%
\indexsyn{reservedop}%

\begin{flushleft}\it\begin{tabbing}
\hspace{0.5in}\=\hspace{3.0in}\=\kill
$\it var$\>\makebox[3.5em]{$\rightarrow$}$\it varid$\>\makebox[3em]{}$\it (variables)$\\ 
$\it con$\>\makebox[3.5em]{$\rightarrow$}$\it conid$\>\makebox[3em]{}$\it (constructors)$\\ 
$\it tyvar$\>\makebox[3.5em]{$\rightarrow$}$\it avarid$\>\makebox[3em]{}$\it (type\ variables)$\\ 
$\it tycon$\>\makebox[3.5em]{$\rightarrow$}$\it aconid$\>\makebox[3em]{}$\it (type\ constructors)$\\ 
$\it tycls$\>\makebox[3.5em]{$\rightarrow$}$\it aconid$\>\makebox[3em]{}$\it (type\ classes)$\\ 
$\it modid$\>\makebox[3.5em]{$\rightarrow$}$\it aconid$\>\makebox[3em]{}$\it (modules)$
\end{tabbing}\end{flushleft}
\indexsyn{var}%
\indexsyn{con}%
\indexsyn{tyvar}%
\indexsyn{tycon}%
\indexsyn{tycls}%
\indexsyn{modid}%

\begin{flushleft}\it\begin{tabbing}
\hspace{0.5in}\=\hspace{3.0in}\=\kill
$\it integer$\>\makebox[3.5em]{$\rightarrow$}$\it digit\{digit\}$\\ 
$\it float$\>\makebox[3.5em]{$\rightarrow$}$\it integer\makebox{\tt .}integer[(\makebox{\tt e}\ |\ \makebox{\tt E})[\makebox{\tt -}\ |\ \makebox{\tt +}]integer]$
\end{tabbing}\end{flushleft}
\indexsyn{integer}%
\indexsyn{float}%

\begin{flushleft}\it\begin{tabbing}
\hspace{0.5in}\=\hspace{3.0in}\=\kill
$\it char$\>\makebox[3.5em]{$\rightarrow$}$\it \fwq\ (graphic_{\{\fwq\ |\ \makebox{\tt {\char'134}}\}}\ |\ space\ |\ escape_{\{\makebox{\tt {\char'134}{\char'46}}\}})\ \fwq$\\ 
$\it string$\>\makebox[3.5em]{$\rightarrow$}$\it \makebox{\tt "}\ \{graphic_{\{\makebox{\tt "}\ |\ \makebox{\tt {\char'134}}\}}\ |\ space\ |\ escape\ |\ gap\}\ \makebox{\tt "}$\\ 
$\it escape$\>\makebox[3.5em]{$\rightarrow$}$\it \makebox{\tt {\char'134}}\ (\ charesc\ |\ ascii\ |\ integer\ |\ \makebox{\tt o}\ octit\{octit\}\ |\ \makebox{\tt x}\ hexit\{hexit\}\ )$\\ 
$\it charesc$\>\makebox[3.5em]{$\rightarrow$}$\it \makebox{\tt a}\ |\ \makebox{\tt b}\ |\ \makebox{\tt f}\ |\ \makebox{\tt n}\ |\ \makebox{\tt r}\ |\ \makebox{\tt t}\ |\ \makebox{\tt v}\ |\ \makebox{\tt {\char'134}}\ |\ \makebox{\tt "}\ |\ \fwq\ |\ \makebox{\tt {\char'46}}$\\ 
$\it ascii$\>\makebox[3.5em]{$\rightarrow$}$\it \makebox{\tt {\char'136}}cntrl\ |\ \makebox{\tt NUL}\ |\ \makebox{\tt SOH}\ |\ \makebox{\tt STX}\ |\ \makebox{\tt ETX}\ |\ \makebox{\tt EOT}\ |\ \makebox{\tt ENQ}\ |\ \makebox{\tt ACK}$\\ 
$\it $\>\makebox[3.5em]{$|$}$\it \makebox{\tt BEL}\ |\ \makebox{\tt BS}\ |\ \makebox{\tt HT}\ |\ \makebox{\tt LF}\ |\ \makebox{\tt VT}\ |\ \makebox{\tt FF}\ |\ \makebox{\tt CR}\ |\ \makebox{\tt SO}\ |\ \makebox{\tt SI}\ |\ \makebox{\tt DLE}$\\ 
$\it $\>\makebox[3.5em]{$|$}$\it \makebox{\tt DC1}\ |\ \makebox{\tt DC2}\ |\ \makebox{\tt DC3}\ |\ \makebox{\tt DC4}\ |\ \makebox{\tt NAK}\ |\ \makebox{\tt SYN}\ |\ \makebox{\tt ETB}\ |\ \makebox{\tt CAN}$\\ 
$\it $\>\makebox[3.5em]{$|$}$\it \makebox{\tt EM}\ |\ \makebox{\tt SUB}\ |\ \makebox{\tt ESC}\ |\ \makebox{\tt FS}\ |\ \makebox{\tt GS}\ |\ \makebox{\tt RS}\ |\ \makebox{\tt US}\ |\ \makebox{\tt SP}\ |\ \makebox{\tt DEL}$\\ 
$\it cntrl$\>\makebox[3.5em]{$\rightarrow$}$\it large\ |\ @\ |\ \makebox{\tt [}\ |\ \makebox{\tt {\char'134}}\ |\ \makebox{\tt ]}\ |\ \makebox{\tt {\char'136}}\ |\ \makebox{\tt {\char'137}}$\\ 
$\it gap$\>\makebox[3.5em]{$\rightarrow$}$\it \makebox{\tt {\char'134}}\ whitechar\ \{whitechar\}\ \makebox{\tt {\char'134}}$\\ 
$\it hexit$\>\makebox[3.5em]{$\rightarrow$}$\it digit\ |\ \makebox{\tt A}\ |\ \makebox{\tt B}\ |\ \makebox{\tt C}\ |\ \makebox{\tt D}\ |\ \makebox{\tt E}\ |\ \makebox{\tt F}\ |\ \makebox{\tt a}\ |\ \makebox{\tt b}\ |\ \makebox{\tt c}\ |\ \makebox{\tt d}\ |\ \makebox{\tt e}\ |\ \makebox{\tt f}$\\ 
$\it octit$\>\makebox[3.5em]{$\rightarrow$}$\it \makebox{\tt 0}\ |\ \makebox{\tt 1}\ |\ \makebox{\tt 2}\ |\ \makebox{\tt 3}\ |\ \makebox{\tt 4}\ |\ \makebox{\tt 5}\ |\ \makebox{\tt 6}\ |\ \makebox{\tt 7}$
\end{tabbing}\end{flushleft}
\indexsyn{char}%
\indexsyn{string}%
\indexsyn{escape}%
\indexsyn{charesc}%
\indexsyn{ascii}%
\indexsyn{cntrl}%
\indexsyn{gap}%
\indexsyn{hexit}%
\indexsyn{octit}%

\subsection{Layout}
\label{layout}
\index{layout}

Definitions: The indentation of a lexeme is the column number
indicating the start of that lexeme; the indentation of a line is the
indentation of its leftmost lexeme.  To determine the column number,
assume a fixed-width font with this tab convention: tab stops
are 8 characters apart, and a tab character causes the insertion of
enough spaces to align the current position with the next tab stop.

In the syntax given in the other parts of the report, {\em declaration
lists} are always preceded by the keyword \mbox{\tt where} or \mbox{\tt of}, and are
enclosed within curly braces (\mbox{\tt {\char'173}\ {\char'175}}) with the individual declarations
separated by semicolons (\mbox{\tt ;}).  For example, the syntax of a \mbox{\tt let}
expression is:
\[
\mbox{$\it let\ \makebox{\tt {\char'173}}\ decl_1\ \makebox{\tt ;}\ decl_2\ \makebox{\tt ;}\ \ldots \ \makebox{\tt ;}\ decl_n\ [\makebox{\tt ;}]\ \makebox{\tt {\char'175}}\ in\ exp$}
\]

%
% $Header$
%
% partain:
% in a separate file, because it is included twice (for now);
%  in intro.verb and syntax.verb

\Haskell{} permits the omission of the braces and semicolons by
using {\em layout} to convey the same information.  This allows both
layout-sensitive and -insensitive styles of coding, which
can be freely mixed within one program.  Because layout is
not required, \Haskell{} programs can be straightforwardly
produced by other programs.
% without worrying about deeply nested layout difficulties.

The layout (or ``off-side'') rule\index{off-side rule} takes effect whenever the
open brace is omitted after the keyword \mbox{\tt where}, \mbox{\tt let} or \mbox{\tt of}.
When this happens, the indentation of the next lexeme (whether or not
on a new line) is remembered and the omitted open brace is inserted
(the whitespace preceding the lexeme may include comments).
For each subsequent line, if it contains only whitespace or is
indented more, then the previous item is continued (nothing is
inserted); if it is indented the same amount, then a new item begins
(a semicolon is inserted); and if it is indented less, then the
declaration list ends (a close brace is inserted).  A close brace is
also inserted whenever the syntactic category containing the
declaration list ends; that is, if an illegal lexeme is encountered at a
point where a close brace would be legal, a close brace is inserted.
The layout rule will match only those open braces
that it has inserted; an
open brace that the user has inserted must be
matched by a close brace inserted by the user.

Given these rules, a single newline may actually terminate several
declaration lists.  Also, these rules permit:
\bprog
\mbox{\tt f\ x\ =\ let\ a\ =\ 1;\ b\ =\ 2\ }\\
\mbox{\tt \ \ \ \ \ \ \ \ \ \ g\ y\ =\ exp2\ in\ exp1}
\eprog
making \mbox{\tt a}, \mbox{\tt b} and \mbox{\tt g} all part of the same declaration
list.

To facilitate the use of layout at the top level of a module
(several modules may reside in one file), the keywords
\mbox{\tt module} and \mbox{\tt interface} and the end-of-file token are assumed to occur in column
0 (whereas normally the first column is 1).  Otherwise, all
top-level declarations would have to be indented.


\subsection{Context-Free Syntax}
\label{bnf}

\begin{flushleft}\it\begin{tabbing}
\hspace{0.5in}\=\hspace{3.0in}\=\kill
$\it module$\>\makebox[3.5em]{$\rightarrow$}$\it \makebox{\tt module}\ modid\ [exports]\ \makebox{\tt where}\ body$\\ 
$\it $\>\makebox[3.5em]{$|$}$\it body$\\ 
$\it body$\>\makebox[3.5em]{$\rightarrow$}$\it \makebox{\tt {\char'173}}\ [impdecls\ \makebox{\tt ;}]\ [[fixdecls\ \makebox{\tt ;}]\ topdecls\ [\makebox{\tt ;}]]\ \makebox{\tt {\char'175}}$\\ 
$\it $\>\makebox[3.5em]{$|$}$\it \makebox{\tt {\char'173}}\ impdecls\ [\makebox{\tt ;}]\ \makebox{\tt {\char'175}}$\\ 
$\it $\\ 
$\it modid$\>\makebox[3.5em]{$\rightarrow$}$\it aconid$\\ 
$\it impdecls$\>\makebox[3.5em]{$\rightarrow$}$\it impdecl_1\ \makebox{\tt ;}\ \ldots \ \makebox{\tt ;}\ impdecl_n$\>\makebox[3em]{}$\it \qquad\ (n\geq 1)$
\end{tabbing}\end{flushleft}
\indexsyn{module}%
\indexsyn{body}%
\indexsyn{modid}%
\indexsyn{impdecls}%

\begin{flushleft}\it\begin{tabbing}
\hspace{0.5in}\=\hspace{3.0in}\=\kill
$\it exports$\>\makebox[3.5em]{$\rightarrow$}$\it \makebox{\tt (}\ export_1\ \makebox{\tt ,}\ \ldots \ \makebox{\tt ,}\ export_n\ \makebox{\tt )}$\>\makebox[3em]{}$\it \qquad\ (n\geq 1)$\\ 
$\it $\\ 
$\it export$\>\makebox[3.5em]{$\rightarrow$}$\it entity$\\ 
$\it $\>\makebox[3.5em]{$|$}$\it modid\ \makebox{\tt ..}$
\end{tabbing}\end{flushleft}
\indexsyn{exports}%
\indexsyn{export}%

\begin{flushleft}\it\begin{tabbing}
\hspace{0.5in}\=\hspace{3.0in}\=\kill
$\it impdecl$\>\makebox[3.5em]{$\rightarrow$}$\it \makebox{\tt import}\ modid\ [impspec]\ [\makebox{\tt renaming}\ renamings]$\\ 
$\it impspec$\>\makebox[3.5em]{$\rightarrow$}$\it \makebox{\tt (}\ import_1\ \makebox{\tt ,}\ \ldots \ \makebox{\tt ,}\ import_n\ \makebox{\tt )}$\>\makebox[3em]{}$\it \qquad\ (n\geq 0)$\\ 
$\it $\>\makebox[3.5em]{$|$}$\it \makebox{\tt hiding}\ \makebox{\tt (}\ import_1\ \makebox{\tt ,}\ \ldots \ \makebox{\tt ,}\ import_n\ \makebox{\tt )}$\>\makebox[3em]{}$\it \qquad\ (n\geq 1)$\\ 
$\it import$\>\makebox[3.5em]{$\rightarrow$}$\it entity$\\ 
$\it renamings$\>\makebox[3.5em]{$\rightarrow$}$\it \makebox{\tt (}\ renaming_1\ \makebox{\tt ,}\ \ldots \ \makebox{\tt ,}\ renaming_n\ \makebox{\tt )}$\>\makebox[3em]{}$\it \qquad\ (n\geq 1)$\\ 
$\it renaming$\>\makebox[3.5em]{$\rightarrow$}$\it varid_1\ \makebox{\tt to}\ varid_2$\\ 
$\it $\>\makebox[3.5em]{$|$}$\it conid_1\ \makebox{\tt to}\ conid_2$\\ 
$\it $\\ 
$\it entity$\>\makebox[3.5em]{$\rightarrow$}$\it varid$\\ 
$\it $\>\makebox[3.5em]{$|$}$\it tycon$\\ 
$\it $\>\makebox[3.5em]{$|$}$\it tycon\ \makebox{\tt (..)}$\\ 
$\it $\>\makebox[3.5em]{$|$}$\it tycon\ \makebox{\tt (}\ conid_1\ \makebox{\tt ,}\ \ldots \ \makebox{\tt ,}\ conid_n\ \makebox{\tt )}$\>\makebox[3em]{}$\it \qquad\ (n\geq 1)$\\ 
$\it $\>\makebox[3.5em]{$|$}$\it tycls\ \makebox{\tt (..)}$\\ 
$\it $\>\makebox[3.5em]{$|$}$\it tycls\ \makebox{\tt (}\ varid_1\ \makebox{\tt ,}\ \ldots \ \makebox{\tt ,}\ varid_n\ \makebox{\tt )}$\>\makebox[3em]{}$\it \qquad\ (n\geq 0)$
\end{tabbing}\end{flushleft}
\indexsyn{impdecl}%
\indexsyn{impspec}%
\indexsyn{import}%
\indexsyn{renamings}%
\indexsyn{renaming}%
\indexsyn{entity}%

\begin{flushleft}\it\begin{tabbing}
\hspace{0.5in}\=\hspace{3.0in}\=\kill
$\it fixdecls$\>\makebox[3.5em]{$\rightarrow$}$\it fix_1\ \makebox{\tt ;}\ \ldots \ \makebox{\tt ;}\ fix_n$\>\makebox[3em]{}$\it \qquad\ (n\geq 1)$\\ 
$\it fix$\>\makebox[3.5em]{$\rightarrow$}$\it \makebox{\tt infixl}\ [digit]\ ops$\\ 
$\it $\>\makebox[3.5em]{$|$}$\it \makebox{\tt infixr}\ [digit]\ ops$\\ 
$\it $\>\makebox[3.5em]{$|$}$\it \makebox{\tt infix\ }\ [digit]\ ops$\\ 
$\it ops$\>\makebox[3.5em]{$\rightarrow$}$\it op_1\ \makebox{\tt ,}\ \ldots \ \makebox{\tt ,}\ op_n$\>\makebox[3em]{}$\it \qquad\ (n\geq 1)$\\ 
$\it op$\>\makebox[3.5em]{$\rightarrow$}$\it varop\ |\ conop$
\end{tabbing}\end{flushleft}
\indexsyn{fixdecls}%
\indexsyn{fix}%
\indexsyn{ops}%
\indexsyn{op}%

\begin{flushleft}\it\begin{tabbing}
\hspace{0.5in}\=\hspace{3.0in}\=\kill
$\it topdecls$\>\makebox[3.5em]{$\rightarrow$}$\it topdecl_1\ \makebox{\tt ;}\ \ldots \ \makebox{\tt ;}\ topdecl_n$\>\makebox[3em]{}$\it \qquad\ (n\geq 1)$\\ 
$\it topdecl$\>\makebox[3.5em]{$\rightarrow$}$\it \makebox{\tt type}\ simple\ \makebox{\tt =}\ type$\\ 
$\it $\>\makebox[3.5em]{$|$}$\it \makebox{\tt data}\ [context\ \makebox{\tt =>}]\ simple\ \makebox{\tt =}\ constrs\ [\makebox{\tt deriving}\ (tycls\ |\ \makebox{\tt (}tyclses\makebox{\tt )})]$\\ 
$\it $\>\makebox[3.5em]{$|$}$\it \makebox{\tt class}\ [context\ \makebox{\tt =>}]\ class\ [\makebox{\tt where}\ \makebox{\tt {\char'173}}\ cbody\ [\makebox{\tt ;}]\ \makebox{\tt {\char'175}}]$\\ 
$\it $\>\makebox[3.5em]{$|$}$\it \makebox{\tt instance}\ [context\ \makebox{\tt =>}]\ tycls\ inst\ [\makebox{\tt where}\ \makebox{\tt {\char'173}}\ valdefs\ [\makebox{\tt ;}]\ \makebox{\tt {\char'175}}]$\\ 
$\it $\>\makebox[3.5em]{$|$}$\it \makebox{\tt default}\ (type\ |\ \makebox{\tt (}type_1\ \makebox{\tt ,}\ \ldots \ \makebox{\tt ,}\ type_n\makebox{\tt )})$\>\makebox[3em]{}$\it \qquad\ (n\geq 0)$\\ 
$\it $\>\makebox[3.5em]{$|$}$\it decl$\\ 
$\it $\\ 
$\it decls$\>\makebox[3.5em]{$\rightarrow$}$\it decl_1\ \makebox{\tt ;}\ \ldots \ \makebox{\tt ;}\ decl_n$\>\makebox[3em]{}$\it \qquad\ (n\geq 0)$\\ 
$\it decl$\>\makebox[3.5em]{$\rightarrow$}$\it vars\ \makebox{\tt ::}\ [context\ \makebox{\tt =>}]\ type$\\ 
$\it $\>\makebox[3.5em]{$|$}$\it valdef$
\end{tabbing}\end{flushleft}
\indexsyn{topdecls}%
\indexsyn{topdecl}%
\indexsyn{decls}%
\indexsyn{decl}%

\begin{flushleft}\it\begin{tabbing}
\hspace{0.5in}\=\hspace{3.0in}\=\kill
$\it type$\>\makebox[3.5em]{$\rightarrow$}$\it atype$\\ 
$\it $\>\makebox[3.5em]{$|$}$\it type_1\ \makebox{\tt ->}\ type_2$\\ 
$\it $\>\makebox[3.5em]{$|$}$\it tycon\ atype_1\ \ldots \ atype_k$\>\makebox[3em]{}$\it (\arity{tycon}=k\geq 1)$\\ 
$\it $\\ 
$\it atype$\>\makebox[3.5em]{$\rightarrow$}$\it tyvar$\\ 
$\it $\>\makebox[3.5em]{$|$}$\it tycon$\>\makebox[3em]{}$\it (\arity{tycon}=0)$\\ 
$\it $\>\makebox[3.5em]{$|$}$\it \makebox{\tt ()}$\>\makebox[3em]{}$\it (\tr{unit\ type})$\\ 
$\it $\>\makebox[3.5em]{$|$}$\it \makebox{\tt (}\ type\ \makebox{\tt )}$\>\makebox[3em]{}$\it (\tr{parenthesised\ type})$\\ 
$\it $\>\makebox[3.5em]{$|$}$\it \makebox{\tt (}\ type_1\ \makebox{\tt ,}\ \ldots \ \makebox{\tt ,}\ type_k\ \makebox{\tt )}$\>\makebox[3em]{}$\it (\tr{tuple\ type},\ k\geq 2)$\\ 
$\it $\>\makebox[3.5em]{$|$}$\it \makebox{\tt [}\ type\ \makebox{\tt ]}$
\end{tabbing}\end{flushleft}
\indexsyn{type}%
\indexsyn{atype}%

\begin{flushleft}\it\begin{tabbing}
\hspace{0.5in}\=\hspace{3.0in}\=\kill
$\it context$\>\makebox[3.5em]{$\rightarrow$}$\it class$\\ 
$\it $\>\makebox[3.5em]{$|$}$\it \makebox{\tt (}\ class_1\ \makebox{\tt ,}\ \ldots \ \makebox{\tt ,}\ class_n\ \makebox{\tt )}$\>\makebox[3em]{}$\it (n\geq 1)$\\ 
$\it class$\>\makebox[3.5em]{$\rightarrow$}$\it tycls\ tyvar$\\ 
$\it $\\ 
$\it $\\ 
$\it cbody$\>\makebox[3.5em]{$\rightarrow$}$\it [csigns\ \makebox{\tt ;}]\ [valdefs]$\\ 
$\it csigns$\>\makebox[3.5em]{$\rightarrow$}$\it csign_1\ \makebox{\tt ;}\ \ldots \ \makebox{\tt ;}\ csign_n$\>\makebox[3em]{}$\it (n\geq 1)$\\ 
$\it csign$\>\makebox[3.5em]{$\rightarrow$}$\it vars\ \makebox{\tt ::}\ [context\ \makebox{\tt =>}]\ type$\\ 
$\it $\\ 
$\it vars$\>\makebox[3.5em]{$\rightarrow$}$\it var_1\ \makebox{\tt ,}\ \ldots \makebox{\tt ,}\ var_n$\>\makebox[3em]{}$\it (n\geq 1)$
\end{tabbing}\end{flushleft}
\indexsyn{context}%
\indexsyn{class}%
\indexsyn{cbody}%
\indexsyn{csigns}%
\indexsyn{csign}%
\indexsyn{vars}%

\begin{flushleft}\it\begin{tabbing}
\hspace{0.5in}\=\hspace{3.0in}\=\kill
$\it simple$\>\makebox[3.5em]{$\rightarrow$}$\it tycon\ tyvar_1\ \ldots \ tyvar_k$\>\makebox[3em]{}$\it (\arity{tycon}=k\geq 0)$\\ 
$\it constrs$\>\makebox[3.5em]{$\rightarrow$}$\it constr_1\ \makebox{\tt |}\ \ldots \ \makebox{\tt |}\ constr_n$\>\makebox[3em]{}$\it (n\geq 1)$\\ 
$\it constr$\>\makebox[3.5em]{$\rightarrow$}$\it con\ atype_1\ \ldots \ atype_k$\>\makebox[3em]{}$\it (\arity{con}=k\geq 0)$\\ 
$\it $\>\makebox[3.5em]{$|$}$\it type_1\ conop\ type_2$\>\makebox[3em]{}$\it (\infix{conop})$\\ 
$\it tyclses$\>\makebox[3.5em]{$\rightarrow$}$\it tycls_1\makebox{\tt ,}\ \ldots \makebox{\tt ,}\ tycls_n$\>\makebox[3em]{}$\it (n\geq 0)$
\end{tabbing}\end{flushleft}
\indexsyn{simple}%
\indexsyn{constrs}%
\indexsyn{constr}%
\indexsyn{tyclses}%

\begin{flushleft}\it\begin{tabbing}
\hspace{0.5in}\=\hspace{3.0in}\=\kill
$\it inst$\>\makebox[3.5em]{$\rightarrow$}$\it tycon$\>\makebox[3em]{}$\it (\arity{tycon}=0)$\\ 
$\it $\>\makebox[3.5em]{$|$}$\it \makebox{\tt (}\ tycon\ tyvar_1\ \ldots \ tyvar_k\ \makebox{\tt )}$\>\makebox[3em]{}$\it (k\geq 1,\ tyvars\ {\rm\ distinct})$\\ 
$\it $\>\makebox[3.5em]{$|$}$\it \makebox{\tt (}\ tyvar_1\ \makebox{\tt ,}\ \ldots \ \makebox{\tt ,}\ tyvar_k\ \makebox{\tt )}$\>\makebox[3em]{}$\it (k\geq 2,\ tyvars\ {\rm\ distinct})$\\ 
$\it $\>\makebox[3.5em]{$|$}$\it \makebox{\tt ()}$\\ 
$\it $\>\makebox[3.5em]{$|$}$\it \makebox{\tt [}\ tyvar\ \makebox{\tt ]}$\\ 
$\it $\>\makebox[3.5em]{$|$}$\it \makebox{\tt (}\ tyvar_1\ \makebox{\tt ->}\ tyvar_2\ \makebox{\tt )}$\>\makebox[3em]{}$\it tyvar_1\ {\rm\ and}\ tyvar_2\ {\rm\ distinct}$
\end{tabbing}\end{flushleft}
\indexsyn{inst}%

\begin{flushleft}\it\begin{tabbing}
\hspace{0.5in}\=\hspace{3.0in}\=\kill
$\it valdefs$\>\makebox[3.5em]{$\rightarrow$}$\it valdef_1\ \makebox{\tt ;}\ \ldots \ \makebox{\tt ;}\ valdef_n$\>\makebox[3em]{}$\it (n\geq 0)$\\ 
$\it valdef$\>\makebox[3.5em]{$\rightarrow$}$\it lhs\ \makebox{\tt =}\ exp\ [\makebox{\tt where}\ \makebox{\tt {\char'173}}\ decls\ [\makebox{\tt ;}]\ \makebox{\tt {\char'175}}]$\\ 
$\it $\>\makebox[3.5em]{$|$}$\it lhs\ gdrhs\ [\makebox{\tt where}\ \makebox{\tt {\char'173}}\ decls\ [\makebox{\tt ;}]\ \makebox{\tt {\char'175}}]$\\ 
$\it $\\ 
$\it lhs$\>\makebox[3.5em]{$\rightarrow$}$\it apat$\\ 
$\it $\>\makebox[3.5em]{$|$}$\it funlhs$\\ 
$\it funlhs$\>\makebox[3.5em]{$\rightarrow$}$\it afunlhs$\\ 
$\it $\>\makebox[3.5em]{$|$}$\it pat^{i+1}_1\ varop^{({\rm\ n},i)}\ pat^{i+1}_2$\>\makebox[3em]{}$\it (0\leq i\leq 9)$\\ 
$\it $\>\makebox[3.5em]{$|$}$\it lpat^i\ varop^{({\rm\ l},i)}\ pat^{i+1}$\>\makebox[3em]{}$\it (0\leq i\leq 9)$\\ 
$\it $\>\makebox[3.5em]{$|$}$\it pat^{i+1}\ varop^{({\rm\ r},i)}\ rpat^i$\>\makebox[3em]{}$\it (0\leq i\leq 9)$\\ 
$\it afunlhs$\>\makebox[3.5em]{$\rightarrow$}$\it var\ apat$\\ 
$\it $\>\makebox[3.5em]{$|$}$\it \makebox{\tt (}\ funlhs\ \makebox{\tt )}\ apat$\\ 
$\it $\>\makebox[3.5em]{$|$}$\it afunlhs\ apat$\\ 
$\it $\\ 
$\it gdrhs$\>\makebox[3.5em]{$\rightarrow$}$\it gd\ \makebox{\tt =}\ exp\ [gdrhs]$\\ 
$\it $\\ 
$\it gd$\>\makebox[3.5em]{$\rightarrow$}$\it \makebox{\tt |}\ exp$
\end{tabbing}\end{flushleft}
\indexsyn{valdefs}%
\indexsyn{valdef}%
\indexsyn{lhs}%
\indexsyn{funlhs}%
\indexsyn{afunlhs}%
\indexsyn{gdrhs}%
\indexsyn{gd}%

\begin{flushleft}\it\begin{tabbing}
\hspace{0.5in}\=\hspace{3.0in}\=\kill
$\it exp$\>\makebox[3.5em]{$\rightarrow$}$\it \makebox{\tt {\char'134}}\ apat_1\ \ldots \ apat_n\ \makebox{\tt ->}\ exp$\>\makebox[3em]{}$\it (\tr{lambda\ abstraction},\ n\geq 1)$\\ 
$\it $\>\makebox[3.5em]{$|$}$\it \makebox{\tt let}\ \makebox{\tt {\char'173}}\ decls\ [\makebox{\tt ;}]\ \makebox{\tt {\char'175}}\ \makebox{\tt in}\ exp$\>\makebox[3em]{}$\it ({\tr{let\ expression}})$\\ 
$\it $\>\makebox[3.5em]{$|$}$\it \makebox{\tt if}\ exp\ \makebox{\tt then}\ exp\ \makebox{\tt else}\ exp$\>\makebox[3em]{}$\it (\tr{conditional})$\\ 
$\it $\>\makebox[3.5em]{$|$}$\it \makebox{\tt case}\ exp\ \makebox{\tt of}\ \makebox{\tt {\char'173}}\ alts\ [\makebox{\tt ;}]\ \makebox{\tt {\char'175}}$\>\makebox[3em]{}$\it (\tr{case\ expression})$\\ 
$\it $\>\makebox[3.5em]{$|$}$\it exp^0\ \makebox{\tt ::}\ [context\ \makebox{\tt =>}]\ atype$\>\makebox[3em]{}$\it (\tr{expression\ type\ signature})$\\ 
$\it $\>\makebox[3.5em]{$|$}$\it exp^0$\\ 
$\it exp^i$\>\makebox[3.5em]{$\rightarrow$}$\it exp^{i+1}\ [op^{({\rm\ n},i)}\ exp^{i+1}]$\>\makebox[3em]{}$\it (0\leq i\leq 9)$\\ 
$\it $\>\makebox[3.5em]{$|$}$\it lexp^i\ op^{({\rm\ l},i)}\ exp^{i+1}$\\ 
$\it $\>\makebox[3.5em]{$|$}$\it exp^{i+1}\ op^{({\rm\ r},i)}\ rexp^i$\\ 
$\it lexp^i$\>\makebox[3.5em]{$\rightarrow$}$\it [lexp^i\ op^{({\rm\ l},i)}]\ exp^{i+1}$\>\makebox[3em]{}$\it (0\leq i\leq 9)$\\ 
$\it lexp^6$\>\makebox[3.5em]{$\rightarrow$}$\it \makebox{\tt -}\ exp^7$\\ 
$\it rexp^i$\>\makebox[3.5em]{$\rightarrow$}$\it exp^{i+1}\ [op^{({\rm\ r},i)}\ rexp^i]$\>\makebox[3em]{}$\it (0\leq i\leq 9)$\\ 
$\it exp^{10}$\>\makebox[3.5em]{$\rightarrow$}$\it exp^{10}\ aexp$\>\makebox[3em]{}$\it (\tr{function\ application})$\\ 
$\it $\>\makebox[3.5em]{$|$}$\it aexp$
\end{tabbing}\end{flushleft}
\indexsyn{exp}%
\index{exp@\mbox{$\it exp^i$}}%
\index{lexp@\mbox{$\it lexp^i$}}%
\index{rexp@\mbox{$\it rexp^i$}}%

\begin{flushleft}\it\begin{tabbing}
\hspace{0.5in}\=\hspace{3.0in}\=\kill
$\it aexp$\>\makebox[3.5em]{$\rightarrow$}$\it var$\>\makebox[3em]{}$\it (\tr{variable})$\\ 
$\it $\>\makebox[3.5em]{$|$}$\it con$\>\makebox[3em]{}$\it (\tr{constructor})$\\ 
$\it $\>\makebox[3.5em]{$|$}$\it literal$\\ 
$\it $\>\makebox[3.5em]{$|$}$\it \makebox{\tt ()}$\>\makebox[3em]{}$\it (\tr{unit})$\\ 
$\it $\>\makebox[3.5em]{$|$}$\it \makebox{\tt (}\ exp\ \makebox{\tt )}$\>\makebox[3em]{}$\it (\tr{parenthesised\ expression})$\\ 
$\it $\>\makebox[3.5em]{$|$}$\it \makebox{\tt (}\ exp_1\ \makebox{\tt ,}\ \ldots \ \makebox{\tt ,}\ exp_k\ \makebox{\tt )}$\>\makebox[3em]{}$\it (\tr{tuple},\ k\geq 2)$\\ 
$\it $\>\makebox[3.5em]{$|$}$\it \makebox{\tt [}\ exp_1\ \makebox{\tt ,}\ \ldots \ \makebox{\tt ,}\ exp_k\ \makebox{\tt ]}$\>\makebox[3em]{}$\it (\tr{list},\ k\geq 0)$\\ 
$\it $\>\makebox[3.5em]{$|$}$\it \makebox{\tt [}\ exp_1\ [\makebox{\tt ,}\ exp_2]\ \makebox{\tt ..}\ [exp_3]\ \makebox{\tt ]}$\>\makebox[3em]{}$\it (\tr{arithmetic\ sequence})$\\ 
$\it $\>\makebox[3.5em]{$|$}$\it \makebox{\tt [}\ exp\ \makebox{\tt |}\ qual_1\ \makebox{\tt ,}\ \ldots \ \makebox{\tt ,}\ qual_n\ \makebox{\tt ]}$\>\makebox[3em]{}$\it (\tr{list\ comprehension},\ n\geq 1)$\\ 
$\it $\>\makebox[3.5em]{$|$}$\it \makebox{\tt (}\ exp^{i+1}\ op^{(a,i)}\ \makebox{\tt )}$\>\makebox[3em]{}$\it (\tr{section},\ 0\leq i\leq 9,\ a\in\{n,l,r\})$\\ 
$\it $\>\makebox[3.5em]{$|$}$\it \makebox{\tt (}\ op^{(a,i)}\ exp^{i+1}\ \makebox{\tt )}$\>\makebox[3em]{}$\it (\tr{section},\ 0\leq i\leq 9,\ a\in\{n,l,r\})$
\end{tabbing}\end{flushleft}
\indexsyn{aexp}%

\begin{flushleft}\it\begin{tabbing}
\hspace{0.5in}\=\hspace{3.0in}\=\kill
$\it qual$\>\makebox[3.5em]{$\rightarrow$}$\it pat\ \makebox{\tt <-}\ exp$\\ 
$\it $\>\makebox[3.5em]{$|$}$\it exp$\\ 
$\it $\\ 
$\it alts$\>\makebox[3.5em]{$\rightarrow$}$\it alt_1\ \makebox{\tt ;}\ \ldots \ \makebox{\tt ;}\ alt_n$\>\makebox[3em]{}$\it (n\geq 0)$\\ 
$\it alt$\>\makebox[3.5em]{$\rightarrow$}$\it pat\ \makebox{\tt ->}\ exp\ [\makebox{\tt where}\ \makebox{\tt {\char'173}}\ decls\ [\makebox{\tt ;}]\ \makebox{\tt {\char'175}}]$\\ 
$\it $\>\makebox[3.5em]{$|$}$\it pat\ gdpat\ [\makebox{\tt where}\ \makebox{\tt {\char'173}}\ decls\ [\makebox{\tt ;}]\ \makebox{\tt {\char'175}}]$\\ 
$\it $\\ 
$\it gdpat$\>\makebox[3.5em]{$\rightarrow$}$\it gd\ \makebox{\tt ->}\ exp\ [\ gdpat\ ]$
\end{tabbing}\end{flushleft}
\indexsyn{qual}%
\indexsyn{alts}%
\indexsyn{alt}%
\indexsyn{gdpat}%

\begin{flushleft}\it\begin{tabbing}
\hspace{0.5in}\=\hspace{3.0in}\=\kill
$\it pat$\>\makebox[3.5em]{$\rightarrow$}$\it pat^0$\\ 
$\it pat^i$\>\makebox[3.5em]{$\rightarrow$}$\it pat^{i+1}_1\ [conop^{({\rm\ n},i)}\ pat^{i+1}_2]$\>\makebox[3em]{}$\it (0\leq i\leq 9)$\\ 
$\it $\>\makebox[3.5em]{$|$}$\it lpat^i\ conop^{({\rm\ l},i)}\ pat^{i+1}$\\ 
$\it $\>\makebox[3.5em]{$|$}$\it pat^{i+1}\ conop^{({\rm\ r},i)}\ rpat^i$\\ 
$\it lpat^i$\>\makebox[3.5em]{$\rightarrow$}$\it [lpat^i\ conop^{({\rm\ l},i)}]\ pat^{i+1}$\>\makebox[3em]{}$\it (0\leq i\leq 9)$\\ 
$\it lpat^6$\>\makebox[3.5em]{$\rightarrow$}$\it lpat^6\ \makebox{\tt +}\ integer$\>\makebox[3em]{}$\it (\tr{successor\ pattern})$\\ 
$\it $\>\makebox[3.5em]{$|$}$\it \makebox{\tt -}\ \{integer\ |\ float\}$\>\makebox[3em]{}$\it (\tr{negative\ literal})$\\ 
$\it rpat^i$\>\makebox[3.5em]{$\rightarrow$}$\it pat^{i+1}\ [conop^{({\rm\ r},i)}\ rpat^i]$\>\makebox[3em]{}$\it (0\leq i\leq 9)$\\ 
$\it pat^{10}$\>\makebox[3.5em]{$\rightarrow$}$\it apat$\\ 
$\it $\>\makebox[3.5em]{$|$}$\it con\ apat_1\ \ldots \ apat_k$\>\makebox[3em]{}$\it (\arity{con}=k\geq 1)$
\end{tabbing}\end{flushleft}
\indexsyn{pat}%
\index{pat@\mbox{$\it pat^i$}}%
\index{lpat@\mbox{$\it lpat^i$}}%
\index{rpat@\mbox{$\it rpat^i$}}%

\begin{flushleft}\it\begin{tabbing}
\hspace{0.5in}\=\hspace{3.0in}\=\kill
$\it apat$\>\makebox[3.5em]{$\rightarrow$}$\it var\ [{\tt\ @}\ apat]$\>\makebox[3em]{}$\it (\tr{as\ pattern})$\\ 
$\it $\>\makebox[3.5em]{$|$}$\it con$\>\makebox[3em]{}$\it (\arity{con}=0)$\\ 
$\it $\>\makebox[3.5em]{$|$}$\it literal$\\ 
$\it $\>\makebox[3.5em]{$|$}$\it \makebox{\tt {\char'137}}$\>\makebox[3em]{}$\it (\tr{wildcard})$\\ 
$\it $\>\makebox[3.5em]{$|$}$\it \makebox{\tt ()}$\>\makebox[3em]{}$\it (\tr{unit\ pattern})$\\ 
$\it $\>\makebox[3.5em]{$|$}$\it \makebox{\tt (}\ pat\ \makebox{\tt )}$\>\makebox[3em]{}$\it (\tr{parenthesised\ pattern})$\\ 
$\it $\>\makebox[3.5em]{$|$}$\it \makebox{\tt (}\ pat_1\ \makebox{\tt ,}\ \ldots \ \makebox{\tt ,}\ pat_k\ \makebox{\tt )}$\>\makebox[3em]{}$\it (\tr{tuple\ pattern},\ k\geq 2)$\\ 
$\it $\>\makebox[3.5em]{$|$}$\it \makebox{\tt [}\ pat_1\ \makebox{\tt ,}\ \ldots \ \makebox{\tt ,}\ pat_k\ \makebox{\tt ]}$\>\makebox[3em]{}$\it (\tr{list\ pattern},\ k\geq 0)$\\ 
$\it $\>\makebox[3.5em]{$|$}$\it \makebox{\tt {\char'176}}\ apat$\>\makebox[3em]{}$\it (\tr{irrefutable\ pattern})$
\end{tabbing}\end{flushleft}
\indexsyn{apat}%

\begin{flushleft}\it\begin{tabbing}
\hspace{0.5in}\=\hspace{3.0in}\=\kill
$\it tycls$\>\makebox[3.5em]{$\rightarrow$}$\it aconid$\\ 
$\it tyvar$\>\makebox[3.5em]{$\rightarrow$}$\it avarid$\\ 
$\it tycon$\>\makebox[3.5em]{$\rightarrow$}$\it aconid$
\end{tabbing}\end{flushleft}
\indexsyn{tycls}%
\indexsyn{tyvar}%
\indexsyn{tycon}%

\subsection{Interface Syntax}
\label{ibnf}

%
%interface-> \mbox{\tt interface} modid \mbox{\tt where} ibody
%
%ibody    -> \mbox{\tt {\char'173}} [ [fixdecls \mbox{\tt ;}] itopdecls] \mbox{\tt {\char'175}}
%itopdecls -> itopdecl_1 \mbox{\tt ;} ... \mbox{\tt ;} itopdecl_n  & \qquad (n>=1) 
%itopdecl  -> \mbox{\tt type} simple \mbox{\tt =} type
%          | \mbox{\tt data} [context \mbox{\tt =>}] simple [\mbox{\tt =} constrs [\mbox{\tt deriving} (tycls | \mbox{\tt (}tyclses\mbox{\tt )})]]
%          | \mbox{\tt class} [context \mbox{\tt =>}] class [\mbox{\tt where} \mbox{\tt {\char'173}} icdecls [\mbox{\tt ;}] \mbox{\tt {\char'175}}]
%          | \mbox{\tt instance} [context \mbox{\tt =>}] tycls inst
%          | vars \mbox{\tt ::} [context \mbox{\tt =>}] type
%icdecls           -> icdecl_1 \mbox{\tt ;} ... \mbox{\tt ;} icdecl_n  & (n>=1)
%icdecl    -> vars \mbox{\tt ::} type
%

\begin{flushleft}\it\begin{tabbing}
\hspace{0.5in}\=\hspace{3.0in}\=\kill
$\it interface$\>\makebox[3.5em]{$\rightarrow$}$\it \makebox{\tt interface}\ modid\ \makebox{\tt where}\ ibody$\\ 
$\it $\\ 
$\it ibody$\>\makebox[3.5em]{$\rightarrow$}$\it \makebox{\tt {\char'173}}\ [iimpdecls\ \makebox{\tt ;}]\ [fixdecls\ \makebox{\tt ;}]\ itopdecls\ [\makebox{\tt ;}]\ \makebox{\tt {\char'175}}$\\ 
$\it $\>\makebox[3.5em]{$|$}$\it \makebox{\tt {\char'173}}\ iimpdecls\ [\makebox{\tt ;}]\ \makebox{\tt {\char'175}}$\\ 
$\it iimpdecls$\>\makebox[3.5em]{$\rightarrow$}$\it iimpdecl_1\ \makebox{\tt ;}\ \ldots \ \makebox{\tt ;}\ iimpdecl_n$\>\makebox[3em]{}$\it \qquad\ (n\geq 1)$\\ 
$\it iimpdecl$\>\makebox[3.5em]{$\rightarrow$}$\it \makebox{\tt import}\ modid\ \makebox{\tt (}\ import_1\ \makebox{\tt ,}\ \ldots \ \makebox{\tt ,}\ import_n\ \makebox{\tt )}$\\ 
$\it $\>\makebox[3em]{}$\it [\makebox{\tt renaming}\ renamings]$\>\makebox[3em]{}$\it \qquad\ (n\geq 1)$\\ 
$\it itopdecls$\>\makebox[3.5em]{$\rightarrow$}$\it itopdecl_1\ \makebox{\tt ;}\ \ldots \ \makebox{\tt ;}\ itopdecl_n$\>\makebox[3em]{}$\it \qquad\ (n\geq 1)$\\ 
$\it itopdecl$\>\makebox[3.5em]{$\rightarrow$}$\it \makebox{\tt type}\ simple\ \makebox{\tt =}\ type$\\ 
$\it $\>\makebox[3.5em]{$|$}$\it \makebox{\tt data}\ [context\ \makebox{\tt =>}]\ simple\ [\makebox{\tt =}\ constrs]\ [\makebox{\tt deriving}\ (tycls\ |\ \makebox{\tt (}tyclses\makebox{\tt )})]$\\ 
$\it $\>\makebox[3.5em]{$|$}$\it \makebox{\tt class}\ [context\ \makebox{\tt =>}]\ class\ [\makebox{\tt where}\ \makebox{\tt {\char'173}}\ icdecls\ [\makebox{\tt ;}]\ \makebox{\tt {\char'175}}]$\\ 
$\it $\>\makebox[3.5em]{$|$}$\it \makebox{\tt instance}\ [context\ \makebox{\tt =>}]\ tycls\ inst$\\ 
$\it $\>\makebox[3.5em]{$|$}$\it vars\ \makebox{\tt ::}\ [context\ \makebox{\tt =>}]\ type$\\ 
$\it icdecls$\>\makebox[3.5em]{$\rightarrow$}$\it icdecl_1\ \makebox{\tt ;}\ \ldots \ \makebox{\tt ;}\ icdecl_n$\>\makebox[3em]{}$\it (n\geq 1)$\\ 
$\it icdecl$\>\makebox[3.5em]{$\rightarrow$}$\it vars\ \makebox{\tt ::}\ type$
\end{tabbing}\end{flushleft}
\indexsyn{interface}%
\indexsyn{ibody}%
\indexsyn{iimpdecls}%
\indexsyn{iimpdecl}%
\indexsyn{itopdecls}%
\indexsyn{itopdecl}%
\indexsyn{icdecls}%
\indexsyn{icdecl}%

%In interface files, the syntax for \mbox{$\it var$}, \mbox{$\it tycon$}, and \mbox{$\it tycls$} is expanded 
%to include full names:\nopagebreak[4]
%
%var    ->  varid [\mbox{\tt {\char'173}}modid [varid]\mbox{\tt {\char'175}}]         & (variables)
%tycon  ->  aconid [\mbox{\tt {\char'173}}modid [aconid]\mbox{\tt {\char'175}}]       & (type constructors)
%tycls  ->  aconid [\mbox{\tt {\char'173}}modid [aconid]\mbox{\tt {\char'175}}]       & (type classes)
%

%\input{syntax-changes-desc}

% Local Variables: 
% mode: latex
% End:
\startnewsection
% \input{short_semantics}\startnewsection
%
% $Header$
%
\section{Input/Output Semantics}
\label{io-semantics}
\index{input/output!semantics}

The behaviour of a \Haskell{} program performing I/O is given within
the environment in which it is running.  That environment will be described
using standard \Haskell{} code augmented with a non-deterministic
merge operator.

The state of the operating system (OS state) that is relevant to
\Haskell{} programs is completely described by the file system and the
channel system.  The channel system is split into two subsystems, the
input channel system and the output channel system.
\bprog
\mbox{\tt type\ State\ =\ (FileSystem,\ ChannelSystem)}\\
\mbox{\tt type\ FileSystem\ \ \ \ =\ String\ ->\ Response}\\
\mbox{\tt type\ ChannelSystem\ =\ (ICs,\ OCs)}\\
\mbox{\tt type\ ICs\ \ \ =\ String\ ->\ (Agent,\ Open)}\\
\mbox{\tt type\ OCs\ \ \ =\ String\ ->\ Response}\\
\mbox{\tt type\ Agent\ =\ (FileSystem,\ OCs)\ ->\ Response}\\
\mbox{\tt type\ Open\ \ =\ PId\ ->\ Bool}\\
\mbox{\tt type\ PId\ \ \ =\ Int}\\
\mbox{\tt type\ PList\ =\ [(PId,[Request->Response])]}
\eprog
An agent maps a list of OS states to responses.  Those
responses will be used as the contents of input channels, and
thus can depend on output channels, other input channels, files, or
any combination thereof.  For example, a valid implementation must
allow the user to act as agent between the standard output
channel and standard input channel.

Each running process (i.e.~program) has a unique \mbox{\tt PId}.
Elements of \mbox{\tt PList} are lists of running programs.

\bprog
\mbox{\tt os\ ::\ TagReqList\ ->\ State\ ->\ (TagRespList,\ State)}\\
\mbox{\tt type\ TagRespList\ =\ [(PId,Response)]}\\
\mbox{\tt type\ TagReqList\ \ =\ [(PId,Request)]}
\eprog
The operating system is modeled as a (non-deterministic) function
\mbox{\tt os}. The \mbox{\tt os} takes a tagged request list and an initial state, and
returns a tagged response list and a final state.  Given a list of
programs \mbox{\tt pList}, \mbox{\tt os} must exhibit this behaviour:
\bprog
\mbox{\tt (tagRespList,\ state')\ =\ os\ tagReqList\ state}\\
\mbox{\tt tagReqList\ =\ merge\ [\ zip\ [pId,pId..]\ (proc\ (untag\ pId\ tagRespList))}\\
\mbox{\tt \ \ \ \ \ \ \ \ \ \ \ \ \ \ \ \ \ \ \ \ \ \ |\ (pId,\ proc)\ <-\ pList\ ]}
\eprog
where \mbox{\tt merge} is a non-deterministic merge of a list of lists, and
\mbox{\tt untag} is:
\bprog
\mbox{\tt untag\ n\ []\ \ \ \ \ \ \ \ \ \ \ \ \ \ \ =\ []}\\
\mbox{\tt untag\ n\ ((m,resp):resps)\ =\ if\ n==m\ then\ resp:(untag\ n\ resps)}\\
\mbox{\tt \ \ \ \ \ \ \ \ \ \ \ \ \ \ \ \ \ \ \ \ \ \ \ \ \ \ \ \ \ \ \ \ \ \ \ else\ untag\ n\ resps}
\eprog
This relationship can be generalised to
include requests such as \mbox{\tt CreateProcess}.

A valid implementation must ensure that the input channel
system is defined at \mbox{\tt stdin} and the output channel system is
defined at \mbox{\tt stdout}, \mbox{\tt stderr}, and \mbox{\tt stdecho}.  If the agent
attached to standard input is called \mbox{\tt user} (i.e.~\mbox{\tt ics\ stdin} has
form \mbox{\tt (user,\ open)}), then \mbox{\tt user} must depend at least on
standard output.  In other words, this constraint must hold:
\bprog
\mbox{\tt user\ [...,\ (fs,(ics,ocs)),\ ...]\ =\ ...\ user'\ (ocs\ stdout)\ ...}
\eprog
where \mbox{\tt user'} is a {\em strict}, but otherwise arbitrary, function
modelling the user.  Its strictness corresponds to the user's
consumption of standard output whilst determining
standard input.

The rest of this section specifies the required behaviour of \mbox{\tt os} in
response to each kind of request.  This semantics is relatively
abstract and omits any reference to hardware errors (e.g.~``bad
sector on disk'') and system dependent errors (e.g.~``access rights
violation'').  Implementation-specific requests (for example the
environment requests) are not shown here.  We describe only
the text version of the requests: the binary version differs
trivially. \mbox{\tt os} is defined by:
\bprog
\mbox{\tt os\ ::\ TagReqList\ ->\ State\ ->\ (TagRespList,State)}\\
\mbox{\tt }\\[-8pt]
\mbox{\tt os\ []\ state\ =\ ([],\ state)}\\
\mbox{\tt os\ ((n,\ ReadChan\ name):es)\ state@(fs,(ics,ocs))\ =}\\
\mbox{\tt \ \ \ \ (alist',state')\ where}\\
\mbox{\tt \ \ \ \ \ \ \ \ \ \ (agent,open)\ =\ ics\ name}\\
\mbox{\tt \ \ \ \ \ \ \ \ \ \ alist'\ =\ (n,\ (if\ open\ n}\\
\mbox{\tt \ \ \ \ \ \ \ \ \ \ \ \ \ \ \ \ \ \ \ \ \ \ \ \ then\ fail}\\
\mbox{\tt \ \ \ \ \ \ \ \ \ \ \ \ \ \ \ \ \ \ \ \ \ \ \ \ else\ (agent\ (fs,ocs))\ ))\ :\ alist}\\
\mbox{\tt \ \ \ \ \ \ \ \ \ \ fail\ =\ Failure\ (OtherError\ "Channel\ already\ open{\char'134}n")}\\
\mbox{\tt \ \ \ \ \ \ \ \ \ \ (alist,state')\ =\ os\ es\ (fs,\ (update\ ics\ name}\\
\mbox{\tt \ \ \ \ \ \ \ \ \ \ \ \ \ \ \ \ \ \ \ \ \ \ \ \ \ \ \ \ \ \ \ \ \ \ \ \ \ \ \ \ \ (agent,\ update\ open\ n\ true),}\\
\mbox{\tt \ \ \ \ \ \ \ \ \ \ \ \ \ \ \ \ \ \ \ \ \ \ \ \ \ \ \ \ \ \ \ \ \ \ \ \ \ \ \ ocs))}
\eprog
where the auxiliary function \mbox{\tt update} is defined by:
\bprog
\mbox{\tt update\ f\ x\ v\ x'\ =\ if\ x==x'\ then\ v\ else\ f\ x}
\eprogNoSkip

If an attempt is made to read a non-existent channel, \mbox{\tt ics}
returns an agent that gives the appropriate error message when
applied to its arguments.  This definition is generalised in the
obvious way for the behaviour of \mbox{\tt ReadChannels}.  In particular, \mbox{\tt ack}
must be created by non-deterministically merging the result of applying
each agent to the stream of future states.
%Applying
%agents to the entire system state is simply meant to capture the most
%general situation where an adept agent might be able to
%observe the entire state of the system.
\bprog
\mbox{\tt os\ ((n,\ AppendChan\ name\ contents):es)\ state@(fs,(ics,ocs))\ =}\\
\mbox{\tt \ \ \ \ (alist',state')\ where}\\
\mbox{\tt \ \ \ \ \ \ \ \ \ \ alist'\ =\ ack:alist}\\
\mbox{\tt \ \ \ \ \ \ \ \ \ \ ack\ =}\\
\mbox{\tt \ \ \ \ \ \ \ \ \ \ \ (n,}\\
\mbox{\tt \ \ \ \ \ \ \ \ \ \ \ \ case\ (ocs\ name)\ of}\\
\mbox{\tt \ \ \ \ \ \ \ \ \ \ \ \ \ Failure\ msg\ ->\ Failure\ (SearchError\ "Nonexistent\ Channel")}\\
\mbox{\tt \ \ \ \ \ \ \ \ \ \ \ \ \ Str\ ochan\ ->\ Success}\\
\mbox{\tt \ \ \ \ \ \ \ \ \ \ \ \ \ Bn\ ochan\ ->\ Failure\ (FormatError\ "format\ error")}\\
\mbox{\tt \ \ \ \ \ \ \ \ \ \ \ \ )}\\
\mbox{\tt \ \ \ \ \ \ \ \ \ \ (alist,state')\ =\ os\ es\ (fs,(ics,}\\
\mbox{\tt \ \ \ \ \ \ \ \ \ \ \ \ \ \ \ \ \ \ \ \ \ \ \ \ \ \ \ \ \ \ \ \ \ \ \ \ \ \ case\ (ocs\ name)\ of}\\
\mbox{\tt \ \ \ \ \ \ \ \ \ \ \ \ \ \ \ \ \ \ \ \ \ \ \ \ \ \ \ \ \ \ \ \ \ \ \ \ \ \ \ \ Failure\ msg\ ->\ ocs}\\
\mbox{\tt \ \ \ \ \ \ \ \ \ \ \ \ \ \ \ \ \ \ \ \ \ \ \ \ \ \ \ \ \ \ \ \ \ \ \ \ \ \ \ \ Str\ ochan\ ->\ update\ ocs\ name}\\
\mbox{\tt \ \ \ \ \ \ \ \ \ \ \ \ \ \ \ \ \ \ \ \ \ \ \ \ \ \ \ \ \ \ \ \ \ \ \ \ \ \ \ \ \ \ \ \ \ \ (Str\ (ochan\ ++\ contents))}\\
\mbox{\tt \ \ \ \ \ \ \ \ \ \ \ \ \ \ \ \ \ \ \ \ \ \ \ \ \ \ \ \ \ \ \ \ \ \ \ \ \ \ \ \ Bn\ ochan\ ->\ ocs}\\
\mbox{\tt \ \ \ \ \ \ \ \ \ \ \ \ \ \ \ \ \ \ \ \ \ \ \ \ \ \ \ \ \ \ \ \ \ \ \ \ \ ))}
\eprogNoSkip
\bprog
\mbox{\tt os\ ((n,\ ReadFile\ name):es)\ state@(fs,(ics,ocs))\ =}\\
\mbox{\tt \ \ \ \ (alist',state')\ where}\\
\mbox{\tt \ \ \ \ \ \ \ \ \ \ alist'\ =\ ack\ :\ alist}\\
\mbox{\tt \ \ \ \ \ \ \ \ \ \ ack\ =\ (n,}\\
\mbox{\tt \ \ \ \ \ \ \ \ \ \ \ \ \ \ \ \ \ case\ (fs\ name)\ of}\\
\mbox{\tt \ \ \ \ \ \ \ \ \ \ \ \ \ \ \ \ \ \ Failure\ msg\ ->\ Failure\ (SearchError\ "File\ not\ found")}\\
\mbox{\tt \ \ \ \ \ \ \ \ \ \ \ \ \ \ \ \ \ \ Str\ string\ ->\ Str\ string}\\
\mbox{\tt \ \ \ \ \ \ \ \ \ \ \ \ \ \ \ \ \ \ Bn\ binary\ ->\ Failure\ (FormatError\ "")}\\
\mbox{\tt \ \ \ \ \ \ \ \ \ \ \ \ \ \ \ \ )}\\
\mbox{\tt \ \ \ \ \ \ \ \ \ \ (alist,state')\ =\ os\ es\ state}\\
\mbox{\tt }\\[-8pt]
\mbox{\tt os\ ((n,\ WriteFile\ name\ contents):es)\ state@(fs,(ics,ocs))\ =}\\
\mbox{\tt \ \ \ \ (alist',state')\ where}\\
\mbox{\tt \ \ \ \ \ \ \ \ \ \ alist'\ =\ (n,\ Success):alist}\\
\mbox{\tt \ \ \ \ \ \ \ \ \ \ (alist,state')\ =\ os\ es\ (update\ fs\ name\ (Str\ contents),}\\
\mbox{\tt \ \ \ \ \ \ \ \ \ \ \ \ \ \ \ \ \ \ \ \ \ \ \ \ \ \ \ \ \ \ \ \ \ \ (ics,ocs))}
\eprogNoSkip
\bprog
\mbox{\tt os\ ((n,\ AppendFile\ name\ contents):es)\ state@(fs,(ics,ocs))\ =}\\
\mbox{\tt \ \ \ \ (alist',state')\ where}\\
\mbox{\tt \ \ \ \ \ \ \ \ \ \ alist'\ =\ ack:alist}\\
\mbox{\tt \ \ \ \ \ \ \ \ \ \ ack\ =\ (n,\ }\\
\mbox{\tt \ \ \ \ \ \ \ \ \ \ \ \ \ \ \ \ case\ (fs\ name)\ of}\\
\mbox{\tt \ \ \ \ \ \ \ \ \ \ \ \ \ \ \ \ \ Failure\ msg\ ->\ Failure\ (SearchError\ "file\ not\ found")}\\
\mbox{\tt \ \ \ \ \ \ \ \ \ \ \ \ \ \ \ \ \ Str\ s\ ->\ Success}\\
\mbox{\tt \ \ \ \ \ \ \ \ \ \ \ \ \ \ \ \ \ Bn\ \ b\ ->\ Failure\ (FormatError\ "")}\\
\mbox{\tt \ \ \ \ \ \ \ \ \ \ \ \ \ \ \ \ )}\\
\mbox{\tt \ \ \ \ \ \ \ \ \ \ (alist,state')\ =\ os\ es\ (newfs,\ (ics,ocs))\ \ where}\\
\mbox{\tt \ \ \ \ \ \ \ \ \ \ \ \ \ \ \ \ \ \ \ \ \ \ \ \ \ \ \ newfs\ =\ case\ (fs\ name)\ of}\\
\mbox{\tt \ \ \ \ \ \ \ \ \ \ \ \ \ \ \ \ \ \ \ \ \ \ \ \ \ \ \ \ \ \ \ \ \ \ \ \ Failure\ msg\ ->\ fs}\\
\mbox{\tt \ \ \ \ \ \ \ \ \ \ \ \ \ \ \ \ \ \ \ \ \ \ \ \ \ \ \ \ \ \ \ \ \ \ \ \ Str\ s\ ->}\\
\mbox{\tt \ \ \ \ \ \ \ \ \ \ \ \ \ \ \ \ \ \ \ \ \ \ \ \ \ \ \ \ \ \ \ \ \ \ \ \ \ update\ fs\ name\ (Str\ (s++contents))}\\
\mbox{\tt \ \ \ \ \ \ \ \ \ \ \ \ \ \ \ \ \ \ \ \ \ \ \ \ \ \ \ \ \ \ \ \ \ \ \ \ Bn\ \ b\ ->\ fs}
\eprogNoSkip
\bprog
\mbox{\tt os\ ((n,\ DeleteFile\ name):es)\ state@(fs,(ics,ocs))\ =}\\
\mbox{\tt \ \ \ \ (alist',state')\ where}\\
\mbox{\tt \ \ \ \ \ \ \ \ \ \ alist'\ =\ ack\ :\ alist}\\
\mbox{\tt \ \ \ \ \ \ \ \ \ \ ack\ =\ (n,}\\
\mbox{\tt \ \ \ \ \ \ \ \ \ \ \ \ \ \ \ \ \ case\ (fs\ name)\ of}\\
\mbox{\tt \ \ \ \ \ \ \ \ \ \ \ \ \ \ \ \ \ \ Failure\ msg\ ->\ Failure\ (SearchError\ "file\ not\ found")}\\
\mbox{\tt \ \ \ \ \ \ \ \ \ \ \ \ \ \ \ \ \ \ Str\ s\ ->\ Success}\\
\mbox{\tt \ \ \ \ \ \ \ \ \ \ \ \ \ \ \ \ \ \ Bn\ b\ ->\ Success}\\
\mbox{\tt \ \ \ \ \ \ \ \ \ \ \ \ \ \ \ \ )}\\
\mbox{\tt \ \ \ \ \ \ \ \ \ \ (alist,state')\ =\ os\ es\ (case\ (fs\ name)\ of}\\
\mbox{\tt \ \ \ \ \ \ \ \ \ \ \ \ \ \ \ \ \ \ \ \ \ \ \ \ \ \ \ \ \ \ \ \ \ \ \ \ Failure\ msg\ ->\ fs}\\
\mbox{\tt \ \ \ \ \ \ \ \ \ \ \ \ \ \ \ \ \ \ \ \ \ \ \ \ \ \ \ \ \ \ \ \ \ \ \ \ Str\ s\ ->\ update\ fs\ name\ fail}\\
\mbox{\tt \ \ \ \ \ \ \ \ \ \ \ \ \ \ \ \ \ \ \ \ \ \ \ \ \ \ \ \ \ \ \ \ \ \ \ \ Bn\ b\ ->\ update\ fs\ name\ fail,}\\
\mbox{\tt \ \ \ \ \ \ \ \ \ \ \ \ \ \ \ \ \ \ \ \ \ \ \ \ \ \ \ \ \ \ \ \ \ \ (ics,ocs))}\\
\mbox{\tt \ \ \ \ \ \ \ \ \ \ fail\ =\ Failure\ (SearchError\ "file\ not\ found")}\\
\mbox{\tt }\\[-8pt]
\mbox{\tt os\ ((n,StatusFile\ name):es)\ state@(fs,(ics,ocs))\ =\ (alist',state')\ where}\\
\mbox{\tt \ \ \ \ \ \ \ \ \ \ alist'\ =\ ack\ :\ alist}\\
\mbox{\tt \ \ \ \ \ \ \ \ \ \ ack\ =\ (n,}\\
\mbox{\tt \ \ \ \ \ \ \ \ \ \ \ \ \ \ \ \ \ case\ (fs\ name)\ of}\\
\mbox{\tt \ \ \ \ \ \ \ \ \ \ \ \ \ \ \ \ \ \ Failure\ msg\ ->\ Failure\ (SearchError\ "File\ not\ found")}\\
\mbox{\tt \ \ \ \ \ \ \ \ \ \ \ \ \ \ \ \ \ \ Str\ string\ ->\ Str\ "t"++(rw\ n\ fs\ name)}\\
\mbox{\tt \ \ \ \ \ \ \ \ \ \ \ \ \ \ \ \ \ \ Bn\ binary\ ->\ Str\ "b"++(rw\ n\ fs\ name)}\\
\mbox{\tt \ \ \ \ \ \ \ \ \ \ \ \ \ \ \ \ )}\\
\mbox{\tt \ \ \ \ \ \ \ \ \ \ (alist,\ state')\ =\ os\ es\ state}
\eprog
where \mbox{\tt rw} is a function that determines the read and write status
of a file for this particular process.

% Local Variables: 
% mode: latex
% End:
\startnewsection
%
% $Header$
%
\subsection{Optional Requests}
\label{io-options}
\index{input/output!optional request}

These optional I/O requests may be useful in a \Haskell{}
implementation.

\begin{itemize}
\item
\mbox{\tt ReadChannels\ [cname1,\ ...,\ cnamek]}\\
\mbox{\tt ReadBinChannels\ [cname1,\ ...,\ cnamek]}

Opens \mbox{\tt cname1} through \mbox{\tt cnamek} for input.  A successful response has
form \mbox{\tt Tag\ vals} [\mbox{\tt BinTag\ vals}] where \mbox{\tt vals} is a list of values
tagged with the name of the channel.  These responses require an
extension to the \mbox{\tt Response} datatype:
\bprog
\mbox{\tt data\ \ Response\ =\ ...}\\
\mbox{\tt \ \ \ \ \ \ \ \ \ \ \ \ \ \ \ |\ Tag\ \ \ \ [(String,Char)]}\\
\mbox{\tt \ \ \ \ \ \ \ \ \ \ \ \ \ \ \ |\ BinTag\ [(String,Bin)]}
\eprog
The tagged list of values is the non-deterministic merge of the values
read from the
individual channels.  If an element of this list has form
\mbox{\tt (cnamei,val)}, then it came from channel \mbox{\tt cnamei}.

If any \mbox{\tt cnamei} does not exist then the response 
\mbox{\tt Failure\ (SearchError\ string)} is induced; all other errors induce
\mbox{\tt Failure\ (ReadError\ string)}.

\item
\mbox{\tt CreateProcess\ prog}

Introduces a new program \mbox{\tt prog} into the operating
system.  \mbox{\tt prog} must have type \mbox{\tt [Response]\ ->\ [Request]}.  Either \mbox{\tt Success}
and \mbox{\tt Failure\ (OtherError\ string)} is induced.

\item
\mbox{\tt CreateDirectory\ name\ string}\\
\mbox{\tt DeleteDirectory\ name\ }

Create or delete directory \mbox{\tt name}.  The \mbox{\tt string} argument to
\mbox{\tt CreateDirectory} is an implementation-dependent specification of the
initial state of the directory.

\item
\mbox{\tt OpenFile\ \ \ \ \ name\ inout}\\
\mbox{\tt OpenBinFile\ \ name\ inout}\\
\mbox{\tt CloseFile\ \ \ \ file}\\
\mbox{\tt ReadVal\ \ \ \ \ \ file}\\
\mbox{\tt ReadBinVal\ \ \ file}\\
\mbox{\tt WriteVal\ \ \ \ \ file\ char}\\
\mbox{\tt WriteBinVal\ \ file\ bin}

These requests emulate traditional file I/O in which
characters are read and written one at a time.
\bprog
\mbox{\tt data\ \ Response\ =\ ...}\\
\mbox{\tt \ \ \ \ \ \ \ \ \ \ \ \ \ \ \ |\ Fil\ File}\\
\mbox{\tt data\ \ File}\\
\mbox{\tt type\ \ Bins\ \ \ \ \ =\ [Bin]}
\eprog
\mbox{\tt OpenFile\ name\ inout} [\mbox{\tt OpenBinFile\ name\ inout}]
opens the file \mbox{\tt name} in text [binary] mode with
direction \mbox{\tt inout} (\mbox{\tt True} for input, \mbox{\tt False} for output).
The response \mbox{\tt Fil\ file} is induced, where \mbox{\tt file} has type \mbox{\tt File}, a
primitive type that represents a handle to a file.
Subsequent use of that file by other requests is via this
handle.

\mbox{\tt CloseFile\ file} closes \mbox{\tt file}.  \mbox{\tt Failure\ (OtherError\ string)} is
induced if \mbox{\tt file} cannot be closed.

\mbox{\tt ReadVal} [\mbox{\tt ReadBinVal}] \mbox{\tt file} reads \mbox{\tt file}, inducing the response 
\mbox{\tt Str\ val} [\mbox{\tt Bins\ val}] or \mbox{\tt Failure\ (ReadError\ string)}.

\mbox{\tt WriteVal\ file\ char} [\mbox{\tt WriteBinVal\ file\ bin}] writes \mbox{\tt char} [\mbox{\tt bin}] to
\mbox{\tt file}.  The response \mbox{\tt Success} or \mbox{\tt Failure\ (WriteError\ string)} is
induced.

\mbox{\tt Failure\ (SearchError\ string)} is induced for \mbox{\tt ReadVal}, \mbox{\tt ReadBinVal},
\mbox{\tt WriteVal}, and \mbox{\tt WriteBinVal} if \mbox{\tt file} is not a text or
binary file, as appropriate.
\end{itemize}

% Local Variables: 
% mode: latex
% End:
\startnewsection
%
% $Header$
%
% The paragraph describing the formats of standard representations might
% be deleted, since the info is already in the Prelude.  
% Note that there is a difference in the way readsPrec and showsPrec are defined.
% showsPrec is exact Haskell text, readsPrec uses an auxiliary function which
% isn't quite Haskell.  

\section{Specification of Derived Instances}
\label{derived-appendix}

If \mbox{$\it T$} is an algebraic datatype declared by:\index{algebraic datatype}
\[\begin{array}{lcl}
\mbox{$\it \makebox{\tt data\ }c\makebox{\tt \ =>}\ T\ u_1\ \ldots \ u_k$}&\mbox{\tt =}&\mbox{$\it K_1\ t_{11}\ \ldots \ t_{1k_1}\ \makebox{\tt |}\ \cdots\ \makebox{\tt |}\ K_n\ t_{n1}\ \ldots \ t_{nk_n}$}\\
& & \mbox{$\it \makebox{\tt deriving\ (}C_1\makebox{\tt ,}\ \ldots \makebox{\tt ,}\ C_m\makebox{\tt )}$}
\end{array}\]
(where \mbox{$\it m\geq0$} and the parentheses may be omitted if \mbox{$\it m=1$}) then
{\em a derived instance declaration is possible} for a class \mbox{$\it C$} 
if and only if these conditions hold:
\begin{enumerate}
\item
\mbox{$\it C$} is one of \mbox{\tt Eq}, \mbox{\tt Ord}, \mbox{\tt Enum}, \mbox{\tt Ix}, \mbox{\tt Text}, or \mbox{\tt Binary}.

\item
There is a context \mbox{$\it c'$} such that \mbox{$\it c'\ \Rightarrow\ C\ t_{ij}$}
holds for each of the constituent types \mbox{$\it t_{ij}$}.

\item
If \mbox{$\it C$} is either \mbox{\tt Ix} or \mbox{\tt Enum}, then further constraints must be
satisfied as described under the paragraphs for \mbox{\tt Ix} and \mbox{\tt Enum}
later in this section.

\item
There must be no explicit instance declaration elsewhere in the module which
makes \mbox{$\it T\ u_1\ \ldots \ u_k$} an instance of \mbox{$\it C$}.
% or of any of \mbox{$\it C$}'s superclasses.
\end{enumerate}

If the \mbox{\tt deriving} form is present (as in the above 
general \mbox{\tt data} declaration),
an instance declaration is automatically generated for \mbox{$\it T\ u_1\ \ldots \ u_k$}
over each class \mbox{$\it C_i$} and each of \mbox{$\it C_i$}'s superclasses.
If the derived instance declaration is impossible for any of the \mbox{$\it C_i$}
then a static error results.
If no derived instances are required, the \mbox{\tt deriving} form may be
omitted or the form \mbox{\tt deriving\ ()} may be used.

% OLD:
%If the \mbox{\tt deriving} form is omitted then instance
%declarations are derived for the datatype in as many of the six
%classes mentioned above as is possible; that is, no static error can occur.
%Since datatypes may be recursive, the implied inclusion in
%these classes may also be recursive, and the largest
%possible set of derived instances is generated.  For example,
%\bprog
%@%@
%data  T1 a = C1 (T2 a) | Nil1
%data  T2 a = C2 (T1 a) | Nil2
%@%@
%\eprog
%Because the \mbox{\tt deriving} form is omitted, we would expect derived
%instances for \mbox{\tt Eq} (for example).  But \mbox{\tt T1} is in \mbox{\tt Eq} only if \mbox{\tt T2}
%is, and \mbox{\tt T2} is in \mbox{\tt Eq} only if \mbox{\tt T1} is.  The largest solution has
%both types in \mbox{\tt Eq}, and thus both derived instances are generated.

Each derived instance declaration will have the form:
\[
\mbox{$\it \makebox{\tt instance\ (}c\makebox{\tt ,}\ C'_1\ u'_1\makebox{\tt ,}\ \ldots \makebox{\tt ,}\ C'_j\ u'_j\ \makebox{\tt )\ =>}\ C_i\ (T\ u_1\ \ldots \ u_k)\ \makebox{\tt where}\ \makebox{\tt {\char'173}}\ d\ \makebox{\tt {\char'175}}$}
\]
where \mbox{$\it d$} is derived automatically depending on \mbox{$\it C_i$} and the data
type declaration for \mbox{$\it T$} (as will be described in the remainder of
this section), and \mbox{$\it u'_1$} through \mbox{$\it u'_j$} form a subset of \mbox{$\it u_1$}
through \mbox{$\it u_k$}.
%% Yale nuked this:
%% The class assertions \mbox{$\it C'\ u'$} are constraints on \mbox{$\it T$}'s
%% type variables that are inferred from the instance declarations of the
%% constituent types \mbox{$\it t_{ij}$}.  For example, consider:
%% \bprog
%% @
%% data  T1 a = C1 (T2 a) deriving Eq
%% data  T2 a = C2 a      deriving ()
%% @
%% \eprog
%% And consider these three different instances for \mbox{\tt T2} in \mbox{\tt Eq}:\nopagebreak
%% \bprog
%% @
%% instance            Eq (T2 a) where (C2 x) == (C2 y)  =  True
%% 
%% instance (Eq  a) => Eq (T2 a) where (C2 x) == (C2 y)  =  x == y
%% 
%% instance (Ord a) => Eq (T2 a) where (C2 x) == (C2 y)  =  x > y
%% @
%% \eprog
%% The corresponding derived instances for \mbox{\tt T1} in \mbox{\tt Eq} are:
%% \bprog
%% @
%% instance            Eq (T1 a) where (C1 x) == (C1 y)  =  x == y
%% 
%% instance (Eq  a) => Eq (T1 a) where (C1 x) == (C1 y)  =  x == y
%% 
%% instance (Ord a) => Eq (T1 a) where (C1 x) == (C1 y)  =  x == y
%% @
%% \eprog
When inferring the context for the derived instances, type synonyms
must be expanded out first.\index{type synonym}
The free variables of the declarations $d$ are all functions
defined in the standard prelude.
%These prelude functions must 
%be in scope whenever \mbox{\tt deriving} instances are used that
%mention them.
The remaining details of the derived
instances for each of the six classes are now given.

\paragraph*{Derived instances of \mbox{\tt Eq} and \mbox{\tt Ord}.}
\index{Eq@{\ptt Eq}!derived instance}
\index{Ord@{\ptt Ord}!derived instance}
The operations automatically introduced by derived instances
of \mbox{\tt Eq} and \mbox{\tt Ord} are \mbox{\tt (==)}\indextt{==}, \mbox{\tt (/=)}\indextt{/=},
\mbox{\tt (<)}\indextt{<}, \mbox{\tt (<=)}\indextt{<=}, \mbox{\tt (>)}\indextt{>},
\mbox{\tt (>=)}\indextt{>=}, \mbox{\tt max}\indextt{max}, and 
\mbox{\tt min}\indextt{min}.  The latter six operators are defined so
as to compare their arguments lexicographically with respect to
the constructor set given, with earlier constructors in the datatype
declaration counting as smaller than later ones.  For example, for the
\mbox{\tt Bool} datatype, we have that \mbox{\tt True\ >\ False\ ==\ True}.

\paragraph*{Derived instances of \mbox{\tt Ix}.}
\index{Ix@{\ptt Ix}!derived instance}
The derived instance declarations for the class \mbox{\tt Ix}
introduce the overloaded functions
\mbox{\tt range}\indextt{range}, \mbox{\tt index}\indextt{index}, and
\mbox{\tt inRange}\indextt{inRange}.  The operation \mbox{\tt range} takes a (lower,
upper) bound pair, and returns a list of all indices in this range, in
ascending order.  The operation \mbox{\tt inRange} is a predicate taking a
(lower, upper) bound pair and an index and returning \mbox{\tt True} if the
index is contained within the specified range.  The operation \mbox{\tt index}
takes a (lower, upper) bound pair and an index and returns an integer,
the position of the index within the range.

Derived instance declarations for the class \mbox{\tt Ix} are only possible
for enumerations\index{enumeration} (i.e.~datatypes having
only nullary constructors) and single-constructor datatypes
(including tuples) whose constituent types are instances of \mbox{\tt Ix}.  
\begin{itemize}
\item
For an {\em enumeration}, the nullary constructors are assumed to be
numbered left-to-right with the indices 0 through $n-1\/$.  For example,
given the datatype:
\bprog
\mbox{\tt data\ \ Colour\ =\ Red\ |\ Orange\ |\ Yellow\ |\ Green\ |\ Blue\ |\ Indigo\ |\ Violet}
\eprog
we would have:
\bprog
\mbox{\tt range\ \ \ (Yellow,Blue)\ \ \ \ \ \ \ \ ==\ \ [Yellow,Green,Blue]}\\
\mbox{\tt index\ \ \ (Yellow,Blue)\ Green\ \ ==\ \ 1}\\
\mbox{\tt inRange\ (Yellow,Blue)\ Red\ \ \ \ ==\ \ False}
\eprog
\item
For {\em single-constructor datatypes}, the derived instance declarations
are created as shown for tuples in
Figure~\ref{prelude-index}.
\end{itemize}

%Instances of the class \mbox{\tt Ix}\indextt{Ix} are typically used as
%indices of arrays; a one-dimensional array might have index type
%\mbox{\tt Int}, a two-dimensional array, \mbox{\tt (Int,Char)}, and so forth.  (See
%Section~\ref{arrays}.)

\begin{figure}
\outline{
\mbox{\tt class\ \ (Ord\ a)\ =>\ Ix\ a\ where}\\
\mbox{\tt \ \ \ \ \ \ \ \ range\ \ \ \ \ \ \ \ \ \ \ ::\ (a,a)\ ->\ [a]}\\
\mbox{\tt \ \ \ \ \ \ \ \ index\ \ \ \ \ \ \ \ \ \ \ ::\ (a,a)\ ->\ a\ ->\ Int}\\
\mbox{\tt \ \ \ \ \ \ \ \ inRange\ \ \ \ \ \ \ \ \ ::\ (a,a)\ ->\ a\ ->\ Bool}\\
\mbox{\tt }\\[-8pt]
\mbox{\tt rangeSize\ \ \ \ \ \ \ \ \ \ \ \ \ \ \ ::\ (Ix\ a)\ =>\ (a,a)\ ->\ Int}\\
\mbox{\tt rangeSize\ (l,u)\ \ \ \ \ \ \ \ \ =\ \ index\ (l,u)\ u\ +\ 1}\\
\mbox{\tt }\\[-8pt]
\mbox{\tt instance\ \ (Ix\ a,\ Ix\ b)\ \ =>\ Ix\ (a,b)\ where}\\
\mbox{\tt \ \ \ \ \ \ \ \ range\ ((l,l'),(u,u'))}\\
\mbox{\tt \ \ \ \ \ \ \ \ \ \ \ \ \ \ \ \ =\ [(i,i')\ |\ i\ <-\ range\ (l,u),\ i'\ <-\ range\ (l',u')]}\\
\mbox{\tt \ \ \ \ \ \ \ \ index\ ((l,l'),(u,u'))\ (i,i')}\\
\mbox{\tt \ \ \ \ \ \ \ \ \ \ \ \ \ \ \ \ =\ \ index\ (l,u)\ i\ *\ rangeSize\ (l',u')\ +\ index\ (l',u')\ i'}\\
\mbox{\tt \ \ \ \ \ \ \ \ inRange\ ((l,l'),(u,u'))\ (i,i')}\\
\mbox{\tt \ \ \ \ \ \ \ \ \ \ \ \ \ \ \ \ =\ inRange\ (l,u)\ i\ {\char'46}{\char'46}\ inRange\ (l',u')\ i'}\\
\mbox{\tt }\\[-8pt]
\mbox{\tt --\ Instances\ for\ other\ tuples\ are\ obtained\ from\ this\ scheme:}\\
\mbox{\tt --}\\
\mbox{\tt --\ \ instance\ \ (Ix\ a1,\ Ix\ a2,\ ...\ ,\ Ix\ ak)\ =>\ Ix\ (a1,a2,...,ak)\ \ where}\\
\mbox{\tt --\ \ \ \ \ \ range\ ((l1,l2,...,lk),(u1,u2,...,uk))\ =}\\
\mbox{\tt --\ \ \ \ \ \ \ \ \ \ [(i1,i2,...,ik)\ |\ i1\ <-\ range\ (l1,u1),}\\
\mbox{\tt --\ \ \ \ \ \ \ \ \ \ \ \ \ \ \ \ \ \ \ \ \ \ \ \ \ \ \ \ i2\ <-\ range\ (l2,u2),}\\
\mbox{\tt --\ \ \ \ \ \ \ \ \ \ \ \ \ \ \ \ \ \ \ \ \ \ \ \ \ \ \ \ ...}\\
\mbox{\tt --\ \ \ \ \ \ \ \ \ \ \ \ \ \ \ \ \ \ \ \ \ \ \ \ \ \ \ \ ik\ <-\ range\ (lk,uk)]}\\
\mbox{\tt --}\\
\mbox{\tt --\ \ \ \ \ \ index\ ((l1,l2,...,lk),(u1,u2,...,uk))\ (i1,i2,...,ik)\ =}\\
\mbox{\tt --\ \ \ \ \ \ \ \ index\ (lk,uk)\ ik\ +\ rangeSize\ (lk,uk)\ *\ (}\\
\mbox{\tt --\ \ \ \ \ \ \ \ \ index\ (lk-1,uk-1)\ ik-1\ +\ rangeSize\ (lk-1,uk-1)\ *\ (}\\
\mbox{\tt --\ \ \ \ \ \ \ \ \ \ ...}\\
\mbox{\tt --\ \ \ \ \ \ \ \ \ \ \ index\ (l1,u1)))}\\
\mbox{\tt --}\\
\mbox{\tt --\ \ \ \ \ \ inRange\ ((l1,l2,...lk),(u1,u2,...,uk))\ (i1,i2,...,ik)\ =}\\
\mbox{\tt --\ \ \ \ \ \ \ \ \ \ inRange\ (l1,u1)\ i1\ {\char'46}{\char'46}\ inRange\ (l2,u2)\ i2\ {\char'46}{\char'46}}\\
\mbox{\tt --\ \ \ \ \ \ \ \ \ \ \ \ \ \ ...\ {\char'46}{\char'46}\ inRange\ (lk,uk)\ ik}
}
\ecaption{Index classes and instances}
\label{prelude-index}
\indextt{Ix}                                                
\indextt{range}\indextt{index}\indextt{inRange}   
\indextt{rangeSize}                                         
\end{figure}

\paragraph*{Derived instances of \mbox{\tt Enum}.}
\index{Enum@{\ptt Enum}!derived instance}
Derived instance declarations for the class \mbox{\tt Enum} are only
possible for enumerations, using the same ordering assumptions made
for \mbox{\tt Ix}.  They introduce the operations
\mbox{\tt enumFrom}\indextt{enumFrom},
\mbox{\tt enumFromThen}\indextt{enumFromThen}, \mbox{\tt enumFromTo}\indextt{enumFromTo}, and
\mbox{\tt enumFromThenTo}\indextt{enumFromThenTo},
which are used to define arithmetic sequences as described
in Section~\ref{arithmetic-sequences}.

\mbox{\tt enumFrom\ n} returns a list corresponding to the complete enumeration
of \mbox{\tt n}'s type starting at the value \mbox{\tt n}.
Similarly, \mbox{\tt enumFromThen\ n\ n'} is the enumeration starting at \mbox{\tt n}, but
with second element \mbox{\tt n'}, and with subsequent elements generated at a
spacing equal to the difference between \mbox{\tt n} and \mbox{\tt n'}.
\mbox{\tt enumFromTo} and \mbox{\tt enumFromThenTo} are as defined by the
default methods
\index{default method}
for \mbox{\tt Enum} (see Figure~\ref{standard-classes},
page~\pageref{standard-classes}).

\paragraph*{Derived instances of \mbox{\tt Binary}.}
\index{Binary@{\ptt Binary}!derived instance}
The \mbox{\tt Binary} class is used primarily for transparent I/O (see
Section~\ref{io-modes}).  The operations automatically introduced
by derived instances of \mbox{\tt Binary} are \mbox{\tt readBin}\indextt{readBin} and
\mbox{\tt showBin}\indextt{showBin}.  They coerce values to and
from the primitive abstract type \mbox{\tt Bin} (see Section~\ref{bin-type}).
An implementation must be able to create derived instances of \mbox{\tt Binary}
for any type \mbox{$\it t$} not containing a function type.

\mbox{\tt showBin} is analogous to \mbox{\tt shows}, taking two arguments: the first
is the value to be coerced, and the second is a \mbox{\tt Bin} value to which
the result is to be concatenated.  \mbox{\tt readBin} is analogous to \mbox{\tt reads},
``parsing'' its argument and returning a pair consisting of the
coerced value and any remaining \mbox{\tt Bin} value.  

Derived versions of \mbox{\tt showBin} and \mbox{\tt readBin} must obey this
property:
\[
\mbox{$\it \makebox{\tt readBin\ (showBin\ }v\ b\makebox{\tt )\ ==\ (}v\makebox{\tt ,}b\makebox{\tt )}$}
\]
for any \mbox{\tt Bin} value \mbox{$\it b$} and value \mbox{$\it v$} whose type is an instance of the
class \mbox{\tt Binary}.

\paragraph*{Derived instances of \mbox{\tt Text}.}
\index{Text@{\ptt Text}!derived instance}
The operations automatically introduced by derived instances
of \mbox{\tt Text} are \mbox{\tt showsPrec}\indextt{showsPrec}, \mbox{\tt readsPrec}\indextt{readsPrec},
\mbox{\tt showList}\indextt{showList} and \mbox{\tt readList}\indextt{readList}.
They are used to coerce values into strings and parse strings into
values.

The function \mbox{\tt showsPrec\ d\ x\ r} accepts a precedence level \mbox{\tt d}
(a number from \mbox{\tt 0} to \mbox{\tt 10}), a value \mbox{\tt x}, and a string \mbox{\tt r}.
It returns a string representing \mbox{\tt x} concatenated to \mbox{\tt r}.
\mbox{\tt showsPrec} satisfies the law:
\[
\mbox{$\it \makebox{\tt showsPrec\ d\ x\ r\ ++\ s\ \ ==\ \ showsPrec\ d\ x\ (r\ ++\ s)}$}
\]
The representation will be enclosed in parentheses if the precedence
of the top-level constructor operator in \mbox{\tt x} is less than \mbox{\tt d}.  Thus,
if \mbox{\tt d} is \mbox{\tt 0} then the result is never surrounded in parentheses; if
\mbox{\tt d} is \mbox{\tt 10} it is always surrounded in parentheses, unless it is an
atomic expression.  The extra parameter \mbox{\tt r} is essential if tree-like
structures are to be printed in linear time rather than time quadratic
in the size of the tree.

The function \mbox{\tt readsPrec\ d\ s} accepts a precedence level \mbox{\tt d} (a number
from \mbox{\tt 0} to \mbox{\tt 10}) and a string \mbox{\tt s}, and returns a list of pairs
\mbox{\tt (x,r)} such that \mbox{\tt showsPrec\ d\ x\ r\ ==\ s}.  \mbox{\tt readsPrec} is a parse
function, returning a list of (parsed value, remaining string) pairs.
If there is no successful parse, the returned list is empty.

\mbox{\tt showList} and \mbox{\tt readList} allow lists of objects to be represented
using non-standard denotations.  This is especially useful for strings
(lists of \mbox{\tt Char}).

%Because a string is a list of characters, \mbox{\tt showsPrec} of a string
%such as \mbox{\tt "abc"} will result in the string 
%\mbox{\tt "[}\fwq\mbox{\tt a}\fwq \mbox{\tt ,} \fwq\mbox{\tt b}\fwq \mbox{\tt ,} \fwq\mbox{\tt c}\fwq \mbox{\tt ]"}.  Because
%\mbox{\tt "{\char'134}"abc{\char'134}""} would be a better representation,
%the operators \mbox{\tt showList}
%and \mbox{\tt readList} are provided in the class \mbox{\tt Text} for coercing {\em
%lists} of values to and from strings.  In particular, \mbox{\tt showsPrec} of a
%string will yield the verbose form above, and \mbox{\tt showList} will yield
%the compact version.  For most other datatypes, \mbox{\tt showList} and
%\mbox{\tt readList} will yield the same result as \mbox{\tt showsPrec} and \mbox{\tt readsPrec}.

For convenience, the standard prelude provides the following auxiliary functions:
\bprog
\mbox{\tt shows\ \ \ \ =\ \ showsPrec\ 0}\\
\mbox{\tt reads\ \ \ \ =\ \ readsPrec\ 0}\\
\mbox{\tt show\ x\ \ \ =\ \ shows\ x\ ""}\\
\mbox{\tt read\ s\ \ \ =\ \ x\ \ where\ [(x,"")]\ =\ reads\ s}
\eprog
\mbox{\tt shows} and \mbox{\tt reads} use a default precedence of 0, and \mbox{\tt show} and \mbox{\tt read}
assume that the result is not being appended to an initial string.

The instances of \mbox{\tt Text} for the standard types \mbox{\tt Int}, \mbox{\tt Integer}, \mbox{\tt Float},
\mbox{\tt Double}, \mbox{\tt Char}, lists, tuples, and rational and complex numbers are defined in the 
standard prelude (see Appendix~\ref{stdprelude}).
For characters and strings, the control characters that have special
representations (\mbox{\tt {\char'134}n} etc.) are shown as such by \mbox{\tt showsPrec};
otherwise, ASCII mnemonics are used.
Non-ASCII characters are shown by decimal escapes.
Floating point numbers are represented by decimal numbers
of sufficient precision to guarantee \mbox{\tt read\ .\ show} is an identity
function.  If $b$ is the floating-point radix and there are
$w$ base-$b$ digits in the floating-point significand,
the number of decimal digits required is
$d = \lceil w \log_{10} b \rceil + 1$ \cite{matula70}.
Numbers are shown in non-exponential format if this requires
only $d$ digits; otherwise, they are shown in exponential format,
with one digit before the decimal point.  \mbox{\tt readsPrec} allows
an exponent to be unsigned or signed with \mbox{\tt +} or \mbox{\tt -}; \mbox{\tt showsPrec}
shows a positive exponent without a sign.

\mbox{\tt readsPrec} will parse any valid representation of the standard types 
apart from lists, for
which only the bracketed form \mbox{\tt [}\ldots\mbox{\tt ]} is accepted. See
Appendix~\ref{stdprelude} for full details.

%Figure~\ref{derived-text} gives the general form of a derived instance
%of \mbox{\tt Text} for a user-defined datatype:
%\[
%\mbox{$\it \makebox{\tt data}\ c\ \makebox{\tt =>}\ T\ u_1\ \ldots \ u_k\ \makebox{\tt =}\ \ldots $}
%\]
%\mbox{\tt showsPrec} and \mbox{\tt readsPrec} are as
%in Appendices~\ref{showsPrec-spec} and \ref{readsPrec-spec}.  The omitted
%definitions of \mbox{\tt readList} and \mbox{\tt showList} in
%Figure~\ref{standard-classes} (page~\pageref{standard-classes})
%are:
%\bprog
%@
%readList:: ReadS [a]
%readList r = [pr | (\mbox{$\it [$},s) <- lex r,
%                 pr      <- readl s    ]
%           where readl s = [([],t) | (\mbox{$\it ]$},t) <- lex s] ++
%                          [(x:xs,v) |  (x,t) <- reads s,
%                                       (\mbox{$\it ,$},u) <- lex t,
%                                       (xs,v) <- readl u       ]
%
%showList:: [a] -> ShowS
%showList xs = showChar '[' . showl xs
%            where
%            showl [] = showChar ']'
%            showl (x:xs) = shows x . showChar ',' . showl xs
%@
%\eprog
%\begin{figure}
%\outline{
%@
%instance (C, Text u1, ..., Text uk) => Text (T u1 ... uk) where
%       showsPrec = ...
%       readsPrec = ...
%@
%}
%\caption{General form of a derived instance of \mbox{\tt Text}}
%\label{derived-text}
%\end{figure}

\subsection{Specification of \mbox{\tt showsPrec}}
\label{showsPrec-spec}
\indextt{showsPrec}

As described in Section~\ref{derived-decls}, \mbox{\tt showsPrec} has the type
\[
\mbox{$\it \makebox{\tt (Text\ a)\ =>\ Int\ ->\ a\ ->\ String\ ->\ String}$}
\]
The first parameter is a
precedence in the range 0 to 10, the second is the value to be
converted into a string, and the third is the string
to append to the end of the result.

For all constructors \mbox{\tt Con} defined by some \mbox{\tt data} declaration
such as:
\[
\mbox{$\it \makebox{\tt data}\ c\ \makebox{\tt =>}\ T\ u_1\ \ldots \ u_k\ \makebox{\tt =}\ \ldots \ \makebox{\tt |\ Con}\ t_1\ \ldots \ t_n\ \makebox{\tt |}\ \ldots $}
\]
the corresponding definition of \mbox{\tt showsPrec} for \mbox{\tt Con} is shown in 
Figure~\ref{showsPrec-infix} for constructors declared in the infix
style and
Figure~\ref{showsPrec-fig} for all other constructors.  
See Appendix~\ref{stdprelude} for details of \mbox{\tt showParen}, \mbox{\tt showChar}, etc.

\begin{figure}
\outline{
\mbox{\tt \ \ \ \ showsPrec\ d\ (e1\ `Con`\ e2)\ =\ showParen\ (d\ >\ p)\ showStr}\\
\mbox{\tt \ \ \ \ \ \ where}\\
\mbox{\tt \ \ \ \ \ \ \ \ \ \ \ \ \ \ p\ =\ `the\ precedence\ of\ Con'}\\
\mbox{\tt \ \ \ \ \ \ \ \ \ \ \ \ \ \ lp\ =\ if\ `Con\ is\ left\ associative'\ then\ p\ else\ p+1}\\
\mbox{\tt \ \ \ \ \ \ \ \ \ \ \ \ \ \ rp\ =\ if\ `Con\ is\ right\ associative'\ then\ p\ else\ p+1}\\
\mbox{\tt \ \ \ \ \ \ \ \ \ \ \ \ \ \ cn\ =\ `the\ original\ name\ of\ Con'}\\
\mbox{\tt }\\[-8pt]
\mbox{\tt \ \ \ \ \ \ \ \ \ \ \ \ \ \ showStr\ =\ showsPrec\ lp\ e1\ .}\\
\mbox{\tt \ \ \ \ \ \ \ \ \ \ \ \ \ \ \ \ \ \ \ \ \ \ \ \ showChar\ '\ '\ .\ showString\ cn\ .\ showChar\ '\ '\ .}\\
\mbox{\tt \ \ \ \ \ \ \ \ \ \ \ \ \ \ \ \ \ \ \ \ \ \ \ \ showsPrec\ rp\ e2}
}
\caption{Specification of \mbox{\tt showsPrec} for Constructors Declared in the Infix Style}
\label{showsPrec-infix}
\end{figure}

\begin{figure}
\outline{
\mbox{\tt \ \ \ \ showsPrec\ d\ (Con\ e1\ ...\ en)\ =\ showParen\ (d\ >=\ 10)\ showStr}\\
\mbox{\tt \ \ \ \ \ \ where}\\
\mbox{\tt \ \ \ \ \ \ \ \ \ \ \ \ \ \ showStr\ =\ showString\ cn\ .\ showChar\ '\ '\ .}\\
\mbox{\tt \ \ \ \ \ \ \ \ \ \ \ \ \ \ \ \ \ \ \ \ \ \ \ \ showsPrec\ 10\ e1\ .\ showChar\ '\ '\ .}\\
\mbox{\tt \ \ \ \ \ \ \ \ \ \ \ \ \ \ \ \ \ \ \ \ \ \ \ \ ...}\\
\mbox{\tt \ \ \ \ \ \ \ \ \ \ \ \ \ \ \ \ \ \ \ \ \ \ \ \ showsPrec\ 10\ en}\\
\mbox{\tt \ \ \ \ \ \ \ \ \ \ \ \ \ \ cn\ =\ `the\ original\ name\ of\ Con'}
}
\caption{General Specification of \mbox{\tt showsPrec} for User-Defined Constructors}
\label{showsPrec-fig}
\end{figure}

\subsection{Specification of \mbox{\tt readsPrec}}
\label{readsPrec-spec}
\indextt{readsPrec}

A {\em lexeme} is exactly as in Section~\ref{lexical-structure}.
\mbox{\tt lex\ ::\ String\ ->\ [(String,String)]} reads the first lexeme from a
string.  If the string begins with a valid lexeme, the lexeme (with
leading whitespace removed) and the remainder of the string are
returned in a singleton list.  If no lexeme is present or the lexeme
is not syntacticly correct, \mbox{\tt []} is returned.  A full definition is
provided in Appendix~\ref{preludetext}.

% \mbox{\tt lex\ ::\ String\ ->\ [(String,\ String)]} parses a string into two
% parts: (1)~a string representing the first lexeme or \mbox{\tt ""} if the input
% is \mbox{\tt ""} or consists entirely of white space, and (2)~the remainder of
% the input after the first lexeme is extracted.  Whitespace (again
% refer to Section~\ref{lexical-structure}) is ignored except within
% strings.  A successful parse results in a singleton list containing
% a pair of strings; if no proper lexeme can be parsed (such as in the case
% of an unrecognised control character), an empty list is returned.
% A full definition is provided in Appendix~\ref{preludetext}.

As described in Section~\ref{derived-decls}, \mbox{\tt readsPrec} has the type
\[
\mbox{$\it \makebox{\tt Text\ a\ =>\ Int\ ->\ String\ ->\ [(a,String)]}$}
\]
Its first parameter is a
precedence in the range 0 to 10, its second is the string to be
parsed.  
Figure~\ref{readsPrec-fig} shows the specification of \mbox{\tt readsPrec} for user-defined 
datatypes of the form:
\[
\mbox{$\it \makebox{\tt data}\ c\ \makebox{\tt =>}\ T\ u_1\ \ldots \ u_k\ \makebox{\tt =}\ K_1\ t_{11}\ \ldots \ t_{1k_1}\ \makebox{\tt |}\ \ldots \ \makebox{\tt |}\ K_n\ t_{n1}\ \ldots \ t_{nk_n}$}
\]

\begin{figure}
\outline{
\mbox{\tt readsPrec\ d\ r\ =\ readCon\ K1\ k1\ `the\ original\ name\ of\ K1'\ r\ ++}\\
\mbox{\tt \ \ \ \ \ \ \ \ \ \ \ ...\ }\\
\mbox{\tt \ \ \ \ \ \ \ \ \ \ \ readCon\ Kn\ kn\ `the\ original\ name\ of\ Kn'\ r}\\
\mbox{\tt \ \ where}\\
\mbox{\tt \ \ \ \ readCon\ con\ n\ cn\ =\ \ \ \ \ \ \ \ \ \ \ \ \ \ \ \ \ \ --\ if\ con\ is\ infix}\\
\mbox{\tt \ \ \ \ \ \ \ \ readParen\ (d\ >\ p)\ readVal\ }\\
\mbox{\tt \ \ \ \ \ \ where}\\
\mbox{\tt \ \ \ \ \ \ \ \ \ \ \ \ \ \ \ \ readVal\ r\ =\ \ [(u\ `con`\ v,\ s2)\ |}\\
\mbox{\tt \ \ \ \ \ \ \ \ \ \ \ \ \ \ \ \ \ \ \ \ \ \ \ \ \ \ \ \ \ (u,s0)\ \ \ <-\ readsPrec\ lp\ r,}\\
\mbox{\tt \ \ \ \ \ \ \ \ \ \ \ \ \ \ \ \ \ \ \ \ \ \ \ \ \ \ \ \ \ (tok,s1)\ <-\ lex\ s0,\ tok\ ==\ cn,}\\
\mbox{\tt \ \ \ \ \ \ \ \ \ \ \ \ \ \ \ \ \ \ \ \ \ \ \ \ \ \ \ \ \ (v,s2)\ \ \ <-\ readsPrec\ rp\ s1]}\\
\mbox{\tt \ \ \ \ \ \ \ \ \ \ \ \ \ \ \ \ p\ =\ `the\ precedence\ of\ con'}\\
\mbox{\tt \ \ \ \ \ \ \ \ \ \ \ \ \ \ \ \ lp\ =\ if\ `con\ is\ left\ associative'\ then\ p\ else\ p+1}\\
\mbox{\tt \ \ \ \ \ \ \ \ \ \ \ \ \ \ \ \ rp\ =\ if\ `con\ is\ right\ associative'\ then\ p\ else\ p+1}\\
\mbox{\tt }\\[-8pt]
\mbox{\tt \ \ \ \ readCon\ con\ n\ cn\ =\ \ \ \ \ \ \ \ \ \ \ \ \ \ \ \ \ \ --\ if\ con\ is\ not\ infix}\\
\mbox{\tt \ \ \ \ \ \ \ \ readParen\ (d\ >\ 9)\ readVal}\\
\mbox{\tt \ \ \ \ \ \ where}\\
\mbox{\tt \ \ \ \ \ \ \ \ \ \ \ \ \ \ \ \ \ readVal\ r\ =\ [(con\ t1\ ...\ tn,\ sn)\ |}\\
\mbox{\tt \ \ \ \ \ \ \ \ \ \ \ \ \ \ \ \ \ \ \ \ \ \ \ \ \ \ \ \ \ (t0,s0)\ <-\ lex\ r,\ t0\ ==\ cn,}\\
\mbox{\tt \ \ \ \ \ \ \ \ \ \ \ \ \ \ \ \ \ \ \ \ \ \ \ \ \ \ \ \ \ (t1,s1)\ <-\ readsPrec\ 10\ s0,}\\
\mbox{\tt \ \ \ \ \ \ \ \ \ \ \ \ \ \ \ \ \ \ \ \ \ \ \ \ \ \ \ \ \ ...}\\
\mbox{\tt \ \ \ \ \ \ \ \ \ \ \ \ \ \ \ \ \ \ \ \ \ \ \ \ \ \ \ \ \ (tn,sn)\ <-\ readsPrec\ 10\ s(n-1)]}
}
\caption{Definition of \mbox{\tt readsPrec} for User-Defined Types}
\label{readsPrec-fig}
\end{figure}

\subsection{An example}

As a complete example, consider a tree datatype:\nopagebreak
%\bprog
%@
%data Tree a = Leaf a | Tree a :^: Tree a
%@
%\eprog\nopagebreak
%Since there is no \mbox{\tt deriving} clause, this is shorthand for:\nopagebreak
\bprog
\mbox{\tt data\ Tree\ a\ =\ Leaf\ a\ |\ Tree\ a\ :{\char'136}:\ Tree\ a}\\
\mbox{\tt \ \ \ \ \ deriving\ (Eq,\ Ord,\ Text,\ Binary)}\\
\mbox{\tt instance\ (Eq\ a)\ =>\ Eq\ (Tree\ a)}\\
\mbox{\tt \ \ where\ ...}\\
\mbox{\tt instance\ (Ord\ a)\ =>\ Ord\ (Tree\ a)}\\
\mbox{\tt \ \ where\ ...}\\
\mbox{\tt instance\ (Text\ a)\ =>\ Text\ (Tree\ a)}\\
\mbox{\tt \ \ where\ ...}\\
\mbox{\tt instance\ (Binary\ a)\ =>\ Binary\ (Tree\ a)}\\
\mbox{\tt \ \ where\ ...}
\eprog
Note the recursive context; the components of the datatype must
themselves be instances of the class.  Automatic derivation of
instance
declarations for \mbox{\tt Ix} and \mbox{\tt Enum} are not possible, as \mbox{\tt Tree} is not
an enumeration or single-constructor datatype.  Except for \mbox{\tt Binary}, the complete
instance declarations for \mbox{\tt Tree} are shown in Figure~\ref{tree-inst},
Note the implicit use of default-method
\index{default method}
definitions---for
example, only \mbox{\tt <=} is defined for \mbox{\tt Ord}, with the other
operations (\mbox{\tt <}, \mbox{\tt >}, \mbox{\tt >=}, \mbox{\tt max}, and \mbox{\tt min}) being defined by the defaults given in
the class declaration shown in Figure~\ref{standard-classes}
(page~\pageref{standard-classes}).

\begin{figure}
\outline{
\mbox{\tt infix\ 4\ :{\char'136}:}\\
\mbox{\tt data\ Tree\ a\ =\ \ Leaf\ a\ \ |\ \ Tree\ a\ :{\char'136}:\ Tree\ a}\\
\mbox{\tt }\\[-8pt]
\mbox{\tt instance\ (Eq\ a)\ =>\ Eq\ (Tree\ a)\ where}\\
\mbox{\tt \ \ \ \ \ \ \ \ Leaf\ m\ ==\ Leaf\ n\ \ =\ \ m==n}\\
\mbox{\tt \ \ \ \ \ \ \ \ u:{\char'136}:v\ \ ==\ x:{\char'136}:y\ \ \ =\ \ u==x\ {\char'46}{\char'46}\ v==y}\\
\mbox{\tt \ \ \ \ \ \ \ \ \ \ \ \ \ {\char'137}\ ==\ {\char'137}\ \ \ \ \ \ \ =\ \ False}\\
\mbox{\tt }\\[-8pt]
\mbox{\tt instance\ (Ord\ a)\ =>\ Ord\ (Tree\ a)\ where}\\
\mbox{\tt \ \ \ \ \ \ \ \ Leaf\ m\ <=\ Leaf\ n\ \ =\ \ m<=n}\\
\mbox{\tt \ \ \ \ \ \ \ \ Leaf\ m\ <=\ x:{\char'136}:y\ \ \ =\ \ True}\\
\mbox{\tt \ \ \ \ \ \ \ \ u:{\char'136}:v\ \ <=\ Leaf\ n\ \ =\ \ False}\\
\mbox{\tt \ \ \ \ \ \ \ \ u:{\char'136}:v\ \ <=\ x:{\char'136}:y\ \ \ =\ \ u<x\ ||\ u==x\ {\char'46}{\char'46}\ v<=y}\\
\mbox{\tt }\\[-8pt]
\mbox{\tt instance\ (Text\ a)\ =>\ Text\ (Tree\ a)\ where}\\
\mbox{\tt }\\[-8pt]
\mbox{\tt \ \ \ \ \ \ \ \ showsPrec\ d\ (Leaf\ m)\ =\ showParen\ (d\ >=\ 10)\ showStr}\\
\mbox{\tt \ \ \ \ \ \ \ \ \ \ where}\\
\mbox{\tt \ \ \ \ \ \ \ \ \ \ \ \ \ showStr\ =\ showString\ "Leaf"\ .\ showChar\ '\ '\ .\ showsPrec\ 10\ m}\\
\mbox{\tt }\\[-8pt]
\mbox{\tt \ \ \ \ \ \ \ \ showsPrec\ d\ (u\ :{\char'136}:\ v)\ =\ showParen\ (d\ >\ 4)\ showStr}\\
\mbox{\tt \ \ \ \ \ \ \ \ \ \ where}\\
\mbox{\tt \ \ \ \ \ \ \ \ \ \ \ \ \ showStr\ =\ showsPrec\ 5\ u\ .\ }\\
\mbox{\tt \ \ \ \ \ \ \ \ \ \ \ \ \ \ \ \ \ \ \ \ \ \ \ showChar\ '\ '\ .\ showString\ ":{\char'136}:"\ .\ showChar\ '\ '\ .}\\
\mbox{\tt \ \ \ \ \ \ \ \ \ \ \ \ \ \ \ \ \ \ \ \ \ \ \ showsPrec\ 5\ v}\\
\mbox{\tt }\\[-8pt]
\mbox{\tt \ \ \ \ \ \ \ \ readsPrec\ d\ r\ =\ \ readParen\ (d\ >\ 4)}\\
\mbox{\tt \ \ \ \ \ \ \ \ \ \ \ \ \ \ \ \ \ \ \ \ \ \ \ \ \ ({\char'134}r\ ->\ [(u:{\char'136}:v,w)\ |}\\
\mbox{\tt \ \ \ \ \ \ \ \ \ \ \ \ \ \ \ \ \ \ \ \ \ \ \ \ \ \ \ \ \ \ \ \ \ (u,s)\ <-\ readsPrec\ 5\ r,}\\
\mbox{\tt \ \ \ \ \ \ \ \ \ \ \ \ \ \ \ \ \ \ \ \ \ \ \ \ \ \ \ \ \ \ \ \ \ (":{\char'136}:",t)\ <-\ lex\ s,}\\
\mbox{\tt \ \ \ \ \ \ \ \ \ \ \ \ \ \ \ \ \ \ \ \ \ \ \ \ \ \ \ \ \ \ \ \ \ (v,w)\ <-\ readsPrec\ 5\ t])\ r}\\
\mbox{\tt }\\[-8pt]
\mbox{\tt \ \ \ \ \ \ \ \ \ \ \ \ \ \ \ \ \ \ \ \ \ \ ++\ readParen\ (d\ >\ 9)}\\
\mbox{\tt \ \ \ \ \ \ \ \ \ \ \ \ \ \ \ \ \ \ \ \ \ \ \ \ \ ({\char'134}r\ ->\ [(Leaf\ m,t)\ |}\\
\mbox{\tt \ \ \ \ \ \ \ \ \ \ \ \ \ \ \ \ \ \ \ \ \ \ \ \ \ \ \ \ \ \ \ \ \ ("Leaf",t)\ <-\ lex\ r,}\\
\mbox{\tt \ \ \ \ \ \ \ \ \ \ \ \ \ \ \ \ \ \ \ \ \ \ \ \ \ \ \ \ \ \ \ \ \ (m,t)\ <-\ readsPrec\ 10\ t])\ r}
}
\ecaption{Example of Derived Instances}
\label{tree-inst}
\end{figure}

% Local Variables: 
% mode: latex
% End:

%
\startnewstuff
% insert the extra indexing things LAST
% \mbox{$\it see\ XXX$} and \mbox{$\it see\ also\ YYY$} index entries
%
% collected here and inserted at the END of the report
% so that those index entries will always appear LAST
% -------------------------------------------------------------------
% but first some ...
%
% NOTES ON INDEXING: by Will Partain (partain@dcs.glasgow.ac.uk)

% Paul Hudak did the initial selection of index entries for the report
% (version 1.0); I haven't changed that, really.  I did the automatic
% indexing of the code in the prelude (Appendix A), and the
% plug-and-chug indexing of the syntax non-terminals.

% The important thing is consistency!  Here's what I've done -- my
% model has been the index for the LaTeX manual.  This doesn't mean
% that there aren't plenty of shortcomings.
%
% 1) I always index the singular form, i.e., \mbox{$\it file$}, not \mbox{$\it files$};
%    \mbox{$\it expression$}, not \mbox{$\it expressions$}.
%
% 2) The tricky part is how to index things that could come under
%    several possible titles.  Here's an example and how I handled it.
%    Let's say we have several index entries for \mbox{$\it list\ expression$},
%    \mbox{$\it conditional\ expression$}, and \mbox{$\it case\ expression$}.
%
%    * we want each index entry to appear under exactly one name
%
%    * we want other likely/interesting entries to at least be
%      cross-referenced.
%
%    So:
%
%    * put each entry under the straightforward un-reordered thing;
%      in the example, \mbox{$\it list\ expression$}, \mbox{$\it case\ expression$}, etc.
%
%    * under \mbox{$\it expression$} (a likely place for someone to look), have:
%
%       expression
%           case, {\em see} case expression
%           conditional, {\em see} conditional expression
%           list, {\em see} list expression
% -------------------------------------------------------------------
%
\index{application!function|see{function application}}
\index{application!operator|see{operator application}}
\index{binding!function|see{function binding}}
\index{binding!pattern|see{pattern binding}}
\index{binding!simple pattern|see{simple pattern binding}}
\index{character set!ASCII|see{ASCII character set}}
\index{character set!transparent|see{transparent character set}}
\index{datatype!abstract|see{abstract datatype}}
\index{datatype!algebraic|see{algebraic datatype}}
\index{datatype!declaration|see{{\ptt data} declaration}}
\index{datatype!recursive|see{recursive datatype}}
\index{declaration!datatype|see{{\ptt data} declaration}}
\index{declaration!class|see{class declaration}}
\index{declaration!instance|see{instance declaration}}
\index{declaration!C-T instance@$C$-$T$ instance|see{$C$-$T$ instance declaration}}
\index{declaration!default|see{{\ptt default} declaration}}
\index{declaration!import|see{import declaration}}
\index{declaration!fixity|see{fixity declaration}}
\index{environment!class|see{class environment}}
\index{environment!type|see{type environment}}
\index{expression!type|see{type expression}}
\index{expression!conditional|see{conditional expression}}
\index{expression!unit|see{unit expression}}
\index{expression!let|see{let expression}}
\index{expression!case|see{case expression}}
\index{expression!simple case|see{simple case expression}}
\index{"@-pattern@{\ptt {\char'100}}|see{as-pattern}}
\index{_@{\ptt {\char'137}}|see{wildcard pattern}}
\index{pattern!"@-pattern@{\ptt {\char'100}}|see{as-pattern}}
\index{pattern!_@{\ptt {\char'137}}|see{wildcard pattern}}
\index{pattern!constructed|see{constructed pattern}}
\index{pattern!integer|see{integer literal pattern}}
\index{pattern!floating|see{floating literal pattern}}
\index{pattern!linear|see{linear pattern}}
\index{pattern!n+k@\mbox{$\it n\makebox{\tt +}k$}|see{\mbox{$\it n\makebox{\tt +}k$} pattern}}
\index{pattern!irrefutable|see{irrefutable pattern}}
\index{pattern!refutable|see{refutable pattern}}
\index{semantics!formal|see{formal semantics}}
\index{semantics!input/output|see{input/output semantics}}
\index{string!transparent|see{transparent string}}
\index{type!ambiguous|see{ambiguous type}}
\index{type!monomorphic|see{monomorphic type}}
\index{type!principal|see{principal type}}
\index{type!trivial|see{trivial type}}
\index{type!numeric|see{numeric type}}
\index{type!function|see{function type}}
\index{type!constructed|see{constructed type}}
\index{type!tuple|see{tuple type}}
\index{type!list|see{list type}}
\index{type signature!for an expression|see{expression type-signature}}
%
\index{(aaa)@{\ptt ()}|see{trivial type and unit expression}}%
\index{-@{\ptt -}|seealso{negation}}
\index{default method|seealso{class method}}
\index{derived instance|seealso{instance declaration}}
\index{if-then-else expression|see{conditional expression}}
\index{instance declaration|seealso{derived instance}}
\index{layout|seealso{off-side rule}}
\index{off-side rule|seealso{layout}}
\index{method|see{class method}}
\index{overloaded pattern|see{pattern-matching}}
\index{precedence|seealso{fixity}}
\index{section|seealso{operator application}}
\index{signature|see{type signature}}
\index{standard prelude|seealso{{\ptt Prelude}}}
\index{synonym|see{type synonym}}
\index{type class|see{class}}
\index{type synonym|seealso{datatype}}
\index{unit datatype|see{trivial type}}

\bibliographystyle{plain}
\bibliography{haskell}
%
\startnewstuff
\printindex
\end{document}

% Local Variables: 
% mode: latex
% End:
