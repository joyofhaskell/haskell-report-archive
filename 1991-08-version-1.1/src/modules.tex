%
% $Header$
%
\section{Modules} 
\label{modules} 
\index{module}

A module defines a collection of values, datatypes, type synonyms,
classes, etc.~(see Section~\ref{declarations}),
and {\em exports} some of these resources, making them available to
other modules.  We use the term {\em entity}\index{entity} to refer to
the values, types, and classes defined in and perhaps exported from a
module.

A \Haskell{} {\em program} is a collection of modules, one of
which, by convention, must be called \mbox{\tt Main}\indexmodule{Main} and must
export the value \mbox{\tt main}\indextt{main}.  The {\em value} of the program
is the value of the identifier \mbox{\tt main} in module \mbox{\tt Main}, and \mbox{\tt main}
must have type \mbox{\tt Dialogue} (see Section~\ref{io}).

% More precisely, if all
% importations, renamings, etc. as described in this section are
% eliminated, and the original declarations (with suitable renamings 
% to prevent name clashes) are collected together into one set of
% declarations called \mbox{$\it decls$}, then the value of the program is 
% \mbox{$\it \makebox{\tt main\ where\ }decls$}.  The semantics of a \mbox{\tt where} clause is defined
% in Section~\ref{let-expressions}.

Modules may reference other modules via explicit
\mbox{\tt import} declarations, each giving the name of a module to be
imported, specifying its entities to be imported, and
optionally renaming some or all of them.  Modules may be mutually
recursive.

The name-space for modules is flat, with each module being associated
with a unique module name (which are \Haskell{} identifiers
beginning with a capital letter; i.e.~$aconid$).  There are two
distinguished modules, \mbox{\tt PreludeCore} and \mbox{\tt Prelude}, both
discussed in Section~\ref{standard-prelude}.

\subsection{Overview}
\label{module-structure}

\subsubsection{Interfaces and Implementations}

A module consists of an {\em interface}\index{interface} and an
{\em implementation}\index{implementation} of that interface.

The interface of a module provides complete information about the
static semantics of that module, including type signatures, class
definitions, and type declarations for the various entities made
available by the module.  This information is complete in this
sense:  If a module $M$ imports modules $M_1, \ldots, M_n$,
then only the interfaces of $M_1, \ldots, M_n$ need be examined in
order to perform static checking on the implementation of {\it M.}  No
implementations of $M_1, \ldots, M_n$ need to exist, nor need any
further interfaces be consulted.  Interfaces are discussed
in Section~\ref{module-interfaces}.

An implementation ``fills in'' the information about a module missing
from the interface.  For example, for each value given a type
signature in the interface the implementation either imports a module
that defines the value or defines the value itself.  Implementations
are discussed in Section~\ref{module-implementations}.

\subsubsection{Original Names}
\label{original-names}
\index{original name}
\index{renaming}

It may be that a particular entity is imported into a module by more
than one route---for example, because it is exported by two modules
both of which are imported by a third module.  It is important that
benign name-clashes of this form are allowed, but that accidental
name-clashes are detected and reported as errors.  This is done as
follows:

Each entity (class, type constructor, value, etc.) has an {\em
original name} that is a pair consisting of the name of the module in
which it was originally declared, and the name it was given in that
declaration.  The original name is carried with the entity wherever it
is exported.  Two entities are the same if and only if they have the
same original name.

\index{renaming!with respect to original name}
Renaming does {\em not} affect the original name; it is a purely
syntactic operation that affects only the name by which the entity is
currently known.  For example, if a class is renamed and a type is
declared to be an instance of the newly-named class, then it is also
an instance of the original class---there is just one class, which
happens to be known by different names in different parts of the
program.  Also, fixity is a property of the original name of an
identifier or operator and is not affected by renaming; the new
name has the same fixity as the old one.

A given entity may be known by at most one name in any scope.
So, for example, a module may not import an entity twice and rename it
differently on each occasion.  Either it must be renamed in the same
way on each import or else not imported twice (for example, by using
a \mbox{\tt hiding} clause).

As there are several name spaces, a single name may identify 
more than one entity. In a \mbox{\tt renaming} clause, such as 
\mbox{$\it \makebox{\tt renaming(}\ldots ,n_1\ \makebox{\tt to}\ n_2,\ldots \makebox{\tt )}$}, {\em all} the entities to
which $n_1$ refers are renamed to $n_2$.

\subsubsection{Closure}
\label{closure}
\index{module!closure}

The implementation together with the interfaces of
the modules it imports must be {\em statically closed} according to
this rule: {\em every value, type, or class referred to
in the text of 
an implementation together with the
entities that it imports, must be 
declared in the implementation or in one of the
imported interfaces}.

It is an error for a module to export a collection of entities that
cannot possibly become closed.  For example, if a module \mbox{\tt A} declares
both the type \mbox{\tt T} and a value \mbox{\tt t} of type \mbox{\tt T}, it may not export \mbox{\tt t}
without also exporting \mbox{\tt T}.  But if
%However, the closure condition applies on {\em import},
%not on {\em export}.  For example, if
another module \mbox{\tt B} imported \mbox{\tt T}
from module \mbox{\tt A}, and declared another value \mbox{\tt s} of type \mbox{\tt T}, it may
export \mbox{\tt s} without exporting \mbox{\tt T}---but any module importing \mbox{\tt B} must
also import the type \mbox{\tt T} by some other route, for example by also
importing \mbox{\tt A}.

%There is one important exception to the closure rule: {\em instance
%declarations in imported interfaces are not subject to it, and are
%ignored if they refer to a class or type constructor which is not in
%scope}.  This allows an interface to export a C-T instance declaration
%(see Section~\ref{instance-decls}) which takes effect when both C and T
%are in scope, but does no harm if one or the other (or both) is not.
%
\subsubsection{The Compilation System}

The task of checking consistency between interfaces and
implementations must be done by the {\em compilation
system}\index{compilation system}.

\Haskell{} does not specify any particular association between
implementations and interfaces on the one hand, and {\em files} on the
other; nor does it specify how implementations and interfaces are
produced.  These matters are determined by the compilation system, and
many variations are possible, depending on the programming
environment.  For example, a compilation system could insist that each
implementation and each interface reside alone in a file, and that the
module name is the same as that of the file, with the implementation
and interface distinguished by a suffix.
%Alternatively, if no such
%restrictions are made, then the compilation system has
%to map module names onto file names.

Similarly, a compilation system may require the programmer to write
the interface, or it may derive the interface from examination of the
implementation, or some hybrid of the two.  \Haskell{} is defined so
that, given the interfaces of all imported modules, it is always
possible to perform a complete static check on the implementation,
and, if it is well-typed, to derive its unique
interface automatically.  However, given a set of mutually recursive
implementations, the compilation system may have to examine several
modules at once to derive the interfaces, which will still be unique
with one exception: because of the shorthand for exporting all
entities from an imported module, the set of exports may not be
unique.  Any set satisfying the consistency constraints is a valid
solution for a well-typed \Haskell{} program, but if an
implementation automatically derives the interface it must derive the
smallest set of exports.

For optimisation across module boundaries, a compilation system may
need more information (e.g., information about strictness, inlining,
uncurrying, etc.) than is provided by the standard interface as
defined in this report.  Draft proposals exist for including such
information as comments in interfaces; for details, contact the
implementors listed in the preface (page~\pageref{implementors}).

\subsection{Module Implementations} 
\label{module-implementations}
\index{module!implementation}

A module implementation\index{implementation} defines a mutually
recursive scope containing declarations for value bindings, data
types, type synonyms, classes, etc. (see Section~\ref{declarations}).

\begin{flushleft}\it\begin{tabbing}
\hspace{0.5in}\=\hspace{3.0in}\=\kill
$\it module$\>\makebox[3.5em]{$\rightarrow$}$\it \makebox{\tt module}\ modid\ [exports]\ \makebox{\tt where}\ body$\\ 
$\it $\>\makebox[3.5em]{$|$}$\it body$\\ 
$\it body$\>\makebox[3.5em]{$\rightarrow$}$\it \makebox{\tt {\char'173}}\ [impdecls\ \makebox{\tt ;}]\ [[fixdecls\ \makebox{\tt ;}]\ topdecls\ [\makebox{\tt ;}]]\ \makebox{\tt {\char'175}}$\\ 
$\it $\>\makebox[3.5em]{$|$}$\it \makebox{\tt {\char'173}}\ impdecls\ [\makebox{\tt ;}]\ \makebox{\tt {\char'175}}$\\ 
$\it $\\ 
$\it modid$\>\makebox[3.5em]{$\rightarrow$}$\it aconid$\\ 
$\it impdecls$\>\makebox[3.5em]{$\rightarrow$}$\it impdecl_1\ \makebox{\tt ;}\ \ldots \ \makebox{\tt ;}\ impdecl_n$\>\makebox[3em]{}$\it \qquad\ (n\geq 1)$\\ 
$\it topdecls$\>\makebox[3.5em]{$\rightarrow$}$\it topdecl_1\ \makebox{\tt ;}\ \ldots \ \makebox{\tt ;}\ topdecl_n$\>\makebox[3em]{}$\it \qquad\ (n\geq 0)$
\end{tabbing}\end{flushleft}
\indexsyn{module}%
\indexsyn{body}%
\indexsyn{modid}%
\indexsyn{impdecls}%
\indexsyn{topdecls}%

% [:: [context =>] type]                TYPE SIGS NOT ALLOWED NOW
% EXPOSE IS OUT
% topdecl        -> expose [modid] entities where {topdecls }
%         |  \ldots \tr{(see Section~\ref{declarations})}

%\indextt{module}
A module implementation begins with a header: the keyword
\mbox{\tt module}, the module name, and a list of entities (enclosed in round
parentheses) to be exported.  The header is followed by an optional
list of \mbox{\tt import} declarations that specify modules to be imported,
optionally restricting and renaming the imported bindings.  This is
followed by an optional list of fixity declarations and the module
body.  The module body is simply a list of top-level declarations
($topdecls$), as described in Section~\ref{declarations}.  

An abbreviated form of module is permitted, which consists only of
the module body.  If this is used, the header is assumed to be
\mbox{\tt module\ Main\ where}.
If the first lexeme in the
abbreviated module is not a \mbox{\tt {\char'173}}, then the layout rule applies
for the top level of the module.
It is inadvisable
for a compilation system to permit 
an abbreviated module to appear in the same file as some
unabbreviated modules.

\subsubsection{Export Lists}
\label{export}
\index{export list}

\begin{flushleft}\it\begin{tabbing}
\hspace{0.5in}\=\hspace{3.0in}\=\kill
$\it exports$\>\makebox[3.5em]{$\rightarrow$}$\it \makebox{\tt (}\ export_1\ \makebox{\tt ,}\ \ldots \ \makebox{\tt ,}\ export_n\ \makebox{\tt )}$\>\makebox[3em]{}$\it \qquad\ (n\geq 1)$\\ 
$\it $\\ 
$\it export$\>\makebox[3.5em]{$\rightarrow$}$\it entity$\\ 
$\it $\>\makebox[3.5em]{$|$}$\it modid\ \makebox{\tt ..}$\\ 
$\it $\\ 
$\it entity$\>\makebox[3.5em]{$\rightarrow$}$\it varid$\\ 
$\it $\>\makebox[3.5em]{$|$}$\it tycon$\\ 
$\it $\>\makebox[3.5em]{$|$}$\it tycon\ \makebox{\tt (..)}$\\ 
$\it $\>\makebox[3.5em]{$|$}$\it tycon\ \makebox{\tt (}\ conid_1\ \makebox{\tt ,}\ \ldots \ \makebox{\tt ,}\ conid_n\ \makebox{\tt )}$\>\makebox[3em]{}$\it \qquad\ (n\geq 1)$\\ 
$\it $\>\makebox[3.5em]{$|$}$\it tycls\ \makebox{\tt (..)}$\\ 
$\it $\>\makebox[3.5em]{$|$}$\it tycls\ \makebox{\tt (}\ varid_1\ \makebox{\tt ,}\ \ldots \ \makebox{\tt ,}\ varid_n\ \makebox{\tt )}$\>\makebox[3em]{}$\it \qquad\ (n\geq 0)$
\end{tabbing}\end{flushleft}
\indexsyn{exports}%
\indexsyn{export}%
\indexsyn{entity}%
% [\mbox{\tt ::} [context \mbox{\tt =>}] type]            TYPE SIGS NOT ALLOWED NOW

An {\em export list} identifies the entities to be exported by a
module declaration.  A module implementation may only export an entity
that it declares, or that it imports from some other module.  If the
export list is omitted, all values, types and classes defined in the
module are exported, {\em but not those that are imported}.

Entities in an export list may be named as follows:
\begin{enumerate}
\item
Ordinary values, whether declared in the implementation
body or imported,
% (including within a named \mbox{\tt expose} declaration) 
may be named by giving the name of the value as a $varid$.
Operators should be enclosed in parentheses to turn them into
$varid$'s.
% Such a name may be annotated with a type signature if it designates an
% ordinary value.
\item
A type synonym $T$ declared by a \mbox{\tt type} declaration
may be named by simply giving the name of the type.
\index{type synonym}
\item
An algebraic datatype $T$ with constructors $K_1,\ldots,K_n$
\index{algebraic datatype}
declared by a \mbox{\tt data} declaration may be named in one of three ways:
\begin{itemize}
\item
The form $T$ names the type {\em but not the constructors}.
The ability to export a type without its constructors allows the
construction of abstract datatypes (see Section~\ref{abstract-types}).
\item
The form $T\mbox{\tt (}K_1\mbox{\tt ,}\ldots\mbox{\tt ,}K_n\mbox{\tt )}$, where
{\em all} and only the constructors are listed without duplications,
names the type and {\em all} its constructors.  
\item
The abbreviated form $T\mbox{\tt (..)}$ also names the type 
and all its constructors.
\end{itemize}
Data constructors may not be named in export lists in any other way.
\item
A class $C$ with operations $f_1,\ldots,f_n$
declared in a \mbox{\tt class} declaration may be named in one of two ways, both of which
name the class together with all its operations:
\index{class declaration}
\begin{itemize}
\item
The form $C\mbox{\tt (}f_1\mbox{\tt ,}\ldots\mbox{\tt ,}f_n\mbox{\tt )}$, where
all and only the operations in that class are listed without duplications.
\item
The abbreviated form $C\mbox{\tt (..)}$.
\end{itemize}
Operators in a class may not be named in export lists in any other way.
\item
The set of all entities brought into scope (after renaming) from a
module $m$ by one or more \mbox{\tt import} declarations may be named by the
form $m\mbox{\tt ..}$, which is equivalent to listing all of the entities
imported from the module.  For example,
\bprog
\mbox{\tt \ \ \ \ \ \ module\ Queue(\ Stack..,\ enqueue,\ dequeue\ )\ where}\\
\mbox{\tt \ \ \ \ \ \ \ \ \ \ import\ Stack}\\
\mbox{\tt \ \ \ \ \ \ \ \ \ \ ...}
\eprog
Here the module \mbox{\tt Queue} uses the module name \mbox{\tt Stack} in its export
list to abbreviate all the entities imported from \mbox{\tt Stack}.

It is a static error to have circular dependencies between
imports/exports using this naming convention.  For example, the
following is not allowed:
\bprog
\mbox{\tt module\ X(\ Y..\ )\ \ \ \ \ --\ ILLEGAL}\\
\mbox{\tt import\ Y\ \ \ \ \ \ \ \ \ \ \ \ --}\\
\mbox{\tt x\ =\ 1\ \ \ \ \ \ \ \ \ \ \ \ \ \ \ --}\\
\mbox{\tt }\\[-8pt]
\mbox{\tt module\ Y(\ X..\ )\ \ \ \ \ --}\\
\mbox{\tt import\ X\ \ \ \ \ \ \ \ \ \ \ \ --}\\
\mbox{\tt y\ =\ 1\ \ \ \ \ \ \ \ \ \ \ \ \ \ \ --}
\eprogNoSkip
\item
A module can name its own local definitions in its export
list using its own name in the \mbox{$\it m\makebox{\tt ..}$} syntax.  For example,
\bprog
\mbox{\tt \ \ \ \ \ \ \ module\ Mod1(Mod1..,\ Mod2..)}\\
\mbox{\tt \ \ \ \ \ \ \ import\ Mod2}\\
\mbox{\tt \ \ \ \ \ \ \ import\ Mod3}
\eprog
Here module \mbox{\tt Mod1} exports all local definitions as well as those
from \mbox{\tt Mod2} but not \mbox{\tt Mod3}.
\end{enumerate}

\subsubsection{Import Declarations}
\label{import}
\index{import declaration}

\begin{flushleft}\it\begin{tabbing}
\hspace{0.5in}\=\hspace{3.0in}\=\kill
$\it impdecl$\>\makebox[3.5em]{$\rightarrow$}$\it \makebox{\tt import}\ modid\ [impspec]\ [\makebox{\tt renaming}\ renamings]$\\ 
$\it impspec$\>\makebox[3.5em]{$\rightarrow$}$\it \makebox{\tt (}\ import_1\ \makebox{\tt ,}\ \ldots \ \makebox{\tt ,}\ import_n\ \makebox{\tt )}$\>\makebox[3em]{}$\it \qquad\ (n\geq 0)$\\ 
$\it $\>\makebox[3.5em]{$|$}$\it \makebox{\tt hiding}\ \makebox{\tt (}\ import_1\ \makebox{\tt ,}\ \ldots \ \makebox{\tt ,}\ import_n\ \makebox{\tt )}$\>\makebox[3em]{}$\it \qquad\ (n\geq 1)$\\ 
$\it import$\>\makebox[3.5em]{$\rightarrow$}$\it entity$\\ 
$\it renamings$\>\makebox[3.5em]{$\rightarrow$}$\it \makebox{\tt (}\ renaming_1\ \makebox{\tt ,}\ \ldots \ \makebox{\tt ,}\ renaming_n\ \makebox{\tt )}$\>\makebox[3em]{}$\it \qquad\ (n\geq 1)$\\ 
$\it renaming$\>\makebox[3.5em]{$\rightarrow$}$\it varid_1\ \makebox{\tt to}\ varid_2$\\ 
$\it $\>\makebox[3.5em]{$|$}$\it conid_1\ \makebox{\tt to}\ conid_2$
\end{tabbing}\end{flushleft}
\indexsyn{impdecl}%
\indexsyn{impspec}%
\indexsyn{import}%
\indexsyn{renamings}%
\indexsyn{renaming}%
% [\mbox{\tt ::} [context \mbox{\tt =>}] type]            TYPE SIGS NOT ALLOWED NOW

The entities exported by a module may be brought into scope in
another module with
an \mbox{\tt import}
declaration at the beginning
of the module.  
The \mbox{\tt import} declaration names the module to be
imported, optionally specifies the entities to be imported,
and optionally provides renamings for imported entities.
A single module may be imported by more
than one \mbox{\tt import} declaration.

Exactly which entities are to be imported can be specified in one
of three ways:\nopagebreak[4]
\begin{enumerate}
\item
The set of entities to be imported can be specified explicitly
by listing them in parentheses.
Items in the list have the same form as those in export lists, except
that the $modid$ abbreviation is not permitted.

The list must name a subset of the 
entities exported by the imported module.
The list may be empty, in which case nothing is imported;
this is only useful in the case of the module \mbox{\tt Prelude} (see
Section~\ref{std-prel-shadowing}).
\item
Specific entities can be excluded by 
using the form \mbox{\tt hiding(} $import_1 \mbox{\tt ,} ... \mbox{\tt ,} import_n$ \mbox{\tt )}, which
specifies that all entities exported by the named module should
be imported apart from those named in the list.
\item
Finally, if $impspec$ is omitted then 
all the entities exported by the specified module are imported.
\end{enumerate}

As instance declarations do not have names, their import cannot be
controlled by the \mbox{$\it impspec$} list.  Instead, the following rule is
used: {\em A $C$-$T$ instance declaration is imported from an
interface if and only if $C$ is imported or $T$ is imported
from that interface.}

Some or all of the imported entities may be renamed,\index{renaming}
thus allowing them to be known by a new name in the importing scope
(see Section~\ref{original-names}).  This is done using the
\mbox{\tt renaming} keyword, with a renaming of the form $oldname\mbox{\tt \ to\ }newname$.
%All renaming is subject to the constraint that each name in a scope
%must refer to exactly one entity; however, a single entity may be given
%more than one name.

\subsection{Module Interfaces}
\label{module-interfaces}
\index{module!interface}

Every module has an {\em interface}\index{interface}
containing all the information needed to do static checks on any
importing module.  All static checks on a module implementation can be
done by inspecting its text and the interfaces of the modules
it imports.

%
%interface-> \mbox{\tt interface} modid \mbox{\tt where} ibody
%
%ibody    -> \mbox{\tt {\char'173}} [ [fixdecls \mbox{\tt ;}] itopdecls] \mbox{\tt {\char'175}}
%itopdecls -> itopdecl_1 \mbox{\tt ;} ... \mbox{\tt ;} itopdecl_n  & \qquad (n>=1) 
%itopdecl  -> \mbox{\tt type} simple \mbox{\tt =} type
%          | \mbox{\tt data} [context \mbox{\tt =>}] simple [\mbox{\tt =} constrs [\mbox{\tt deriving} (tycls | \mbox{\tt (}tyclses\mbox{\tt )})]]
%          | \mbox{\tt class} [context \mbox{\tt =>}] class [\mbox{\tt where} \mbox{\tt {\char'173}} icdecls [\mbox{\tt ;}] \mbox{\tt {\char'175}}]
%          | \mbox{\tt instance} [context \mbox{\tt =>}] tycls inst
%          | vars \mbox{\tt ::} [context \mbox{\tt =>}] type
%icdecls           -> icdecl_1 \mbox{\tt ;} ... \mbox{\tt ;} icdecl_n  & (n>=1)
%icdecl    -> vars \mbox{\tt ::} type
%

\begin{flushleft}\it\begin{tabbing}
\hspace{0.5in}\=\hspace{3.0in}\=\kill
$\it interface$\>\makebox[3.5em]{$\rightarrow$}$\it \makebox{\tt interface}\ modid\ \makebox{\tt where}\ ibody$\\ 
$\it $\\ 
$\it ibody$\>\makebox[3.5em]{$\rightarrow$}$\it \makebox{\tt {\char'173}}\ [iimpdecls\ \makebox{\tt ;}]\ [fixdecls\ \makebox{\tt ;}]\ itopdecls\ [\makebox{\tt ;}]\ \makebox{\tt {\char'175}}$\\ 
$\it $\>\makebox[3.5em]{$|$}$\it \makebox{\tt {\char'173}}\ iimpdecls\ [\makebox{\tt ;}]\ \makebox{\tt {\char'175}}$\\ 
$\it iimpdecls$\>\makebox[3.5em]{$\rightarrow$}$\it iimpdecl_1\ \makebox{\tt ;}\ \ldots \ \makebox{\tt ;}\ iimpdecl_n$\>\makebox[3em]{}$\it \qquad\ (n\geq 1)$\\ 
$\it iimpdecl$\>\makebox[3.5em]{$\rightarrow$}$\it \makebox{\tt import}\ modid\ \makebox{\tt (}\ import_1\ \makebox{\tt ,}\ \ldots \ \makebox{\tt ,}\ import_n\ \makebox{\tt )}$\\ 
$\it $\>\makebox[3em]{}$\it [\makebox{\tt renaming}\ renamings]$\>\makebox[3em]{}$\it \qquad\ (n\geq 1)$\\ 
$\it itopdecls$\>\makebox[3.5em]{$\rightarrow$}$\it itopdecl_1\ \makebox{\tt ;}\ \ldots \ \makebox{\tt ;}\ itopdecl_n$\>\makebox[3em]{}$\it \qquad\ (n\geq 1)$\\ 
$\it itopdecl$\>\makebox[3.5em]{$\rightarrow$}$\it \makebox{\tt type}\ simple\ \makebox{\tt =}\ type$\\ 
$\it $\>\makebox[3.5em]{$|$}$\it \makebox{\tt data}\ [context\ \makebox{\tt =>}]\ simple\ [\makebox{\tt =}\ constrs]\ [\makebox{\tt deriving}\ (tycls\ |\ \makebox{\tt (}tyclses\makebox{\tt )})]$\\ 
$\it $\>\makebox[3.5em]{$|$}$\it \makebox{\tt class}\ [context\ \makebox{\tt =>}]\ class\ [\makebox{\tt where}\ \makebox{\tt {\char'173}}\ icdecls\ [\makebox{\tt ;}]\ \makebox{\tt {\char'175}}]$\\ 
$\it $\>\makebox[3.5em]{$|$}$\it \makebox{\tt instance}\ [context\ \makebox{\tt =>}]\ tycls\ inst$\\ 
$\it $\>\makebox[3.5em]{$|$}$\it vars\ \makebox{\tt ::}\ [context\ \makebox{\tt =>}]\ type$\\ 
$\it icdecls$\>\makebox[3.5em]{$\rightarrow$}$\it icdecl_1\ \makebox{\tt ;}\ \ldots \ \makebox{\tt ;}\ icdecl_n$\>\makebox[3em]{}$\it (n\geq 1)$\\ 
$\it icdecl$\>\makebox[3.5em]{$\rightarrow$}$\it vars\ \makebox{\tt ::}\ type$
\end{tabbing}\end{flushleft}
\indexsyn{interface}%
\indexsyn{ibody}%
\indexsyn{iimpdecls}%
\indexsyn{iimpdecl}%
\indexsyn{itopdecls}%
\indexsyn{itopdecl}%
\indexsyn{icdecls}%
\indexsyn{icdecl}%

The syntax of \mbox{\tt interface} is similar to that of \mbox{\tt module}, except:
\begin{itemize}
\item
There is no export list: everything in the interface is exported.
\item
\mbox{\tt import} declarations have a
slightly different purpose from those
in implementations (see Section~\ref{interface-imports}).
The list of entities to be imported is always specified explicitly.
\item
\mbox{\tt data} declarations appear without their constructors if these
are not exported.
\item
There is no implementation part to \mbox{\tt instance} declarations.
\item
Value declarations do not appear at all; for exported values, type signatures
take their place.
\end{itemize}


%The syntax of \mbox{\tt interface} is similar to that of \mbox{\tt module}, except:
%\begin{itemize}
%\item
%In interface files, the syntax for \mbox{$\it var$}, \mbox{$\it tycon$}, and \mbox{$\it tycls$} is expanded 
%to include full names, and thus the productions in Section~\ref{ids} should
%be replaced by:
%
%var    ->  varid [\mbox{\tt {\char'173}}modid [varid]\mbox{\tt {\char'175}}]         & (variables)
%tycon  ->  aconid [\mbox{\tt {\char'173}}modid [aconid]\mbox{\tt {\char'175}}]       & (type constructors)
%tycls  ->  aconid [\mbox{\tt {\char'173}}modid [aconid]\mbox{\tt {\char'175}}]       & (type classes)
%
%which are BNF rules (i.e.~not part of the lexical syntax) and thus
%whitespace is allowed between lexemes.
%\item
%There is no export list: everything in the interface is exported.
%\item
%There are no \mbox{\tt import} declarations.
%\item
%\mbox{\tt data} declarations appear without their constructors if these
%are not exported.
%\item
%There is no implementation part to \mbox{\tt instance} declarations.
%\item
%Value declarations do not appear at all; for exported values, type signatures
%take their place.
%\end{itemize}

\subsubsection{Consistency}
\label{consistency}

The interface and implementation of a module must obey certain
constraints.
(In the following,
the phrase ``in the implementation'' refers to something
either declared within the implementation or imported by it.)
\begin{enumerate}
%\item
%The names of entities exported by a module, and types or classes not
%exported but 
%referred to in an interface must appear as full names. (See
%Section~\ref{interface-imports}).
\item
Every entity given a declaration in an interface must either have an
import declaration for the entity in the interface (the import
specifies the module that defines it) or have a definition of the
entity in the implementation.  Furthermore, if an interface A imports
an entity X from module B (perhaps renaming it), then the interface
for B must define X but not import it.

\item
A class, type synonym, algebraic datatype, or value appears in the
\index{algebraic datatype}
\index{type synonym}
interface exactly when its name appears in the implementation's export
list or, if the export list is omitted, when it is {\em declared} in
the implementation.

\item
A type signature appears in the interface for every value that the
implementation exports.  
%Apart from any required use of full names, 
This
type signature must be the same as that 
in the implementation (see Section~\ref{type-semantics}), where
the latter is obtained from the explicit type signature in the
implementation (when present) or is the most general type
inferred from the declaration of the value.

\item
%Apart from any required use of full names, 
A \mbox{\tt type} declaration in an
interface must be identical to that in the implementation.

\item
%Apart from any required use of full names,  
A \mbox{\tt class} declaration in an
interface must be identical to that in the implementation, except that
default-method declarations are omitted.
\index{default method}

\item
If the constructors of a \mbox{\tt data} declaration are not exported, 
%then, apart from any required use of full names,
then the \mbox{\tt data} declaration in the interface differs from that in the
implementation by omitting everything after (and including) the \mbox{\tt =}
sign.  If the \mbox{\tt data} declaration in the implementation uses the
\mbox{\tt deriving} mechanism to derive instance declarations for the type, a
separate \mbox{\tt instance} declaration must appear in the interface for each
class of which the type is made an instance of.  Hence, the
information that certain instances are derived is hidden when the
constructors are hidden, since in this case the type is abstract (see
Section~\ref{abstract-types}).
% The interface 
% should not change in any way if the implementation were to change an
% instance from being derived to being explicit, or vice versa.

\item
If the constructors of a \mbox{\tt data} declaration are to be exported, then
the \mbox{\tt data} declaration in the interface is identical to that in the
implementation including the \mbox{\tt deriving} part.\footnote{ It is important
to retain the information about which instances are derived and which
are not, because the importing module ``knows'' more about derived
instances.}

\item
If a $C$-$T$ instance is declared in a module or imported by it, then the
instance declaration appears in the interface (omitting the \mbox{\tt where}
part) if {\em either} $C$ is exported {\em or} $T$ is exported.
Instance declarations are not named explicitly in export or
import lists.  This rule ensures that, if $C$ and $T$ are both in
scope, then the (unique) $C$-$T$ instance declaration will also be in
scope.\footnote{The reverse also applies.  For example,
suppose that a new type $T$ is declared and made an instance of an
imported class $C$.  The instance declaration will be exported along
with $T$, and so the closure rule (Section~\ref{closure}) will require
that $C$ is also in scope in every importing scope.}

No explicit instance declaration should appear in the interface for
instances that are specified by the \mbox{\tt deriving} part of a \mbox{\tt data}
declaration in the interface.

\item
A fixity declaration for a value or constructor appears in an
interface exactly when (a)~the value or constructor is declared by the
interface, and (b)~the identical fixity declaration appears either in
the implementation or in an imported interface.
\end{enumerate}

%\subsubsection{Full Names and Original Names in Interfaces}
%\label{interface-imports}
%\index{original name!in an interface}
%\index{full name!in an interface}
%
%An entity defined in module \mbox{\tt A} as \mbox{\tt f} has an {\em original name} which is
%a pair of \mbox{\tt A} and \mbox{\tt f}, but is never written except as part of a {\em full
%name}. 
%
%Full names, or permitted abbreviations of full names are used in
%interfaces for the names of type synonyms, datatypes, classes and values.
%
%The {\em full name} of an entity defines its original name and its
%current renaming, if any. The {\em full name} of an entity defined in
%module \mbox{\tt A} as \mbox{\tt f}, and subsequently renamed as \mbox{\tt g}, is \mbox{\tt g\ {\char'173}A\ f{\char'175}}. The
%{\em full name} of an entity defined in module \mbox{\tt A} as
%\mbox{\tt f}, and not subsequently renamed, is \mbox{\tt f\ {\char'173}A\ f{\char'175}}. The {\em local name}
%\index{local name}
%of an entity is the first component of its full name, i.e. excluding
%the part in braces.
%
%In the following, a {\em defining occurrence} of a name is an occurence in
%one of the following positions in an {\em itopdecl}, indicating that the
%entity associated with the name is exported.
%\begin{enumerate}
%\item
%Immediately following the keyword {\it type}, or {\em data} or {\em class},
%or immediately following the \mbox{\tt =>} terminating a context immediately
%following one of these keywords.
%\item
%As a {\em var} before the \mbox{\tt ::} in a type signature corresponding to a value
%declaration in the module implementation.
%\end{enumerate}
%
%Full names may be abbreviated in some circumstances:
%\begin{enumerate}
%\item
%For an entity that has not been renamed, \mbox{\tt f\ {\char'173}A\ f{\char'175}} may be written as \mbox{\tt f\ {\char'173}A{\char'175}}.
%\item
%For an entity defined in the current module, no module name need be given,
%so in the interface for module \mbox{\tt A}, \mbox{\tt f\ {\char'173}A\ f{\char'175}} may be written as \mbox{\tt f}.
%\item
%If the full name \mbox{\tt g\ {\char'173}A\ f{\char'175}} or any abbreviation of it appears in a defining
%occurence in the current interface, all other occurrences of that name in the
%interface may use the local name \mbox{\tt g}.
%\item
%The names of data constructors (in {\em constrs}) and class
%operations (in {\em icdels}) may be abbreviated to the local name.
%
%\item
%Entities imported from the standard prelude without renaming may always use
%the local name.
%\end{enumerate}

\subsubsection{Imports and Original Names}
\label{interface-imports}
\index{original name!in an interface}

The original-name information is carried in the interface file using
\mbox{\tt import} declarations in a special way.

Suppose that a module \mbox{\tt A} exports an entity \mbox{\tt x}; the
interface for \mbox{\tt A} will contain static information about \mbox{\tt x}.  
If \mbox{\tt x} was originally defined in \mbox{\tt A}, then this is
all that appears.
But, suppose that \mbox{\tt x} was imported by \mbox{\tt A} from some other module \mbox{\tt B}
and that \mbox{\tt x} was originally defined in module \mbox{\tt C}
with name \mbox{\tt y}; this declaration must appear in the interface for \mbox{\tt A}:
\bprog
\mbox{\tt import\ C(y)\ renaming\ (\ y\ to\ x\ )}
\eprog
No reference to \mbox{\tt B} remains in the interface.  {\em The
\mbox{\tt import} declaration in the interface serves only to convey to the
importing module the original name of \mbox{\tt x}}, and does {\em not} imply
that module \mbox{\tt C}'s interface must be consulted when reading module \mbox{\tt A}'s
interface.  Multiple imports from a single original module may
optionally be grouped in a single import declaration in the interface.

A module may export a value whose typing involves a type and/or class
that is not exported.
(Any importing module would have to import the type or class by some
other route.)
{\em Nevertheless, it is still required that the interface contain the
import declaration required to give the original name of the type or class.}

In summary, for every entity \mbox{\tt e1} mentioned in the interface 
of a module \mbox{\tt M} whose original name is \mbox{\tt e2} in module \mbox{\tt N},
\mbox{\tt M}'s interface must
contain the \mbox{\tt import} declaration
\bprog
\mbox{\tt \ \ \ \ \ \ import\ N(e2)\ renaming\ (\ e2\ to\ e1\ )}
\eprog
The word ``mentioned'' includes mention in the type signature
of an exported value, as discussed above.

This example illustrates most of these constraints; first, the
interface:
\bprog
\mbox{\tt interface\ A\ where}\\
\mbox{\tt infixr\ 4\ }\bkqB\mbox{\tt sameShape}\bkqA\mbox{\tt \ \ \ \ \ \ \ \ }\\
\mbox{\tt import\ PreludeList(sum)\ renaming\ (\ sum\ to\ oldSum\ )}\\
\mbox{\tt data\ \ BinTree\ a\ =\ Empty\ |\ Branch\ a\ (BinTree\ a)\ (BinTree\ a)\ \ \ \ \ \ }\\
\mbox{\tt class\ Tree\ a\ where\ }\\
\mbox{\tt \ \ \ \ \ \ \ \ sameShape\ ::\ a\ ->\ a\ ->\ Bool}\\
\mbox{\tt instance\ Tree\ (BinTree\ a)}\\
\mbox{\tt sum\ ::\ Num\ a\ =>\ BinTree\ a\ ->\ a}\\
\mbox{\tt oldSum\ ::\ Num\ a\ =>\ [a]\ ->\ a}
\eprog
Now the implementation:
\bprog
\mbox{\tt module\ A(\ BinTree(..),\ Tree(..),\ sum,\ oldSum\ )\ where}\\
\mbox{\tt import\ Prelude\ renaming\ (\ sum\ to\ oldSum\ )}\\
\mbox{\tt infixr\ 4\ }\bkqB\mbox{\tt sameShape}\bkqA\mbox{\tt }\\
\mbox{\tt \ \ \ \ \ \ \ \ --\ }\bkqB\mbox{\tt sameShape}\bkqA\mbox{\tt \ is\ an\ operation\ of\ class\ C\ below}\\
\mbox{\tt }\\[-8pt]
\mbox{\tt data\ BinTree\ a\ =\ Empty\ |\ Branch\ a\ (BinTree\ a)\ (BinTree\ a)}\\
\mbox{\tt }\\[-8pt]
\mbox{\tt class\ Tree\ a\ where\ }\\
\mbox{\tt \ \ \ \ \ \ sameShape\ ::\ a\ ->\ a\ ->\ Bool}\\
\mbox{\tt \ \ \ \ \ \ t1\ }\bkqB\mbox{\tt sameShape}\bkqA\mbox{\tt \ t2\ =\ False\ \ \ \ \ --\ Default\ method}\\
\mbox{\tt }\\[-8pt]
\mbox{\tt instance\ Tree\ (BinTree\ a)\ where\ }\\
\mbox{\tt \ \ \ \ \ \ \ \ \ Empty\ }\bkqB\mbox{\tt sameShape}\bkqA\mbox{\tt \ Empty\ \ =\ \ True}\\
\mbox{\tt \ \ \ \ \ \ \ \ \ (Branch\ {\char'137}\ t1\ t2)\ }\bkqB\mbox{\tt sameShape}\bkqA\mbox{\tt \ (Branch\ {\char'137}\ t1'\ t2')\ \ }\\
\mbox{\tt \ \ \ \ \ \ \ \ \ \ \ \ =\ \ (t1\ }\bkqB\mbox{\tt sameShape}\bkqA\mbox{\tt \ t1')\ {\char'46}{\char'46}\ (t2\ }\bkqB\mbox{\tt sameShape}\bkqA\mbox{\tt \ t2')}\\
\mbox{\tt \ \ \ \ \ \ \ \ \ t1\ }\bkqB\mbox{\tt sameShape}\bkqA\mbox{\tt \ t2\ =\ False}\\
\mbox{\tt }\\[-8pt]
\mbox{\tt sum\ \ Empty\ \ \ \ \ \ \ \ \ \ \ =\ 0}\\
\mbox{\tt sum\ (Branch\ n\ t1\ t2)\ =\ n\ +\ sum\ t1\ +\ sum\ t2}
\eprogNoSkip

\subsection{Standard Prelude}
\label{standard-prelude}
\index{standard prelude}

Many of the features of \Haskell{} are defined in \Haskell{}
itself, as a library of standard datatypes, classes and
functions, called the ``standard prelude.''  In
\Haskell{}, the standard prelude is specified as two distinct modules
(in the technical sense of this chapter), \mbox{\tt PreludeCore} and
\mbox{\tt Prelude}.\indexmodule{Prelude}
\indexmodule{PreludeCore}

\mbox{\tt PreludeCore} and \mbox{\tt Prelude} differ from other modules in that
{\em their interfaces,
and the semantics of the entities defined by those interfaces, are
part of the \Haskell{} language definition.} This means, for example, that a
compiler may optimise calls to functions in the
standard prelude, because it knows their semantics as well as their
interface.

Each of these modules is structured into submodules.
To avoid name-clashes with these sub-modules,
user-defined module names must not begin with the prefix \mbox{\tt Prelude}.

\subsubsection{The \mbox{\tt PreludeCore} Module}
\indexmodule{PreludeCore}

The \mbox{\tt PreludeCore} module contains {\em all the algebraic datatypes,
type synonyms, classes and instance declarations} specified by the
standard prelude.
\index{algebraic datatype}
\index{type synonym}
\index{class declaration}
\index{instance declaration}

\mbox{\tt PreludeCore} is {\em always implicitly imported}, so it
is not possible to import only part of it or to rename any of the
entities that it defines.

The semantics of the entities defined by \mbox{\tt PreludeCore} is specified by
an implementation written in \Haskell{}, in
Appendix~\ref{preludecore}.
A \Haskell{} system need not implement \mbox{\tt PreludeCore} in this way.
The interface for \mbox{\tt PreludeCore} may
be inferred from the implementation in
Appendix~\ref{preludecore}.

Some datatypes (such as \mbox{\tt Int}) and functions (such as addition of
\mbox{\tt Int}s) cannot be specified directly in \Haskell{}.  This is
expressed in the \mbox{\tt PreludeCore} implementation by importing these
built-in types and values from
\mbox{\tt PreludeBuiltin}.\indexmodule{PreludeBuiltin}
The semantics of the built-in datatypes and
functions is given as English text in
Appendix~\ref{preludebuiltin}.

The implementation for \mbox{\tt PreludeCore} is incomplete in its
treatment of tuples: there should be an infinite family of instance
declarations for tuples, but the implementation only gives a scheme.

The alert reader may notice that the implementation of
\mbox{\tt PreludeCore} given in Appendix~\ref{preludecore} uses
some functions defined in \mbox{\tt Prelude} (see next section).  There is no
conflict; \mbox{\tt PreludeCore} and \mbox{\tt Prelude} are mutually
recursive.

\subsubsection{The \mbox{\tt Prelude} Module}
\indexmodule{Prelude}

The \mbox{\tt Prelude} module contains all the {\em value} declarations
in the standard prelude.

The \mbox{\tt Prelude} module is imported automatically if and only if it is
not imported with an explicit \mbox{\tt import} declaration.  This provision
for explicit import allows values defined in the standard prelude to
be renamed or not imported at all.

The semantics of the entities in \mbox{\tt Prelude} is specified by
an implementation of \mbox{\tt Prelude} written in \Haskell{}, given in
Appendix~\ref{stdprelude}.  As for \mbox{\tt PreludeCore},
a \Haskell{} system may implement the \mbox{\tt Prelude} module
as it pleases, provided it maintains the semantics in
Appendix~\ref{stdprelude}.  The interface can be inferred
from this implementation.

\subsubsection{Shadowing Prelude Names and Non-Standard Preludes}
\label{std-prel-shadowing}

The rules about the standard prelude have been cast so that it is
possible to use standard prelude names for nonstandard purposes; however,
every module that does so will have an \mbox{\tt import} declaration
that makes this nonstandard usage explicit.  For example:
\bprog
\mbox{\tt \ \ \ \ \ module\ A\ where}\\
\mbox{\tt \ \ \ \ \ import\ Prelude\ hiding\ (map)}\\
\mbox{\tt \ \ \ \ \ map\ f\ x\ =\ x\ f}
\eprog
Module \mbox{\tt A} redefines \mbox{\tt map}, but it must indicate this by
importing \mbox{\tt Prelude} without \mbox{\tt map}.  Furthermore, \mbox{\tt A} exports \mbox{\tt map},
but every module that imports \mbox{\tt map} from \mbox{\tt A} must also hide
\mbox{\tt map} from \mbox{\tt Prelude} just as \mbox{\tt A} does.  Thus there is little danger of accidentally
shadowing standard prelude names.

It is possible to construct and use a different \mbox{\tt Prelude} module:
\bprog
\mbox{\tt \ \ \ \ \ \ module\ B\ where}\\
\mbox{\tt \ \ \ \ \ \ import\ Prelude()}\\
\mbox{\tt \ \ \ \ \ \ import\ MyPrelude}\\
\mbox{\tt \ \ \ \ \ \ ...}
\eprog
\mbox{\tt B} imports nothing from \mbox{\tt Prelude}, but the
explicit \mbox{\tt import\ Prelude} declaration prevents the automatic import of
\mbox{\tt Prelude}.  \mbox{\tt import\ MyPrelude} brings the
non-standard prelude into scope.  As before, the
standard prelude names are hidden explicitly.

%It is not possible to rename or hide the entities imported
%from \mbox{\tt PreludeCore}.

\subsection{Example}

As an example, here are two small modules:\nopagebreak
\bprog
\mbox{\tt module\ A(\ Tree(..),\ depth\ )\ where}\\
\mbox{\tt data\ Tree\ a\ =\ Leaf\ a\ |\ Branch\ (Tree\ a)\ (Tree\ a)}\\
\mbox{\tt depth\ (Leaf\ a)\ \ \ \ \ \ \ \ =\ \ 0}\\
\mbox{\tt depth\ (Branch\ xt\ yt)\ \ =\ \ (depth\ xt\ }\bkqB\mbox{\tt max}\bkqA\mbox{\tt \ depth\ yt)\ +\ 1}\\
\mbox{\tt }\\[-8pt]
\mbox{\tt module\ B(\ leaves\ )\ where}\\
\mbox{\tt import\ A}\\
\mbox{\tt leaves\ (Leaf\ a)\ \ \ \ \ \ \ \ =\ \ [a]}\\
\mbox{\tt leaves\ (Branch\ xt\ yt)\ \ =\ \ leaves\ xt\ ++\ leaves\ yt}
\eprog
Module \mbox{\tt A} must export \mbox{\tt Tree} because it exports \mbox{\tt depth}, and \mbox{\tt Tree}
could not be made visible in any other way.  However, \mbox{\tt B} is not
required to export \mbox{\tt Tree}, since a module importing \mbox{\tt B} could import
\mbox{\tt A} in order to satisfy the closure constraints.

Modules may be used to combine the resources of other modules.  For
example, one might use renaming to make trees available to
French speakers:
\bprog
\mbox{\tt module\ C(\ Arbre(..),\ fond,\ feuilles\ )\ where}\\
\mbox{\tt import\ A\ renaming\ (\ Tree\ to\ Arbre,\ Leaf\ to\ Feuille,\ Branch\ to\ Branche,}\\
\mbox{\tt \ \ \ \ \ \ \ \ \ \ \ \ \ \ \ \ \ \ \ \ depth\ to\ fond\ )}\\
\mbox{\tt import\ B\ renaming\ (\ leaves\ to\ feuilles\ )}
\eprogNoSkip

\subsection{Abstract Datatypes}
\label{abstract-types}

\index{abstract datatype}
The ability to export a datatype without its constructors
allows the construction of abstract datatypes (ADTs).  For example,
an ADT for stacks could be defined as:
\bprog
\mbox{\tt module\ Stack(\ StkType,\ push,\ pop,\ empty\ )\ where}\\
\mbox{\tt \ \ \ \ \ \ \ \ data\ StkType\ a\ =\ EmptyStk\ |\ Stk\ a\ (StkType\ a)}\\
\mbox{\tt \ \ \ \ \ \ \ \ push\ x\ s\ =\ Stk\ x\ s}\\
\mbox{\tt \ \ \ \ \ \ \ \ pop\ (Stk\ {\char'137}\ s)\ =\ s}\\
\mbox{\tt \ \ \ \ \ \ \ \ empty\ =\ EmptyStk}
\eprog
Modules importing \mbox{\tt Stack} cannot construct values of type \mbox{\tt StkType}
because they do not have access to the constructors of the type.

It is also possible to build an ADT on top of an existing type by
using a \mbox{\tt data} declaration with a single constructor with only one
field.  For example, stacks can be defined with lists:
\bprog
\mbox{\tt module\ Stack(\ StkType,\ push,\ pop,\ empty\ )\ where}\\
\mbox{\tt \ \ \ \ \ \ \ \ data\ StkType\ a\ =\ Stk\ [a]}\\
\mbox{\tt \ \ \ \ \ \ \ \ push\ x\ (Stk\ s)\ =\ Stk\ (x:s)}\\
\mbox{\tt \ \ \ \ \ \ \ \ pop\ (Stk\ (x:s))\ =\ Stk\ s}\\
\mbox{\tt \ \ \ \ \ \ \ \ empty\ =\ Stk\ []}
\eprogNoSkip

{\em Note 1.} Every ADT must be a module (but a
\Haskell{} compilation system may allow multiple modules in a single file).

%Arguably there should be a separate scope-limiting construct to allow
%constructors to be hidden within modules.  This was considered but
%finally discarded on grounds of simplicity.  Remember that

{\em Note 2.} Using a single-constructor single-field \mbox{\tt data}
declaration to create an isomorphic type introduces an unwanted extra
element to the new type, namely \mbox{\mbox{\tt (Stk\ }$\perp$\mbox{\tt )}}, with the
risk of an accompanying small inefficiency in the implementation.  

%It would be possible to introduce a variant of \mbox{\tt data} to avoid this
%problem, but this was omitted on grounds of simplicity.

\subsection{Fixity Declarations}
\index{fixity declaration}
\label{fixity}

\begin{flushleft}\it\begin{tabbing}
\hspace{0.5in}\=\hspace{3.0in}\=\kill
$\it fixdecls$\>\makebox[3.5em]{$\rightarrow$}$\it fix_1\ \makebox{\tt ;}\ \ldots \ \makebox{\tt ;}\ fix_n$\>\makebox[3em]{}$\it \qquad\ (n\geq 1)$\\ 
$\it fix$\>\makebox[3.5em]{$\rightarrow$}$\it \makebox{\tt infixl}\ [digit]\ ops$\\ 
$\it $\>\makebox[3.5em]{$|$}$\it \makebox{\tt infixr}\ [digit]\ ops$\\ 
$\it $\>\makebox[3.5em]{$|$}$\it \makebox{\tt infix\ }\ [digit]\ ops$\\ 
$\it ops$\>\makebox[3.5em]{$\rightarrow$}$\it op_1\ \makebox{\tt ,}\ \ldots \ \makebox{\tt ,}\ op_n$\>\makebox[3em]{}$\it \qquad\ (n\geq 1)$\\ 
$\it op$\>\makebox[3.5em]{$\rightarrow$}$\it varop\ |\ conop$
\end{tabbing}\end{flushleft}
\indexsyn{fixdecls}%
\indexsyn{fix}%
\indexsyn{ops}%
\indexsyn{op}%
%\indextt{infixl}\indextt{infixr}\indextt{infix}
A fixity declaration gives the fixity and binding
precedence of a set of operators.  Fixity declarations must appear only
at the start of a
module
%\footnote{This is to avoid parsing problems that arise when
%fixity declarations appear lexically after the operators to which they
%refer.}
and may only be given for identifiers defined in that module.
Fixity declarations cannot subsequently be overridden, and an
identifier can only have one fixity definition.

There are three kinds of fixity, non-, left- and right-associativity
(\mbox{\tt infix}, \mbox{\tt infixl}, and \mbox{\tt infixr}, respectively), and ten precedence
levels, 0 through 9 (level 0 binds least tightly, and level 9
binds most tightly).  If the \mbox{$\it digit$} is omitted, level 9 is assumed.
Any operator lacking a fixity declaration
is assumed to be \mbox{\tt infixl\ 9} (See Section~\ref{expressions} for more on
the use of fixities).
Figure~\ref{prelude-fixities} lists the fixities and precedences of
the operators defined in the standard prelude.
\begin{figure}
\centerline{
\begin{tabular}{|c|l|l|}\hline
Precedence & Fixity & Operators \\ \hline
9 & \mbox{\tt infixl} & \mbox{\tt !}, \mbox{\tt !!}, \mbox{\tt //} \\
  & \mbox{\tt infixr} & \mbox{\tt .} \\
8 & \mbox{\tt infixr} & \mbox{\tt **}, \mbox{\tt {\char'136}}, \mbox{\tt {\char'136}{\char'136}} \\
7 & \mbox{\tt infixl} & \mbox{\tt {\char'45}}, \mbox{\tt *}, \mbox{\tt :{\char'45}} \\
  & \mbox{\tt infix}  & \mbox{\tt /}, \mbox{\tt `div`}, \mbox{\tt `mod`}, \mbox{\tt `rem`} \\
6 & \mbox{\tt infixl} & \mbox{\tt +}, \mbox{\tt -} \\
  & \mbox{\tt infix}  & \mbox{\tt :+} \\
5 & \mbox{\tt infixr} & \mbox{\tt :}, \mbox{\tt ++} \\
  & \mbox{\tt infix}  & \mbox{\tt {\char'134}{\char'134}} \\
4 & \mbox{\tt infix}  & \mbox{\tt /=}, \mbox{\tt <}, \mbox{\tt <=}, \mbox{\tt ==}, \mbox{\tt >}, \mbox{\tt >=}, \mbox{\tt `elem`}, \mbox{\tt `notElem`} \\
3 & \mbox{\tt infixr} & \mbox{\tt {\char'46}{\char'46}} \\
2 & \mbox{\tt infixr} & \mbox{\tt ||} \\
1 & \mbox{\tt infix}  & \mbox{\tt :=} \\ \hline
\end{tabular}}
\ecaption{Precedences and fixities of prelude-defined operators}
\label{prelude-fixities}
\end{figure}

Fixity is a property of the original name of an identifier or operator
(see Section~\ref{original-names}).  Fixity is not affected by
renaming; the new name has the same fixity as the old one.  The same
fixity attaches to every occurrence of an operator name in a module,
whether at the top level or rebound at an inner level.  For example:
\bprog
\mbox{\tt module\ Foo}\\
\mbox{\tt import\ Bar}\\
\mbox{\tt infix\ 3\ `op`}\\
\mbox{\tt }\\[-8pt]
\mbox{\tt f\ x\ =\ ...\ where\ p\ `op`\ q\ =\ ...}
\eprog
Here \mbox{\tt `op`} has fixity 3 wherever it is in scope, provided \mbox{\tt Bar} does not
export the identifier \mbox{\tt op}.  If \mbox{\tt Bar} does export \mbox{\tt op}, then the example
becomes illegal, because the fixity (or lack thereof) of \mbox{\tt op} is defined 
in \mbox{\tt Bar} (or wherever \mbox{\tt Bar} imported \mbox{\tt op} from).

% Local Variables: 
% mode: latex
% End:
