%
% $Header$
%
% partain:
% in a separate file, because it is included twice (for now);
%  in intro.verb and syntax.verb

\Haskell{} permits the omission of the braces and semicolons by
using {\em layout} to convey the same information.  This allows both
layout-sensitive and -insensitive styles of coding, which
can be freely mixed within one program.  Because layout is
not required, \Haskell{} programs can be straightforwardly
produced by other programs.
% without worrying about deeply nested layout difficulties.

The layout (or ``off-side'') rule\index{off-side rule} takes effect whenever the
open brace is omitted after the keyword \mbox{\tt where}, \mbox{\tt let} or \mbox{\tt of}.
When this happens, the indentation of the next lexeme (whether or not
on a new line) is remembered and the omitted open brace is inserted
(the whitespace preceding the lexeme may include comments).
For each subsequent line, if it contains only whitespace or is
indented more, then the previous item is continued (nothing is
inserted); if it is indented the same amount, then a new item begins
(a semicolon is inserted); and if it is indented less, then the
declaration list ends (a close brace is inserted).  A close brace is
also inserted whenever the syntactic category containing the
declaration list ends; that is, if an illegal lexeme is encountered at a
point where a close brace would be legal, a close brace is inserted.
The layout rule will match only those open braces
that it has inserted; an
open brace that the user has inserted must be
matched by a close brace inserted by the user.

Given these rules, a single newline may actually terminate several
declaration lists.  Also, these rules permit:
\bprog
\mbox{\tt f\ x\ =\ let\ a\ =\ 1;\ b\ =\ 2\ }\\
\mbox{\tt \ \ \ \ \ \ \ \ \ \ g\ y\ =\ exp2\ in\ exp1}
\eprog
making \mbox{\tt a}, \mbox{\tt b} and \mbox{\tt g} all part of the same declaration
list.

To facilitate the use of layout at the top level of a module
(several modules may reside in one file), the keywords
\mbox{\tt module} and \mbox{\tt interface} and the end-of-file token are assumed to occur in column
0 (whereas normally the first column is 1).  Otherwise, all
top-level declarations would have to be indented.
